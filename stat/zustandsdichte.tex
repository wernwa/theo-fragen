\input{../headers/header_script.tex}
\usepackage{amsmath} 



\begin{document}

\textit{29. März 2012}
\input{../headers/authors.tex}

\section*{Zustandsdichte}

Die Zustandsdichte ist definiert 

\begin{align}
  \label{eq:1}
  \boxed{ D(\epsilon) = \sum_{\vec k} \delta(\epsilon - \epsilon(\vec k)) }
\end{align}

Sie gibt die Anzahl der Systemzustäde pro Energieeinheit an. Als praktisch erweist sich eine Zustandsdichte pro Volumen

\begin{align}
  \label{eq:2}
  \mathcal N(\epsilon) = \frac{1}{V} \sum_{\vec k} \delta(\epsilon - \epsilon(\vec k))
\end{align}
Im folgenden wollen wir die Zustandsdichte für spinlose Teilchen in 1, 2 und 3 Dimensionen berechnen. Die Polar- bzw. Kugelkoordianten für \(dk\) lauten in verschiedenen Dimensionen

\begin{align}
  d&=1:\quad dk = dk \label{eq:10.1} \\
  d&=2:\quad d^2k = kdkd\phi \label{eq:10.2}\\
  d&=3:\quad d^3k = k^2dk \sin(\theta)d\theta d\phi \label{eq:10.3}
\end{align}

\subsection*{1D Zustandsdichte}

Für 1 Dimension gilt für die Zustandsdichte

\begin{align}
  \label{eq:3}
  \mathcal N(\epsilon) = \frac{1}{V} \sum_{\vec k} \delta(\epsilon - \epsilon(\vec k))
\end{align}
Wir nehmen an dass die Energiezustände dicht bei einander liegen, deswegen können wir die Summe als ein Integral ausdrücken. Im thermodynamischen Limes gilt

\begin{align}
  \label{eq:4}
\boxed{  \frac{1}{L^d}\sum_{\vec k} \xrightarrow{L\to \infty} \int \frac{d^d k}{(2\pi)^d} }
\end{align}

Für 1-Dimension können wir die Formel (\ref{eq:3}) schreiben (mit Hilfe (\ref{eq:10.3}))

\begin{align}
  \label{eq:5}
  \mathcal N(\epsilon) =  \int_{-\infty}^{\infty} \frac{d k}{(2\pi)} \delta(\epsilon - \epsilon(\vec k))
\end{align}

Wir möchten das Integral nach \(\epsilon\) ausdrücken. Die Dispersion eines freien Teilchens lautet

\begin{align}
  \label{eq:6}
  \epsilon(\vec k) = \frac{\hbar^2 \vec k^2}{2m} 
\end{align}

Nach \(\vec k\) umgestellt

\begin{align}
  \label{eq:7}
  k = \sqrt{\frac{2m \epsilon}{\hbar^2}}
\end{align}
Differenziert nach \(d\epsilon\)

\begin{align}
  \label{eq:8}
  \frac{d k}{d\epsilon} = \frac{1}{2} \sqrt{\frac{2m}{\hbar^2  \epsilon}} \qquad \Leftrightarrow \qquad dk = \frac{1}{2} \sqrt{\frac{2m}{\hbar^2  \epsilon}} d\epsilon
\end{align}

Eingesetzt in Gleichung (\ref{eq:5}) unter Beachtung dass das die Energie \(d\epsilon\) nicht negativ werden kann, folgt die Integration \(2\cdot\int_{0}^\infty\)

\begin{align}
  \label{eq:9}
   \mathcal N(\epsilon) &= \int_{0}^{\infty} d\epsilon  \sqrt{\frac{2m}{\hbar^2  \epsilon}}  \frac{ 1}{(2\pi)} \delta(\epsilon - \epsilon(\vec k)) \qquad \\
 &= \frac{1}{2\pi} \sqrt{\frac{2m}{\hbar^2}} \int_{-\infty}^{\infty} d\epsilon \frac{1}{\sqrt{\epsilon}}   \delta(\epsilon - \epsilon(\vec k)) \qquad \\
 &= \frac{\sqrt{2m}}{2\pi\hbar}  \epsilon^{-\frac{1}{2}}(\vec k)  \qquad \\
&\sim \epsilon^{-\frac{1}{2}}
\end{align}

\subsection*{2D Zustandsdichte}

Im 2 Dimensionalen Fall lauten die Gleichung (\ref{eq:2}) folgendermaßen

\begin{align}
  \label{eq:11}
  \mathcal N(\epsilon) = \frac{1}{L^2} \sum_{\vec k} \delta(\epsilon - \epsilon(\vec k))
\end{align}
wir ersetzen die Summe durch das Integral laut (\ref{eq:4}) lautet die Gleichung nun

\begin{align}
  \label{eq:12}
  \mathcal N(\epsilon) =  \int \frac{d^2 k}{(2\pi)^2}  \delta(\epsilon - \epsilon(\vec k))
\end{align}

Mit der Relation (\ref{eq:10.2}) für \(dk\) in 2D

\begin{align}
  \label{eq:13}
  \mathcal N(\epsilon) &=\frac{1}{(2\pi)^2} \int_0^{2\pi}d\phi \int_0^{\infty} dk \,\, k   \delta(\epsilon - \epsilon(\vec k))  \notag\\
   &=\frac{1}{2\pi} \int_0^{\infty} dk \,\, k   \delta(\epsilon - \epsilon(\vec k)) \qquad \text{ mit }dk = \frac{1}{2} \sqrt{\frac{2m}{\hbar^2  \epsilon}} d\epsilon \text{ und }  k = \sqrt{\frac{2m \epsilon}{\hbar^2}} \notag\\
   &=\frac{m}{2\pi\hbar^2 } \int_0^{\infty} d\epsilon \,\,  \delta(\epsilon - \epsilon(\vec k)) \notag\\
 &=\frac{m}{2\pi\hbar^2 } = \text{const}
\end{align}

\subsection*{3D Zustandsdichte}

Im 3 Dimensionalen Fall lauten die Gleichung (\ref{eq:2}) folgendermaßen

\begin{align}
  \label{eq:14}
  \mathcal N(\epsilon) = \frac{1}{L^3} \sum_{\vec k} \delta(\epsilon - \epsilon(\vec k))
\end{align}

wir ersetzen die Summe durch das Integral laut (\ref{eq:4}) lautet die Gleichung nun

\begin{align}
  \label{eq:15}
  \mathcal N(\epsilon) =  \int \frac{d^3 k}{(2\pi)^3}  \delta(\epsilon - \epsilon(\vec k))
\end{align}

Mit der Relation (\ref{eq:10.3}) für \(dk\) in 3D

\begin{align}
  \label{eq:16}
  \mathcal N(\epsilon) &=\frac{1}{(2\pi)^3} \underbr{\int_0^{2\pi}d\phi \int_0^{\pi}\sin\theta d\theta }_{4\pi} \int_0^{\infty} dk \,\, k^2   \delta(\epsilon - \epsilon(\vec k))  \notag\\
&=\frac{2}{(2\pi)^2}\int_0^{\infty} dk \,\, k^2   \delta(\epsilon - \epsilon(\vec k))  \qquad \text{ mit }dk = \frac{1}{2} \sqrt{\frac{2m}{\hbar^2  \epsilon}} d\epsilon \text{ und }  k^2 = \frac{2m \epsilon}{\hbar^2} \notag\\
&=\frac{2}{(2\pi)^2}\int_0^{\infty} \frac{1}{2} \sqrt{\frac{2m}{\hbar^2  \epsilon}} d\epsilon \,\, \frac{2m \epsilon}{\hbar^2}  \delta(\epsilon - \epsilon(\vec k))  \notag\\
&=\frac{1}{(2\pi)^2} \left(\frac{2m}{\hbar^2}\right)^{\frac{3}{2}}  \int_0^{\infty}  d\epsilon \,\, \sqrt{\epsilon} \,\, \delta(\epsilon - \epsilon(\vec k))  \notag\\
&=\frac{1}{(2\pi)^2} \left(\frac{2m}{\hbar^2}\right)^{\frac{3}{2}} \sqrt{\epsilon(\vec k)}\notag\\
&\sim \sqrt{\epsilon}
\end{align}


Die Zustandsdichte in 1,2 und 3 Dimensionen zusammengefasst

\begin{equation}
  \label{eq:10}
 \boxed{ \begin{aligned}
       d&=1: \quad \mathcal N(\epsilon) = \frac{\sqrt{2m}}{2\pi\hbar}  \epsilon^{-\frac{1}{2}}(\vec k) \sim \frac{1}{\sqrt{\epsilon}} \\
d&=2: \quad \mathcal N(\epsilon) = \frac{m}{2\pi\hbar^2 } = \text{const} \\
d&=3: \quad \mathcal N(\epsilon) = \frac{1}{(2\pi)^2} \left(\frac{2m}{\hbar^2}\right)^{\frac{3}{2}} \sqrt{\epsilon(\vec k)}\sim \sqrt{\epsilon}
  \end{aligned}  }
\end{equation}


Betrachtet man noch den Spin des Teilchens, so lautet die Gleichung (\ref{eq:2})

\begin{align}
  \label{eq:17}
  \mathcal N(\epsilon) = (2s+1)\frac{1}{V} \sum_{\vec k} \delta(\epsilon - \epsilon(\vec k))
\end{align}
Damit ändert sich die Zustandsdichte jeweils um den Faktor \(2s+1\). D.h. für Teilchein mit Spin \(\frac{1}{2}\) folgt

\begin{align}
  \label{eq:19}
   \boxed{ \begin{aligned}
       d&=1: \quad \mathcal N(\epsilon) = 2\frac{\sqrt{2m}}{2\pi\hbar}  \epsilon^{-\frac{1}{2}}(\vec k) \sim \frac{1}{\sqrt{\epsilon}} \\
d&=2: \quad \mathcal N(\epsilon) = \frac{m}{\pi\hbar^2 } = \text{const} \\
d&=3: \quad \mathcal N(\epsilon) = \frac{1}{2\pi^2} \left(\frac{2m}{\hbar^2}\right)^{\frac{3}{2}} \sqrt{\epsilon(\vec k)}\sim \sqrt{\epsilon}
  \end{aligned}  }
\end{align}





\subsection*{Referenzen}
\begin{itemize}
\item \url{http://www.tfp.uni-karlsruhe.de/Lehre/SS2010/Uebungen_TheorieF.shtml}
\end{itemize}

\end{document}
