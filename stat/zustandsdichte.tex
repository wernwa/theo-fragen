\input{../headers/header_script.tex}
\usepackage{amsmath} 



\begin{document}

\section*{Zustandsdichte}

Die Zustandsdichte ist definiert 

\begin{align}
  \label{eq:1}
  \boxed{ D(\epsilon) = \sum_{\vec k} \delta(\epsilon - \epsilon(\vec k)) }
\end{align}

Sie gibt die Anzahl der Systemzustäde pro Energieeinheit an. Als praktisch erweist sich eine Zustandsdichte pro Volumen und spin einzuführen

\begin{align}
  \label{eq:2}
  \mathcal N(\epsilon) = \frac{1}{2s+1}\frac{1}{V} \sum_{\vec k} \delta(\epsilon - \epsilon(\vec k))
\end{align}
Im folgenden wollen wir die Zustandsdichte für Fermionen in 1, 2 und 3 Dimensionen berechnen.

\subsection*{1D Zustandsdichte}

Für Spin \(\frac{1}{2}\)-Fermionen gilt

\begin{align}
  \label{eq:3}
  \mathcal N(\epsilon) = \frac{1}{2V} \sum_{\vec k} \delta(\epsilon - \epsilon(\vec k))
\end{align}
Wir nehmen an dass die Energiezustände dicht bei einander liegen, deswegen können wir die Summe als ein Integral ausdrücken. Im thermodynamischen Limes gilt

\begin{align}
  \label{eq:4}
\boxed{  \frac{1}{L^d}\sum_{\vec k} \xrightarrow{L\to \infty} \int \frac{d^d k}{(2\pi)^d} }
\end{align}

Für 1-Dimension können wir die Formel (\ref{eq:3}) schreiben

\begin{align}
  \label{eq:5}
  \mathcal N(\epsilon) = \frac{1}{2}  \int_{-\infty}^{\infty} \frac{d k}{(2\pi)} \delta(\epsilon - \epsilon(\vec k))
\end{align}

Wir möchten das Integral nach \(\epsilon\) ausdrücken. Die Dispersion eines freien Teilchens lautet

\begin{align}
  \label{eq:6}
  \epsilon(\vec k) = \frac{\hbar^2 \vec k^2}{2m} 
\end{align}

Nach \(\vec k\) umgestellt

\begin{align}
  \label{eq:7}
  k = \sqrt{\frac{2m \epsilon}{\hbar^2}}
\end{align}
Differenziert nach \(d\epsilon\)

\begin{align}
  \label{eq:8}
  \frac{d k}{d\epsilon} = \frac{1}{2} \sqrt{\frac{2m}{\hbar^2  \epsilon}} \qquad \Leftrightarrow \qquad dk = \frac{1}{2} \sqrt{\frac{2m}{\hbar^2  \epsilon}} d\epsilon
\end{align}

Eingesetzt in Gleichung (\ref{eq:5}) unter Beachtung dass das die Energie \(d\epsilon\) nicht negativ werden kann, folgt die Integration \(2\cdot\int_{0}^\infty\)

\begin{align}
  \label{eq:9}
   \mathcal N(\epsilon) &= \int_{0}^{\infty} d\epsilon \frac{1}{2} \sqrt{\frac{2m}{\hbar^2  \epsilon}}  \frac{ 1}{(2\pi)} \delta(\epsilon - \epsilon(\vec k)) \qquad \\
 &= \frac{1}{4\pi} \sqrt{\frac{2m}{\hbar^2}} \int_{-\infty}^{\infty} d\epsilon \frac{1}{\sqrt{\epsilon}}   \delta(\epsilon - \epsilon(\vec k)) \qquad \\
 &= \frac{\sqrt{2m}}{4\pi\hbar}  \epsilon^{-\frac{1}{2}}(\vec k)  \qquad \\
&\sim \epsilon^{-\frac{1}{2}}
\end{align}

\subsection*{2D Zustandssumme TODO}








\subsection*{Referenzen}
\begin{itemize}
\item Claude Cohen-Tannoudji Quantenmechanik Band 2
\item Zettili Quanten Mehanics
\item Rollnik Quantentheorie 2
\end{itemize}

\end{document}
