\input{../headers/header_script.tex}
\usepackage{amsmath} 



\begin{document}

\section*{Ising Modell}

Das Ising Modell ist ein wichtiges Modell der Statistischen Physik, was die Phänomäne des Magnetismus insbesondere Ferromagnetismus beschreiben soll. Damit werden auch Phasenübergänge untersucht, d.h. ob es eine bestimmte kritische Temperatur gibt bei dem ein Phasenübergang in einem Material stattfindet. Mit dem Ising Modell ist es möglich analytisch ein und zwei dimensionale Probleme exakt beschreiben. Für dreidimensionale Probleme werden nummerische approximationen verwendet. Das Ising Modell gilt als das einzige halbwegs realistische Modell für ein Viel-Teilchen-System mit dem man Phasenübergänge mathematisch behandeln kann.

Bei vielen meisten Problemuntersuchungen wird immer die gleiche Vorgehensweise verwendet. Man bestimmt die Zustandssumme eines Systems von der aus ist es möglich weitere termondynamische Größen zu bestimmen. Um die Zustandssumme bestimmen zu können benötigt muss man die Gesamtenergie des Systems wissen.

\subsection*{Herleitung der Hamiltonfunktion}

Wir betrachten ein System aus vielen Teilchen, die in einer Dimension nebeneinander angeordnet sind mit jeweils einem positiven bzw. einem negativen Spin besitzen (siehe Abbildung \ref{fig:1}). Beim Ising Modell wird nur die Spin-Wechselwirkung von benachbarten Teilchen betrachtet.


\begin{figure}
  \centering
  \includegraphics[scale=0.5]{./ising-pics/ising01.png}
  \caption{1-D Ising-Modell-Kette (Quelle: W. Nolting - Grundkurs Theo-
retische Physik: Band 6)
}
  \label{fig:1}
\end{figure}


Die Hamiltonfunktion dieses Systems lautet

\begin{align}
  \label{eq:1}
  H = -\sum_{ij}J_{ij}S_iS_j - \vec \mu \vec B \sum_i S_i
\end{align}

Dabei ist \(J_{ij} \) eine Wechselwirkungskonstante die die magnetische Wechselwirkung zwischen den Teilchen \(i\) und \(j\) beschreibt. \(S_i=\pm 1\) ist dazu da um das Vorzeichen des Spins darzustellen. \(\vec mu\) ist dabei das Magnetische Moment und \(\vec B\) die magnetische Flussdichte. 

Wir betrachten das Magnetfeld in z-Richtung, d.h. \(\vec B = (0,0,B_0)^T\), weiterhin gilt im Ising-Modell nur die Wechselwirkung zwischen benachbarten Teilchen \(j=i+1\), vereinfacht sich die Gleichung (\ref{eq:1}) insgesamt zu

\begin{align}
  \label{eq:2}
  H = -\sum_{i}^{N-1}J_{i}S_iS_{i+1} - \mu B_0 \sum_i S_i
\end{align}

\subsection*{1-D Ising-Modell ohne äußeres Magnetfeld ($B_0=0$)}

In diesem Abschnitt wollen wir untersuchen ob es bei einer kritischen Temperatur \(T_C\) zu einem Phasenübergang kommt. D.h. ob sich eine spontane Magnetisiertung einstellt. Dies war der ursprüngliche Plan von Ernst Ising, als dieses Modell erarbeitet hat.

Wir bestimmen zunächst die kanonische Zustandssumme, die allgemein lautet

\begin{align}
  \label{eq:3}
  Z = \sum_{\{\alpha\}} e^{-\beta E_\alpha} \text{ mit }\beta = \frac{1}{k_B T}
\end{align}

Setzen wir nun den Hamiltonoperator aus Gleichung (\ref{eq:2}) mit \(B_0=0\) in die Zustandssumme (\ref{eq:3}) ein, so lautet die von Teilchenzahl abhängige Zustandssumme

\begin{align}
  \label{eq:4}
    Z_N &= \sum_{S_1=\pm 1}\cdot\sum_{S_2=\pm 1}\cdots\sum_{S_N=\pm 1}  e^{\beta \sum_{i}^{N-1}J_{i}S_iS_{i+1}} \notag\\
&= \sum_{S_1=\pm 1}\cdot\sum_{S_2=\pm 1} e^{\beta J_{1}S_{1}S_2}\sum_{S_3=\pm 1}e^{\beta J_{2} S_2 S_3} \cdots\sum_{S_N=\pm 1}  e^{\beta J_{N-1}S_{N-1}S_{N}} \notag\\
&= \sum_{S_1=\pm 1}\prod_{i=2}^N\left( \sum_{S_i=\pm 1} e^{\beta J_{i-1}S_{i-1}S_i}\right) \notag\\
&= \sum_{S_1=\pm 1}\prod_{i=2}^N\left( e^{ + \beta J_{i-1}S_{i-1}}+ e^{ - \beta J_{i-1}S_{i-1}} \right)\notag\\
&= \sum_{S_1=\pm 1}\prod_{i=2}^N2\cosh( \beta J_{i-1}S_{i-1})
\end{align}


Nun möchten wir die erste Summe \(\sum_{S_1=\pm 1}\) berechnen. Das ist die Summe für das erste Teilchen das ohne Partner, d.h. ohne Wechselwirkung mit der Wechselwirkungskonstenen \(J_0=0\) und aus dem Grund müsste Die Exponentialfunktion lauten \(e^0\) und die Summe \(\sum_{S_1}e^0 = 2\). Rein mathematisch können wir aus der Gleichung (\ref{eq:4}) ein Produkt ausklammern

\begin{align}
  \label{eq:6}
   Z_N &=\sum_{S_1=\pm 1}\left( 2\cosh(\beta J_1 S_1) \prod_{i=3}^N2\cosh( \beta J_{i-1}S_{i-1}) \right) \notag\\
&=2\cosh(\beta J_1 1)\prod_{i=3}^N2\cosh( \beta J_{i-1}S_{i-1}) + 2\cosh(\beta J_1 (-1))\prod_{i=3}^N2\cosh( \beta J_{i-1}S_{i-1})
\end{align}

Mit der Relation  \(\cosh(\pm x) = \cosh(x) \) können wir die zwei Terme wieder zusammenfassen

\begin{align}
  \label{eq:7}
   Z_N &=2\cdot  2\cosh(\beta J_1)\prod_{i=3}^N2\cosh( \beta J_{i-1}S_{i-1})  \notag\\
&=2\cdot\prod_{i=2}^N2\cosh( \beta J_{i-1}S_{i-1}) = 2\cdot\prod_{i=2}^N2\cosh(\pm \beta J_{i-1}) \notag\\
&=  2\cdot\prod_{i=2}^N2\cosh(\beta J_{i-1})
\end{align}

Um die Gleichung (\ref{eq:7}) weiter zu vereinfachen, betrachten wir eine isotrope Wechselwirkung zwischen den Spins, d.h es gilt \(J_1=J_2=\dots=J_j=J\) mit \(j=2...N\). Desweiteren gilt \(\cosh(\beta J S_i ) = \cosh(\pm \beta J) =  \cosh(\beta J) \). Mit diesen Vereinfachungen lautet die Gleichung (\ref{eq:7})

\begin{align}
  \label{eq:5}
   Z_N =2\cdot \prod_{i=2}^N2\cosh( \beta J ) = 2\cdot \prod_{i=1}^{N-1}2\cosh( \beta J ) = 2\cdot 2^{N-1}\cosh^{N-1}( \beta J ) 
\end{align}

Aus Gleichung (\ref{eq:5}) erhalten wir schlussendlich eine Zustandssumme für ein 1-Dim. Ising-Modell ohne äußeres Magnetfeld

\begin{align}
  \label{eq:8}
  \boxed{Z_N = 2^{N}\cosh^{N-1}( \beta J )  }
\end{align}

\subsubsection*{Manetisierung}

Um die Magnetisierung bestimmen zu können, müssen wir den Erwartungswert, bzw. den Mittelwert von Spineinstellung \(\langle S_i\rangle =\langle S\rangle \) bestimmen, da er laut der Beziehung

\begin{align}
  \label{eq:9}
  M_S(T) = \mu \langle S\rangle
\end{align}

Direkt proportional zu der Magnetisierung ist. Es ist geschickt den Mittelwert bzw. den Erwartungswert der Spineinstellung \(S_i\) nicht direkt zu bestimmen sondern über die Spin-Spin-korrelation-Funktion

\begin{align}
  \label{eq:10}
  g_{i,i+j} = \langle S_i,S_{i+j}\rangle - \langle S_i\rangle\langle S_{i+j}\rangle
\end{align}

Den es gilt für weit auseinander liegende Spins, dass diese unkorreliert sind, d.h \(j\to\infty\) geht \(g_{i,i+j}\to 0\)

\begin{align}
  \label{eq:11}
   \lim_{j\to\infty} g_{i,i+j} = \lim_{j\to\infty}\left( \langle S_iS_{i+j}\rangle - \langle S_i\rangle\langle S_{i+j}\rangle\right) = 0 \notag\\
\Leftrightarrow \langle S_iS_{i+j}\rangle = \langle S_i\rangle\langle S_{i+j}\rangle = \langle S \rangle^2
\end{align}

Für den Erwartungswert \(\langle S_iS_{i+j}\rangle\) gilt allgemein

\begin{align}
  \label{eq:12}
  \langle S_iS_{i+j}\rangle &= \frac{1}{Z_N} \sum_{\{\alpha\}} S_i S_{i+j}  e^{\beta \sum_i^{N-1}J_iS_i S_{i+j} } \notag\\
&= \frac{1}{Z_N} \sum_{\{\alpha\}} S_i \cdot 1\cdot S_{i+j}  e^{\beta \sum_i^{N-1}J_iS_i S_{i+j} } \notag\\
&= \frac{1}{Z_N} \sum_{\{\alpha\}} (S_i\underbr{S_{i+1})(S_{i+1}}_{=1}\underbr{S_{i+2})(S_{i+2}}_{=1}\underbr{S_{i+3})}_{=1}\cdots\underbr{(S_{i+j-1}}_{=1} S_{i+j})  e^{\beta \sum_i^{N-1}J_iS_i S_{i+j} } 
\end{align}

Den Erwartungwert können wir durch ableiten der Zustandsumme nach \(\beta J_i\) gewinnen. Um die Wechselwirkung zwischen Spin \(i\) und Spin \(j\) zu bekommen, muss man alle Ableitungen zwischen dem \(i\)-ten und dem \((i+j-1)\)-ten Spin auf die Zustandsumme anwenden.

\begin{align}
  \label{eq:13}
  \langle S_iS_{i+j}\rangle &=  \frac{1}{Z_N} \pdiff_{(\beta J_i)}\pdiff_{(\beta J_{i+1})}\cdots \pdiff_{(\beta J_{i+j-1})} Z_N \notag\\
&\stackrel{~(\ref{eq:7})}=  \frac{1}{Z_N} \pdiff_{(\beta J_i)}\pdiff_{(\beta J_{i+1})}\cdots \pdiff_{(\beta J_{i+j-1})} \left(2\cdot\prod_{i=2}^N2\cosh( \beta J_{i-1})\right)  \notag\\
&=\cancel{\frac{\cosh(\beta J_{1})\cdots \cosh(\beta J_{i-1})}{ \cosh(\beta J_{1})\cdots \cosh(\beta J_{i-1})} } \times\notag \\
&\qquad\times \frac{ \sinh(\beta J_{i})\cdots\sinh(\beta J_{i+j-1})}{\cosh(\beta J_{i})\cdots\cosh(\beta J_{i+j-1})}\notag \\
&\qquad \times \cancel{ \frac{  \cosh(\beta J_{i+j}) \cdots \cosh(\beta J_{N})  }{   \cosh(\beta J_{i+j}) \cdots \cosh(\beta J_{N}) }} \notag\\
%
&=\prod_{k=i}^{i+j-1} \tanh(\beta J_{k}) = \prod_{k=1}^{j} \tanh(\beta J_{k})
\end{align}

Wegen Isotroper Wechselwirkung können wir das Produkt aus Gleichung (\ref{eq:13}) als Potenz schreiben

\begin{align}
  \label{eq:14}
   \langle S_iS_{i+j}\rangle =  \tanh^j(\beta J)
\end{align}

Laut Gleichung (\ref{eq:10}) für \(j\to\infty\) gilt

\begin{align}
  \label{eq:15}
\langle S \rangle^2 =  \langle S_i\rangle\langle S_{i+j}\rangle = \lim_{j\to\infty} \tanh^j(\beta J)
\end{align}

Betrachten wir die Magnetisierung aus Gleichung (\ref{eq:9}), so können wir schreiben

\begin{align}
  \label{eq:16}
  M(T) = \mu \lim_{j\to\infty} \tanh^{j/2}(\beta J)
\end{align}

\begin{itemize}
\item Die Magnetisierung für \(T>0\) ist immer Null, da \(\lim_{y\to\infty}\tanh^y(x)=0\)
\item Die Magnetisierung für \(T=0\) ist gleich \(\mu\) da \(\lim_{y\to\infty}\tanh^y(\infty)=1^\infty=1\)
\end{itemize}

Somit stellt sich \textit{keine spontane Magnetisierung} für endliche Temperaturen ein.\\

Der Erwartungswert aus Gleichung (\ref{eq:15}) ist Null für \(\beta J=\frac{1}{k_BT} J \ne \infty \) , d.h. für jede Temperatur \(T\ne 0\). Daraus folgt dass der gemittelte Spinzustand \(\langle S \rangle\)  auch Null ist. Was zu erwarten ist, da bei endlichen Temperaturen die Spinzustände zufällig chaotisch zwischen \(+ 1\) und \(-1\) verteilt sind. Somit muss der statistische Mittelwert bei Null liegen.

\end{document}
