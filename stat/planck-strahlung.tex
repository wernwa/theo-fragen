\documentclass[10pt,a4paper,oneside,fleqn]{article}
\usepackage{geometry}
\geometry{a4paper,left=20mm,right=20mm,top=1cm,bottom=2cm}
\usepackage[utf8]{inputenc}
%\usepackage{ngerman}
\usepackage{amsmath}                % brauche ich um dir Formel zu umrahmen.
\usepackage{amsfonts}                % brauche ich für die Mengensymbole
\usepackage{graphicx}
\setlength{\parindent}{0px}
\setlength{\mathindent}{10mm}
\usepackage{bbold}                    %brauche ich für die doppel Zahlen Darstellung (Einheitsmatrix z.B)



\usepackage{color}
\usepackage{titlesec} %sudo apt-get install texlive-latex-extra

\definecolor{darkblue}{rgb}{0.1,0.1,0.55}
\definecolor{verydarkblue}{rgb}{0.1,0.1,0.35}
\definecolor{darkred}{rgb}{0.55,0.2,0.2}

%hyperref Link color
\usepackage[colorlinks=true,
        linkcolor=darkblue,
        citecolor=darkblue,
        filecolor=darkblue,
        pagecolor=darkblue,
        urlcolor=darkblue,
        bookmarks=true,
        bookmarksopen=true,
        bookmarksopenlevel=3,
        plainpages=false,
        pdfpagelabels=true]{hyperref}

\titleformat{\chapter}[display]{\color{darkred}\normalfont\huge\bfseries}{\chaptertitlename\
\thechapter}{20pt}{\Huge}

\titleformat{\section}{\color{darkblue}\normalfont\Large\bfseries}{\thesection}{1em}{}
\titleformat{\subsection}{\color{verydarkblue}\normalfont\large\bfseries}{\thesubsection}{1em}{}

% Notiz Box
\usepackage{fancybox}
\newcommand{\notiz}[1]{\vspace{5mm}\ovalbox{\begin{minipage}{1\textwidth}#1\end{minipage}}\vspace{5mm}}

\usepackage{cancel}
\setcounter{secnumdepth}{3}
\setcounter{tocdepth}{3}





%-------------------------------------------------------------------------------
%Diff-Makro:
%Das Diff-Makro stellt einen Differentialoperator da.
%
%Benutzung:
% \diff  ->  d
% \diff f  ->  df
% \diff^2 f  ->  d^2 f
% \diff_x  ->  d/dx
% \diff^2_x  ->  d^2/dx^2
% \diff f_x  ->  df/dx
% \diff^2 f_x  ->  d^2f/dx^2
% \diff^2{f(x^5)}_x  ->  d^2(f(x^5))/dx^2
%
%Ersetzt man \diff durch \pdiff, so wird der partieller
%Differentialoperator dargestellt.
%
\makeatletter
\def\diff@n^#1{\@ifnextchar{_}{\diff@n@d^#1}{\diff@n@fun^#1}}
\def\diff@n@d^#1_#2{\frac{\textrm{d}^#1}{\textrm{d}#2^#1}}
\def\diff@n@fun^#1#2{\@ifnextchar{_}{\diff@n@fun@d^#1#2}{\textrm{d}^#1#2}}
\def\diff@n@fun@d^#1#2_#3{\frac{\textrm{d}^#1 #2}{\textrm{d}#3^#1}}
\def\diff@one@d_#1{\frac{\textrm{d}}{\textrm{d}#1}}
\def\diff@one@fun#1{\@ifnextchar{_}{\diff@one@fun@d #1}{\textrm{d}#1}}
\def\diff@one@fun@d#1_#2{\frac{\textrm{d}#1}{\textrm{d}#2}}
\newcommand*{\diff}{\@ifnextchar{^}{\diff@n}
  {\@ifnextchar{_}{\diff@one@d}{\diff@one@fun}}}
%
%Partieller Diff-Operator.
\def\pdiff@n^#1{\@ifnextchar{_}{\pdiff@n@d^#1}{\pdiff@n@fun^#1}}
\def\pdiff@n@d^#1_#2{\frac{\partial^#1}{\partial#2^#1}}
\def\pdiff@n@fun^#1#2{\@ifnextchar{_}{\pdiff@n@fun@d^#1#2}{\partial^#1#2}}
\def\pdiff@n@fun@d^#1#2_#3{\frac{\partial^#1 #2}{\partial#3^#1}}
\def\pdiff@one@d_#1{\frac{\partial}{\partial #1}}
\def\pdiff@one@fun#1{\@ifnextchar{_}{\pdiff@one@fun@d #1}{\partial#1}}
\def\pdiff@one@fun@d#1_#2{\frac{\partial#1}{\partial#2}}
\newcommand*{\pdiff}{\@ifnextchar{^}{\pdiff@n}
  {\@ifnextchar{_}{\pdiff@one@d}{\pdiff@one@fun}}}
\makeatother
%
%Das gleich nur mit etwas andere Syntax. Die Potenz der Differentiation wird erst
%zum Schluss angegeben. Somit lautet die Syntax:
%
% \diff_x^2  ->  d^2/dx^2
% \diff f_x^2  ->  d^2f/dx^2
% \diff{f(x^5)}_x^2  ->  d^2(f(x^5))/dx^2
% Ansonsten wie Oben.
%
%Ersetzt man \diff durch \pdiff, so wird der partieller
%Differentialoperator dargestellt.
%
%\makeatletter
%\def\diff@#1{\@ifnextchar{_}{\diff@fun#1}{\textrm{d} #1}}
%\def\diff@one_#1{\@ifnextchar{^}{\diff@n{#1}}%
%  {\frac{\textrm d}{\textrm{d} #1}}}
%\def\diff@fun#1_#2{\@ifnextchar{^}{\diff@fun@n#1_#2}%
%  {\frac{\textrm d #1}{\textrm{d} #2}}}
%\def\diff@n#1^#2{\frac{\textrm d^#2}{\textrm{d}#1^#2}}
%\def\diff@fun@n#1_#2^#3{\frac{\textrm d^#3 #1}%
%  {\textrm{d}#2^#3}}
%\def\diff{\@ifnextchar{_}{\diff@one}{\diff@}}
%\newcommand*{\diff}{\@ifnextchar{_}{\diff@one}{\diff@}}
%
%Partieller Diff-Operator.
%\def\pdiff@#1{\@ifnextchar{_}{\pdiff@fun#1}{\partial #1}}
%\def\pdiff@one_#1{\@ifnextchar{^}{\pdiff@n{#1}}%
%  {\frac{\partial}{\partial #1}}}
%\def\pdiff@fun#1_#2{\@ifnextchar{^}{\pdiff@fun@n#1_#2}%
%  {\frac{\partial #1}{\partial #2}}}
%\def\pdiff@n#1^#2{\frac{\partial^#2}{\partial #1^#2}}
%\def\pdiff@fun@n#1_#2^#3{\frac{\partial^#3 #1}%
%  {\partial #2^#3}}
%\newcommand*{\pdiff}{\@ifnextchar{_}{\pdiff@one}{\pdiff@}}
%\makeatother

%-------------------------------------------------------------------------------
%%Nützliche Makros um in der Quantenmechanik Bras, Kets und das Skalarprodukt
%%zwischen den beiden darzustellen.
%%Benutzung:
%% \bra{x}  ->    < x |
%% \ket{x}  ->    | x >
%% \braket{x}{y} ->   < x | y >

\newcommand\bra[1]{\left\langle #1 \right|}
\newcommand\ket[1]{\left| #1 \right\rangle}
\newcommand\braket[2]{%
  \left\langle #1\vphantom{#2} \right.%
  \left|\vphantom{#1#2}\right.%
  \left. \vphantom{#1}#2 \right\rangle}%

%-------------------------------------------------------------------------------
%%Aus dem Buch:
%%Titel:  Latex in Naturwissenschaften und Mathematik
%%Autor:  Herbert Voß
%%Verlag: Franzis Verlag, 2006
%%ISBN:   3772374190, 9783772374197
%%
%%Hier werden drei Makros definiert:\mathllap, \mathclap und \mathrlap, welche
%%analog zu den aus Latex bekannten \rlap und \llap arbeiten, d.h. selbst
%%keinerlei horizontalen Platz benötigen, aber dennoch zentriert zum aktuellen
%%Punkt erscheinen.

\newcommand*\mathllap{\mathstrut\mathpalette\mathllapinternal}
\newcommand*\mathllapinternal[2]{\llap{$\mathsurround=0pt#1{#2}$}}
\newcommand*\clap[1]{\hbox to 0pt{\hss#1\hss}}
\newcommand*\mathclap{\mathpalette\mathclapinternal}
\newcommand*\mathclapinternal[2]{\clap{$\mathsurround=0pt#1{#2}$}}
\newcommand*\mathrlap{\mathpalette\mathrlapinternal}
\newcommand*\mathrlapinternal[2]{\rlap{$\mathsurround=0pt#1{#2}$}}

%%Das Gleiche nur mit \def statt \newcommand.
%\def\mathllap{\mathpalette\mathllapinternal}
%\def\mathllapinternal#1#2{%
%  \llap{$\mathsurround=0pt#1{#2}$}% $
%}
%\def\clap#1{\hbox to 0pt{\hss#1\hss}}
%\def\mathclap{\mathpalette\mathclapinternal}
%\def\mathclapinternal#1#2{%
%  \clap{$\mathsurround=0pt#1{#2}$}%
%}
%\def\mathrlap{\mathpalette\mathrlapinternal}
%\def\mathrlapinternal#1#2{%
%  \rlap{$\mathsurround=0pt#1{#2}$}% $
%}

%-------------------------------------------------------------------------------
%%Hier werden zwei neue Makros definiert \overbr und \underbr welche analog zu
%%\overbrace und \underbrace funktionieren jedoch die Gleichung nicht
%%'zerreißen'. Dies wird ermöglicht durch das \mathclap Makro.

\def\overbr#1^#2{\overbrace{#1}^{\mathclap{#2}}}
\def\underbr#1_#2{\underbrace{#1}_{\mathclap{#2}}}
\usepackage{amsmath} 



\begin{document}

\textit{29. März 2012}
\input{../headers/authors.tex}

\section*{Planck Strahlungsgesetz}

Wir möchten das Planck Strahlungsgesetz aus der Photonen-Zustandsdichte herleiten. Wir betrachen Bosonenteilchen  für die gilt

\begin{align}
  \label{eq:1}
  \omega = c |\vec k| \qquad \epsilon = \hbar \omega = \hbar c k
\end{align}

Die Definition der Zustandsdichte für 3 Dimensionen lautet

\begin{align}
  \label{eq:2}
  \mathcal N (\epsilon) = \frac{1}{V}\sum_{\vec k} \delta(\epsilon - \epsilon(\vec k))
\end{align}
Machen wir die klassische Ersetzung der Summe durch das Integral so folgt

\begin{align}
  \label{eq:3}
  \mathcal N (\epsilon) = \int \frac{d^3k}{(2\pi)^3} \delta(\epsilon - \epsilon(\vec k))
\end{align}
Wir nehmen dazu die Kugelkoordinaten \(d^3k = k^2dk \sin\theta d\theta d\phi\) und da die Zustandsdichte nicht von Winkeln abhängt, liefert die Integration \( \int \sin\theta d\theta d\phi=4\pi\). Dies in Gleichung (\ref{eq:3}) eingesetzt ergibt

\begin{align}
  \label{eq:4}
   \mathcal N (\epsilon) &= \frac{2}{(2\pi)^2}  \int dk k^2 \delta(\epsilon - \epsilon(\vec k)) \qquad \text{mit }k=\frac{\epsilon}{\hbar c},\quad dk = \frac{1}{\hbar c}d\epsilon \notag\\
 &= \frac{2}{(2\pi)^2}  \int \frac{1}{\hbar c}d\epsilon \frac{\epsilon^2}{\hbar^2 c^2} \delta(\epsilon - \epsilon(\vec k)) \notag\\
&= \frac{2}{(2\pi)^2 \hbar^3 c^3}  \int d\epsilon \epsilon^2 \delta(\epsilon - \epsilon(\vec k)) \notag\\
&= \frac{2 \epsilon^2}{(2\pi)^2 \hbar^3 c^3} 
\end{align}

Wir müssen noch berücksichtigen dass es 2 Polarisationen für das Licht gibt, deswegen müssen wir das Ergebnis mit Faktor 2 multiplizieren und erhalten somit

\begin{align}
  \label{eq:5}
   \mathcal N (\epsilon) &= \frac{ \epsilon^2}{\pi^2 \hbar^3 c^3} 
\end{align}

Für die weitere Berechnung ist es geschickt die Zustandsdichte in Abhängigkeit der Kreisfrequenz \(\omega\) auszudrücken. Mit Hilfe der Gleichung (\ref{eq:1}) erhalten wir

\begin{align}
  \label{eq:6}
   \mathcal N (\omega) &= \frac{ \omega^2}{\pi^2 \hbar c^3} 
\end{align}

Das Plancksche-Gesetz beschreibt die Energiedichte pro Frequenzintervall. Die Innere Energie lässt sich bestimmen mit

\begin{align}
  \label{eq:7}
  U = V\int d\epsilon\,\mathcal N(\epsilon) \epsilon \frac{1}{e^{\beta \epsilon}-1} 
\end{align}
Ersetzen wir die Abhängigkeit von \(\epsilon\) durch die Abhängigkeit von \(\omega\) so folgt

\begin{align}
  \label{eq:8}
  U =\hbar^2 V\int d\omega\,\mathcal N(\epsilon) \omega \frac{1}{e^{\beta \hbar\omega}-1} \qquad \text{mit }\epsilon =\hbar\omega,\quad d\epsilon = \hbar d\omega 
\end{align}

Betrachten wir ferner die Innere Energie pro Frequenz-Intervall

\begin{align}
  \label{eq:9}
  u(\omega) = \frac{dU}{d\omega} =  \hbar^2 V \mathcal N(\epsilon) \omega \frac{1}{e^{\beta \hbar\omega}-1}
\end{align}
Setzen wir noch die Zustandsdichte aus Gleichung (\ref{eq:6}) ein

\begin{align}
  \label{eq:10}
   u(\omega) &= \hbar^2 V \frac{ \omega^2}{\pi^2 \hbar c^3}  \omega \frac{1}{e^{\beta \hbar\omega}-1} \notag \\
 &=  V \frac{\hbar  \omega^3}{\pi^2  c^3} \frac{1}{e^{\beta \hbar\omega}-1} 
\end{align}

Im wesentlichen stellt die Gleichung (\ref{eq:10}) das schon aus der klassischen Physik bekannte \textbf{plancksche Strahlungsgesetz} dar. In der Literatur wird das plancksche Strahlungsgesetz oft als Energiedichte pro Frequenzintervall und pro Volumen angegeben. Damit ändert sich die Gleichung (\ref{eq:10}) zu

\begin{align}
  \label{eq:11}
  \boxed{    u(\omega) = \frac{\hbar  \omega^3}{\pi^2  c^3} \frac{1}{e^{\beta \hbar\omega}-1}   }
\end{align}

Als weiteres betrachten wir zwei Grenzfälle.

\begin{itemize}
\item[\(\hbar\omega\ll k_B T\)] Hieraus folgt


  \begin{align}
    \label{eq:12}
   u(\omega) &=  \frac{\hbar  \omega^3}{\pi^2  c^3} \frac{1}{\underbr{\exp\{\frac{\hbar\omega}{k_B T}}_{\approx 1+\frac{\hbar\omega}{k_B T}+\cdots} \}-1} \notag\\
&\approx  \frac{\hbar  \omega^3}{\pi^2  c^3} \frac{1}{  1 + \frac{\hbar\omega}{k_B T} -1} \notag\\
&=  \frac{k_B T \omega^2}{\pi^2  c^3} \notag\\
  \end{align}
Die Gleichung (\ref{eq:12}) ist als \textbf{Rayleigh-Jeans-Gesetz} bekannt. Es zeigt die sogenannte Ultraviolett-Katastrophe (UV-Katastrophe), da es für große Frequenzen \(\omega\) divergiert also \(u(\omega\to \infty)= \infty\). Dies würde bedeuten dass unendlich viel Leistung bei einem Schwarz-Körper abgestrahlt würde. Was gegen die Energieerhaltung spricht.

\item[\(\hbar\omega \gg k_B T\)] Hierraus folgt

  \begin{align}
    \label{eq:13}
    u(\omega) &=  \frac{\hbar  \omega^3}{\pi^2  c^3} \frac{1}{\underbr{\exp\{\frac{\hbar\omega}{k_B T}}_{\text{groß gegenüber der -1, somit vernachlässige -1}} \}-1} \notag\\
 &\approx  \frac{\hbar  \omega^3}{\pi^2  c^3} e^{-\frac{\hbar\omega}{k_B T}} \notag\\
  \end{align}
Diese Gleichung (\ref{eq:13}) ist unter dem Begriff \textbf{Wiensches Strahlungsgesetz} bekannt. 
\end{itemize}

Die Funktion \(\frac{x^3}{e^x-1}\) hat bei \(x=2.82\) ihr einziges Maximum. Daraus folgt das Wiensche Verschiebungsgesetz

\begin{align}
  \label{eq:15}
  \hbar\omega_{\text{max}}  = 2.82 k_B T
\end{align}

das eine strickte Proportionalität zwischen \(\omega_{\text{max}}\) und \(T\). 


Das Wiensche Verschiebungsgesetz gibt an, bei welcher Frequenz \(\omega_{\text{max}}\) ein nach dem planckschen Strahlungsgesetz strahlender schwarzer Körper je nach seiner Temperatur die größte Strahlungsleistung oder die größte Photonenrate abgibt.

Als Beispiel betrachte die Sonne als näherungsweise Schwarzen-Strahler. Die maximale Frequenz der von der Sonne abgestrahlten Lichts beträgt

\begin{align}
  \label{eq:14}
  \omega_{max}=2\pi f = 2\pi \cdot 3.4\cdot 10^{14}Hz
\end{align}

In die Beziehung (\ref{eq:15}) eingesetzt und nach T umgestellt ergibt

\begin{align}
  \label{eq:16}
  T = \frac{\hbar \omega_{\text{max}}}{2.82\cdot k_B} =  \frac{2\pi \hbar\cdot 3.4\cdot 10^{14} Hz }{2.82\cdot k_B} \approx 5789 K
\end{align}

Damit erhalten die Oberflächentemperatur der Sonne die ca. bei \(T=5800K\) liegt. Die tatsächliche Temperatur dürfte davon abweichen, weil die Sonne kein idealer schwarzer Körper ist.


\subsection*{Referenzen}
\begin{itemize}
\item \url{http://t1.physik.tu-dortmund.de/uhrig/teaching/tus_ws0910/tus-ws0910.pdf}
\item \url{http://www.tkm.uni-karlsruhe.de/~rachel/MV_StatPhys.pdf}
\item \url{http://de.wikipedia.org/wiki/Sonne}
\item \url{http://de.wikipedia.org/wiki/Wiensches_Verschiebungsgesetz}
\item \url{http://matheplanet.com/matheplanet/nuke/html/viewtopic.php?topic=99503}
\end{itemize}

\end{document}
