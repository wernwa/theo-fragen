\input{../headers/header_script.tex}
\usepackage{amsmath} 



\begin{document}

\section*{Zustandssumme und Freie Energie}

Wir möchten auf den Zusammenhang zwischen der Zustandssumme und der Freien Energie kommen. Dazu betrachten wir zunächst den dichte Operator für die kanonische Gesamtheit

\begin{align}
  \label{eq:1}
  \rho = \frac{1}{Z}e^{-\beta E}
\end{align}

mit \(Z\) der Zustandssumme und \(\beta=\frac{1}{k_B T}\). Man bildet von beiden Seiten der Gleichung (\ref{eq:1}) den Logarithmus

\begin{align}
  \label{eq:2}
  \ln \rho &= \ln\left( \frac{1}{Z}e^{-\beta E} \right) = \ln\frac{1}{Z} + \ln\left( e^{-\beta E}\right) = - \ln Z -\beta E =  - \ln Z - \frac{E}{k_B T} \quad | \cdot k_B T \notag \\
k_B T \ln \rho &=   - k_B T \ln Z - E
\end{align}

Mit der Beziehung zwischen der Entropie und dem Dichteoperator

\begin{align}
  \label{eq:3}
  S = -k_B \ln\rho
\end{align}

eingesetzt in die Gleichung (\ref{eq:2}) folgt

\begin{align}
  \label{eq:4}
  -TS &=  - k_B T \ln Z - E \notag \\
\Leftrightarrow E - TS &=  - k_B T \ln Z
\end{align}

Die definition aus der Legendre-Transformation der Freien Energie lautet

\begin{align}
  \label{eq:5}
  F = E - TS
\end{align}

Dies nun in Gleichung (\ref{eq:4}) eingesetzt ergibt unsere gesuchte Beziehung für die Freie Energie

\begin{align}
  \label{eq:6}
\boxed{  F =  - k_B T \ln Z }
\end{align}







\subsection*{Referenzen}
\begin{itemize}
\item 
\end{itemize}

\end{document}
