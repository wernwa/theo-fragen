\input{../headers/header_script.tex}
\usepackage{amsmath} 



\begin{document}

\textit{29. März 2012}
\input{../headers/authors.tex}

\section*{Zustandssumme und Freie Energie}

Wir möchten auf den Zusammenhang zwischen der Zustandssumme und der Freien Energie kommen. Dazu betrachten wir zunächst den dichte Operator für die kanonische Gesamtheit


\begin{align}
  \label{eq:7}
  \rho = \sum_n p_n \ket{\psi_n}\bra{\psi_n}
\end{align}

Nach der Boltzmann-Statistik gilt für die Wahrscheinlichkeit eines nicht entarteten Zustandes \(p_n\)

\begin{align}
  \label{eq:8}
  p_n = \frac{1}{Z}e^{-\beta E_n}
\end{align}

somit ergibt sich für den Dichteoperator

\begin{align}
  \label{eq:1}
  \rho = \sum_n \frac{1}{Z}e^{-\beta E_n} \ket{\psi_n}\bra{\psi_n} = \frac{1}{Z}e^{-\beta H}
\end{align}

Per definition mit \(H=E\) mit \(Z\) der Zustandssumme und \(\beta=\frac{1}{k_B T}\). Man bildet von beiden Seiten der Gleichung (\ref{eq:1}) den Logarithmus

\begin{align}
  \label{eq:2}
  \ln \rho &= \ln\left( \frac{1}{Z}e^{-\beta E} \right) = \ln\frac{1}{Z} + \ln\left( e^{-\beta E}\right) = - \ln Z -\beta E =  - \ln Z - \frac{E}{k_B T} \quad | \cdot k_B T \notag \\
k_B T \ln \rho &=   - k_B T \ln Z - E
\end{align}

Mit der Beziehung zwischen der Entropie und dem Dichteoperator

\begin{align}
  \label{eq:3}
  S = -k_B \ln\rho
\end{align}

eingesetzt in die Gleichung (\ref{eq:2}) folgt

\begin{align}
  \label{eq:4}
  -TS &=  - k_B T \ln Z - E \notag \\
\Leftrightarrow E - TS &=  - k_B T \ln Z
\end{align}

Die definition aus der Legendre-Transformation der Freien Energie lautet

\begin{align}
  \label{eq:5}
  F = E - TS
\end{align}

Dies nun in Gleichung (\ref{eq:4}) eingesetzt ergibt unsere gesuchte Beziehung für die Freie Energie

\begin{align}
  \label{eq:6}
\boxed{  F =  - k_B T \ln Z }
\end{align}


\subsection*{alternative Herleitung}

Definition der Statistischen Entropie lautet

\begin{align}
  \label{eq:9}
  S = - k_B\sum W_n \ln W_n
\end{align}

mit

\begin{align}
  \label{eq:10}
  W_n = \frac{1}{Z}e^{-\beta E_n}\qquad \text{ mit } \sum_n W_n = \frac{1}{Z}\sum_ne^{-\beta E_n} = 1
\end{align}

Daraus folgt die Definition für die Zustandssumme

\begin{align}
  \label{eq:11}
  Z = \sum_n e^{-\beta E_n}
\end{align}
Die Definition der Inneren Energie

\begin{align}
  \label{eq:12}
  U = \sum_n W_n E_n
\end{align}

Die Gleichung (\ref{eq:10}) in die Gleichung (\ref{eq:12}) eingesetzt folgt

\begin{align}
  \label{eq:13}
  U = \frac{1}{Z}\sum_n E_ne^{-\beta E_n}
\end{align}

Die Gleichung (\ref{eq:9}) ausgeschrieben lautet

\begin{align}
  \label{eq:15}
  S &= -k_B \sum_n \frac{1}{Z}e^{-\beta E_n} \ln(\frac{1}{Z}e^{-\beta E_n}) \notag\\
&= -k_B \sum_n \frac{1}{Z}e^{-\beta E_n} (-\beta E_n - \ln Z ) \notag\\
&= k_B \beta \underbr{\sum_n \frac{E_n}{Z}e^{-\beta E_n}}_{\equiv U\,\, (\ref{eq:13})} + k_B \underbr{\sum_n \frac{1}{Z}e^{-\beta E_n}}_{\equiv 1\,\, (\ref{eq:10})} \ln Z  \notag\\
&= k_B \beta U + k_B \ln Z
\end{align}

Setzen wir \(\beta=\frac{1}{k_B T}\) in die Gleichung (\ref{eq:15}) ein so ergibt sich

\begin{align}
  \label{eq:16}
S &= k_B \frac{1}{k_B T} U + k_B \ln Z \qquad |\cdot T \notag \\
TS &=  U + k_B T \ln Z \qquad |\cdot (-1) +U \notag \\
\underbr{U - TS}_{F} &= - k_B T\ln Z 
\end{align}
Daraus folgt unsere gesuchte Formel für die freie Energie

\begin{align}
  \label{eq:17}
  \boxed{ F= - k_BT\ln Z }
\end{align}




\subsection*{Referenzen}


\begin{itemize}
\item \url{http://www.fsmpi.uni-bayreuth.de/thermo/entropie.html};
\end{itemize}

\end{document}
