\documentclass[10pt,a4paper,oneside,fleqn]{article}
\usepackage{geometry}
\geometry{a4paper,left=20mm,right=20mm,top=1cm,bottom=2cm}
\usepackage[utf8]{inputenc}
%\usepackage{ngerman}
\usepackage{amsmath}                % brauche ich um dir Formel zu umrahmen.
\usepackage{amsfonts}                % brauche ich für die Mengensymbole
\usepackage{graphicx}
\setlength{\parindent}{0px}
\setlength{\mathindent}{10mm}
\usepackage{bbold}                    %brauche ich für die doppel Zahlen Darstellung (Einheitsmatrix z.B)



\usepackage{color}
\usepackage{titlesec} %sudo apt-get install texlive-latex-extra

\definecolor{darkblue}{rgb}{0.1,0.1,0.55}
\definecolor{verydarkblue}{rgb}{0.1,0.1,0.35}
\definecolor{darkred}{rgb}{0.55,0.2,0.2}

%hyperref Link color
\usepackage[colorlinks=true,
        linkcolor=darkblue,
        citecolor=darkblue,
        filecolor=darkblue,
        pagecolor=darkblue,
        urlcolor=darkblue,
        bookmarks=true,
        bookmarksopen=true,
        bookmarksopenlevel=3,
        plainpages=false,
        pdfpagelabels=true]{hyperref}

\titleformat{\chapter}[display]{\color{darkred}\normalfont\huge\bfseries}{\chaptertitlename\
\thechapter}{20pt}{\Huge}

\titleformat{\section}{\color{darkblue}\normalfont\Large\bfseries}{\thesection}{1em}{}
\titleformat{\subsection}{\color{verydarkblue}\normalfont\large\bfseries}{\thesubsection}{1em}{}

% Notiz Box
\usepackage{fancybox}
\newcommand{\notiz}[1]{\vspace{5mm}\ovalbox{\begin{minipage}{1\textwidth}#1\end{minipage}}\vspace{5mm}}

\usepackage{cancel}
\setcounter{secnumdepth}{3}
\setcounter{tocdepth}{3}





%-------------------------------------------------------------------------------
%Diff-Makro:
%Das Diff-Makro stellt einen Differentialoperator da.
%
%Benutzung:
% \diff  ->  d
% \diff f  ->  df
% \diff^2 f  ->  d^2 f
% \diff_x  ->  d/dx
% \diff^2_x  ->  d^2/dx^2
% \diff f_x  ->  df/dx
% \diff^2 f_x  ->  d^2f/dx^2
% \diff^2{f(x^5)}_x  ->  d^2(f(x^5))/dx^2
%
%Ersetzt man \diff durch \pdiff, so wird der partieller
%Differentialoperator dargestellt.
%
\makeatletter
\def\diff@n^#1{\@ifnextchar{_}{\diff@n@d^#1}{\diff@n@fun^#1}}
\def\diff@n@d^#1_#2{\frac{\textrm{d}^#1}{\textrm{d}#2^#1}}
\def\diff@n@fun^#1#2{\@ifnextchar{_}{\diff@n@fun@d^#1#2}{\textrm{d}^#1#2}}
\def\diff@n@fun@d^#1#2_#3{\frac{\textrm{d}^#1 #2}{\textrm{d}#3^#1}}
\def\diff@one@d_#1{\frac{\textrm{d}}{\textrm{d}#1}}
\def\diff@one@fun#1{\@ifnextchar{_}{\diff@one@fun@d #1}{\textrm{d}#1}}
\def\diff@one@fun@d#1_#2{\frac{\textrm{d}#1}{\textrm{d}#2}}
\newcommand*{\diff}{\@ifnextchar{^}{\diff@n}
  {\@ifnextchar{_}{\diff@one@d}{\diff@one@fun}}}
%
%Partieller Diff-Operator.
\def\pdiff@n^#1{\@ifnextchar{_}{\pdiff@n@d^#1}{\pdiff@n@fun^#1}}
\def\pdiff@n@d^#1_#2{\frac{\partial^#1}{\partial#2^#1}}
\def\pdiff@n@fun^#1#2{\@ifnextchar{_}{\pdiff@n@fun@d^#1#2}{\partial^#1#2}}
\def\pdiff@n@fun@d^#1#2_#3{\frac{\partial^#1 #2}{\partial#3^#1}}
\def\pdiff@one@d_#1{\frac{\partial}{\partial #1}}
\def\pdiff@one@fun#1{\@ifnextchar{_}{\pdiff@one@fun@d #1}{\partial#1}}
\def\pdiff@one@fun@d#1_#2{\frac{\partial#1}{\partial#2}}
\newcommand*{\pdiff}{\@ifnextchar{^}{\pdiff@n}
  {\@ifnextchar{_}{\pdiff@one@d}{\pdiff@one@fun}}}
\makeatother
%
%Das gleich nur mit etwas andere Syntax. Die Potenz der Differentiation wird erst
%zum Schluss angegeben. Somit lautet die Syntax:
%
% \diff_x^2  ->  d^2/dx^2
% \diff f_x^2  ->  d^2f/dx^2
% \diff{f(x^5)}_x^2  ->  d^2(f(x^5))/dx^2
% Ansonsten wie Oben.
%
%Ersetzt man \diff durch \pdiff, so wird der partieller
%Differentialoperator dargestellt.
%
%\makeatletter
%\def\diff@#1{\@ifnextchar{_}{\diff@fun#1}{\textrm{d} #1}}
%\def\diff@one_#1{\@ifnextchar{^}{\diff@n{#1}}%
%  {\frac{\textrm d}{\textrm{d} #1}}}
%\def\diff@fun#1_#2{\@ifnextchar{^}{\diff@fun@n#1_#2}%
%  {\frac{\textrm d #1}{\textrm{d} #2}}}
%\def\diff@n#1^#2{\frac{\textrm d^#2}{\textrm{d}#1^#2}}
%\def\diff@fun@n#1_#2^#3{\frac{\textrm d^#3 #1}%
%  {\textrm{d}#2^#3}}
%\def\diff{\@ifnextchar{_}{\diff@one}{\diff@}}
%\newcommand*{\diff}{\@ifnextchar{_}{\diff@one}{\diff@}}
%
%Partieller Diff-Operator.
%\def\pdiff@#1{\@ifnextchar{_}{\pdiff@fun#1}{\partial #1}}
%\def\pdiff@one_#1{\@ifnextchar{^}{\pdiff@n{#1}}%
%  {\frac{\partial}{\partial #1}}}
%\def\pdiff@fun#1_#2{\@ifnextchar{^}{\pdiff@fun@n#1_#2}%
%  {\frac{\partial #1}{\partial #2}}}
%\def\pdiff@n#1^#2{\frac{\partial^#2}{\partial #1^#2}}
%\def\pdiff@fun@n#1_#2^#3{\frac{\partial^#3 #1}%
%  {\partial #2^#3}}
%\newcommand*{\pdiff}{\@ifnextchar{_}{\pdiff@one}{\pdiff@}}
%\makeatother

%-------------------------------------------------------------------------------
%%Nützliche Makros um in der Quantenmechanik Bras, Kets und das Skalarprodukt
%%zwischen den beiden darzustellen.
%%Benutzung:
%% \bra{x}  ->    < x |
%% \ket{x}  ->    | x >
%% \braket{x}{y} ->   < x | y >

\newcommand\bra[1]{\left\langle #1 \right|}
\newcommand\ket[1]{\left| #1 \right\rangle}
\newcommand\braket[2]{%
  \left\langle #1\vphantom{#2} \right.%
  \left|\vphantom{#1#2}\right.%
  \left. \vphantom{#1}#2 \right\rangle}%

%-------------------------------------------------------------------------------
%%Aus dem Buch:
%%Titel:  Latex in Naturwissenschaften und Mathematik
%%Autor:  Herbert Voß
%%Verlag: Franzis Verlag, 2006
%%ISBN:   3772374190, 9783772374197
%%
%%Hier werden drei Makros definiert:\mathllap, \mathclap und \mathrlap, welche
%%analog zu den aus Latex bekannten \rlap und \llap arbeiten, d.h. selbst
%%keinerlei horizontalen Platz benötigen, aber dennoch zentriert zum aktuellen
%%Punkt erscheinen.

\newcommand*\mathllap{\mathstrut\mathpalette\mathllapinternal}
\newcommand*\mathllapinternal[2]{\llap{$\mathsurround=0pt#1{#2}$}}
\newcommand*\clap[1]{\hbox to 0pt{\hss#1\hss}}
\newcommand*\mathclap{\mathpalette\mathclapinternal}
\newcommand*\mathclapinternal[2]{\clap{$\mathsurround=0pt#1{#2}$}}
\newcommand*\mathrlap{\mathpalette\mathrlapinternal}
\newcommand*\mathrlapinternal[2]{\rlap{$\mathsurround=0pt#1{#2}$}}

%%Das Gleiche nur mit \def statt \newcommand.
%\def\mathllap{\mathpalette\mathllapinternal}
%\def\mathllapinternal#1#2{%
%  \llap{$\mathsurround=0pt#1{#2}$}% $
%}
%\def\clap#1{\hbox to 0pt{\hss#1\hss}}
%\def\mathclap{\mathpalette\mathclapinternal}
%\def\mathclapinternal#1#2{%
%  \clap{$\mathsurround=0pt#1{#2}$}%
%}
%\def\mathrlap{\mathpalette\mathrlapinternal}
%\def\mathrlapinternal#1#2{%
%  \rlap{$\mathsurround=0pt#1{#2}$}% $
%}

%-------------------------------------------------------------------------------
%%Hier werden zwei neue Makros definiert \overbr und \underbr welche analog zu
%%\overbrace und \underbrace funktionieren jedoch die Gleichung nicht
%%'zerreißen'. Dies wird ermöglicht durch das \mathclap Makro.

\def\overbr#1^#2{\overbrace{#1}^{\mathclap{#2}}}
\def\underbr#1_#2{\underbrace{#1}_{\mathclap{#2}}}
\usepackage{amsmath} 



\begin{document}

\section*{Bose-Einstein-Kondensation (BEC)}

Unter Bose-Einstein-Kondensation versteht man dass sich Teilchen eines idealen Bosegases mit der Dispersionsrelation \(\epsilon = \frac{\hbar^2 k^2}{2m}\) unterhalb einer bestimmten kritischen Temperatur \(T_C\) in dem niedrigsten Energiezustand versammeln. Dies ist ein rein quantenmechanischer Effekt da für eine Kondensation im klassischen Sinne die Teilchen miteinander wechselwirken können müssen z.B. Gitter aufbauen können, was bei einem idealen Gas nicht der Fall ist.

Wir möchten nun die kritische Temperatur \(T_C\) bestimmen. Dazu nutzen wir die Tatsache aus dass die Teilchenzahl bei jeder Temperatur erhalten bleiben muss. Die Anzahl der Teilchen lässt sich nach folgender Formel bestimmen

\begin{align}
  \label{eq:1}
  N(T,V,\mu) = V\int d\epsilon \mathcal N(\epsilon) n(\epsilon-\mu)
\end{align}

Wobei \( n(\epsilon-\mu) \) die Bose-Einstein Verteilungsfunktion ist und \(\mathcal N(\epsilon)\) ist die Zustandsdichte. Die Zustandsdichte für ein Gas im 3-dimensionalen Raum.

\begin{align}
  \label{eq:2}
  \mathcal N(\epsilon) = \frac{1}{4\pi^2}\left( \frac{2m}{\hbar^2}  \right)^{\frac{3}{2}}\sqrt{\epsilon}
\end{align}

Die Bose-Einstein-Verteilungsfunktion die da lautet

\begin{align}
  \label{eq:3}
  n(\epsilon-\mu) = \frac{1}{\exp\left(\frac{\epsilon-\mu}{k_B T}\right)-1}
\end{align}

gibt die Anzahl der Teilchen in einem Zustand mit der Energie \(\epsilon = \frac{\hbar^2 k^2}{2m}\) an. Betrachtet man den Exponenten in der Gleichung (\ref{eq:3}) so stellt man fest, dass das chemische Potential \(\mu\) immer kleiner seien muss als \(\epsilon\), wenn das nicht der Fall ist würde sich eine negative Teilchenzahl ergeben. D.h. es muss gelten

\begin{align}
  \label{eq:4}
  \mu \le \epsilon
\end{align}

Da die kleinst mögliche Energie \(\epsilon(k=0) = 0\) mit dem Impuls \(k=0\) ist, gilt für die Gleichung (\ref{eq:4})

\begin{align}
  \label{eq:5}
  \mu \le 0
\end{align}

Dies ist eine wichtige Bedingung für das chemische Potential, dass die Energie beschreibt das man aufwenden muss um ein zusätzliches Teilchen dem System hinzuzufügen. 

Betrachtet man weiterhin Gleichung (\ref{eq:3}) für sehr tiefe Termperaturen \(T\to 0\) stellt man fest, dass \(n(\epsilon-\mu)\to 0\) ebenfalls gegen Null geht. Was dazu führt dass die Anzahl der Teilchen in Gleichung (\ref{eq:1}) ebenfalls gegen Null gehen lässt was mit der Teilchenzahl-Erhaltung im Widerspruch steht. Deswegen Teilt man die Gleichung (\ref{eq:1}) in zwei Summanden wie folgt auf

\begin{align}
  \label{eq:6}
   N(T,V,\mu) = N_0 + N_T = N_0 +  V\int d\epsilon \mathcal N(\epsilon) n(\epsilon-\mu)
\end{align}

Dabei repräsentiert \(N_0\) die Anzahl der Teilchen im Grundzustand \(\epsilon(k=0)\) und der zweite Summand \(N_T\) die Teilchen in allen anderen möglichen Zuständen. \(N_0\) ist für hohe Termperaturen zu vernachlässigen, da man davon ausgeht, dass die meisten Teilchen nicht im Grundzustand sind, für \(N\to\infty\) gilt \(\frac{N_0}{N} \to 0\). Die Quintessenz bei BEC ist, dass bei sinkender Temperatur das chemische Potential im zweiten Summanden dafür sorgt dass die Teilchenanzahl bis zu einer kritischen Temperatur \(T_C\)erhalten bleibt. Ist die Bedingung in Gleichung (\ref{eq:5}) erreicht, kann das chemische Potential nicht mehr für Teilchenerhalt sorgen. D.h. damit die Teilchen dennoch erhalten bleiben, müssen sie sich im ersten Summanden \(N_0\) versammeln und somit den gleichen quantenmechanischen energetischen Zustand \(\epsilon_0 = 0\) einnehmen. Was man als Bose-Einstein-Kondensation bezeichnet. Kurz vor der kritischen Temperatur muss gelten \(N_0\) immer noch Null und \(\mu\) fast Null. Damit können wir die Gleichung (\ref{eq:6}) vereinfachen zu

\begin{align}
  \label{eq:7}
  N(T_C,V,\mu=0) = V\int d\epsilon \mathcal N(\epsilon) n(\epsilon)
\end{align}

Mit den Gleichungen (\ref{eq:2}) und (\ref{eq:3}) ergibt sich

\begin{align}
  \label{eq:8}
   N(T_C,V,\mu=0) = \frac{1}{4\pi^2}\left( \frac{2m}{\hbar^2}  \right)^{\frac{3}{2}} V \int_0^\infty d\epsilon  \frac{\sqrt{\epsilon}}{\exp\left(\frac{\epsilon}{k_B T_C}\right)-1}
\end{align}

Mit der Substitution \(u=\frac{\epsilon}{k_B T_C}\) und \(\frac{du}{d\epsilon}=\frac{1}{k_B T_C} \rightarrow d\epsilon=k_B T_C du\) eingesetzt in Gleichung (\ref{eq:8})

\begin{align}
  \label{eq:9}
    N(T_C,V,\mu=0) &= \frac{1}{4\pi^2}\left( \frac{2m}{\hbar^2}  \right)^{\frac{3}{2}} V \int_0^\infty k_B T_C du  \frac{\sqrt{u k_B T_C}}{e^{u}-1} \notag\\
&=\frac{1}{4\pi^2}\left( \frac{2m k_B T_C}{\hbar^2}  \right)^{\frac{3}{2}} V  \int_0^\infty du  \frac{\sqrt{u}}{e^{u}-1} \notag\\
\end{align}

Das Integral lässt sich mit Hilfe der \(\zeta\)-Funktion bestimmen, es gilt

\begin{align}
  \label{eq:10}
  \zeta(s) = \frac{1}{\Gamma(s)} \int_0^\infty \frac{u^{s-1}}{e^u-1}du
\end{align}

Somit lautet die Gleichung (\ref{eq:9})

\begin{align}
  \label{eq:11}
   N(T_C,V,\mu=0) &= \frac{1}{4\pi^2}\left( \frac{2m k_B T_C}{\hbar^2}  \right)^{\frac{3}{2}} V \zeta(\frac{3}{2})\Gamma(\frac{3}{2})   \notag\\
\end{align}

Um die kritische Temperatur bei der die Kondensation beginnt zu bestimmten stellen wir die Gleichung (\ref{eq:11}) nach \(T_C\) um und setzen \(\Gamma(\frac{3}{2}) = \frac{\sqrt{\pi}}{2}\) ein

\begin{align}
  \label{eq:12}
  T_C = \frac{\hbar^2}{2m k_B}\left( \frac{N4\pi^2}{V\zeta(\frac{3}{2})\Gamma(\frac{3}{2})  } \right)^{\frac{2}{3}} = \frac{ 2\pi \hbar^2}{m k_B }\left( \frac{N}{V\zeta(\frac{3}{2})  } \right)^{\frac{2}{3}}
\end{align}

Damit lässt sich die kritische Temperatur mit der folgenden Gleichung bestimmen

\begin{align}
  \label{eq:13}
  \boxed{ T_C = \frac{ 2\pi \hbar^2}{m k_B }\frac{1}{\zeta(\frac{3}{2})^{\frac{3}{2}}} \left( \frac{N}{V} \right)^{\frac{2}{3}} 
= \frac{ h^2}{2\pi m k_B }\frac{1}{\zeta(\frac{3}{2})^{\frac{3}{2}}} \left( \frac{N}{V} \right)^{\frac{2}{3}} }
\end{align}

Es lässt sich weiterhin  \( \zeta(\frac{3}{2}) \approx 2,61\) ungefähr angeben.


Für das Verhältnis zwischen Teilchen im Grundzustand und den Gesamtanzahl der Teilchen gilt

\begin{align}
  \label{eq:14}
  \frac{N_0}{N} \stackrel{~(\ref{eq:6})}= \frac{N-N_T} N = 1-\frac {N_T} N 
\end{align}

Im Falle \(T<T_C\) sammeln sich die meisten Teilchen im Grundzustand \(N_0\) so dass gilt \(N\approx N_0\) und somit erhalten wir folgendes Verhältnis

\begin{align}
  \label{eq:15}
 \frac{N_0}{N} = 1 - \left( \frac{T}{T_C}\right)^{\frac{3}{2}} \qquad \text{für } T<T_C
\end{align}

Die Gleichung (\ref{eq:15}) ist in Abbildung \ref{fig:1} dargestellt. Man sieht dass bei \(\frac{T}{T_C}=1\) es zu eine Knick kommt. Dieses Verhalten ist typisch für einen Phasenübergang. Das gleiche beobachtet man auch z.B. bei spezifischer Wärmekapazität. 

\begin{figure}
  \centering
  % GNUPLOT: LaTeX picture
\setlength{\unitlength}{0.240900pt}
\ifx\plotpoint\undefined\newsavebox{\plotpoint}\fi
\sbox{\plotpoint}{\rule[-0.200pt]{0.400pt}{0.400pt}}%
\begin{picture}(1500,900)(0,0)
\sbox{\plotpoint}{\rule[-0.200pt]{0.400pt}{0.400pt}}%
\put(140.0,82.0){\rule[-0.200pt]{4.818pt}{0.400pt}}
\put(120,82){\makebox(0,0)[r]{ 0}}
\put(1419.0,82.0){\rule[-0.200pt]{4.818pt}{0.400pt}}
\put(140.0,152.0){\rule[-0.200pt]{4.818pt}{0.400pt}}
\put(120,152){\makebox(0,0)[r]{ 0.1}}
\put(1419.0,152.0){\rule[-0.200pt]{4.818pt}{0.400pt}}
\put(140.0,221.0){\rule[-0.200pt]{4.818pt}{0.400pt}}
\put(120,221){\makebox(0,0)[r]{ 0.2}}
\put(1419.0,221.0){\rule[-0.200pt]{4.818pt}{0.400pt}}
\put(140.0,291.0){\rule[-0.200pt]{4.818pt}{0.400pt}}
\put(120,291){\makebox(0,0)[r]{ 0.3}}
\put(1419.0,291.0){\rule[-0.200pt]{4.818pt}{0.400pt}}
\put(140.0,360.0){\rule[-0.200pt]{4.818pt}{0.400pt}}
\put(120,360){\makebox(0,0)[r]{ 0.4}}
\put(1419.0,360.0){\rule[-0.200pt]{4.818pt}{0.400pt}}
\put(140.0,430.0){\rule[-0.200pt]{4.818pt}{0.400pt}}
\put(120,430){\makebox(0,0)[r]{ 0.5}}
\put(1419.0,430.0){\rule[-0.200pt]{4.818pt}{0.400pt}}
\put(140.0,499.0){\rule[-0.200pt]{4.818pt}{0.400pt}}
\put(120,499){\makebox(0,0)[r]{ 0.6}}
\put(1419.0,499.0){\rule[-0.200pt]{4.818pt}{0.400pt}}
\put(140.0,568.0){\rule[-0.200pt]{4.818pt}{0.400pt}}
\put(120,568){\makebox(0,0)[r]{ 0.7}}
\put(1419.0,568.0){\rule[-0.200pt]{4.818pt}{0.400pt}}
\put(140.0,638.0){\rule[-0.200pt]{4.818pt}{0.400pt}}
\put(120,638){\makebox(0,0)[r]{ 0.8}}
\put(1419.0,638.0){\rule[-0.200pt]{4.818pt}{0.400pt}}
\put(140.0,707.0){\rule[-0.200pt]{4.818pt}{0.400pt}}
\put(120,707){\makebox(0,0)[r]{ 0.9}}
\put(1419.0,707.0){\rule[-0.200pt]{4.818pt}{0.400pt}}
\put(140.0,777.0){\rule[-0.200pt]{4.818pt}{0.400pt}}
\put(120,777){\makebox(0,0)[r]{ 1}}
\put(1419.0,777.0){\rule[-0.200pt]{4.818pt}{0.400pt}}
\put(140.0,82.0){\rule[-0.200pt]{0.400pt}{4.818pt}}
\put(140,41){\makebox(0,0){ 0}}
\put(140.0,757.0){\rule[-0.200pt]{0.400pt}{4.818pt}}
\put(465.0,82.0){\rule[-0.200pt]{0.400pt}{4.818pt}}
\put(465,41){\makebox(0,0){ 0.5}}
\put(465.0,757.0){\rule[-0.200pt]{0.400pt}{4.818pt}}
\put(790.0,82.0){\rule[-0.200pt]{0.400pt}{4.818pt}}
\put(790,41){\makebox(0,0){ 1}}
\put(790.0,757.0){\rule[-0.200pt]{0.400pt}{4.818pt}}
\put(1114.0,82.0){\rule[-0.200pt]{0.400pt}{4.818pt}}
\put(1114,41){\makebox(0,0){ 1.5}}
\put(1114.0,757.0){\rule[-0.200pt]{0.400pt}{4.818pt}}
\put(1439.0,82.0){\rule[-0.200pt]{0.400pt}{4.818pt}}
\put(1439,41){\makebox(0,0){ 2}}
\put(1439.0,757.0){\rule[-0.200pt]{0.400pt}{4.818pt}}
\put(140.0,82.0){\rule[-0.200pt]{0.400pt}{167.425pt}}
\put(140.0,82.0){\rule[-0.200pt]{312.929pt}{0.400pt}}
\put(1439.0,82.0){\rule[-0.200pt]{0.400pt}{167.425pt}}
\put(140.0,777.0){\rule[-0.200pt]{312.929pt}{0.400pt}}
\put(789,839){\makebox(0,0){$\frac{N_0}{N}$ über $\frac{T}{T_C}$}}
\put(140,777){\usebox{\plotpoint}}
\put(140,777){\usebox{\plotpoint}}
\put(140,777){\usebox{\plotpoint}}
\put(140,777){\usebox{\plotpoint}}
\put(140,777){\usebox{\plotpoint}}
\put(140,777){\usebox{\plotpoint}}
\put(140,777){\usebox{\plotpoint}}
\put(140,777){\usebox{\plotpoint}}
\put(140,777){\usebox{\plotpoint}}
\put(140,777){\usebox{\plotpoint}}
\put(140,777){\usebox{\plotpoint}}
\put(140,777){\usebox{\plotpoint}}
\put(140,777){\usebox{\plotpoint}}
\put(140,777){\usebox{\plotpoint}}
\put(140,777){\usebox{\plotpoint}}
\put(140,777){\usebox{\plotpoint}}
\put(140,777){\usebox{\plotpoint}}
\put(140,777){\usebox{\plotpoint}}
\put(140,777){\usebox{\plotpoint}}
\put(140,777){\usebox{\plotpoint}}
\put(140,777){\usebox{\plotpoint}}
\put(140,777){\usebox{\plotpoint}}
\put(140,777){\usebox{\plotpoint}}
\put(140,777){\usebox{\plotpoint}}
\put(140,777){\usebox{\plotpoint}}
\put(140,777){\usebox{\plotpoint}}
\put(140,777){\usebox{\plotpoint}}
\put(140,777){\usebox{\plotpoint}}
\put(140,777){\usebox{\plotpoint}}
\put(140,777){\usebox{\plotpoint}}
\put(140,777){\usebox{\plotpoint}}
\put(140,777){\usebox{\plotpoint}}
\put(140,777){\usebox{\plotpoint}}
\put(140,777){\usebox{\plotpoint}}
\put(140,777){\usebox{\plotpoint}}
\put(140,777){\usebox{\plotpoint}}
\put(140,777){\usebox{\plotpoint}}
\put(140,777){\usebox{\plotpoint}}
\put(140,777){\usebox{\plotpoint}}
\put(140.0,777.0){\rule[-0.200pt]{1.204pt}{0.400pt}}
\put(145.0,776.0){\usebox{\plotpoint}}
\put(145.0,776.0){\rule[-0.200pt]{1.445pt}{0.400pt}}
\put(151.0,775.0){\usebox{\plotpoint}}
\put(151.0,775.0){\rule[-0.200pt]{0.964pt}{0.400pt}}
\put(155.0,774.0){\usebox{\plotpoint}}
\put(155.0,774.0){\rule[-0.200pt]{0.964pt}{0.400pt}}
\put(159.0,773.0){\usebox{\plotpoint}}
\put(159.0,773.0){\rule[-0.200pt]{0.964pt}{0.400pt}}
\put(163.0,772.0){\usebox{\plotpoint}}
\put(163.0,772.0){\rule[-0.200pt]{0.723pt}{0.400pt}}
\put(166.0,771.0){\usebox{\plotpoint}}
\put(166.0,771.0){\rule[-0.200pt]{0.723pt}{0.400pt}}
\put(169.0,770.0){\usebox{\plotpoint}}
\put(169.0,770.0){\rule[-0.200pt]{0.723pt}{0.400pt}}
\put(172.0,769.0){\usebox{\plotpoint}}
\put(172.0,769.0){\rule[-0.200pt]{0.482pt}{0.400pt}}
\put(174.0,768.0){\usebox{\plotpoint}}
\put(174.0,768.0){\rule[-0.200pt]{0.723pt}{0.400pt}}
\put(177.0,767.0){\usebox{\plotpoint}}
\put(177.0,767.0){\rule[-0.200pt]{0.723pt}{0.400pt}}
\put(180.0,766.0){\usebox{\plotpoint}}
\put(180.0,766.0){\rule[-0.200pt]{0.482pt}{0.400pt}}
\put(182.0,765.0){\usebox{\plotpoint}}
\put(182.0,765.0){\rule[-0.200pt]{0.723pt}{0.400pt}}
\put(185.0,764.0){\usebox{\plotpoint}}
\put(185.0,764.0){\rule[-0.200pt]{0.482pt}{0.400pt}}
\put(187.0,763.0){\usebox{\plotpoint}}
\put(187.0,763.0){\rule[-0.200pt]{0.482pt}{0.400pt}}
\put(189.0,762.0){\usebox{\plotpoint}}
\put(189.0,762.0){\rule[-0.200pt]{0.482pt}{0.400pt}}
\put(191.0,761.0){\usebox{\plotpoint}}
\put(191.0,761.0){\rule[-0.200pt]{0.723pt}{0.400pt}}
\put(194.0,760.0){\usebox{\plotpoint}}
\put(194.0,760.0){\rule[-0.200pt]{0.482pt}{0.400pt}}
\put(196.0,759.0){\usebox{\plotpoint}}
\put(196.0,759.0){\rule[-0.200pt]{0.482pt}{0.400pt}}
\put(198.0,758.0){\usebox{\plotpoint}}
\put(198.0,758.0){\rule[-0.200pt]{0.482pt}{0.400pt}}
\put(200.0,757.0){\usebox{\plotpoint}}
\put(200.0,757.0){\rule[-0.200pt]{0.482pt}{0.400pt}}
\put(202.0,756.0){\usebox{\plotpoint}}
\put(202.0,756.0){\rule[-0.200pt]{0.482pt}{0.400pt}}
\put(204.0,755.0){\usebox{\plotpoint}}
\put(204.0,755.0){\rule[-0.200pt]{0.482pt}{0.400pt}}
\put(206.0,754.0){\usebox{\plotpoint}}
\put(206.0,754.0){\rule[-0.200pt]{0.482pt}{0.400pt}}
\put(208.0,753.0){\usebox{\plotpoint}}
\put(208.0,753.0){\rule[-0.200pt]{0.482pt}{0.400pt}}
\put(210.0,752.0){\usebox{\plotpoint}}
\put(210.0,752.0){\rule[-0.200pt]{0.482pt}{0.400pt}}
\put(212.0,751.0){\usebox{\plotpoint}}
\put(212.0,751.0){\rule[-0.200pt]{0.482pt}{0.400pt}}
\put(214.0,750.0){\usebox{\plotpoint}}
\put(214.0,750.0){\usebox{\plotpoint}}
\put(215.0,749.0){\usebox{\plotpoint}}
\put(215.0,749.0){\rule[-0.200pt]{0.482pt}{0.400pt}}
\put(217.0,748.0){\usebox{\plotpoint}}
\put(217.0,748.0){\rule[-0.200pt]{0.482pt}{0.400pt}}
\put(219.0,747.0){\usebox{\plotpoint}}
\put(219.0,747.0){\rule[-0.200pt]{0.482pt}{0.400pt}}
\put(221.0,746.0){\usebox{\plotpoint}}
\put(221.0,746.0){\rule[-0.200pt]{0.482pt}{0.400pt}}
\put(223.0,745.0){\usebox{\plotpoint}}
\put(223.0,745.0){\usebox{\plotpoint}}
\put(224.0,744.0){\usebox{\plotpoint}}
\put(224.0,744.0){\rule[-0.200pt]{0.482pt}{0.400pt}}
\put(226.0,743.0){\usebox{\plotpoint}}
\put(226.0,743.0){\rule[-0.200pt]{0.482pt}{0.400pt}}
\put(228.0,742.0){\usebox{\plotpoint}}
\put(228.0,742.0){\usebox{\plotpoint}}
\put(229.0,741.0){\usebox{\plotpoint}}
\put(229.0,741.0){\rule[-0.200pt]{0.482pt}{0.400pt}}
\put(231.0,740.0){\usebox{\plotpoint}}
\put(231.0,740.0){\rule[-0.200pt]{0.482pt}{0.400pt}}
\put(233.0,739.0){\usebox{\plotpoint}}
\put(233.0,739.0){\usebox{\plotpoint}}
\put(234.0,738.0){\usebox{\plotpoint}}
\put(234.0,738.0){\rule[-0.200pt]{0.482pt}{0.400pt}}
\put(236.0,737.0){\usebox{\plotpoint}}
\put(236.0,737.0){\rule[-0.200pt]{0.482pt}{0.400pt}}
\put(238.0,736.0){\usebox{\plotpoint}}
\put(238.0,736.0){\usebox{\plotpoint}}
\put(239.0,735.0){\usebox{\plotpoint}}
\put(239.0,735.0){\rule[-0.200pt]{0.482pt}{0.400pt}}
\put(241.0,734.0){\usebox{\plotpoint}}
\put(241.0,734.0){\usebox{\plotpoint}}
\put(242.0,733.0){\usebox{\plotpoint}}
\put(242.0,733.0){\rule[-0.200pt]{0.482pt}{0.400pt}}
\put(244.0,732.0){\usebox{\plotpoint}}
\put(245,730.67){\rule{0.241pt}{0.400pt}}
\multiput(245.00,731.17)(0.500,-1.000){2}{\rule{0.120pt}{0.400pt}}
\put(244.0,732.0){\usebox{\plotpoint}}
\put(246,731){\usebox{\plotpoint}}
\put(246,731){\usebox{\plotpoint}}
\put(246,731){\usebox{\plotpoint}}
\put(246,731){\usebox{\plotpoint}}
\put(246,731){\usebox{\plotpoint}}
\put(246,731){\usebox{\plotpoint}}
\put(246,731){\usebox{\plotpoint}}
\put(246,731){\usebox{\plotpoint}}
\put(246,731){\usebox{\plotpoint}}
\put(246,731){\usebox{\plotpoint}}
\put(246,731){\usebox{\plotpoint}}
\put(246,731){\usebox{\plotpoint}}
\put(246,731){\usebox{\plotpoint}}
\put(246,731){\usebox{\plotpoint}}
\put(246,731){\usebox{\plotpoint}}
\put(246,731){\usebox{\plotpoint}}
\put(246,731){\usebox{\plotpoint}}
\put(246,731){\usebox{\plotpoint}}
\put(246,731){\usebox{\plotpoint}}
\put(246,731){\usebox{\plotpoint}}
\put(246,731){\usebox{\plotpoint}}
\put(246,731){\usebox{\plotpoint}}
\put(246,731){\usebox{\plotpoint}}
\put(246,731){\usebox{\plotpoint}}
\put(246,731){\usebox{\plotpoint}}
\put(246,731){\usebox{\plotpoint}}
\put(246,731){\usebox{\plotpoint}}
\put(246,731){\usebox{\plotpoint}}
\put(246,731){\usebox{\plotpoint}}
\put(246,731){\usebox{\plotpoint}}
\put(246,731){\usebox{\plotpoint}}
\put(246,731){\usebox{\plotpoint}}
\put(246,731){\usebox{\plotpoint}}
\put(246,731){\usebox{\plotpoint}}
\put(246,731){\usebox{\plotpoint}}
\put(246,731){\usebox{\plotpoint}}
\put(246,731){\usebox{\plotpoint}}
\put(246,731){\usebox{\plotpoint}}
\put(246,731){\usebox{\plotpoint}}
\put(246,731){\usebox{\plotpoint}}
\put(246,731){\usebox{\plotpoint}}
\put(246,731){\usebox{\plotpoint}}
\put(246,731){\usebox{\plotpoint}}
\put(246,731){\usebox{\plotpoint}}
\put(246,731){\usebox{\plotpoint}}
\put(246,731){\usebox{\plotpoint}}
\put(246,731){\usebox{\plotpoint}}
\put(246,731){\usebox{\plotpoint}}
\put(246,731){\usebox{\plotpoint}}
\put(246,731){\usebox{\plotpoint}}
\put(246,731){\usebox{\plotpoint}}
\put(246,731){\usebox{\plotpoint}}
\put(246,731){\usebox{\plotpoint}}
\put(246,731){\usebox{\plotpoint}}
\put(246,731){\usebox{\plotpoint}}
\put(246,731){\usebox{\plotpoint}}
\put(246,731){\usebox{\plotpoint}}
\put(246,731){\usebox{\plotpoint}}
\put(246,731){\usebox{\plotpoint}}
\put(246,731){\usebox{\plotpoint}}
\put(246,731){\usebox{\plotpoint}}
\put(246,731){\usebox{\plotpoint}}
\put(246,731){\usebox{\plotpoint}}
\put(246,731){\usebox{\plotpoint}}
\put(246,731){\usebox{\plotpoint}}
\put(246,731){\usebox{\plotpoint}}
\put(246,731){\usebox{\plotpoint}}
\put(246,731){\usebox{\plotpoint}}
\put(246,731){\usebox{\plotpoint}}
\put(246,731){\usebox{\plotpoint}}
\put(246,731){\usebox{\plotpoint}}
\put(246,731){\usebox{\plotpoint}}
\put(246,731){\usebox{\plotpoint}}
\put(246,731){\usebox{\plotpoint}}
\put(246,731){\usebox{\plotpoint}}
\put(246,731){\usebox{\plotpoint}}
\put(246.0,731.0){\usebox{\plotpoint}}
\put(247.0,730.0){\usebox{\plotpoint}}
\put(247.0,730.0){\rule[-0.200pt]{0.482pt}{0.400pt}}
\put(249.0,729.0){\usebox{\plotpoint}}
\put(249.0,729.0){\usebox{\plotpoint}}
\put(250.0,728.0){\usebox{\plotpoint}}
\put(250.0,728.0){\rule[-0.200pt]{0.482pt}{0.400pt}}
\put(252.0,727.0){\usebox{\plotpoint}}
\put(252.0,727.0){\usebox{\plotpoint}}
\put(253.0,726.0){\usebox{\plotpoint}}
\put(253.0,726.0){\rule[-0.200pt]{0.482pt}{0.400pt}}
\put(255.0,725.0){\usebox{\plotpoint}}
\put(255.0,725.0){\usebox{\plotpoint}}
\put(256.0,724.0){\usebox{\plotpoint}}
\put(256.0,724.0){\rule[-0.200pt]{0.482pt}{0.400pt}}
\put(258.0,723.0){\usebox{\plotpoint}}
\put(258.0,723.0){\usebox{\plotpoint}}
\put(259.0,722.0){\usebox{\plotpoint}}
\put(259.0,722.0){\usebox{\plotpoint}}
\put(260.0,721.0){\usebox{\plotpoint}}
\put(260.0,721.0){\rule[-0.200pt]{0.482pt}{0.400pt}}
\put(262.0,720.0){\usebox{\plotpoint}}
\put(262.0,720.0){\usebox{\plotpoint}}
\put(263.0,719.0){\usebox{\plotpoint}}
\put(263.0,719.0){\rule[-0.200pt]{0.482pt}{0.400pt}}
\put(265.0,718.0){\usebox{\plotpoint}}
\put(265.0,718.0){\usebox{\plotpoint}}
\put(266.0,717.0){\usebox{\plotpoint}}
\put(266.0,717.0){\rule[-0.200pt]{0.482pt}{0.400pt}}
\put(268.0,716.0){\usebox{\plotpoint}}
\put(268.0,716.0){\usebox{\plotpoint}}
\put(269.0,715.0){\usebox{\plotpoint}}
\put(269.0,715.0){\usebox{\plotpoint}}
\put(270.0,714.0){\usebox{\plotpoint}}
\put(270.0,714.0){\rule[-0.200pt]{0.482pt}{0.400pt}}
\put(272.0,713.0){\usebox{\plotpoint}}
\put(272.0,713.0){\usebox{\plotpoint}}
\put(273.0,712.0){\usebox{\plotpoint}}
\put(274,710.67){\rule{0.241pt}{0.400pt}}
\multiput(274.00,711.17)(0.500,-1.000){2}{\rule{0.120pt}{0.400pt}}
\put(273.0,712.0){\usebox{\plotpoint}}
\put(275,711){\usebox{\plotpoint}}
\put(275,711){\usebox{\plotpoint}}
\put(275,711){\usebox{\plotpoint}}
\put(275,711){\usebox{\plotpoint}}
\put(275,711){\usebox{\plotpoint}}
\put(275,711){\usebox{\plotpoint}}
\put(275,711){\usebox{\plotpoint}}
\put(275,711){\usebox{\plotpoint}}
\put(275,711){\usebox{\plotpoint}}
\put(275,711){\usebox{\plotpoint}}
\put(275,711){\usebox{\plotpoint}}
\put(275,711){\usebox{\plotpoint}}
\put(275,711){\usebox{\plotpoint}}
\put(275,711){\usebox{\plotpoint}}
\put(275,711){\usebox{\plotpoint}}
\put(275,711){\usebox{\plotpoint}}
\put(275,711){\usebox{\plotpoint}}
\put(275,711){\usebox{\plotpoint}}
\put(275,711){\usebox{\plotpoint}}
\put(275,711){\usebox{\plotpoint}}
\put(275,711){\usebox{\plotpoint}}
\put(275,711){\usebox{\plotpoint}}
\put(275,711){\usebox{\plotpoint}}
\put(275,711){\usebox{\plotpoint}}
\put(275,711){\usebox{\plotpoint}}
\put(275,711){\usebox{\plotpoint}}
\put(275,711){\usebox{\plotpoint}}
\put(275,711){\usebox{\plotpoint}}
\put(275,711){\usebox{\plotpoint}}
\put(275,711){\usebox{\plotpoint}}
\put(275,711){\usebox{\plotpoint}}
\put(275,711){\usebox{\plotpoint}}
\put(275,711){\usebox{\plotpoint}}
\put(275,711){\usebox{\plotpoint}}
\put(275,711){\usebox{\plotpoint}}
\put(275,711){\usebox{\plotpoint}}
\put(275,711){\usebox{\plotpoint}}
\put(275,711){\usebox{\plotpoint}}
\put(275,711){\usebox{\plotpoint}}
\put(275,711){\usebox{\plotpoint}}
\put(275,711){\usebox{\plotpoint}}
\put(275,711){\usebox{\plotpoint}}
\put(275,711){\usebox{\plotpoint}}
\put(275,711){\usebox{\plotpoint}}
\put(275,711){\usebox{\plotpoint}}
\put(275,711){\usebox{\plotpoint}}
\put(275,711){\usebox{\plotpoint}}
\put(275,711){\usebox{\plotpoint}}
\put(275,711){\usebox{\plotpoint}}
\put(275,711){\usebox{\plotpoint}}
\put(275,711){\usebox{\plotpoint}}
\put(275,711){\usebox{\plotpoint}}
\put(275,711){\usebox{\plotpoint}}
\put(275,711){\usebox{\plotpoint}}
\put(275,711){\usebox{\plotpoint}}
\put(275,711){\usebox{\plotpoint}}
\put(275,711){\usebox{\plotpoint}}
\put(275,711){\usebox{\plotpoint}}
\put(275,711){\usebox{\plotpoint}}
\put(275,711){\usebox{\plotpoint}}
\put(275,711){\usebox{\plotpoint}}
\put(275,711){\usebox{\plotpoint}}
\put(275,711){\usebox{\plotpoint}}
\put(275,711){\usebox{\plotpoint}}
\put(275,711){\usebox{\plotpoint}}
\put(275,711){\usebox{\plotpoint}}
\put(275,711){\usebox{\plotpoint}}
\put(275,711){\usebox{\plotpoint}}
\put(275,711){\usebox{\plotpoint}}
\put(275,711){\usebox{\plotpoint}}
\put(275,711){\usebox{\plotpoint}}
\put(275,711){\usebox{\plotpoint}}
\put(275,711){\usebox{\plotpoint}}
\put(275,711){\usebox{\plotpoint}}
\put(275,711){\usebox{\plotpoint}}
\put(275.0,711.0){\usebox{\plotpoint}}
\put(276.0,710.0){\usebox{\plotpoint}}
\put(276.0,710.0){\usebox{\plotpoint}}
\put(277.0,709.0){\usebox{\plotpoint}}
\put(277.0,709.0){\rule[-0.200pt]{0.482pt}{0.400pt}}
\put(279.0,708.0){\usebox{\plotpoint}}
\put(279.0,708.0){\usebox{\plotpoint}}
\put(280.0,707.0){\usebox{\plotpoint}}
\put(280.0,707.0){\usebox{\plotpoint}}
\put(281.0,706.0){\usebox{\plotpoint}}
\put(281.0,706.0){\rule[-0.200pt]{0.482pt}{0.400pt}}
\put(283.0,705.0){\usebox{\plotpoint}}
\put(283.0,705.0){\usebox{\plotpoint}}
\put(284.0,704.0){\usebox{\plotpoint}}
\put(284.0,704.0){\usebox{\plotpoint}}
\put(285.0,703.0){\usebox{\plotpoint}}
\put(285.0,703.0){\rule[-0.200pt]{0.482pt}{0.400pt}}
\put(287.0,702.0){\usebox{\plotpoint}}
\put(287.0,702.0){\usebox{\plotpoint}}
\put(288.0,701.0){\usebox{\plotpoint}}
\put(288.0,701.0){\usebox{\plotpoint}}
\put(289.0,700.0){\usebox{\plotpoint}}
\put(289.0,700.0){\usebox{\plotpoint}}
\put(290.0,699.0){\usebox{\plotpoint}}
\put(290.0,699.0){\rule[-0.200pt]{0.482pt}{0.400pt}}
\put(292.0,698.0){\usebox{\plotpoint}}
\put(292.0,698.0){\usebox{\plotpoint}}
\put(293.0,697.0){\usebox{\plotpoint}}
\put(293.0,697.0){\usebox{\plotpoint}}
\put(294.0,696.0){\usebox{\plotpoint}}
\put(294.0,696.0){\rule[-0.200pt]{0.482pt}{0.400pt}}
\put(296.0,695.0){\usebox{\plotpoint}}
\put(296.0,695.0){\usebox{\plotpoint}}
\put(297.0,694.0){\usebox{\plotpoint}}
\put(297.0,694.0){\usebox{\plotpoint}}
\put(298.0,693.0){\usebox{\plotpoint}}
\put(298.0,693.0){\usebox{\plotpoint}}
\put(299.0,692.0){\usebox{\plotpoint}}
\put(299.0,692.0){\rule[-0.200pt]{0.482pt}{0.400pt}}
\put(301.0,691.0){\usebox{\plotpoint}}
\put(301.0,691.0){\usebox{\plotpoint}}
\put(302.0,690.0){\usebox{\plotpoint}}
\put(302.0,690.0){\usebox{\plotpoint}}
\put(303.0,689.0){\usebox{\plotpoint}}
\put(303.0,689.0){\usebox{\plotpoint}}
\put(304.0,688.0){\usebox{\plotpoint}}
\put(304.0,688.0){\rule[-0.200pt]{0.482pt}{0.400pt}}
\put(306.0,687.0){\usebox{\plotpoint}}
\put(306.0,687.0){\usebox{\plotpoint}}
\put(307.0,686.0){\usebox{\plotpoint}}
\put(307.0,686.0){\usebox{\plotpoint}}
\put(308.0,685.0){\usebox{\plotpoint}}
\put(308.0,685.0){\usebox{\plotpoint}}
\put(309.0,684.0){\usebox{\plotpoint}}
\put(309.0,684.0){\rule[-0.200pt]{0.482pt}{0.400pt}}
\put(311.0,683.0){\usebox{\plotpoint}}
\put(311.0,683.0){\usebox{\plotpoint}}
\put(312.0,682.0){\usebox{\plotpoint}}
\put(312.0,682.0){\usebox{\plotpoint}}
\put(313.0,681.0){\usebox{\plotpoint}}
\put(313.0,681.0){\usebox{\plotpoint}}
\put(314.0,680.0){\usebox{\plotpoint}}
\put(314.0,680.0){\usebox{\plotpoint}}
\put(315.0,679.0){\usebox{\plotpoint}}
\put(315.0,679.0){\rule[-0.200pt]{0.482pt}{0.400pt}}
\put(317.0,678.0){\usebox{\plotpoint}}
\put(317.0,678.0){\usebox{\plotpoint}}
\put(318.0,677.0){\usebox{\plotpoint}}
\put(318.0,677.0){\usebox{\plotpoint}}
\put(319.0,676.0){\usebox{\plotpoint}}
\put(319.0,676.0){\usebox{\plotpoint}}
\put(320.0,675.0){\usebox{\plotpoint}}
\put(320.0,675.0){\usebox{\plotpoint}}
\put(321.0,674.0){\usebox{\plotpoint}}
\put(321.0,674.0){\usebox{\plotpoint}}
\put(322.0,673.0){\usebox{\plotpoint}}
\put(322.0,673.0){\rule[-0.200pt]{0.482pt}{0.400pt}}
\put(324.0,672.0){\usebox{\plotpoint}}
\put(324.0,672.0){\usebox{\plotpoint}}
\put(325.0,671.0){\usebox{\plotpoint}}
\put(325.0,671.0){\usebox{\plotpoint}}
\put(326.0,670.0){\usebox{\plotpoint}}
\put(326.0,670.0){\usebox{\plotpoint}}
\put(327.0,669.0){\usebox{\plotpoint}}
\put(327.0,669.0){\usebox{\plotpoint}}
\put(328.0,668.0){\usebox{\plotpoint}}
\put(328.0,668.0){\usebox{\plotpoint}}
\put(329.0,667.0){\usebox{\plotpoint}}
\put(329.0,667.0){\rule[-0.200pt]{0.482pt}{0.400pt}}
\put(331.0,666.0){\usebox{\plotpoint}}
\put(331.0,666.0){\usebox{\plotpoint}}
\put(332.0,665.0){\usebox{\plotpoint}}
\put(332.0,665.0){\usebox{\plotpoint}}
\put(333.0,664.0){\usebox{\plotpoint}}
\put(333.0,664.0){\usebox{\plotpoint}}
\put(334.0,663.0){\usebox{\plotpoint}}
\put(334.0,663.0){\usebox{\plotpoint}}
\put(335.0,662.0){\usebox{\plotpoint}}
\put(335.0,662.0){\usebox{\plotpoint}}
\put(336.0,661.0){\usebox{\plotpoint}}
\put(336.0,661.0){\usebox{\plotpoint}}
\put(337.0,660.0){\usebox{\plotpoint}}
\put(337.0,660.0){\rule[-0.200pt]{0.482pt}{0.400pt}}
\put(339.0,659.0){\usebox{\plotpoint}}
\put(339.0,659.0){\usebox{\plotpoint}}
\put(340.0,658.0){\usebox{\plotpoint}}
\put(340.0,658.0){\usebox{\plotpoint}}
\put(341.0,657.0){\usebox{\plotpoint}}
\put(341.0,657.0){\usebox{\plotpoint}}
\put(342.0,656.0){\usebox{\plotpoint}}
\put(342.0,656.0){\usebox{\plotpoint}}
\put(343.0,655.0){\usebox{\plotpoint}}
\put(343.0,655.0){\usebox{\plotpoint}}
\put(344.0,654.0){\usebox{\plotpoint}}
\put(344.0,654.0){\usebox{\plotpoint}}
\put(345.0,653.0){\usebox{\plotpoint}}
\put(345.0,653.0){\usebox{\plotpoint}}
\put(346.0,652.0){\usebox{\plotpoint}}
\put(347,650.67){\rule{0.241pt}{0.400pt}}
\multiput(347.00,651.17)(0.500,-1.000){2}{\rule{0.120pt}{0.400pt}}
\put(346.0,652.0){\usebox{\plotpoint}}
\put(348,651){\usebox{\plotpoint}}
\put(348,651){\usebox{\plotpoint}}
\put(348,651){\usebox{\plotpoint}}
\put(348,651){\usebox{\plotpoint}}
\put(348,651){\usebox{\plotpoint}}
\put(348,651){\usebox{\plotpoint}}
\put(348,651){\usebox{\plotpoint}}
\put(348,651){\usebox{\plotpoint}}
\put(348,651){\usebox{\plotpoint}}
\put(348,651){\usebox{\plotpoint}}
\put(348,651){\usebox{\plotpoint}}
\put(348,651){\usebox{\plotpoint}}
\put(348,651){\usebox{\plotpoint}}
\put(348,651){\usebox{\plotpoint}}
\put(348,651){\usebox{\plotpoint}}
\put(348,651){\usebox{\plotpoint}}
\put(348,651){\usebox{\plotpoint}}
\put(348,651){\usebox{\plotpoint}}
\put(348,651){\usebox{\plotpoint}}
\put(348,651){\usebox{\plotpoint}}
\put(348,651){\usebox{\plotpoint}}
\put(348,651){\usebox{\plotpoint}}
\put(348,651){\usebox{\plotpoint}}
\put(348,651){\usebox{\plotpoint}}
\put(348,651){\usebox{\plotpoint}}
\put(348,651){\usebox{\plotpoint}}
\put(348,651){\usebox{\plotpoint}}
\put(348,651){\usebox{\plotpoint}}
\put(348,651){\usebox{\plotpoint}}
\put(348,651){\usebox{\plotpoint}}
\put(348,651){\usebox{\plotpoint}}
\put(348,651){\usebox{\plotpoint}}
\put(348,651){\usebox{\plotpoint}}
\put(348,651){\usebox{\plotpoint}}
\put(348,651){\usebox{\plotpoint}}
\put(348,651){\usebox{\plotpoint}}
\put(348,651){\usebox{\plotpoint}}
\put(348,651){\usebox{\plotpoint}}
\put(348,651){\usebox{\plotpoint}}
\put(348,651){\usebox{\plotpoint}}
\put(348,651){\usebox{\plotpoint}}
\put(348,651){\usebox{\plotpoint}}
\put(348,651){\usebox{\plotpoint}}
\put(348,651){\usebox{\plotpoint}}
\put(348,651){\usebox{\plotpoint}}
\put(348,651){\usebox{\plotpoint}}
\put(348,651){\usebox{\plotpoint}}
\put(348,651){\usebox{\plotpoint}}
\put(348,651){\usebox{\plotpoint}}
\put(348,651){\usebox{\plotpoint}}
\put(348,651){\usebox{\plotpoint}}
\put(348,651){\usebox{\plotpoint}}
\put(348,651){\usebox{\plotpoint}}
\put(348,651){\usebox{\plotpoint}}
\put(348,651){\usebox{\plotpoint}}
\put(348,651){\usebox{\plotpoint}}
\put(348,651){\usebox{\plotpoint}}
\put(348,651){\usebox{\plotpoint}}
\put(348,651){\usebox{\plotpoint}}
\put(348,651){\usebox{\plotpoint}}
\put(348,651){\usebox{\plotpoint}}
\put(348,651){\usebox{\plotpoint}}
\put(348,651){\usebox{\plotpoint}}
\put(348,651){\usebox{\plotpoint}}
\put(348,651){\usebox{\plotpoint}}
\put(348,651){\usebox{\plotpoint}}
\put(348,651){\usebox{\plotpoint}}
\put(348,651){\usebox{\plotpoint}}
\put(348,651){\usebox{\plotpoint}}
\put(348,651){\usebox{\plotpoint}}
\put(348,651){\usebox{\plotpoint}}
\put(348,651){\usebox{\plotpoint}}
\put(348,651){\usebox{\plotpoint}}
\put(348,651){\usebox{\plotpoint}}
\put(348,651){\usebox{\plotpoint}}
\put(348,651){\usebox{\plotpoint}}
\put(348.0,651.0){\usebox{\plotpoint}}
\put(349.0,650.0){\usebox{\plotpoint}}
\put(349.0,650.0){\usebox{\plotpoint}}
\put(350.0,649.0){\usebox{\plotpoint}}
\put(350.0,649.0){\usebox{\plotpoint}}
\put(351.0,648.0){\usebox{\plotpoint}}
\put(351.0,648.0){\usebox{\plotpoint}}
\put(352.0,647.0){\usebox{\plotpoint}}
\put(352.0,647.0){\usebox{\plotpoint}}
\put(353.0,646.0){\usebox{\plotpoint}}
\put(353.0,646.0){\usebox{\plotpoint}}
\put(354.0,645.0){\usebox{\plotpoint}}
\put(354.0,645.0){\usebox{\plotpoint}}
\put(355.0,644.0){\usebox{\plotpoint}}
\put(355.0,644.0){\usebox{\plotpoint}}
\put(356.0,643.0){\usebox{\plotpoint}}
\put(356.0,643.0){\usebox{\plotpoint}}
\put(357.0,642.0){\usebox{\plotpoint}}
\put(357.0,642.0){\usebox{\plotpoint}}
\put(358.0,641.0){\usebox{\plotpoint}}
\put(358.0,641.0){\usebox{\plotpoint}}
\put(359.0,640.0){\usebox{\plotpoint}}
\put(359.0,640.0){\rule[-0.200pt]{0.482pt}{0.400pt}}
\put(361.0,639.0){\usebox{\plotpoint}}
\put(361.0,639.0){\usebox{\plotpoint}}
\put(362.0,638.0){\usebox{\plotpoint}}
\put(362.0,638.0){\usebox{\plotpoint}}
\put(363.0,637.0){\usebox{\plotpoint}}
\put(363.0,637.0){\usebox{\plotpoint}}
\put(364.0,636.0){\usebox{\plotpoint}}
\put(364.0,636.0){\usebox{\plotpoint}}
\put(365.0,635.0){\usebox{\plotpoint}}
\put(365.0,635.0){\usebox{\plotpoint}}
\put(366.0,634.0){\usebox{\plotpoint}}
\put(366.0,634.0){\usebox{\plotpoint}}
\put(367.0,633.0){\usebox{\plotpoint}}
\put(367.0,633.0){\usebox{\plotpoint}}
\put(368.0,632.0){\usebox{\plotpoint}}
\put(368.0,632.0){\usebox{\plotpoint}}
\put(369.0,631.0){\usebox{\plotpoint}}
\put(369.0,631.0){\usebox{\plotpoint}}
\put(370.0,630.0){\usebox{\plotpoint}}
\put(370.0,630.0){\usebox{\plotpoint}}
\put(371.0,629.0){\usebox{\plotpoint}}
\put(371.0,629.0){\usebox{\plotpoint}}
\put(372.0,628.0){\usebox{\plotpoint}}
\put(372.0,628.0){\usebox{\plotpoint}}
\put(373.0,627.0){\usebox{\plotpoint}}
\put(373.0,627.0){\usebox{\plotpoint}}
\put(374.0,626.0){\usebox{\plotpoint}}
\put(374.0,626.0){\usebox{\plotpoint}}
\put(375.0,625.0){\usebox{\plotpoint}}
\put(375.0,625.0){\usebox{\plotpoint}}
\put(376.0,624.0){\usebox{\plotpoint}}
\put(376.0,624.0){\usebox{\plotpoint}}
\put(377.0,623.0){\usebox{\plotpoint}}
\put(377.0,623.0){\usebox{\plotpoint}}
\put(378.0,622.0){\usebox{\plotpoint}}
\put(378.0,622.0){\usebox{\plotpoint}}
\put(379.0,621.0){\usebox{\plotpoint}}
\put(379.0,621.0){\usebox{\plotpoint}}
\put(380.0,620.0){\usebox{\plotpoint}}
\put(380.0,620.0){\usebox{\plotpoint}}
\put(381.0,619.0){\usebox{\plotpoint}}
\put(381.0,619.0){\usebox{\plotpoint}}
\put(382.0,618.0){\usebox{\plotpoint}}
\put(382.0,618.0){\usebox{\plotpoint}}
\put(383.0,617.0){\usebox{\plotpoint}}
\put(383.0,617.0){\usebox{\plotpoint}}
\put(384.0,616.0){\usebox{\plotpoint}}
\put(385,614.67){\rule{0.241pt}{0.400pt}}
\multiput(385.00,615.17)(0.500,-1.000){2}{\rule{0.120pt}{0.400pt}}
\put(384.0,616.0){\usebox{\plotpoint}}
\put(386,615){\usebox{\plotpoint}}
\put(386,615){\usebox{\plotpoint}}
\put(386,615){\usebox{\plotpoint}}
\put(386,615){\usebox{\plotpoint}}
\put(386,615){\usebox{\plotpoint}}
\put(386,615){\usebox{\plotpoint}}
\put(386,615){\usebox{\plotpoint}}
\put(386,615){\usebox{\plotpoint}}
\put(386,615){\usebox{\plotpoint}}
\put(386,615){\usebox{\plotpoint}}
\put(386,615){\usebox{\plotpoint}}
\put(386,615){\usebox{\plotpoint}}
\put(386,615){\usebox{\plotpoint}}
\put(386,615){\usebox{\plotpoint}}
\put(386,615){\usebox{\plotpoint}}
\put(386,615){\usebox{\plotpoint}}
\put(386,615){\usebox{\plotpoint}}
\put(386,615){\usebox{\plotpoint}}
\put(386,615){\usebox{\plotpoint}}
\put(386,615){\usebox{\plotpoint}}
\put(386,615){\usebox{\plotpoint}}
\put(386,615){\usebox{\plotpoint}}
\put(386,615){\usebox{\plotpoint}}
\put(386,615){\usebox{\plotpoint}}
\put(386,615){\usebox{\plotpoint}}
\put(386,615){\usebox{\plotpoint}}
\put(386,615){\usebox{\plotpoint}}
\put(386,615){\usebox{\plotpoint}}
\put(386,615){\usebox{\plotpoint}}
\put(386,615){\usebox{\plotpoint}}
\put(386,615){\usebox{\plotpoint}}
\put(386,615){\usebox{\plotpoint}}
\put(386,615){\usebox{\plotpoint}}
\put(386,615){\usebox{\plotpoint}}
\put(386,615){\usebox{\plotpoint}}
\put(386,615){\usebox{\plotpoint}}
\put(386,615){\usebox{\plotpoint}}
\put(386,615){\usebox{\plotpoint}}
\put(386,615){\usebox{\plotpoint}}
\put(386,615){\usebox{\plotpoint}}
\put(386,615){\usebox{\plotpoint}}
\put(386,615){\usebox{\plotpoint}}
\put(386,615){\usebox{\plotpoint}}
\put(386,615){\usebox{\plotpoint}}
\put(386,615){\usebox{\plotpoint}}
\put(386,615){\usebox{\plotpoint}}
\put(386,615){\usebox{\plotpoint}}
\put(386,615){\usebox{\plotpoint}}
\put(386,615){\usebox{\plotpoint}}
\put(386,615){\usebox{\plotpoint}}
\put(386,615){\usebox{\plotpoint}}
\put(386,615){\usebox{\plotpoint}}
\put(386,615){\usebox{\plotpoint}}
\put(386,615){\usebox{\plotpoint}}
\put(386,615){\usebox{\plotpoint}}
\put(386,615){\usebox{\plotpoint}}
\put(386,615){\usebox{\plotpoint}}
\put(386,615){\usebox{\plotpoint}}
\put(386,615){\usebox{\plotpoint}}
\put(386,615){\usebox{\plotpoint}}
\put(386,615){\usebox{\plotpoint}}
\put(386,615){\usebox{\plotpoint}}
\put(386,615){\usebox{\plotpoint}}
\put(386,615){\usebox{\plotpoint}}
\put(386,615){\usebox{\plotpoint}}
\put(386,615){\usebox{\plotpoint}}
\put(386,615){\usebox{\plotpoint}}
\put(386,615){\usebox{\plotpoint}}
\put(386,615){\usebox{\plotpoint}}
\put(386,615){\usebox{\plotpoint}}
\put(386,615){\usebox{\plotpoint}}
\put(386,615){\usebox{\plotpoint}}
\put(386,615){\usebox{\plotpoint}}
\put(386,615){\usebox{\plotpoint}}
\put(386,615){\usebox{\plotpoint}}
\put(386,615){\usebox{\plotpoint}}
\put(386.0,615.0){\usebox{\plotpoint}}
\put(387.0,614.0){\usebox{\plotpoint}}
\put(387.0,614.0){\usebox{\plotpoint}}
\put(388.0,613.0){\usebox{\plotpoint}}
\put(388.0,613.0){\usebox{\plotpoint}}
\put(389.0,612.0){\usebox{\plotpoint}}
\put(389.0,612.0){\usebox{\plotpoint}}
\put(390.0,611.0){\usebox{\plotpoint}}
\put(390.0,611.0){\usebox{\plotpoint}}
\put(391.0,610.0){\usebox{\plotpoint}}
\put(391.0,610.0){\usebox{\plotpoint}}
\put(392.0,609.0){\usebox{\plotpoint}}
\put(392.0,609.0){\usebox{\plotpoint}}
\put(393.0,608.0){\usebox{\plotpoint}}
\put(393.0,608.0){\usebox{\plotpoint}}
\put(394.0,607.0){\usebox{\plotpoint}}
\put(394.0,607.0){\usebox{\plotpoint}}
\put(395.0,606.0){\usebox{\plotpoint}}
\put(395.0,606.0){\usebox{\plotpoint}}
\put(396.0,605.0){\usebox{\plotpoint}}
\put(396.0,605.0){\usebox{\plotpoint}}
\put(397.0,604.0){\usebox{\plotpoint}}
\put(397.0,604.0){\usebox{\plotpoint}}
\put(398,601.67){\rule{0.241pt}{0.400pt}}
\multiput(398.00,602.17)(0.500,-1.000){2}{\rule{0.120pt}{0.400pt}}
\put(398.0,603.0){\usebox{\plotpoint}}
\put(399,602){\usebox{\plotpoint}}
\put(399,602){\usebox{\plotpoint}}
\put(399,602){\usebox{\plotpoint}}
\put(399,602){\usebox{\plotpoint}}
\put(399,602){\usebox{\plotpoint}}
\put(399,602){\usebox{\plotpoint}}
\put(399,602){\usebox{\plotpoint}}
\put(399,602){\usebox{\plotpoint}}
\put(399,602){\usebox{\plotpoint}}
\put(399,602){\usebox{\plotpoint}}
\put(399,602){\usebox{\plotpoint}}
\put(399,602){\usebox{\plotpoint}}
\put(399,602){\usebox{\plotpoint}}
\put(399,602){\usebox{\plotpoint}}
\put(399,602){\usebox{\plotpoint}}
\put(399,602){\usebox{\plotpoint}}
\put(399,602){\usebox{\plotpoint}}
\put(399,602){\usebox{\plotpoint}}
\put(399,602){\usebox{\plotpoint}}
\put(399,602){\usebox{\plotpoint}}
\put(399,602){\usebox{\plotpoint}}
\put(399,602){\usebox{\plotpoint}}
\put(399,602){\usebox{\plotpoint}}
\put(399,602){\usebox{\plotpoint}}
\put(399,602){\usebox{\plotpoint}}
\put(399,602){\usebox{\plotpoint}}
\put(399,602){\usebox{\plotpoint}}
\put(399,602){\usebox{\plotpoint}}
\put(399,602){\usebox{\plotpoint}}
\put(399,602){\usebox{\plotpoint}}
\put(399,602){\usebox{\plotpoint}}
\put(399,602){\usebox{\plotpoint}}
\put(399,602){\usebox{\plotpoint}}
\put(399,602){\usebox{\plotpoint}}
\put(399,602){\usebox{\plotpoint}}
\put(399,602){\usebox{\plotpoint}}
\put(399,602){\usebox{\plotpoint}}
\put(399,602){\usebox{\plotpoint}}
\put(399,602){\usebox{\plotpoint}}
\put(399,602){\usebox{\plotpoint}}
\put(399,602){\usebox{\plotpoint}}
\put(399,602){\usebox{\plotpoint}}
\put(399,602){\usebox{\plotpoint}}
\put(399,602){\usebox{\plotpoint}}
\put(399,602){\usebox{\plotpoint}}
\put(399,602){\usebox{\plotpoint}}
\put(399,602){\usebox{\plotpoint}}
\put(399,602){\usebox{\plotpoint}}
\put(399,602){\usebox{\plotpoint}}
\put(399,602){\usebox{\plotpoint}}
\put(399,602){\usebox{\plotpoint}}
\put(399,602){\usebox{\plotpoint}}
\put(399,602){\usebox{\plotpoint}}
\put(399,602){\usebox{\plotpoint}}
\put(399,602){\usebox{\plotpoint}}
\put(399,602){\usebox{\plotpoint}}
\put(399,602){\usebox{\plotpoint}}
\put(399,602){\usebox{\plotpoint}}
\put(399,602){\usebox{\plotpoint}}
\put(399,602){\usebox{\plotpoint}}
\put(399,602){\usebox{\plotpoint}}
\put(399,602){\usebox{\plotpoint}}
\put(399,602){\usebox{\plotpoint}}
\put(399,602){\usebox{\plotpoint}}
\put(399,602){\usebox{\plotpoint}}
\put(399,602){\usebox{\plotpoint}}
\put(399,602){\usebox{\plotpoint}}
\put(399,602){\usebox{\plotpoint}}
\put(399,602){\usebox{\plotpoint}}
\put(399,602){\usebox{\plotpoint}}
\put(399,602){\usebox{\plotpoint}}
\put(399,602){\usebox{\plotpoint}}
\put(399,602){\usebox{\plotpoint}}
\put(399,602){\usebox{\plotpoint}}
\put(399,602){\usebox{\plotpoint}}
\put(399.0,601.0){\usebox{\plotpoint}}
\put(399.0,601.0){\usebox{\plotpoint}}
\put(400.0,600.0){\usebox{\plotpoint}}
\put(400.0,600.0){\usebox{\plotpoint}}
\put(401.0,599.0){\usebox{\plotpoint}}
\put(401.0,599.0){\usebox{\plotpoint}}
\put(402.0,598.0){\usebox{\plotpoint}}
\put(402.0,598.0){\usebox{\plotpoint}}
\put(403.0,597.0){\usebox{\plotpoint}}
\put(403.0,597.0){\usebox{\plotpoint}}
\put(404.0,596.0){\usebox{\plotpoint}}
\put(404.0,596.0){\usebox{\plotpoint}}
\put(405.0,595.0){\usebox{\plotpoint}}
\put(405.0,595.0){\usebox{\plotpoint}}
\put(406.0,594.0){\usebox{\plotpoint}}
\put(406.0,594.0){\usebox{\plotpoint}}
\put(407.0,593.0){\usebox{\plotpoint}}
\put(407.0,593.0){\usebox{\plotpoint}}
\put(408.0,592.0){\usebox{\plotpoint}}
\put(408.0,592.0){\usebox{\plotpoint}}
\put(409.0,591.0){\usebox{\plotpoint}}
\put(409.0,591.0){\usebox{\plotpoint}}
\put(410.0,590.0){\usebox{\plotpoint}}
\put(410.0,590.0){\usebox{\plotpoint}}
\put(411.0,589.0){\usebox{\plotpoint}}
\put(411.0,589.0){\usebox{\plotpoint}}
\put(412.0,588.0){\usebox{\plotpoint}}
\put(412.0,588.0){\usebox{\plotpoint}}
\put(413.0,587.0){\usebox{\plotpoint}}
\put(413.0,587.0){\usebox{\plotpoint}}
\put(414.0,586.0){\usebox{\plotpoint}}
\put(414.0,586.0){\usebox{\plotpoint}}
\put(415.0,585.0){\usebox{\plotpoint}}
\put(415.0,585.0){\usebox{\plotpoint}}
\put(416.0,584.0){\usebox{\plotpoint}}
\put(416.0,584.0){\usebox{\plotpoint}}
\put(417.0,583.0){\usebox{\plotpoint}}
\put(417.0,583.0){\usebox{\plotpoint}}
\put(418.0,582.0){\usebox{\plotpoint}}
\put(418.0,582.0){\usebox{\plotpoint}}
\put(419.0,581.0){\usebox{\plotpoint}}
\put(419.0,581.0){\usebox{\plotpoint}}
\put(420.0,580.0){\usebox{\plotpoint}}
\put(420.0,580.0){\usebox{\plotpoint}}
\put(421.0,579.0){\usebox{\plotpoint}}
\put(421.0,579.0){\usebox{\plotpoint}}
\put(422.0,578.0){\usebox{\plotpoint}}
\put(422.0,578.0){\usebox{\plotpoint}}
\put(423.0,577.0){\usebox{\plotpoint}}
\put(423.0,577.0){\usebox{\plotpoint}}
\put(424.0,576.0){\usebox{\plotpoint}}
\put(424.0,576.0){\usebox{\plotpoint}}
\put(425.0,574.0){\rule[-0.200pt]{0.400pt}{0.482pt}}
\put(425.0,574.0){\usebox{\plotpoint}}
\put(426.0,573.0){\usebox{\plotpoint}}
\put(426.0,573.0){\usebox{\plotpoint}}
\put(427.0,572.0){\usebox{\plotpoint}}
\put(427.0,572.0){\usebox{\plotpoint}}
\put(428.0,571.0){\usebox{\plotpoint}}
\put(428.0,571.0){\usebox{\plotpoint}}
\put(429.0,570.0){\usebox{\plotpoint}}
\put(429.0,570.0){\usebox{\plotpoint}}
\put(430.0,569.0){\usebox{\plotpoint}}
\put(430.0,569.0){\usebox{\plotpoint}}
\put(431.0,568.0){\usebox{\plotpoint}}
\put(431.0,568.0){\usebox{\plotpoint}}
\put(432.0,567.0){\usebox{\plotpoint}}
\put(432.0,567.0){\usebox{\plotpoint}}
\put(433.0,566.0){\usebox{\plotpoint}}
\put(433.0,566.0){\usebox{\plotpoint}}
\put(434.0,565.0){\usebox{\plotpoint}}
\put(434.0,565.0){\usebox{\plotpoint}}
\put(435.0,564.0){\usebox{\plotpoint}}
\put(435.0,564.0){\usebox{\plotpoint}}
\put(436.0,563.0){\usebox{\plotpoint}}
\put(436.0,563.0){\usebox{\plotpoint}}
\put(437.0,562.0){\usebox{\plotpoint}}
\put(437.0,562.0){\usebox{\plotpoint}}
\put(438.0,560.0){\rule[-0.200pt]{0.400pt}{0.482pt}}
\put(438.0,560.0){\usebox{\plotpoint}}
\put(439.0,559.0){\usebox{\plotpoint}}
\put(439.0,559.0){\usebox{\plotpoint}}
\put(440.0,558.0){\usebox{\plotpoint}}
\put(440.0,558.0){\usebox{\plotpoint}}
\put(441.0,557.0){\usebox{\plotpoint}}
\put(441.0,557.0){\usebox{\plotpoint}}
\put(442.0,556.0){\usebox{\plotpoint}}
\put(442.0,556.0){\usebox{\plotpoint}}
\put(443.0,555.0){\usebox{\plotpoint}}
\put(443.0,555.0){\usebox{\plotpoint}}
\put(444.0,554.0){\usebox{\plotpoint}}
\put(444.0,554.0){\usebox{\plotpoint}}
\put(445.0,553.0){\usebox{\plotpoint}}
\put(445.0,553.0){\usebox{\plotpoint}}
\put(446.0,552.0){\usebox{\plotpoint}}
\put(446.0,552.0){\usebox{\plotpoint}}
\put(447.0,551.0){\usebox{\plotpoint}}
\put(447.0,551.0){\usebox{\plotpoint}}
\put(448,548.67){\rule{0.241pt}{0.400pt}}
\multiput(448.00,549.17)(0.500,-1.000){2}{\rule{0.120pt}{0.400pt}}
\put(448.0,550.0){\usebox{\plotpoint}}
\put(449,549){\usebox{\plotpoint}}
\put(449,549){\usebox{\plotpoint}}
\put(449,549){\usebox{\plotpoint}}
\put(449,549){\usebox{\plotpoint}}
\put(449,549){\usebox{\plotpoint}}
\put(449,549){\usebox{\plotpoint}}
\put(449,549){\usebox{\plotpoint}}
\put(449,549){\usebox{\plotpoint}}
\put(449,549){\usebox{\plotpoint}}
\put(449,549){\usebox{\plotpoint}}
\put(449,549){\usebox{\plotpoint}}
\put(449,549){\usebox{\plotpoint}}
\put(449,549){\usebox{\plotpoint}}
\put(449,549){\usebox{\plotpoint}}
\put(449,549){\usebox{\plotpoint}}
\put(449,549){\usebox{\plotpoint}}
\put(449,549){\usebox{\plotpoint}}
\put(449,549){\usebox{\plotpoint}}
\put(449,549){\usebox{\plotpoint}}
\put(449,549){\usebox{\plotpoint}}
\put(449,549){\usebox{\plotpoint}}
\put(449,549){\usebox{\plotpoint}}
\put(449,549){\usebox{\plotpoint}}
\put(449,549){\usebox{\plotpoint}}
\put(449,549){\usebox{\plotpoint}}
\put(449,549){\usebox{\plotpoint}}
\put(449,549){\usebox{\plotpoint}}
\put(449,549){\usebox{\plotpoint}}
\put(449,549){\usebox{\plotpoint}}
\put(449,549){\usebox{\plotpoint}}
\put(449,549){\usebox{\plotpoint}}
\put(449,549){\usebox{\plotpoint}}
\put(449,549){\usebox{\plotpoint}}
\put(449,549){\usebox{\plotpoint}}
\put(449,549){\usebox{\plotpoint}}
\put(449,549){\usebox{\plotpoint}}
\put(449,549){\usebox{\plotpoint}}
\put(449,549){\usebox{\plotpoint}}
\put(449,549){\usebox{\plotpoint}}
\put(449,549){\usebox{\plotpoint}}
\put(449,549){\usebox{\plotpoint}}
\put(449,549){\usebox{\plotpoint}}
\put(449,549){\usebox{\plotpoint}}
\put(449,549){\usebox{\plotpoint}}
\put(449,549){\usebox{\plotpoint}}
\put(449,549){\usebox{\plotpoint}}
\put(449,549){\usebox{\plotpoint}}
\put(449,549){\usebox{\plotpoint}}
\put(449,549){\usebox{\plotpoint}}
\put(449,549){\usebox{\plotpoint}}
\put(449,549){\usebox{\plotpoint}}
\put(449,549){\usebox{\plotpoint}}
\put(449,549){\usebox{\plotpoint}}
\put(449,549){\usebox{\plotpoint}}
\put(449,549){\usebox{\plotpoint}}
\put(449,549){\usebox{\plotpoint}}
\put(449,549){\usebox{\plotpoint}}
\put(449,549){\usebox{\plotpoint}}
\put(449,549){\usebox{\plotpoint}}
\put(449,549){\usebox{\plotpoint}}
\put(449,549){\usebox{\plotpoint}}
\put(449,549){\usebox{\plotpoint}}
\put(449,549){\usebox{\plotpoint}}
\put(449,549){\usebox{\plotpoint}}
\put(449,549){\usebox{\plotpoint}}
\put(449,549){\usebox{\plotpoint}}
\put(449,549){\usebox{\plotpoint}}
\put(449,549){\usebox{\plotpoint}}
\put(449.0,548.0){\usebox{\plotpoint}}
\put(449.0,548.0){\usebox{\plotpoint}}
\put(450.0,547.0){\usebox{\plotpoint}}
\put(450.0,547.0){\usebox{\plotpoint}}
\put(451.0,546.0){\usebox{\plotpoint}}
\put(451.0,546.0){\usebox{\plotpoint}}
\put(452.0,545.0){\usebox{\plotpoint}}
\put(452.0,545.0){\usebox{\plotpoint}}
\put(453.0,544.0){\usebox{\plotpoint}}
\put(453.0,544.0){\usebox{\plotpoint}}
\put(454.0,543.0){\usebox{\plotpoint}}
\put(454.0,543.0){\usebox{\plotpoint}}
\put(455.0,542.0){\usebox{\plotpoint}}
\put(455.0,542.0){\usebox{\plotpoint}}
\put(456.0,541.0){\usebox{\plotpoint}}
\put(456.0,541.0){\usebox{\plotpoint}}
\put(457.0,539.0){\rule[-0.200pt]{0.400pt}{0.482pt}}
\put(457.0,539.0){\usebox{\plotpoint}}
\put(458.0,538.0){\usebox{\plotpoint}}
\put(458.0,538.0){\usebox{\plotpoint}}
\put(459.0,537.0){\usebox{\plotpoint}}
\put(459.0,537.0){\usebox{\plotpoint}}
\put(460.0,536.0){\usebox{\plotpoint}}
\put(460.0,536.0){\usebox{\plotpoint}}
\put(461.0,535.0){\usebox{\plotpoint}}
\put(461.0,535.0){\usebox{\plotpoint}}
\put(462.0,534.0){\usebox{\plotpoint}}
\put(462.0,534.0){\usebox{\plotpoint}}
\put(463.0,533.0){\usebox{\plotpoint}}
\put(463.0,533.0){\usebox{\plotpoint}}
\put(464.0,532.0){\usebox{\plotpoint}}
\put(464.0,532.0){\usebox{\plotpoint}}
\put(465.0,530.0){\rule[-0.200pt]{0.400pt}{0.482pt}}
\put(465.0,530.0){\usebox{\plotpoint}}
\put(466.0,529.0){\usebox{\plotpoint}}
\put(466.0,529.0){\usebox{\plotpoint}}
\put(467.0,528.0){\usebox{\plotpoint}}
\put(467.0,528.0){\usebox{\plotpoint}}
\put(468.0,527.0){\usebox{\plotpoint}}
\put(468.0,527.0){\usebox{\plotpoint}}
\put(469.0,526.0){\usebox{\plotpoint}}
\put(469.0,526.0){\usebox{\plotpoint}}
\put(470.0,525.0){\usebox{\plotpoint}}
\put(470.0,525.0){\usebox{\plotpoint}}
\put(471.0,524.0){\usebox{\plotpoint}}
\put(471.0,524.0){\usebox{\plotpoint}}
\put(472.0,522.0){\rule[-0.200pt]{0.400pt}{0.482pt}}
\put(472.0,522.0){\usebox{\plotpoint}}
\put(473.0,521.0){\usebox{\plotpoint}}
\put(473.0,521.0){\usebox{\plotpoint}}
\put(474.0,520.0){\usebox{\plotpoint}}
\put(474.0,520.0){\usebox{\plotpoint}}
\put(475.0,519.0){\usebox{\plotpoint}}
\put(475.0,519.0){\usebox{\plotpoint}}
\put(476.0,518.0){\usebox{\plotpoint}}
\put(476.0,518.0){\usebox{\plotpoint}}
\put(477.0,517.0){\usebox{\plotpoint}}
\put(477.0,517.0){\usebox{\plotpoint}}
\put(478,514.67){\rule{0.241pt}{0.400pt}}
\multiput(478.00,515.17)(0.500,-1.000){2}{\rule{0.120pt}{0.400pt}}
\put(478.0,516.0){\usebox{\plotpoint}}
\put(479,515){\usebox{\plotpoint}}
\put(479,515){\usebox{\plotpoint}}
\put(479,515){\usebox{\plotpoint}}
\put(479,515){\usebox{\plotpoint}}
\put(479,515){\usebox{\plotpoint}}
\put(479,515){\usebox{\plotpoint}}
\put(479,515){\usebox{\plotpoint}}
\put(479,515){\usebox{\plotpoint}}
\put(479,515){\usebox{\plotpoint}}
\put(479,515){\usebox{\plotpoint}}
\put(479,515){\usebox{\plotpoint}}
\put(479,515){\usebox{\plotpoint}}
\put(479,515){\usebox{\plotpoint}}
\put(479,515){\usebox{\plotpoint}}
\put(479,515){\usebox{\plotpoint}}
\put(479,515){\usebox{\plotpoint}}
\put(479,515){\usebox{\plotpoint}}
\put(479,515){\usebox{\plotpoint}}
\put(479,515){\usebox{\plotpoint}}
\put(479,515){\usebox{\plotpoint}}
\put(479,515){\usebox{\plotpoint}}
\put(479,515){\usebox{\plotpoint}}
\put(479,515){\usebox{\plotpoint}}
\put(479,515){\usebox{\plotpoint}}
\put(479,515){\usebox{\plotpoint}}
\put(479,515){\usebox{\plotpoint}}
\put(479,515){\usebox{\plotpoint}}
\put(479,515){\usebox{\plotpoint}}
\put(479,515){\usebox{\plotpoint}}
\put(479,515){\usebox{\plotpoint}}
\put(479,515){\usebox{\plotpoint}}
\put(479,515){\usebox{\plotpoint}}
\put(479,515){\usebox{\plotpoint}}
\put(479,515){\usebox{\plotpoint}}
\put(479,515){\usebox{\plotpoint}}
\put(479,515){\usebox{\plotpoint}}
\put(479,515){\usebox{\plotpoint}}
\put(479,515){\usebox{\plotpoint}}
\put(479,515){\usebox{\plotpoint}}
\put(479,515){\usebox{\plotpoint}}
\put(479,515){\usebox{\plotpoint}}
\put(479,515){\usebox{\plotpoint}}
\put(479,515){\usebox{\plotpoint}}
\put(479,515){\usebox{\plotpoint}}
\put(479,515){\usebox{\plotpoint}}
\put(479,515){\usebox{\plotpoint}}
\put(479,515){\usebox{\plotpoint}}
\put(479,515){\usebox{\plotpoint}}
\put(479,515){\usebox{\plotpoint}}
\put(479,515){\usebox{\plotpoint}}
\put(479,515){\usebox{\plotpoint}}
\put(479,515){\usebox{\plotpoint}}
\put(479,515){\usebox{\plotpoint}}
\put(479,515){\usebox{\plotpoint}}
\put(479,515){\usebox{\plotpoint}}
\put(479,515){\usebox{\plotpoint}}
\put(479,515){\usebox{\plotpoint}}
\put(479,515){\usebox{\plotpoint}}
\put(479,515){\usebox{\plotpoint}}
\put(479,515){\usebox{\plotpoint}}
\put(479,515){\usebox{\plotpoint}}
\put(479,515){\usebox{\plotpoint}}
\put(479,515){\usebox{\plotpoint}}
\put(479,515){\usebox{\plotpoint}}
\put(479,515){\usebox{\plotpoint}}
\put(479,515){\usebox{\plotpoint}}
\put(479.0,514.0){\usebox{\plotpoint}}
\put(479.0,514.0){\usebox{\plotpoint}}
\put(480.0,513.0){\usebox{\plotpoint}}
\put(480.0,513.0){\usebox{\plotpoint}}
\put(481.0,512.0){\usebox{\plotpoint}}
\put(481.0,512.0){\usebox{\plotpoint}}
\put(482.0,511.0){\usebox{\plotpoint}}
\put(482.0,511.0){\usebox{\plotpoint}}
\put(483.0,510.0){\usebox{\plotpoint}}
\put(483.0,510.0){\usebox{\plotpoint}}
\put(484.0,509.0){\usebox{\plotpoint}}
\put(484.0,509.0){\usebox{\plotpoint}}
\put(485.0,507.0){\rule[-0.200pt]{0.400pt}{0.482pt}}
\put(485.0,507.0){\usebox{\plotpoint}}
\put(486.0,506.0){\usebox{\plotpoint}}
\put(486.0,506.0){\usebox{\plotpoint}}
\put(487.0,505.0){\usebox{\plotpoint}}
\put(487.0,505.0){\usebox{\plotpoint}}
\put(488.0,504.0){\usebox{\plotpoint}}
\put(488.0,504.0){\usebox{\plotpoint}}
\put(489.0,503.0){\usebox{\plotpoint}}
\put(489.0,503.0){\usebox{\plotpoint}}
\put(490.0,501.0){\rule[-0.200pt]{0.400pt}{0.482pt}}
\put(490.0,501.0){\usebox{\plotpoint}}
\put(491.0,500.0){\usebox{\plotpoint}}
\put(491.0,500.0){\usebox{\plotpoint}}
\put(492.0,499.0){\usebox{\plotpoint}}
\put(492.0,499.0){\usebox{\plotpoint}}
\put(493.0,498.0){\usebox{\plotpoint}}
\put(493.0,498.0){\usebox{\plotpoint}}
\put(494.0,497.0){\usebox{\plotpoint}}
\put(494.0,497.0){\usebox{\plotpoint}}
\put(495.0,496.0){\usebox{\plotpoint}}
\put(495.0,496.0){\usebox{\plotpoint}}
\put(496.0,494.0){\rule[-0.200pt]{0.400pt}{0.482pt}}
\put(496.0,494.0){\usebox{\plotpoint}}
\put(497.0,493.0){\usebox{\plotpoint}}
\put(497.0,493.0){\usebox{\plotpoint}}
\put(498.0,492.0){\usebox{\plotpoint}}
\put(498.0,492.0){\usebox{\plotpoint}}
\put(499.0,491.0){\usebox{\plotpoint}}
\put(499.0,491.0){\usebox{\plotpoint}}
\put(500.0,490.0){\usebox{\plotpoint}}
\put(500.0,490.0){\usebox{\plotpoint}}
\put(501.0,488.0){\rule[-0.200pt]{0.400pt}{0.482pt}}
\put(501.0,488.0){\usebox{\plotpoint}}
\put(502.0,487.0){\usebox{\plotpoint}}
\put(502.0,487.0){\usebox{\plotpoint}}
\put(503.0,486.0){\usebox{\plotpoint}}
\put(503.0,486.0){\usebox{\plotpoint}}
\put(504.0,485.0){\usebox{\plotpoint}}
\put(504.0,485.0){\usebox{\plotpoint}}
\put(505.0,484.0){\usebox{\plotpoint}}
\put(505.0,484.0){\usebox{\plotpoint}}
\put(506.0,482.0){\rule[-0.200pt]{0.400pt}{0.482pt}}
\put(506.0,482.0){\usebox{\plotpoint}}
\put(507.0,481.0){\usebox{\plotpoint}}
\put(507.0,481.0){\usebox{\plotpoint}}
\put(508.0,480.0){\usebox{\plotpoint}}
\put(508.0,480.0){\usebox{\plotpoint}}
\put(509.0,479.0){\usebox{\plotpoint}}
\put(509.0,479.0){\usebox{\plotpoint}}
\put(510.0,478.0){\usebox{\plotpoint}}
\put(510.0,478.0){\usebox{\plotpoint}}
\put(511.0,476.0){\rule[-0.200pt]{0.400pt}{0.482pt}}
\put(511.0,476.0){\usebox{\plotpoint}}
\put(512.0,475.0){\usebox{\plotpoint}}
\put(512.0,475.0){\usebox{\plotpoint}}
\put(513.0,474.0){\usebox{\plotpoint}}
\put(513.0,474.0){\usebox{\plotpoint}}
\put(514.0,473.0){\usebox{\plotpoint}}
\put(514.0,473.0){\usebox{\plotpoint}}
\put(515.0,471.0){\rule[-0.200pt]{0.400pt}{0.482pt}}
\put(515.0,471.0){\usebox{\plotpoint}}
\put(516.0,470.0){\usebox{\plotpoint}}
\put(516.0,470.0){\usebox{\plotpoint}}
\put(517.0,469.0){\usebox{\plotpoint}}
\put(517.0,469.0){\usebox{\plotpoint}}
\put(518.0,468.0){\usebox{\plotpoint}}
\put(518.0,468.0){\usebox{\plotpoint}}
\put(519.0,467.0){\usebox{\plotpoint}}
\put(519.0,467.0){\usebox{\plotpoint}}
\put(520.0,465.0){\rule[-0.200pt]{0.400pt}{0.482pt}}
\put(520.0,465.0){\usebox{\plotpoint}}
\put(521.0,464.0){\usebox{\plotpoint}}
\put(521.0,464.0){\usebox{\plotpoint}}
\put(522.0,463.0){\usebox{\plotpoint}}
\put(522.0,463.0){\usebox{\plotpoint}}
\put(523.0,462.0){\usebox{\plotpoint}}
\put(523.0,462.0){\usebox{\plotpoint}}
\put(524.0,460.0){\rule[-0.200pt]{0.400pt}{0.482pt}}
\put(524.0,460.0){\usebox{\plotpoint}}
\put(525.0,459.0){\usebox{\plotpoint}}
\put(525.0,459.0){\usebox{\plotpoint}}
\put(526.0,458.0){\usebox{\plotpoint}}
\put(526.0,458.0){\usebox{\plotpoint}}
\put(527.0,457.0){\usebox{\plotpoint}}
\put(527.0,457.0){\usebox{\plotpoint}}
\put(528.0,455.0){\rule[-0.200pt]{0.400pt}{0.482pt}}
\put(528.0,455.0){\usebox{\plotpoint}}
\put(529.0,454.0){\usebox{\plotpoint}}
\put(529.0,454.0){\usebox{\plotpoint}}
\put(530.0,453.0){\usebox{\plotpoint}}
\put(530.0,453.0){\usebox{\plotpoint}}
\put(531.0,452.0){\usebox{\plotpoint}}
\put(531.0,452.0){\usebox{\plotpoint}}
\put(532,449.67){\rule{0.241pt}{0.400pt}}
\multiput(532.00,450.17)(0.500,-1.000){2}{\rule{0.120pt}{0.400pt}}
\put(532.0,451.0){\usebox{\plotpoint}}
\put(533,450){\usebox{\plotpoint}}
\put(533,450){\usebox{\plotpoint}}
\put(533,450){\usebox{\plotpoint}}
\put(533,450){\usebox{\plotpoint}}
\put(533,450){\usebox{\plotpoint}}
\put(533,450){\usebox{\plotpoint}}
\put(533,450){\usebox{\plotpoint}}
\put(533,450){\usebox{\plotpoint}}
\put(533,450){\usebox{\plotpoint}}
\put(533,450){\usebox{\plotpoint}}
\put(533,450){\usebox{\plotpoint}}
\put(533,450){\usebox{\plotpoint}}
\put(533,450){\usebox{\plotpoint}}
\put(533,450){\usebox{\plotpoint}}
\put(533,450){\usebox{\plotpoint}}
\put(533,450){\usebox{\plotpoint}}
\put(533,450){\usebox{\plotpoint}}
\put(533,450){\usebox{\plotpoint}}
\put(533,450){\usebox{\plotpoint}}
\put(533,450){\usebox{\plotpoint}}
\put(533,450){\usebox{\plotpoint}}
\put(533,450){\usebox{\plotpoint}}
\put(533,450){\usebox{\plotpoint}}
\put(533,450){\usebox{\plotpoint}}
\put(533,450){\usebox{\plotpoint}}
\put(533,450){\usebox{\plotpoint}}
\put(533,450){\usebox{\plotpoint}}
\put(533,450){\usebox{\plotpoint}}
\put(533,450){\usebox{\plotpoint}}
\put(533,450){\usebox{\plotpoint}}
\put(533,450){\usebox{\plotpoint}}
\put(533,450){\usebox{\plotpoint}}
\put(533,450){\usebox{\plotpoint}}
\put(533,450){\usebox{\plotpoint}}
\put(533,450){\usebox{\plotpoint}}
\put(533,450){\usebox{\plotpoint}}
\put(533,450){\usebox{\plotpoint}}
\put(533,450){\usebox{\plotpoint}}
\put(533,450){\usebox{\plotpoint}}
\put(533,450){\usebox{\plotpoint}}
\put(533,450){\usebox{\plotpoint}}
\put(533,450){\usebox{\plotpoint}}
\put(533,450){\usebox{\plotpoint}}
\put(533,450){\usebox{\plotpoint}}
\put(533,450){\usebox{\plotpoint}}
\put(533,450){\usebox{\plotpoint}}
\put(533,450){\usebox{\plotpoint}}
\put(533,450){\usebox{\plotpoint}}
\put(533,450){\usebox{\plotpoint}}
\put(533,450){\usebox{\plotpoint}}
\put(533,450){\usebox{\plotpoint}}
\put(533,450){\usebox{\plotpoint}}
\put(533,450){\usebox{\plotpoint}}
\put(533,450){\usebox{\plotpoint}}
\put(533,450){\usebox{\plotpoint}}
\put(533,450){\usebox{\plotpoint}}
\put(533,450){\usebox{\plotpoint}}
\put(533,450){\usebox{\plotpoint}}
\put(533,450){\usebox{\plotpoint}}
\put(533,450){\usebox{\plotpoint}}
\put(533,450){\usebox{\plotpoint}}
\put(533.0,449.0){\usebox{\plotpoint}}
\put(533.0,449.0){\usebox{\plotpoint}}
\put(534.0,448.0){\usebox{\plotpoint}}
\put(534.0,448.0){\usebox{\plotpoint}}
\put(535.0,447.0){\usebox{\plotpoint}}
\put(535.0,447.0){\usebox{\plotpoint}}
\put(536,444.67){\rule{0.241pt}{0.400pt}}
\multiput(536.00,445.17)(0.500,-1.000){2}{\rule{0.120pt}{0.400pt}}
\put(536.0,446.0){\usebox{\plotpoint}}
\put(537,445){\usebox{\plotpoint}}
\put(537,445){\usebox{\plotpoint}}
\put(537,445){\usebox{\plotpoint}}
\put(537,445){\usebox{\plotpoint}}
\put(537,445){\usebox{\plotpoint}}
\put(537,445){\usebox{\plotpoint}}
\put(537,445){\usebox{\plotpoint}}
\put(537,445){\usebox{\plotpoint}}
\put(537,445){\usebox{\plotpoint}}
\put(537,445){\usebox{\plotpoint}}
\put(537,445){\usebox{\plotpoint}}
\put(537,445){\usebox{\plotpoint}}
\put(537,445){\usebox{\plotpoint}}
\put(537,445){\usebox{\plotpoint}}
\put(537,445){\usebox{\plotpoint}}
\put(537,445){\usebox{\plotpoint}}
\put(537,445){\usebox{\plotpoint}}
\put(537,445){\usebox{\plotpoint}}
\put(537,445){\usebox{\plotpoint}}
\put(537,445){\usebox{\plotpoint}}
\put(537,445){\usebox{\plotpoint}}
\put(537,445){\usebox{\plotpoint}}
\put(537,445){\usebox{\plotpoint}}
\put(537,445){\usebox{\plotpoint}}
\put(537,445){\usebox{\plotpoint}}
\put(537,445){\usebox{\plotpoint}}
\put(537,445){\usebox{\plotpoint}}
\put(537,445){\usebox{\plotpoint}}
\put(537,445){\usebox{\plotpoint}}
\put(537,445){\usebox{\plotpoint}}
\put(537,445){\usebox{\plotpoint}}
\put(537,445){\usebox{\plotpoint}}
\put(537,445){\usebox{\plotpoint}}
\put(537,445){\usebox{\plotpoint}}
\put(537,445){\usebox{\plotpoint}}
\put(537,445){\usebox{\plotpoint}}
\put(537,445){\usebox{\plotpoint}}
\put(537,445){\usebox{\plotpoint}}
\put(537,445){\usebox{\plotpoint}}
\put(537,445){\usebox{\plotpoint}}
\put(537,445){\usebox{\plotpoint}}
\put(537,445){\usebox{\plotpoint}}
\put(537,445){\usebox{\plotpoint}}
\put(537,445){\usebox{\plotpoint}}
\put(537,445){\usebox{\plotpoint}}
\put(537,445){\usebox{\plotpoint}}
\put(537,445){\usebox{\plotpoint}}
\put(537,445){\usebox{\plotpoint}}
\put(537,445){\usebox{\plotpoint}}
\put(537,445){\usebox{\plotpoint}}
\put(537,445){\usebox{\plotpoint}}
\put(537,445){\usebox{\plotpoint}}
\put(537,445){\usebox{\plotpoint}}
\put(537,445){\usebox{\plotpoint}}
\put(537,445){\usebox{\plotpoint}}
\put(537,445){\usebox{\plotpoint}}
\put(537,445){\usebox{\plotpoint}}
\put(537,445){\usebox{\plotpoint}}
\put(537,445){\usebox{\plotpoint}}
\put(537,445){\usebox{\plotpoint}}
\put(537.0,444.0){\usebox{\plotpoint}}
\put(537.0,444.0){\usebox{\plotpoint}}
\put(538.0,443.0){\usebox{\plotpoint}}
\put(538.0,443.0){\usebox{\plotpoint}}
\put(539.0,442.0){\usebox{\plotpoint}}
\put(539.0,442.0){\usebox{\plotpoint}}
\put(540.0,440.0){\rule[-0.200pt]{0.400pt}{0.482pt}}
\put(540.0,440.0){\usebox{\plotpoint}}
\put(541.0,439.0){\usebox{\plotpoint}}
\put(541.0,439.0){\usebox{\plotpoint}}
\put(542.0,438.0){\usebox{\plotpoint}}
\put(542.0,438.0){\usebox{\plotpoint}}
\put(543.0,437.0){\usebox{\plotpoint}}
\put(543.0,437.0){\usebox{\plotpoint}}
\put(544.0,435.0){\rule[-0.200pt]{0.400pt}{0.482pt}}
\put(544.0,435.0){\usebox{\plotpoint}}
\put(545.0,434.0){\usebox{\plotpoint}}
\put(545.0,434.0){\usebox{\plotpoint}}
\put(546.0,433.0){\usebox{\plotpoint}}
\put(546.0,433.0){\usebox{\plotpoint}}
\put(547.0,432.0){\usebox{\plotpoint}}
\put(547.0,432.0){\usebox{\plotpoint}}
\put(548.0,430.0){\rule[-0.200pt]{0.400pt}{0.482pt}}
\put(548.0,430.0){\usebox{\plotpoint}}
\put(549.0,429.0){\usebox{\plotpoint}}
\put(549.0,429.0){\usebox{\plotpoint}}
\put(550.0,428.0){\usebox{\plotpoint}}
\put(550.0,428.0){\usebox{\plotpoint}}
\put(551.0,427.0){\usebox{\plotpoint}}
\put(551.0,427.0){\usebox{\plotpoint}}
\put(552.0,425.0){\rule[-0.200pt]{0.400pt}{0.482pt}}
\put(552.0,425.0){\usebox{\plotpoint}}
\put(553.0,424.0){\usebox{\plotpoint}}
\put(553.0,424.0){\usebox{\plotpoint}}
\put(554.0,423.0){\usebox{\plotpoint}}
\put(554.0,423.0){\usebox{\plotpoint}}
\put(555.0,421.0){\rule[-0.200pt]{0.400pt}{0.482pt}}
\put(555.0,421.0){\usebox{\plotpoint}}
\put(556.0,420.0){\usebox{\plotpoint}}
\put(556.0,420.0){\usebox{\plotpoint}}
\put(557.0,419.0){\usebox{\plotpoint}}
\put(557.0,419.0){\usebox{\plotpoint}}
\put(558.0,418.0){\usebox{\plotpoint}}
\put(558.0,418.0){\usebox{\plotpoint}}
\put(559.0,416.0){\rule[-0.200pt]{0.400pt}{0.482pt}}
\put(559.0,416.0){\usebox{\plotpoint}}
\put(560.0,415.0){\usebox{\plotpoint}}
\put(560.0,415.0){\usebox{\plotpoint}}
\put(561.0,414.0){\usebox{\plotpoint}}
\put(561.0,414.0){\usebox{\plotpoint}}
\put(562.0,412.0){\rule[-0.200pt]{0.400pt}{0.482pt}}
\put(562.0,412.0){\usebox{\plotpoint}}
\put(563.0,411.0){\usebox{\plotpoint}}
\put(563.0,411.0){\usebox{\plotpoint}}
\put(564.0,410.0){\usebox{\plotpoint}}
\put(564.0,410.0){\usebox{\plotpoint}}
\put(565.0,408.0){\rule[-0.200pt]{0.400pt}{0.482pt}}
\put(565.0,408.0){\usebox{\plotpoint}}
\put(566.0,407.0){\usebox{\plotpoint}}
\put(566.0,407.0){\usebox{\plotpoint}}
\put(567.0,406.0){\usebox{\plotpoint}}
\put(567.0,406.0){\usebox{\plotpoint}}
\put(568.0,405.0){\usebox{\plotpoint}}
\put(568.0,405.0){\usebox{\plotpoint}}
\put(569.0,403.0){\rule[-0.200pt]{0.400pt}{0.482pt}}
\put(569.0,403.0){\usebox{\plotpoint}}
\put(570.0,402.0){\usebox{\plotpoint}}
\put(570.0,402.0){\usebox{\plotpoint}}
\put(571.0,401.0){\usebox{\plotpoint}}
\put(571.0,401.0){\usebox{\plotpoint}}
\put(572.0,399.0){\rule[-0.200pt]{0.400pt}{0.482pt}}
\put(572.0,399.0){\usebox{\plotpoint}}
\put(573.0,398.0){\usebox{\plotpoint}}
\put(573.0,398.0){\usebox{\plotpoint}}
\put(574.0,397.0){\usebox{\plotpoint}}
\put(574.0,397.0){\usebox{\plotpoint}}
\put(575.0,395.0){\rule[-0.200pt]{0.400pt}{0.482pt}}
\put(575.0,395.0){\usebox{\plotpoint}}
\put(576.0,394.0){\usebox{\plotpoint}}
\put(576.0,394.0){\usebox{\plotpoint}}
\put(577.0,393.0){\usebox{\plotpoint}}
\put(577.0,393.0){\usebox{\plotpoint}}
\put(578.0,391.0){\rule[-0.200pt]{0.400pt}{0.482pt}}
\put(578.0,391.0){\usebox{\plotpoint}}
\put(579.0,390.0){\usebox{\plotpoint}}
\put(579.0,390.0){\usebox{\plotpoint}}
\put(580.0,389.0){\usebox{\plotpoint}}
\put(580.0,389.0){\usebox{\plotpoint}}
\put(581,386.67){\rule{0.241pt}{0.400pt}}
\multiput(581.00,387.17)(0.500,-1.000){2}{\rule{0.120pt}{0.400pt}}
\put(581.0,388.0){\usebox{\plotpoint}}
\put(582,387){\usebox{\plotpoint}}
\put(582,387){\usebox{\plotpoint}}
\put(582,387){\usebox{\plotpoint}}
\put(582,387){\usebox{\plotpoint}}
\put(582,387){\usebox{\plotpoint}}
\put(582,387){\usebox{\plotpoint}}
\put(582,387){\usebox{\plotpoint}}
\put(582,387){\usebox{\plotpoint}}
\put(582,387){\usebox{\plotpoint}}
\put(582,387){\usebox{\plotpoint}}
\put(582,387){\usebox{\plotpoint}}
\put(582,387){\usebox{\plotpoint}}
\put(582,387){\usebox{\plotpoint}}
\put(582,387){\usebox{\plotpoint}}
\put(582,387){\usebox{\plotpoint}}
\put(582,387){\usebox{\plotpoint}}
\put(582,387){\usebox{\plotpoint}}
\put(582,387){\usebox{\plotpoint}}
\put(582,387){\usebox{\plotpoint}}
\put(582,387){\usebox{\plotpoint}}
\put(582,387){\usebox{\plotpoint}}
\put(582,387){\usebox{\plotpoint}}
\put(582,387){\usebox{\plotpoint}}
\put(582,387){\usebox{\plotpoint}}
\put(582,387){\usebox{\plotpoint}}
\put(582,387){\usebox{\plotpoint}}
\put(582,387){\usebox{\plotpoint}}
\put(582,387){\usebox{\plotpoint}}
\put(582,387){\usebox{\plotpoint}}
\put(582,387){\usebox{\plotpoint}}
\put(582,387){\usebox{\plotpoint}}
\put(582,387){\usebox{\plotpoint}}
\put(582,387){\usebox{\plotpoint}}
\put(582,387){\usebox{\plotpoint}}
\put(582,387){\usebox{\plotpoint}}
\put(582,387){\usebox{\plotpoint}}
\put(582,387){\usebox{\plotpoint}}
\put(582,387){\usebox{\plotpoint}}
\put(582,387){\usebox{\plotpoint}}
\put(582,387){\usebox{\plotpoint}}
\put(582,387){\usebox{\plotpoint}}
\put(582,387){\usebox{\plotpoint}}
\put(582,387){\usebox{\plotpoint}}
\put(582,387){\usebox{\plotpoint}}
\put(582,387){\usebox{\plotpoint}}
\put(582,387){\usebox{\plotpoint}}
\put(582,387){\usebox{\plotpoint}}
\put(582,387){\usebox{\plotpoint}}
\put(582,387){\usebox{\plotpoint}}
\put(582,387){\usebox{\plotpoint}}
\put(582,387){\usebox{\plotpoint}}
\put(582,387){\usebox{\plotpoint}}
\put(582,387){\usebox{\plotpoint}}
\put(582,387){\usebox{\plotpoint}}
\put(582,387){\usebox{\plotpoint}}
\put(582,387){\usebox{\plotpoint}}
\put(582,387){\usebox{\plotpoint}}
\put(582.0,386.0){\usebox{\plotpoint}}
\put(582.0,386.0){\usebox{\plotpoint}}
\put(583.0,385.0){\usebox{\plotpoint}}
\put(583.0,385.0){\usebox{\plotpoint}}
\put(584.0,384.0){\usebox{\plotpoint}}
\put(584.0,384.0){\usebox{\plotpoint}}
\put(585.0,382.0){\rule[-0.200pt]{0.400pt}{0.482pt}}
\put(585.0,382.0){\usebox{\plotpoint}}
\put(586.0,381.0){\usebox{\plotpoint}}
\put(586.0,381.0){\usebox{\plotpoint}}
\put(587.0,380.0){\usebox{\plotpoint}}
\put(587.0,380.0){\usebox{\plotpoint}}
\put(588.0,378.0){\rule[-0.200pt]{0.400pt}{0.482pt}}
\put(588.0,378.0){\usebox{\plotpoint}}
\put(589.0,377.0){\usebox{\plotpoint}}
\put(589.0,377.0){\usebox{\plotpoint}}
\put(590.0,376.0){\usebox{\plotpoint}}
\put(590.0,376.0){\usebox{\plotpoint}}
\put(591.0,374.0){\rule[-0.200pt]{0.400pt}{0.482pt}}
\put(591.0,374.0){\usebox{\plotpoint}}
\put(592.0,373.0){\usebox{\plotpoint}}
\put(592.0,373.0){\usebox{\plotpoint}}
\put(593,370.67){\rule{0.241pt}{0.400pt}}
\multiput(593.00,371.17)(0.500,-1.000){2}{\rule{0.120pt}{0.400pt}}
\put(593.0,372.0){\usebox{\plotpoint}}
\put(594,371){\usebox{\plotpoint}}
\put(594,371){\usebox{\plotpoint}}
\put(594,371){\usebox{\plotpoint}}
\put(594,371){\usebox{\plotpoint}}
\put(594,371){\usebox{\plotpoint}}
\put(594,371){\usebox{\plotpoint}}
\put(594,371){\usebox{\plotpoint}}
\put(594,371){\usebox{\plotpoint}}
\put(594,371){\usebox{\plotpoint}}
\put(594,371){\usebox{\plotpoint}}
\put(594,371){\usebox{\plotpoint}}
\put(594,371){\usebox{\plotpoint}}
\put(594,371){\usebox{\plotpoint}}
\put(594,371){\usebox{\plotpoint}}
\put(594,371){\usebox{\plotpoint}}
\put(594,371){\usebox{\plotpoint}}
\put(594,371){\usebox{\plotpoint}}
\put(594,371){\usebox{\plotpoint}}
\put(594,371){\usebox{\plotpoint}}
\put(594,371){\usebox{\plotpoint}}
\put(594,371){\usebox{\plotpoint}}
\put(594,371){\usebox{\plotpoint}}
\put(594,371){\usebox{\plotpoint}}
\put(594,371){\usebox{\plotpoint}}
\put(594,371){\usebox{\plotpoint}}
\put(594,371){\usebox{\plotpoint}}
\put(594,371){\usebox{\plotpoint}}
\put(594,371){\usebox{\plotpoint}}
\put(594,371){\usebox{\plotpoint}}
\put(594,371){\usebox{\plotpoint}}
\put(594,371){\usebox{\plotpoint}}
\put(594,371){\usebox{\plotpoint}}
\put(594,371){\usebox{\plotpoint}}
\put(594,371){\usebox{\plotpoint}}
\put(594,371){\usebox{\plotpoint}}
\put(594,371){\usebox{\plotpoint}}
\put(594,371){\usebox{\plotpoint}}
\put(594,371){\usebox{\plotpoint}}
\put(594,371){\usebox{\plotpoint}}
\put(594,371){\usebox{\plotpoint}}
\put(594,371){\usebox{\plotpoint}}
\put(594,371){\usebox{\plotpoint}}
\put(594,371){\usebox{\plotpoint}}
\put(594,371){\usebox{\plotpoint}}
\put(594,371){\usebox{\plotpoint}}
\put(594,371){\usebox{\plotpoint}}
\put(594,371){\usebox{\plotpoint}}
\put(594,371){\usebox{\plotpoint}}
\put(594,371){\usebox{\plotpoint}}
\put(594,371){\usebox{\plotpoint}}
\put(594,371){\usebox{\plotpoint}}
\put(594,371){\usebox{\plotpoint}}
\put(594,371){\usebox{\plotpoint}}
\put(594,371){\usebox{\plotpoint}}
\put(594,371){\usebox{\plotpoint}}
\put(594,371){\usebox{\plotpoint}}
\put(594.0,370.0){\usebox{\plotpoint}}
\put(594.0,370.0){\usebox{\plotpoint}}
\put(595.0,369.0){\usebox{\plotpoint}}
\put(595.0,369.0){\usebox{\plotpoint}}
\put(596.0,367.0){\rule[-0.200pt]{0.400pt}{0.482pt}}
\put(596.0,367.0){\usebox{\plotpoint}}
\put(597.0,366.0){\usebox{\plotpoint}}
\put(597.0,366.0){\usebox{\plotpoint}}
\put(598.0,365.0){\usebox{\plotpoint}}
\put(598.0,365.0){\usebox{\plotpoint}}
\put(599.0,363.0){\rule[-0.200pt]{0.400pt}{0.482pt}}
\put(599.0,363.0){\usebox{\plotpoint}}
\put(600.0,362.0){\usebox{\plotpoint}}
\put(600.0,362.0){\usebox{\plotpoint}}
\put(601.0,361.0){\usebox{\plotpoint}}
\put(601.0,361.0){\usebox{\plotpoint}}
\put(602.0,359.0){\rule[-0.200pt]{0.400pt}{0.482pt}}
\put(602.0,359.0){\usebox{\plotpoint}}
\put(603.0,358.0){\usebox{\plotpoint}}
\put(603.0,358.0){\usebox{\plotpoint}}
\put(604.0,357.0){\usebox{\plotpoint}}
\put(604.0,357.0){\usebox{\plotpoint}}
\put(605.0,355.0){\rule[-0.200pt]{0.400pt}{0.482pt}}
\put(605.0,355.0){\usebox{\plotpoint}}
\put(606.0,354.0){\usebox{\plotpoint}}
\put(606.0,354.0){\usebox{\plotpoint}}
\put(607.0,353.0){\usebox{\plotpoint}}
\put(607.0,353.0){\usebox{\plotpoint}}
\put(608.0,351.0){\rule[-0.200pt]{0.400pt}{0.482pt}}
\put(608.0,351.0){\usebox{\plotpoint}}
\put(609.0,350.0){\usebox{\plotpoint}}
\put(609.0,350.0){\usebox{\plotpoint}}
\put(610,347.67){\rule{0.241pt}{0.400pt}}
\multiput(610.00,348.17)(0.500,-1.000){2}{\rule{0.120pt}{0.400pt}}
\put(610.0,349.0){\usebox{\plotpoint}}
\put(611,348){\usebox{\plotpoint}}
\put(611,348){\usebox{\plotpoint}}
\put(611,348){\usebox{\plotpoint}}
\put(611,348){\usebox{\plotpoint}}
\put(611,348){\usebox{\plotpoint}}
\put(611,348){\usebox{\plotpoint}}
\put(611,348){\usebox{\plotpoint}}
\put(611,348){\usebox{\plotpoint}}
\put(611,348){\usebox{\plotpoint}}
\put(611,348){\usebox{\plotpoint}}
\put(611,348){\usebox{\plotpoint}}
\put(611,348){\usebox{\plotpoint}}
\put(611,348){\usebox{\plotpoint}}
\put(611,348){\usebox{\plotpoint}}
\put(611,348){\usebox{\plotpoint}}
\put(611,348){\usebox{\plotpoint}}
\put(611,348){\usebox{\plotpoint}}
\put(611,348){\usebox{\plotpoint}}
\put(611,348){\usebox{\plotpoint}}
\put(611,348){\usebox{\plotpoint}}
\put(611,348){\usebox{\plotpoint}}
\put(611,348){\usebox{\plotpoint}}
\put(611,348){\usebox{\plotpoint}}
\put(611,348){\usebox{\plotpoint}}
\put(611,348){\usebox{\plotpoint}}
\put(611,348){\usebox{\plotpoint}}
\put(611,348){\usebox{\plotpoint}}
\put(611,348){\usebox{\plotpoint}}
\put(611,348){\usebox{\plotpoint}}
\put(611,348){\usebox{\plotpoint}}
\put(611,348){\usebox{\plotpoint}}
\put(611,348){\usebox{\plotpoint}}
\put(611,348){\usebox{\plotpoint}}
\put(611,348){\usebox{\plotpoint}}
\put(611,348){\usebox{\plotpoint}}
\put(611,348){\usebox{\plotpoint}}
\put(611,348){\usebox{\plotpoint}}
\put(611,348){\usebox{\plotpoint}}
\put(611,348){\usebox{\plotpoint}}
\put(611,348){\usebox{\plotpoint}}
\put(611,348){\usebox{\plotpoint}}
\put(611,348){\usebox{\plotpoint}}
\put(611,348){\usebox{\plotpoint}}
\put(611,348){\usebox{\plotpoint}}
\put(611,348){\usebox{\plotpoint}}
\put(611,348){\usebox{\plotpoint}}
\put(611,348){\usebox{\plotpoint}}
\put(611,348){\usebox{\plotpoint}}
\put(611,348){\usebox{\plotpoint}}
\put(611,348){\usebox{\plotpoint}}
\put(611,348){\usebox{\plotpoint}}
\put(611,348){\usebox{\plotpoint}}
\put(611,348){\usebox{\plotpoint}}
\put(611,348){\usebox{\plotpoint}}
\put(611,348){\usebox{\plotpoint}}
\put(611.0,347.0){\usebox{\plotpoint}}
\put(611.0,347.0){\usebox{\plotpoint}}
\put(612.0,346.0){\usebox{\plotpoint}}
\put(612.0,346.0){\usebox{\plotpoint}}
\put(613.0,344.0){\rule[-0.200pt]{0.400pt}{0.482pt}}
\put(613.0,344.0){\usebox{\plotpoint}}
\put(614.0,343.0){\usebox{\plotpoint}}
\put(614.0,343.0){\usebox{\plotpoint}}
\put(615.0,342.0){\usebox{\plotpoint}}
\put(615.0,342.0){\usebox{\plotpoint}}
\put(616.0,340.0){\rule[-0.200pt]{0.400pt}{0.482pt}}
\put(616.0,340.0){\usebox{\plotpoint}}
\put(617.0,339.0){\usebox{\plotpoint}}
\put(617.0,339.0){\usebox{\plotpoint}}
\put(618.0,338.0){\usebox{\plotpoint}}
\put(618.0,338.0){\usebox{\plotpoint}}
\put(619.0,336.0){\rule[-0.200pt]{0.400pt}{0.482pt}}
\put(619.0,336.0){\usebox{\plotpoint}}
\put(620.0,335.0){\usebox{\plotpoint}}
\put(620.0,335.0){\usebox{\plotpoint}}
\put(621.0,333.0){\rule[-0.200pt]{0.400pt}{0.482pt}}
\put(621.0,333.0){\usebox{\plotpoint}}
\put(622.0,332.0){\usebox{\plotpoint}}
\put(622.0,332.0){\usebox{\plotpoint}}
\put(623.0,331.0){\usebox{\plotpoint}}
\put(623.0,331.0){\usebox{\plotpoint}}
\put(624.0,329.0){\rule[-0.200pt]{0.400pt}{0.482pt}}
\put(624.0,329.0){\usebox{\plotpoint}}
\put(625.0,328.0){\usebox{\plotpoint}}
\put(625.0,328.0){\usebox{\plotpoint}}
\put(626.0,326.0){\rule[-0.200pt]{0.400pt}{0.482pt}}
\put(626.0,326.0){\usebox{\plotpoint}}
\put(627.0,325.0){\usebox{\plotpoint}}
\put(627.0,325.0){\usebox{\plotpoint}}
\put(628.0,324.0){\usebox{\plotpoint}}
\put(628.0,324.0){\usebox{\plotpoint}}
\put(629.0,322.0){\rule[-0.200pt]{0.400pt}{0.482pt}}
\put(629.0,322.0){\usebox{\plotpoint}}
\put(630.0,321.0){\usebox{\plotpoint}}
\put(630.0,321.0){\usebox{\plotpoint}}
\put(631.0,319.0){\rule[-0.200pt]{0.400pt}{0.482pt}}
\put(631.0,319.0){\usebox{\plotpoint}}
\put(632.0,318.0){\usebox{\plotpoint}}
\put(632.0,318.0){\usebox{\plotpoint}}
\put(633.0,317.0){\usebox{\plotpoint}}
\put(633.0,317.0){\usebox{\plotpoint}}
\put(634.0,315.0){\rule[-0.200pt]{0.400pt}{0.482pt}}
\put(634.0,315.0){\usebox{\plotpoint}}
\put(635.0,314.0){\usebox{\plotpoint}}
\put(635.0,314.0){\usebox{\plotpoint}}
\put(636.0,312.0){\rule[-0.200pt]{0.400pt}{0.482pt}}
\put(636.0,312.0){\usebox{\plotpoint}}
\put(637.0,311.0){\usebox{\plotpoint}}
\put(637.0,311.0){\usebox{\plotpoint}}
\put(638.0,310.0){\usebox{\plotpoint}}
\put(638.0,310.0){\usebox{\plotpoint}}
\put(639.0,308.0){\rule[-0.200pt]{0.400pt}{0.482pt}}
\put(639.0,308.0){\usebox{\plotpoint}}
\put(640.0,307.0){\usebox{\plotpoint}}
\put(640.0,307.0){\usebox{\plotpoint}}
\put(641.0,305.0){\rule[-0.200pt]{0.400pt}{0.482pt}}
\put(641.0,305.0){\usebox{\plotpoint}}
\put(642.0,304.0){\usebox{\plotpoint}}
\put(642.0,304.0){\usebox{\plotpoint}}
\put(643.0,303.0){\usebox{\plotpoint}}
\put(643.0,303.0){\usebox{\plotpoint}}
\put(644.0,301.0){\rule[-0.200pt]{0.400pt}{0.482pt}}
\put(644.0,301.0){\usebox{\plotpoint}}
\put(645.0,300.0){\usebox{\plotpoint}}
\put(645.0,300.0){\usebox{\plotpoint}}
\put(646.0,298.0){\rule[-0.200pt]{0.400pt}{0.482pt}}
\put(646.0,298.0){\usebox{\plotpoint}}
\put(647.0,297.0){\usebox{\plotpoint}}
\put(647.0,297.0){\usebox{\plotpoint}}
\put(648.0,296.0){\usebox{\plotpoint}}
\put(648.0,296.0){\usebox{\plotpoint}}
\put(649.0,294.0){\rule[-0.200pt]{0.400pt}{0.482pt}}
\put(649.0,294.0){\usebox{\plotpoint}}
\put(650.0,293.0){\usebox{\plotpoint}}
\put(650.0,293.0){\usebox{\plotpoint}}
\put(651.0,291.0){\rule[-0.200pt]{0.400pt}{0.482pt}}
\put(651.0,291.0){\usebox{\plotpoint}}
\put(652.0,290.0){\usebox{\plotpoint}}
\put(652.0,290.0){\usebox{\plotpoint}}
\put(653.0,288.0){\rule[-0.200pt]{0.400pt}{0.482pt}}
\put(653.0,288.0){\usebox{\plotpoint}}
\put(654.0,287.0){\usebox{\plotpoint}}
\put(654.0,287.0){\usebox{\plotpoint}}
\put(655.0,286.0){\usebox{\plotpoint}}
\put(655.0,286.0){\usebox{\plotpoint}}
\put(656.0,284.0){\rule[-0.200pt]{0.400pt}{0.482pt}}
\put(656.0,284.0){\usebox{\plotpoint}}
\put(657.0,283.0){\usebox{\plotpoint}}
\put(657.0,283.0){\usebox{\plotpoint}}
\put(658.0,281.0){\rule[-0.200pt]{0.400pt}{0.482pt}}
\put(658.0,281.0){\usebox{\plotpoint}}
\put(659.0,280.0){\usebox{\plotpoint}}
\put(659.0,280.0){\usebox{\plotpoint}}
\put(660.0,278.0){\rule[-0.200pt]{0.400pt}{0.482pt}}
\put(660.0,278.0){\usebox{\plotpoint}}
\put(661.0,277.0){\usebox{\plotpoint}}
\put(661.0,277.0){\usebox{\plotpoint}}
\put(662.0,276.0){\usebox{\plotpoint}}
\put(662.0,276.0){\usebox{\plotpoint}}
\put(663.0,274.0){\rule[-0.200pt]{0.400pt}{0.482pt}}
\put(663.0,274.0){\usebox{\plotpoint}}
\put(664.0,273.0){\usebox{\plotpoint}}
\put(664.0,273.0){\usebox{\plotpoint}}
\put(665.0,271.0){\rule[-0.200pt]{0.400pt}{0.482pt}}
\put(665.0,271.0){\usebox{\plotpoint}}
\put(666.0,270.0){\usebox{\plotpoint}}
\put(666.0,270.0){\usebox{\plotpoint}}
\put(667.0,268.0){\rule[-0.200pt]{0.400pt}{0.482pt}}
\put(667.0,268.0){\usebox{\plotpoint}}
\put(668.0,267.0){\usebox{\plotpoint}}
\put(668.0,267.0){\usebox{\plotpoint}}
\put(669.0,265.0){\rule[-0.200pt]{0.400pt}{0.482pt}}
\put(669.0,265.0){\usebox{\plotpoint}}
\put(670.0,264.0){\usebox{\plotpoint}}
\put(670.0,264.0){\usebox{\plotpoint}}
\put(671.0,263.0){\usebox{\plotpoint}}
\put(671.0,263.0){\usebox{\plotpoint}}
\put(672.0,261.0){\rule[-0.200pt]{0.400pt}{0.482pt}}
\put(672.0,261.0){\usebox{\plotpoint}}
\put(673.0,260.0){\usebox{\plotpoint}}
\put(673.0,260.0){\usebox{\plotpoint}}
\put(674.0,258.0){\rule[-0.200pt]{0.400pt}{0.482pt}}
\put(674.0,258.0){\usebox{\plotpoint}}
\put(675.0,257.0){\usebox{\plotpoint}}
\put(675.0,257.0){\usebox{\plotpoint}}
\put(676.0,255.0){\rule[-0.200pt]{0.400pt}{0.482pt}}
\put(676.0,255.0){\usebox{\plotpoint}}
\put(677.0,254.0){\usebox{\plotpoint}}
\put(677.0,254.0){\usebox{\plotpoint}}
\put(678.0,252.0){\rule[-0.200pt]{0.400pt}{0.482pt}}
\put(678.0,252.0){\usebox{\plotpoint}}
\put(679.0,251.0){\usebox{\plotpoint}}
\put(679.0,251.0){\usebox{\plotpoint}}
\put(680.0,249.0){\rule[-0.200pt]{0.400pt}{0.482pt}}
\put(680.0,249.0){\usebox{\plotpoint}}
\put(681.0,248.0){\usebox{\plotpoint}}
\put(681.0,248.0){\usebox{\plotpoint}}
\put(682.0,246.0){\rule[-0.200pt]{0.400pt}{0.482pt}}
\put(682.0,246.0){\usebox{\plotpoint}}
\put(683.0,245.0){\usebox{\plotpoint}}
\put(683.0,245.0){\usebox{\plotpoint}}
\put(684.0,244.0){\usebox{\plotpoint}}
\put(684.0,244.0){\usebox{\plotpoint}}
\put(685.0,242.0){\rule[-0.200pt]{0.400pt}{0.482pt}}
\put(685.0,242.0){\usebox{\plotpoint}}
\put(686.0,241.0){\usebox{\plotpoint}}
\put(686.0,241.0){\usebox{\plotpoint}}
\put(687.0,239.0){\rule[-0.200pt]{0.400pt}{0.482pt}}
\put(687.0,239.0){\usebox{\plotpoint}}
\put(688.0,238.0){\usebox{\plotpoint}}
\put(688.0,238.0){\usebox{\plotpoint}}
\put(689.0,236.0){\rule[-0.200pt]{0.400pt}{0.482pt}}
\put(689.0,236.0){\usebox{\plotpoint}}
\put(690.0,235.0){\usebox{\plotpoint}}
\put(690.0,235.0){\usebox{\plotpoint}}
\put(691.0,233.0){\rule[-0.200pt]{0.400pt}{0.482pt}}
\put(691.0,233.0){\usebox{\plotpoint}}
\put(692.0,232.0){\usebox{\plotpoint}}
\put(692.0,232.0){\usebox{\plotpoint}}
\put(693.0,230.0){\rule[-0.200pt]{0.400pt}{0.482pt}}
\put(693.0,230.0){\usebox{\plotpoint}}
\put(694.0,229.0){\usebox{\plotpoint}}
\put(694.0,229.0){\usebox{\plotpoint}}
\put(695.0,227.0){\rule[-0.200pt]{0.400pt}{0.482pt}}
\put(695.0,227.0){\usebox{\plotpoint}}
\put(696.0,226.0){\usebox{\plotpoint}}
\put(696.0,226.0){\usebox{\plotpoint}}
\put(697.0,224.0){\rule[-0.200pt]{0.400pt}{0.482pt}}
\put(697.0,224.0){\usebox{\plotpoint}}
\put(698.0,223.0){\usebox{\plotpoint}}
\put(698.0,223.0){\usebox{\plotpoint}}
\put(699.0,221.0){\rule[-0.200pt]{0.400pt}{0.482pt}}
\put(699.0,221.0){\usebox{\plotpoint}}
\put(700.0,220.0){\usebox{\plotpoint}}
\put(700.0,220.0){\usebox{\plotpoint}}
\put(701.0,218.0){\rule[-0.200pt]{0.400pt}{0.482pt}}
\put(701.0,218.0){\usebox{\plotpoint}}
\put(702.0,217.0){\usebox{\plotpoint}}
\put(702.0,217.0){\usebox{\plotpoint}}
\put(703.0,215.0){\rule[-0.200pt]{0.400pt}{0.482pt}}
\put(703.0,215.0){\usebox{\plotpoint}}
\put(704.0,214.0){\usebox{\plotpoint}}
\put(704.0,214.0){\usebox{\plotpoint}}
\put(705.0,212.0){\rule[-0.200pt]{0.400pt}{0.482pt}}
\put(705.0,212.0){\usebox{\plotpoint}}
\put(706.0,211.0){\usebox{\plotpoint}}
\put(706.0,211.0){\usebox{\plotpoint}}
\put(707.0,209.0){\rule[-0.200pt]{0.400pt}{0.482pt}}
\put(707.0,209.0){\usebox{\plotpoint}}
\put(708.0,208.0){\usebox{\plotpoint}}
\put(708.0,208.0){\usebox{\plotpoint}}
\put(709.0,206.0){\rule[-0.200pt]{0.400pt}{0.482pt}}
\put(709.0,206.0){\usebox{\plotpoint}}
\put(710.0,205.0){\usebox{\plotpoint}}
\put(710.0,205.0){\usebox{\plotpoint}}
\put(711.0,203.0){\rule[-0.200pt]{0.400pt}{0.482pt}}
\put(711.0,203.0){\usebox{\plotpoint}}
\put(712.0,202.0){\usebox{\plotpoint}}
\put(712.0,202.0){\usebox{\plotpoint}}
\put(713.0,200.0){\rule[-0.200pt]{0.400pt}{0.482pt}}
\put(713.0,200.0){\usebox{\plotpoint}}
\put(714.0,199.0){\usebox{\plotpoint}}
\put(714.0,199.0){\usebox{\plotpoint}}
\put(715.0,197.0){\rule[-0.200pt]{0.400pt}{0.482pt}}
\put(715.0,197.0){\usebox{\plotpoint}}
\put(716.0,196.0){\usebox{\plotpoint}}
\put(716.0,196.0){\usebox{\plotpoint}}
\put(717.0,194.0){\rule[-0.200pt]{0.400pt}{0.482pt}}
\put(717.0,194.0){\usebox{\plotpoint}}
\put(718.0,193.0){\usebox{\plotpoint}}
\put(718.0,193.0){\usebox{\plotpoint}}
\put(719.0,191.0){\rule[-0.200pt]{0.400pt}{0.482pt}}
\put(719.0,191.0){\usebox{\plotpoint}}
\put(720.0,190.0){\usebox{\plotpoint}}
\put(720.0,190.0){\usebox{\plotpoint}}
\put(721.0,188.0){\rule[-0.200pt]{0.400pt}{0.482pt}}
\put(721.0,188.0){\usebox{\plotpoint}}
\put(722.0,187.0){\usebox{\plotpoint}}
\put(722.0,187.0){\usebox{\plotpoint}}
\put(723.0,185.0){\rule[-0.200pt]{0.400pt}{0.482pt}}
\put(723.0,185.0){\usebox{\plotpoint}}
\put(724.0,184.0){\usebox{\plotpoint}}
\put(724.0,184.0){\usebox{\plotpoint}}
\put(725.0,182.0){\rule[-0.200pt]{0.400pt}{0.482pt}}
\put(725.0,182.0){\usebox{\plotpoint}}
\put(726.0,181.0){\usebox{\plotpoint}}
\put(726.0,181.0){\usebox{\plotpoint}}
\put(727.0,179.0){\rule[-0.200pt]{0.400pt}{0.482pt}}
\put(727.0,179.0){\usebox{\plotpoint}}
\put(728.0,178.0){\usebox{\plotpoint}}
\put(728.0,178.0){\usebox{\plotpoint}}
\put(729.0,176.0){\rule[-0.200pt]{0.400pt}{0.482pt}}
\put(729.0,176.0){\usebox{\plotpoint}}
\put(730.0,175.0){\usebox{\plotpoint}}
\put(730.0,175.0){\usebox{\plotpoint}}
\put(731.0,173.0){\rule[-0.200pt]{0.400pt}{0.482pt}}
\put(731.0,173.0){\usebox{\plotpoint}}
\put(732.0,171.0){\rule[-0.200pt]{0.400pt}{0.482pt}}
\put(732.0,171.0){\usebox{\plotpoint}}
\put(733.0,170.0){\usebox{\plotpoint}}
\put(733.0,170.0){\usebox{\plotpoint}}
\put(734.0,168.0){\rule[-0.200pt]{0.400pt}{0.482pt}}
\put(734.0,168.0){\usebox{\plotpoint}}
\put(735.0,167.0){\usebox{\plotpoint}}
\put(735.0,167.0){\usebox{\plotpoint}}
\put(736.0,165.0){\rule[-0.200pt]{0.400pt}{0.482pt}}
\put(736.0,165.0){\usebox{\plotpoint}}
\put(737.0,164.0){\usebox{\plotpoint}}
\put(737.0,164.0){\usebox{\plotpoint}}
\put(738.0,162.0){\rule[-0.200pt]{0.400pt}{0.482pt}}
\put(738.0,162.0){\usebox{\plotpoint}}
\put(739.0,161.0){\usebox{\plotpoint}}
\put(739.0,161.0){\usebox{\plotpoint}}
\put(740.0,159.0){\rule[-0.200pt]{0.400pt}{0.482pt}}
\put(740.0,159.0){\usebox{\plotpoint}}
\put(741.0,158.0){\usebox{\plotpoint}}
\put(741.0,158.0){\usebox{\plotpoint}}
\put(742.0,156.0){\rule[-0.200pt]{0.400pt}{0.482pt}}
\put(742.0,156.0){\usebox{\plotpoint}}
\put(743,153.67){\rule{0.241pt}{0.400pt}}
\multiput(743.00,154.17)(0.500,-1.000){2}{\rule{0.120pt}{0.400pt}}
\put(743.0,155.0){\usebox{\plotpoint}}
\put(744,154){\usebox{\plotpoint}}
\put(744,154){\usebox{\plotpoint}}
\put(744,154){\usebox{\plotpoint}}
\put(744,154){\usebox{\plotpoint}}
\put(744,154){\usebox{\plotpoint}}
\put(744,154){\usebox{\plotpoint}}
\put(744,154){\usebox{\plotpoint}}
\put(744,154){\usebox{\plotpoint}}
\put(744,154){\usebox{\plotpoint}}
\put(744,154){\usebox{\plotpoint}}
\put(744,154){\usebox{\plotpoint}}
\put(744,154){\usebox{\plotpoint}}
\put(744,154){\usebox{\plotpoint}}
\put(744,154){\usebox{\plotpoint}}
\put(744,154){\usebox{\plotpoint}}
\put(744,154){\usebox{\plotpoint}}
\put(744,154){\usebox{\plotpoint}}
\put(744,154){\usebox{\plotpoint}}
\put(744,154){\usebox{\plotpoint}}
\put(744,154){\usebox{\plotpoint}}
\put(744,154){\usebox{\plotpoint}}
\put(744,154){\usebox{\plotpoint}}
\put(744,154){\usebox{\plotpoint}}
\put(744,154){\usebox{\plotpoint}}
\put(744,154){\usebox{\plotpoint}}
\put(744,154){\usebox{\plotpoint}}
\put(744,154){\usebox{\plotpoint}}
\put(744,154){\usebox{\plotpoint}}
\put(744,154){\usebox{\plotpoint}}
\put(744,154){\usebox{\plotpoint}}
\put(744,154){\usebox{\plotpoint}}
\put(744,154){\usebox{\plotpoint}}
\put(744,154){\usebox{\plotpoint}}
\put(744,154){\usebox{\plotpoint}}
\put(744,154){\usebox{\plotpoint}}
\put(744,154){\usebox{\plotpoint}}
\put(744,154){\usebox{\plotpoint}}
\put(744,154){\usebox{\plotpoint}}
\put(744,154){\usebox{\plotpoint}}
\put(744,154){\usebox{\plotpoint}}
\put(744,154){\usebox{\plotpoint}}
\put(744,154){\usebox{\plotpoint}}
\put(744,154){\usebox{\plotpoint}}
\put(744,154){\usebox{\plotpoint}}
\put(744,154){\usebox{\plotpoint}}
\put(744,154){\usebox{\plotpoint}}
\put(744,154){\usebox{\plotpoint}}
\put(744,154){\usebox{\plotpoint}}
\put(744,154){\usebox{\plotpoint}}
\put(744.0,153.0){\usebox{\plotpoint}}
\put(744.0,153.0){\usebox{\plotpoint}}
\put(745.0,151.0){\rule[-0.200pt]{0.400pt}{0.482pt}}
\put(745.0,151.0){\usebox{\plotpoint}}
\put(746.0,150.0){\usebox{\plotpoint}}
\put(746.0,150.0){\usebox{\plotpoint}}
\put(747.0,148.0){\rule[-0.200pt]{0.400pt}{0.482pt}}
\put(747.0,148.0){\usebox{\plotpoint}}
\put(748.0,147.0){\usebox{\plotpoint}}
\put(748.0,147.0){\usebox{\plotpoint}}
\put(749.0,145.0){\rule[-0.200pt]{0.400pt}{0.482pt}}
\put(749.0,145.0){\usebox{\plotpoint}}
\put(750.0,144.0){\usebox{\plotpoint}}
\put(750.0,144.0){\usebox{\plotpoint}}
\put(751.0,142.0){\rule[-0.200pt]{0.400pt}{0.482pt}}
\put(751.0,142.0){\usebox{\plotpoint}}
\put(752.0,141.0){\usebox{\plotpoint}}
\put(752.0,141.0){\usebox{\plotpoint}}
\put(753.0,139.0){\rule[-0.200pt]{0.400pt}{0.482pt}}
\put(753.0,139.0){\usebox{\plotpoint}}
\put(754.0,137.0){\rule[-0.200pt]{0.400pt}{0.482pt}}
\put(754.0,137.0){\usebox{\plotpoint}}
\put(755.0,136.0){\usebox{\plotpoint}}
\put(755.0,136.0){\usebox{\plotpoint}}
\put(756.0,134.0){\rule[-0.200pt]{0.400pt}{0.482pt}}
\put(756.0,134.0){\usebox{\plotpoint}}
\put(757.0,133.0){\usebox{\plotpoint}}
\put(757.0,133.0){\usebox{\plotpoint}}
\put(758.0,131.0){\rule[-0.200pt]{0.400pt}{0.482pt}}
\put(758.0,131.0){\usebox{\plotpoint}}
\put(759.0,130.0){\usebox{\plotpoint}}
\put(759.0,130.0){\usebox{\plotpoint}}
\put(760.0,128.0){\rule[-0.200pt]{0.400pt}{0.482pt}}
\put(760.0,128.0){\usebox{\plotpoint}}
\put(761.0,126.0){\rule[-0.200pt]{0.400pt}{0.482pt}}
\put(761.0,126.0){\usebox{\plotpoint}}
\put(762.0,125.0){\usebox{\plotpoint}}
\put(762.0,125.0){\usebox{\plotpoint}}
\put(763.0,123.0){\rule[-0.200pt]{0.400pt}{0.482pt}}
\put(763.0,123.0){\usebox{\plotpoint}}
\put(764.0,122.0){\usebox{\plotpoint}}
\put(764.0,122.0){\usebox{\plotpoint}}
\put(765.0,120.0){\rule[-0.200pt]{0.400pt}{0.482pt}}
\put(765.0,120.0){\usebox{\plotpoint}}
\put(766.0,119.0){\usebox{\plotpoint}}
\put(766.0,119.0){\usebox{\plotpoint}}
\put(767.0,117.0){\rule[-0.200pt]{0.400pt}{0.482pt}}
\put(767.0,117.0){\usebox{\plotpoint}}
\put(768.0,115.0){\rule[-0.200pt]{0.400pt}{0.482pt}}
\put(768.0,115.0){\usebox{\plotpoint}}
\put(769.0,114.0){\usebox{\plotpoint}}
\put(769.0,114.0){\usebox{\plotpoint}}
\put(770.0,112.0){\rule[-0.200pt]{0.400pt}{0.482pt}}
\put(770.0,112.0){\usebox{\plotpoint}}
\put(771.0,111.0){\usebox{\plotpoint}}
\put(771.0,111.0){\usebox{\plotpoint}}
\put(772.0,109.0){\rule[-0.200pt]{0.400pt}{0.482pt}}
\put(772.0,109.0){\usebox{\plotpoint}}
\put(773.0,108.0){\usebox{\plotpoint}}
\put(773.0,108.0){\usebox{\plotpoint}}
\put(774.0,106.0){\rule[-0.200pt]{0.400pt}{0.482pt}}
\put(774.0,106.0){\usebox{\plotpoint}}
\put(775.0,104.0){\rule[-0.200pt]{0.400pt}{0.482pt}}
\put(775.0,104.0){\usebox{\plotpoint}}
\put(776.0,103.0){\usebox{\plotpoint}}
\put(776.0,103.0){\usebox{\plotpoint}}
\put(777.0,101.0){\rule[-0.200pt]{0.400pt}{0.482pt}}
\put(777.0,101.0){\usebox{\plotpoint}}
\put(778.0,100.0){\usebox{\plotpoint}}
\put(778.0,100.0){\usebox{\plotpoint}}
\put(779.0,98.0){\rule[-0.200pt]{0.400pt}{0.482pt}}
\put(779.0,98.0){\usebox{\plotpoint}}
\put(780.0,96.0){\rule[-0.200pt]{0.400pt}{0.482pt}}
\put(780.0,96.0){\usebox{\plotpoint}}
\put(781.0,95.0){\usebox{\plotpoint}}
\put(781.0,95.0){\usebox{\plotpoint}}
\put(782.0,93.0){\rule[-0.200pt]{0.400pt}{0.482pt}}
\put(782.0,93.0){\usebox{\plotpoint}}
\put(783.0,92.0){\usebox{\plotpoint}}
\put(783.0,92.0){\usebox{\plotpoint}}
\put(784.0,90.0){\rule[-0.200pt]{0.400pt}{0.482pt}}
\put(784.0,90.0){\usebox{\plotpoint}}
\put(785.0,88.0){\rule[-0.200pt]{0.400pt}{0.482pt}}
\put(785.0,88.0){\usebox{\plotpoint}}
\put(786.0,87.0){\usebox{\plotpoint}}
\put(786.0,87.0){\usebox{\plotpoint}}
\put(787.0,85.0){\rule[-0.200pt]{0.400pt}{0.482pt}}
\put(787.0,85.0){\usebox{\plotpoint}}
\put(788.0,84.0){\usebox{\plotpoint}}
\put(788.0,84.0){\usebox{\plotpoint}}
\put(789,81.67){\rule{0.241pt}{0.400pt}}
\multiput(789.00,81.17)(0.500,1.000){2}{\rule{0.120pt}{0.400pt}}
\put(789.0,82.0){\rule[-0.200pt]{0.400pt}{0.482pt}}
\put(790,83){\usebox{\plotpoint}}
\put(790,83){\usebox{\plotpoint}}
\put(790,83){\usebox{\plotpoint}}
\put(790,83){\usebox{\plotpoint}}
\put(790,83){\usebox{\plotpoint}}
\put(790,83){\usebox{\plotpoint}}
\put(790,83){\usebox{\plotpoint}}
\put(790,83){\usebox{\plotpoint}}
\put(790,83){\usebox{\plotpoint}}
\put(790,83){\usebox{\plotpoint}}
\put(790,83){\usebox{\plotpoint}}
\put(790,83){\usebox{\plotpoint}}
\put(790,83){\usebox{\plotpoint}}
\put(790,83){\usebox{\plotpoint}}
\put(790,83){\usebox{\plotpoint}}
\put(790,83){\usebox{\plotpoint}}
\put(790,83){\usebox{\plotpoint}}
\put(790,83){\usebox{\plotpoint}}
\put(790,83){\usebox{\plotpoint}}
\put(790,83){\usebox{\plotpoint}}
\put(790,83){\usebox{\plotpoint}}
\put(790,83){\usebox{\plotpoint}}
\put(790,83){\usebox{\plotpoint}}
\put(790,83){\usebox{\plotpoint}}
\put(790,83){\usebox{\plotpoint}}
\put(790,83){\usebox{\plotpoint}}
\put(790,83){\usebox{\plotpoint}}
\put(790,83){\usebox{\plotpoint}}
\put(790,83){\usebox{\plotpoint}}
\put(790,83){\usebox{\plotpoint}}
\put(790,83){\usebox{\plotpoint}}
\put(790,83){\usebox{\plotpoint}}
\put(790,83){\usebox{\plotpoint}}
\put(790,83){\usebox{\plotpoint}}
\put(790,83){\usebox{\plotpoint}}
\put(790,83){\usebox{\plotpoint}}
\put(790,83){\usebox{\plotpoint}}
\put(790,83){\usebox{\plotpoint}}
\put(790,83){\usebox{\plotpoint}}
\put(790,83){\usebox{\plotpoint}}
\put(790,83){\usebox{\plotpoint}}
\put(790,83){\usebox{\plotpoint}}
\put(790,83){\usebox{\plotpoint}}
\put(790,83){\usebox{\plotpoint}}
\put(790,83){\usebox{\plotpoint}}
\put(790,83){\usebox{\plotpoint}}
\put(790,83){\usebox{\plotpoint}}
\put(790,83){\usebox{\plotpoint}}
\put(790,83){\usebox{\plotpoint}}
\put(790,83){\usebox{\plotpoint}}
\put(790,83){\usebox{\plotpoint}}
\put(790,83){\usebox{\plotpoint}}
\put(790,83){\usebox{\plotpoint}}
\put(790,83){\usebox{\plotpoint}}
\put(790,83){\usebox{\plotpoint}}
\put(790,83){\usebox{\plotpoint}}
\put(790,83){\usebox{\plotpoint}}
\put(790,83){\usebox{\plotpoint}}
\put(790,83){\usebox{\plotpoint}}
\put(790,83){\usebox{\plotpoint}}
\put(790,83){\usebox{\plotpoint}}
\put(790,83){\usebox{\plotpoint}}
\put(790,83){\usebox{\plotpoint}}
\put(790,83){\usebox{\plotpoint}}
\put(790,83){\usebox{\plotpoint}}
\put(790,83){\usebox{\plotpoint}}
\put(790,83){\usebox{\plotpoint}}
\put(790,83){\usebox{\plotpoint}}
\put(790,83){\usebox{\plotpoint}}
\put(790,83){\usebox{\plotpoint}}
\put(790,83){\usebox{\plotpoint}}
\put(790,83){\usebox{\plotpoint}}
\put(790,83){\usebox{\plotpoint}}
\put(790,83){\usebox{\plotpoint}}
\put(790,83){\usebox{\plotpoint}}
\put(790,83){\usebox{\plotpoint}}
\put(790.0,83.0){\rule[-0.200pt]{156.344pt}{0.400pt}}
\put(140.0,82.0){\rule[-0.200pt]{0.400pt}{167.425pt}}
\put(140.0,82.0){\rule[-0.200pt]{312.929pt}{0.400pt}}
\put(1439.0,82.0){\rule[-0.200pt]{0.400pt}{167.425pt}}
\put(140.0,777.0){\rule[-0.200pt]{312.929pt}{0.400pt}}
\end{picture}

  \caption{Der Plot beschreibt das Verhältnis  der kondensierten Teilchen mit dem Impuls \(k=0\) zu Gesamtteilchenzahl in Abhängigkeit der Temperatur. Man sieht dass bei \(T=0\) alle Teilchen in ihrem Grundzustand sind.}
  \label{fig:1}
\end{figure}






\subsection*{Phänomenologische Herleitung}

Betrachte die De-Brouglie-Beziehung zwischen Welle und Impuls

\begin{align}
  \label{eq:16}
  \lambda = \frac{h}{p}
\end{align}

Für den Impuls eines Teilchen in einem Gas kann die Kinetische Energie wie folgt angenommen werden

\begin{align}
  \label{eq:17}
  E=\frac{p^2}{2m} \Leftrightarrow p = \sqrt{2mE}
\end{align}

Die Energie \(E\) in Gleichung (\ref{eq:17}) ersetzt man nun mit der thermischen Energie, damit eine Abhängigkeit von der Temperatur in die Beziehung aufgenommen wird. Am einfachsten ist der Ansatz \(E\equiv U=\frac{f}{2}k_B T=\frac{3}{2}k_B T\) welches eine brauchbare Näherung liefert. Die thermische Wellenlänge lautet somit aus Gleichung (\ref{eq:16})

\begin{align}
  \label{eq:18}
  \lambda_{\text{th}} = \frac{h}{\sqrt{3mk_B T}}
\end{align}

Eine genauere Approximation für die thermische Wellenlänge gibt uns dagegen die Zustandsumme eines ideales Gases. Dabei kommt in der Rechnung eine Energie \(E\equiv U=\pi k_B T \) heraus. Für die thermische Wellenlänge wird deswegen im weiteren folgende Gleichung verwendet

\begin{align}
  \label{eq:19}
  \boxed{ \lambda_{\text{th}} = \frac{h}{\sqrt{2\pi mk_B T}} }
\end{align}

Uns interessiert die kritische Temperatur. Die Gleichung (\ref{eq:19}) nach \(T\) umgestellt lautet dann

\begin{align}
  \label{eq:20}
  T = \frac{h^2}{2\pi mk_B \lambda^2_{\text{th}}}
\end{align}

Die einzige noch uns unbekannte Größe in der Gleichung (\ref{eq:20}) ist die thermische Wellenlänge\(\lambda_{\text{th}}\) bei welcher quantenmechanische Effekte auftreten sollen. Sinnvoller weise nimmt man eine Wellenlänge an die mindestens so groß ist wie der Abstand zum nächsten Atom, weil dann eine Überlappung der Wellen-Teilchen erst stattfinden kann. Die Bedingung sollte also lauten

\begin{align}
  \label{eq:21}
  \lambda_{\text{th}} \gtrsim d
\end{align}

Man überlegt sich den Abstand \(d\) folgendermaßen. In einem Würfel das \(N\)-Teilchen beinhaltet, muss eine Würfelkante \(^3\sqrt{N}\) Teilchen haben. Das gleiche gilt für die Würfelstrecke eines Volumens, nämlich \(^3\sqrt{V}\). Das Verhältnis zwischen Kantenlänge und Teilchen auf ihr gibt uns den Abstand zwischen den Teilchen

\begin{align}
  \label{eq:22}
  d = \left( \frac{V}{N}\right)^{1/3}
\end{align}

Diese Beziehung in Gleichung (\ref{eq:20}) eingesetzt ergibt 

\begin{align}
  \label{eq:23}
  T_C = \frac{h^2}{2\pi mk_B} \left( \frac{N}{V}\right)^{2/3}
\end{align}

Vergleicht man nun mit Gleichung (\ref{eq:13}) muss man feststellen, dass die Approximation in (\ref{eq:23}) um den Faktor \(\frac{1}{\zeta(\frac{3}{2})^{\frac{3}{2}}}\approx 0,237 \) abweicht.

Aus Gleichung (\ref{eq:19}) sieht man dass bei hohen Temperaturen \(T\gg T_C\) die Wellenlänge der Teilchen sehr kurz ist. Somit sind die Teilchen sehr scharf lokalisierbar. Wird die Temperatur unter der kritischen Temperatur \(T_C\) immer weiter abgesenkt, fangen die Wellenfunktionen der einzelnen Teilchen sich zu überlagern, bis sie zum Schluss bei \(\lambda_{\text{th}} \gtrsim d\) nur noch eine einzige Wellenfunktion bilden und somit alle in dem gleichen Grundzustand \(N_0\) sind. Dazu muss man sagen, dass der Zustand sich nicht im Ortsraum sondern im Impulsraum befindet und die Teilchen sehr wohl einen mittleren Ortsabstand voneinander haben. Jedoch da sie sich alle in gleichen Zustand befinden ist es so als wären sie eine große Einheit. D.h. bestrahlt man ein Teilchen mit Laserlicht, ist es so als ob man alle anderen Teilchen mitbestrahlt. Man sagt dazu auch dass die Teilchen eine Identitätskrise haben.


\end{document}
