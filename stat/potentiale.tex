\input{../headers/header_script.tex}
\usepackage{amsmath} 



\begin{document}

\section*{Thermodynamische Potentiale}

Wir betrachten homogene Systeme, deren Gleichgewichtslage durch 2 Zustandsvariablen eindeutig festgelegt ist. Es ergeben sich folgende Variablenpaare für die Größen \(S,T,p,V\), wobei Kombinationen zwischen \(S,T\) und \(p,V\) wegen Abhängigkeit voneinander ausgeschlossen ist.

\begin{align}
  \label{eq:1}
  U(S,V),\,\, F(T,V),\,\, H(S,p),\,\, G(T,p)
\end{align}

Die Zustandsgrößen \(U,F,H,G\) sind Thermodynamische Potentiale. Man nennt Zustandsvariablen, als Funktionen deren eine Größe zum thermondynamischen Potential wird ihre \textbf{natürliche Variable} genannt. Im Falle von \(U\) sind \(S\) und \(V\) ihre natürlichen Variablen.

Die Thermodynamischen Potentiale haben folgende Eigenschaften

\begin{itemize}
\item[1] Sie sind Zustandsgrößen von der Dimension einer Energie
\item[2] Ihre partiellen Ableitungen nach den natürlichen Variablen liefern wieder einfache Zustandsgrößen
\item[3] Sie liefern vollständige thermodynamische Informationen über das betrachtete System
\end{itemize}

Vergleiche die schon aus der Mechanik oder Elektronik bekannten Potentiale wie die Kraft \(\vec F=-\vec \nabla V\) bzw \(\vec E = -\vec \nabla\phi - \pdiff_t \vec A\) oder \(\vec B = \nabla\times \vec A\).

Betrachten wir die Zustandsfunktion \(U = U(S,V)\) und bilden das Vollständige Differential

\begin{align}
  \label{eq:2}
  dU = \left(\pdiff U_S \right)_V dS +  \left(\pdiff U_V \right)_S dV
\end{align}

Vergleiche mit dem vollständigen Differential der inneren Energie aus dem 1.Hauptsatz

\begin{align}
  \label{eq:3}
  dU = TdS - pdV
\end{align}
so ergibt sich

\begin{align}
  \label{eq:4}
  T = \left(\pdiff U_S \right)_V \qquad p = - \left(\pdiff U_V \right)_S
\end{align}

Damit erfüllt die Innere Energie den zweiten Punkt von den verlangten Eigenschaften für thermodynamische Potentiale. Wobei die anderen Kriterien auch erfüllt sind. D.h. die Innere Energie ist ein thermodynamisches Potential

\begin{align}
  \label{eq:5}
  U=U(S,V)
\end{align}

Mit Hilfe der Legendre-Transformation der Zustandsvariablen kann man vom Potential \(U(S,V)\) zu einem anderen Potential \(F(T,V)\) kommen.

\begin{align}
  \label{eq:6}
 F(T,V) = U - TS
\end{align}

Das thermodynamisches Potential in der Gleichung (\ref{eq:6}) nennt man die helmholtsche freie Energie. Wir bilden das totale Differential

\begin{align}
  \label{eq:8}
  dF &= dU - TdS - SdT \notag\\
&=  \cancel{TdS} - pdV - \cancel{TdS} - SdT \notag\\
&=-pdV - SdT
\end{align}

Bildet man das vollständige Differential einer Funktion \(F(T,V)\) so erhält man

\begin{align}
  \label{eq:7}
  dF = \left(\pdiff F_T \right)_S dT +  \left(\pdiff F_V \right)_T dV
\end{align}

Durch vergleich der Koeffizienten sieht man

\begin{align}
  \label{eq:9}
  p = - \left(\pdiff F_V \right)_T  \qquad S = - \left(\pdiff F_T \right)_S 
\end{align}

Also ist die freie Energie ebenfalls ein thermodynamisches Potential.


Analog erhält man noch zwei weitere wichtige thermodynamische Potentiale die \textbf{Enthalpie} \(H(S,p)\) 

\begin{align}
  \label{eq:10}
  H(S,p) = U + pV
\end{align}

Aus der sich folgende Zustandsvariablen ableiten lassen

\begin{align}
  \label{eq:11}
  T= \left(\pdiff H_S \right)_p  \qquad V = - \left(\pdiff H_p \right)_S 
\end{align}
und die \textbf{freie Enthalpie} bzw das \textbf{Gibbs-Potential} \(G(T,p)\)

\begin{align}
  \label{eq:12}
  G(T,p) = U - TS + pV
\end{align}

Aus der sich folgende Zustandsvariablen ableiten lassen

\begin{align}
  \label{eq:13}
  S= -\left(\pdiff G_T \right)_p  \qquad V =  \left(\pdiff G_p \right)_T
\end{align}

Alle vier Potentiale aus Gleichung (\ref{eq:1}) zusammengefasst:


\begin{table}[h]
  \centering
  \begin{tabular}{lll}
    Name&Funktion&differentielle Form\\
\hline
innere Energie&\(U(S,V)=U\)&\(dU = TdS-pdV\)\\
freie Energie&\(F(T,V)=U-TS\)&\(dF=-SdT-pdV\)\\
Entalpie & \(H(S,p)=U+PV\)&\(dH = TdS+VdP\)\\
freie Entalpie & \(G(T,p)=U-TS+pV\)&\(dG = -SdT+Vdp\)
  \end{tabular}
\end{table}


Die thermodynamischen Potentiale lassen sich auf den Fall von mehreren äußeren Parametern \(x= (x_1,...,x_n)\) verallgemeinern. 

\begin{align}
  \label{eq:14}
  \diff U = T\diff S - \sum_{i=1}^n X_i \diff{} x_i
\end{align}

Dies zum Beispiel ist die Verallgemeinerung von der differentiellen Form der inneren Energie (\ref{eq:2}), denn neben den Volumen \(V\) können zum Beispiel Teilchenzahl \(N\) und ein Magnetfeld \(B\) als äußerer Parameter auftreten. In diesem Fall wäre die differentielle Darstellung der inneren Energie

\begin{align}
  \label{eq:15}
  \diff U = T\diff S - P\diff V - M\diff B + \mu \diff N
\end{align}

Durch Legendre-Transformation lassen sich neue thermodynamische Potentiale herleiten. Aus diesem Grund existieren so viele Potentiale wie Kombinationsmöglichkeiten der Zustandsvariablen \(n\) von Systemen.

\subsection*{Maxwellrelationen}

Da man die partiellen Ableitungen bei stetig differentierbaren Funktionen vertauschen kann d.h. \(\pdiff_y \pdiff f_x = \pdiff_x \pdiff f_y \), kann man die so genannten Maxwellrelationen herleiten. Mit Hilfe der Gleichungen (\ref{eq:3}) und (\ref{eq:4}) folgt

 \begin{align}
  \label{eq:16}
  \left(\pdiff U_S\right)_V = T \qquad   \left(\pdiff U_V\right)_S = -p
\end{align}

\begin{align}
  \label{eq:17}
  \pdiff_V \underbr{\left(\pdiff U_S\right)}_{T} &= \pdiff_S \underbr{\left(\pdiff U_V\right)}_{-p} \\
\left(\pdiff T_V\right) &= -\left(\pdiff p_S\right)
\end{align}

Analog lassen sich weitere Maxwellrelationen aus den uns bekannten Potentialen herleiten. Hier die Zusammenstellung aller möglichen Maxwellrelationen aus den hier vier besprochenen Potentialen

\begin{align}
  \label{eq:18}
  -\left(\pdiff T_V\right)_S &= \left(\pdiff p_S\right)_V \qquad \text{ aus }dU \\
  \left(\pdiff S_V\right)_T &= \left(\pdiff p_T\right)_V \qquad \text{ aus }dF  \\
 \left(\pdiff T_p\right)_S &= \left(\pdiff V_S\right)_p \qquad \text{ aus }dH \\
 - \left(\pdiff S_p\right)_T &= \left(\pdiff V_T\right)_p \qquad \text{ aus }dG 
\end{align}


\subsection*{Referenzen}
\begin{itemize}
\item Fließbach
\item \url{http://141.20.44.172/ede/06theophys2/060630.pdf}
\end{itemize}

\end{document}
