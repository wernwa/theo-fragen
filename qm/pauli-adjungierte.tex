\input{../headers/header_script.tex}
\usepackage{amsmath} 



\begin{document}

\section*{adjungierter Pauli-Spinor}


Um den Adjungierten Pauli-Spinor zu bestimmen wollen wir die adjungierte Dirac-Gleichung herleiten. Dazu gehen wir von der nicht adjungierten freien Dirac-Gleichung aus

\begin{align}
  \label{eq:1}
  i\gamma^\mu\partial_\mu\psi - \frac{mc}{\hbar}\psi = 0
\end{align}

Nun adjungieren wir diese Gleichung

\begin{align}
  \label{eq:2}
   (i\gamma^\mu\partial_\mu\psi)^\dagger - \frac{mc}{\hbar}\psi^\dagger &= 0 \notag\\
 -i(\partial_\mu\psi)^\dagger(\gamma^\mu)^\dagger - \frac{mc}{\hbar}\psi^\dagger &= 0
\end{align}

Nun machen wir eine kleine Nebenrechnung

\begin{align}
  \label{eq:3}
  (\gamma^\mu)^\dagger &= (\beta,\beta\alpha_1,\beta\alpha_2,\beta\alpha_3)^\dagger \notag \\
&= (\beta^\dagger,(\beta\alpha_1)^\dagger,(\beta\alpha_2)^\dagger,(\beta\alpha_3)^\dagger) \notag \\
&= (\beta^\dagger,(\alpha_1)^\dagger(\beta)^\dagger,(\alpha_2)^\dagger(\beta)^\dagger,(\alpha_3)^\dagger(\beta)^\dagger)\quad \text{mit }\beta^\dagger=\beta \text{ und }(\alpha_i)^\dagger=\alpha_i \notag \\
&= (\mathds 1\cdot \beta,\mathds 1\cdot \alpha_1\beta,\mathds 1\cdot\alpha_2\beta,\mathds 1\cdot\alpha_3\beta)   \notag \\
&= (\beta\cdot\beta\cdot \beta,\beta\cdot\beta\cdot \alpha_1\beta,\beta\cdot\beta\cdot\alpha_2\beta,\beta\cdot\beta\cdot\alpha_3\beta)   \notag \\
&= \beta\cdot\underbr{(\beta,\beta\cdot \alpha_1,\beta\cdot\alpha_2,\beta\cdot\alpha_3)}_{\gamma^\mu} \cdot\beta 
\end{align}
Wir bekommen aus der Gleichung (\ref{eq:3}) eine wichtige Relation mit \(\beta=\gamma^0\)

\begin{align}
  \label{eq:4}
  \boxed{ (\gamma^\mu)^\dagger = \gamma^0\gamma^\mu\gamma^0  }
\end{align}

Setzen wir nun diese Relation in die Gleichung (\ref{eq:2}) ein und multiplizieren von rechts mit \(\gamma^0\)

\begin{align}
  \label{eq:5}
  -i(\partial_\mu\psi)^\dagger\gamma^0\gamma^\mu\gamma^0   - \frac{mc}{\hbar}\psi^\dagger &= 0 \qquad |\cdot\gamma^0 \notag\\
-i(\partial_\mu\underbr{\psi^\dagger\gamma^0}_{\overline \psi})\gamma^\mu\mathds 1  - \frac{mc}{\hbar}\underbr{\psi^\dagger\gamma^0}_{\overline \psi} &= 0 
\end{align}
Damit sieht unsere adjungierte Dirac-Gleichung wie folgt aus

\begin{align}
  \label{eq:6}
  -i(\partial_\mu\overline \psi)\gamma^\mu  - \frac{mc}{\hbar}\overline \psi &= 0 
\end{align}
Mit dem \textbf{adjungierten Paulispinor} 

\begin{align}
  \label{eq:7}
  \boxed{\overline \psi = \psi^\dagger\gamma^0}
\end{align}

Unter Lorenztranformation verhält sich der adjungierte Paulispinor \(\overline \psi\)  invers zu dem Dirac-Spinor \(\psi\)

\begin{align}
  \label{eq:8}
  \overline \psi'= \psi^{'\dagger}\gamma^0  = (S(\Lambda)\psi)^\dagger\gamma^0 = \psi^\dagger S(\Lambda)^\dagger \gamma^0 = \psi^\dagger\mathds 1 S(\Lambda)^\dagger \gamma^0 = \underbr{\psi^\dagger \gamma^0}_{\overline\psi}\gamma^0 S(\Lambda)^\dagger \gamma^0  =  \overline\psi \gamma^0 S(\Lambda)^\dagger \gamma^0 
\end{align}







\subsection*{Referenzen}
\begin{itemize}
\item Claude Cohen-Tannoudji Quantenmechanik Band 2
\item Zettili Quanten Mehanics
\item Rollnik Quantentheorie 2
\end{itemize}

\end{document}
