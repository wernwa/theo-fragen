\documentclass[10pt,a4paper,oneside,fleqn]{article}
\usepackage{geometry}
\geometry{a4paper,left=20mm,right=20mm,top=1cm,bottom=2cm}
\usepackage[utf8]{inputenc}
%\usepackage{ngerman}
\usepackage{amsmath}                % brauche ich um dir Formel zu umrahmen.
\usepackage{amsfonts}                % brauche ich für die Mengensymbole
\usepackage{graphicx}
\setlength{\parindent}{0px}
\setlength{\mathindent}{10mm}
\usepackage{bbold}                    %brauche ich für die doppel Zahlen Darstellung (Einheitsmatrix z.B)



\usepackage{color}
\usepackage{titlesec} %sudo apt-get install texlive-latex-extra

\definecolor{darkblue}{rgb}{0.1,0.1,0.55}
\definecolor{verydarkblue}{rgb}{0.1,0.1,0.35}
\definecolor{darkred}{rgb}{0.55,0.2,0.2}

%hyperref Link color
\usepackage[colorlinks=true,
        linkcolor=darkblue,
        citecolor=darkblue,
        filecolor=darkblue,
        pagecolor=darkblue,
        urlcolor=darkblue,
        bookmarks=true,
        bookmarksopen=true,
        bookmarksopenlevel=3,
        plainpages=false,
        pdfpagelabels=true]{hyperref}

\titleformat{\chapter}[display]{\color{darkred}\normalfont\huge\bfseries}{\chaptertitlename\
\thechapter}{20pt}{\Huge}

\titleformat{\section}{\color{darkblue}\normalfont\Large\bfseries}{\thesection}{1em}{}
\titleformat{\subsection}{\color{verydarkblue}\normalfont\large\bfseries}{\thesubsection}{1em}{}

% Notiz Box
\usepackage{fancybox}
\newcommand{\notiz}[1]{\vspace{5mm}\ovalbox{\begin{minipage}{1\textwidth}#1\end{minipage}}\vspace{5mm}}

\usepackage{cancel}
\setcounter{secnumdepth}{3}
\setcounter{tocdepth}{3}





%-------------------------------------------------------------------------------
%Diff-Makro:
%Das Diff-Makro stellt einen Differentialoperator da.
%
%Benutzung:
% \diff  ->  d
% \diff f  ->  df
% \diff^2 f  ->  d^2 f
% \diff_x  ->  d/dx
% \diff^2_x  ->  d^2/dx^2
% \diff f_x  ->  df/dx
% \diff^2 f_x  ->  d^2f/dx^2
% \diff^2{f(x^5)}_x  ->  d^2(f(x^5))/dx^2
%
%Ersetzt man \diff durch \pdiff, so wird der partieller
%Differentialoperator dargestellt.
%
\makeatletter
\def\diff@n^#1{\@ifnextchar{_}{\diff@n@d^#1}{\diff@n@fun^#1}}
\def\diff@n@d^#1_#2{\frac{\textrm{d}^#1}{\textrm{d}#2^#1}}
\def\diff@n@fun^#1#2{\@ifnextchar{_}{\diff@n@fun@d^#1#2}{\textrm{d}^#1#2}}
\def\diff@n@fun@d^#1#2_#3{\frac{\textrm{d}^#1 #2}{\textrm{d}#3^#1}}
\def\diff@one@d_#1{\frac{\textrm{d}}{\textrm{d}#1}}
\def\diff@one@fun#1{\@ifnextchar{_}{\diff@one@fun@d #1}{\textrm{d}#1}}
\def\diff@one@fun@d#1_#2{\frac{\textrm{d}#1}{\textrm{d}#2}}
\newcommand*{\diff}{\@ifnextchar{^}{\diff@n}
  {\@ifnextchar{_}{\diff@one@d}{\diff@one@fun}}}
%
%Partieller Diff-Operator.
\def\pdiff@n^#1{\@ifnextchar{_}{\pdiff@n@d^#1}{\pdiff@n@fun^#1}}
\def\pdiff@n@d^#1_#2{\frac{\partial^#1}{\partial#2^#1}}
\def\pdiff@n@fun^#1#2{\@ifnextchar{_}{\pdiff@n@fun@d^#1#2}{\partial^#1#2}}
\def\pdiff@n@fun@d^#1#2_#3{\frac{\partial^#1 #2}{\partial#3^#1}}
\def\pdiff@one@d_#1{\frac{\partial}{\partial #1}}
\def\pdiff@one@fun#1{\@ifnextchar{_}{\pdiff@one@fun@d #1}{\partial#1}}
\def\pdiff@one@fun@d#1_#2{\frac{\partial#1}{\partial#2}}
\newcommand*{\pdiff}{\@ifnextchar{^}{\pdiff@n}
  {\@ifnextchar{_}{\pdiff@one@d}{\pdiff@one@fun}}}
\makeatother
%
%Das gleich nur mit etwas andere Syntax. Die Potenz der Differentiation wird erst
%zum Schluss angegeben. Somit lautet die Syntax:
%
% \diff_x^2  ->  d^2/dx^2
% \diff f_x^2  ->  d^2f/dx^2
% \diff{f(x^5)}_x^2  ->  d^2(f(x^5))/dx^2
% Ansonsten wie Oben.
%
%Ersetzt man \diff durch \pdiff, so wird der partieller
%Differentialoperator dargestellt.
%
%\makeatletter
%\def\diff@#1{\@ifnextchar{_}{\diff@fun#1}{\textrm{d} #1}}
%\def\diff@one_#1{\@ifnextchar{^}{\diff@n{#1}}%
%  {\frac{\textrm d}{\textrm{d} #1}}}
%\def\diff@fun#1_#2{\@ifnextchar{^}{\diff@fun@n#1_#2}%
%  {\frac{\textrm d #1}{\textrm{d} #2}}}
%\def\diff@n#1^#2{\frac{\textrm d^#2}{\textrm{d}#1^#2}}
%\def\diff@fun@n#1_#2^#3{\frac{\textrm d^#3 #1}%
%  {\textrm{d}#2^#3}}
%\def\diff{\@ifnextchar{_}{\diff@one}{\diff@}}
%\newcommand*{\diff}{\@ifnextchar{_}{\diff@one}{\diff@}}
%
%Partieller Diff-Operator.
%\def\pdiff@#1{\@ifnextchar{_}{\pdiff@fun#1}{\partial #1}}
%\def\pdiff@one_#1{\@ifnextchar{^}{\pdiff@n{#1}}%
%  {\frac{\partial}{\partial #1}}}
%\def\pdiff@fun#1_#2{\@ifnextchar{^}{\pdiff@fun@n#1_#2}%
%  {\frac{\partial #1}{\partial #2}}}
%\def\pdiff@n#1^#2{\frac{\partial^#2}{\partial #1^#2}}
%\def\pdiff@fun@n#1_#2^#3{\frac{\partial^#3 #1}%
%  {\partial #2^#3}}
%\newcommand*{\pdiff}{\@ifnextchar{_}{\pdiff@one}{\pdiff@}}
%\makeatother

%-------------------------------------------------------------------------------
%%Nützliche Makros um in der Quantenmechanik Bras, Kets und das Skalarprodukt
%%zwischen den beiden darzustellen.
%%Benutzung:
%% \bra{x}  ->    < x |
%% \ket{x}  ->    | x >
%% \braket{x}{y} ->   < x | y >

\newcommand\bra[1]{\left\langle #1 \right|}
\newcommand\ket[1]{\left| #1 \right\rangle}
\newcommand\braket[2]{%
  \left\langle #1\vphantom{#2} \right.%
  \left|\vphantom{#1#2}\right.%
  \left. \vphantom{#1}#2 \right\rangle}%

%-------------------------------------------------------------------------------
%%Aus dem Buch:
%%Titel:  Latex in Naturwissenschaften und Mathematik
%%Autor:  Herbert Voß
%%Verlag: Franzis Verlag, 2006
%%ISBN:   3772374190, 9783772374197
%%
%%Hier werden drei Makros definiert:\mathllap, \mathclap und \mathrlap, welche
%%analog zu den aus Latex bekannten \rlap und \llap arbeiten, d.h. selbst
%%keinerlei horizontalen Platz benötigen, aber dennoch zentriert zum aktuellen
%%Punkt erscheinen.

\newcommand*\mathllap{\mathstrut\mathpalette\mathllapinternal}
\newcommand*\mathllapinternal[2]{\llap{$\mathsurround=0pt#1{#2}$}}
\newcommand*\clap[1]{\hbox to 0pt{\hss#1\hss}}
\newcommand*\mathclap{\mathpalette\mathclapinternal}
\newcommand*\mathclapinternal[2]{\clap{$\mathsurround=0pt#1{#2}$}}
\newcommand*\mathrlap{\mathpalette\mathrlapinternal}
\newcommand*\mathrlapinternal[2]{\rlap{$\mathsurround=0pt#1{#2}$}}

%%Das Gleiche nur mit \def statt \newcommand.
%\def\mathllap{\mathpalette\mathllapinternal}
%\def\mathllapinternal#1#2{%
%  \llap{$\mathsurround=0pt#1{#2}$}% $
%}
%\def\clap#1{\hbox to 0pt{\hss#1\hss}}
%\def\mathclap{\mathpalette\mathclapinternal}
%\def\mathclapinternal#1#2{%
%  \clap{$\mathsurround=0pt#1{#2}$}%
%}
%\def\mathrlap{\mathpalette\mathrlapinternal}
%\def\mathrlapinternal#1#2{%
%  \rlap{$\mathsurround=0pt#1{#2}$}% $
%}

%-------------------------------------------------------------------------------
%%Hier werden zwei neue Makros definiert \overbr und \underbr welche analog zu
%%\overbrace und \underbrace funktionieren jedoch die Gleichung nicht
%%'zerreißen'. Dies wird ermöglicht durch das \mathclap Makro.

\def\overbr#1^#2{\overbrace{#1}^{\mathclap{#2}}}
\def\underbr#1_#2{\underbrace{#1}_{\mathclap{#2}}}
\usepackage{amsmath} 



\begin{document}

\section*{adjungierter Pauli-Spinor}


Um den Adjungierten Pauli-Spinor zu bestimmen wollen wir die adjungierte Dirac-Gleichung herleiten. Dazu gehen wir von der nicht adjungierten freien Dirac-Gleichung aus

\begin{align}
  \label{eq:1}
  i\gamma^\mu\partial_\mu\psi - \frac{mc}{\hbar}\psi = 0
\end{align}

Nun adjungieren wir diese Gleichung

\begin{align}
  \label{eq:2}
   (i\gamma^\mu\partial_\mu\psi)^\dagger - \frac{mc}{\hbar}\psi^\dagger &= 0 \notag\\
 -i(\partial_\mu\psi)^\dagger(\gamma^\mu)^\dagger - \frac{mc}{\hbar}\psi^\dagger &= 0
\end{align}

Nun machen wir eine kleine Nebenrechnung

\begin{align}
  \label{eq:3}
  (\gamma^\mu)^\dagger &= (\beta,\beta\alpha_1,\beta\alpha_2,\beta\alpha_3)^\dagger \notag \\
&= (\beta^\dagger,(\beta\alpha_1)^\dagger,(\beta\alpha_2)^\dagger,(\beta\alpha_3)^\dagger) \notag \\
&= (\beta^\dagger,(\alpha_1)^\dagger(\beta)^\dagger,(\alpha_2)^\dagger(\beta)^\dagger,(\alpha_3)^\dagger(\beta)^\dagger)\quad \text{mit }\beta^\dagger=\beta \text{ und }(\alpha_i)^\dagger=\alpha_i \notag \\
&= (\mathds 1\cdot \beta,\mathds 1\cdot \alpha_1\beta,\mathds 1\cdot\alpha_2\beta,\mathds 1\cdot\alpha_3\beta)   \notag \\
&= (\beta\cdot\beta\cdot \beta,\beta\cdot\beta\cdot \alpha_1\beta,\beta\cdot\beta\cdot\alpha_2\beta,\beta\cdot\beta\cdot\alpha_3\beta)   \notag \\
&= \beta\cdot\underbr{(\beta,\beta\cdot \alpha_1,\beta\cdot\alpha_2,\beta\cdot\alpha_3)}_{\gamma^\mu} \cdot\beta 
\end{align}
Wir bekommen aus der Gleichung (\ref{eq:3}) eine wichtige Relation mit \(\beta=\gamma^0\)

\begin{align}
  \label{eq:4}
  \boxed{ (\gamma^\mu)^\dagger = \gamma^0\gamma^\mu\gamma^0  }
\end{align}

Setzen wir nun diese Relation in die Gleichung (\ref{eq:2}) ein und multiplizieren von rechts mit \(\gamma^0\)

\begin{align}
  \label{eq:5}
  -i(\partial_\mu\psi)^\dagger\gamma^0\gamma^\mu\gamma^0   - \frac{mc}{\hbar}\psi^\dagger &= 0 \qquad |\cdot\gamma^0 \notag\\
-i(\partial_\mu\underbr{\psi^\dagger\gamma^0}_{\overline \psi})\gamma^\mu\mathds 1  - \frac{mc}{\hbar}\underbr{\psi^\dagger\gamma^0}_{\overline \psi} &= 0 
\end{align}
Damit sieht unsere adjungierte Dirac-Gleichung wie folgt aus

\begin{align}
  \label{eq:6}
  -i(\partial_\mu\overline \psi)\gamma^\mu  - \frac{mc}{\hbar}\overline \psi &= 0 
\end{align}
Mit dem \textbf{adjungierten Paulispinor} 

\begin{align}
  \label{eq:7}
  \boxed{\overline \psi = \psi^\dagger\gamma^0}
\end{align}

Unter Lorenztranformation verhält sich der adjungierte Paulispinor \(\overline \psi\)  invers zu dem Dirac-Spinor \(\psi\)

\begin{align}
  \label{eq:8}
  \overline \psi'= \psi^{'\dagger}\gamma^0  = (S(\Lambda)\psi)^\dagger\gamma^0 = \psi^\dagger S(\Lambda)^\dagger \gamma^0 = \psi^\dagger\mathds 1 S(\Lambda)^\dagger \gamma^0 = \underbr{\psi^\dagger \gamma^0}_{\overline\psi}\gamma^0 S(\Lambda)^\dagger \gamma^0  =  \overline\psi \gamma^0 S(\Lambda)^\dagger \gamma^0 
\end{align}







\subsection*{Referenzen}
\begin{itemize}
\item Claude Cohen-Tannoudji Quantenmechanik Band 2
\item Zettili Quanten Mehanics
\item Rollnik Quantentheorie 2
\end{itemize}

\end{document}
