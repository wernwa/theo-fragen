\input{../headers/header_script.tex}
 



\begin{document}

\section*{Die Dirac Gleichung}

 Die Möglichkeit \(\rho=\psi^*\psi\) für die Wahrscheinlichkeitsdichte zu schreiben folgt aus der Tatsache, dass die Zeitableitung in der nicht relativistische Schrödinger-Gleichung nur in 1-Ordnung auftritt. Im Vergleich nimmt \(\rho\) in der Klein-Gordon Gleichung auch negative Werte an, dort ist die Zeitlichte Ableitung von 2-Ordnung.\\
\\
Es gilt nun eine Differential Gleichung 1-Ordung in der Zeit der Form

\begin{align}
  \label{eq:1}
  i\hbar \pdiff_t \psi = H \psi
\end{align}

die relativistische Enegie-Impuls-Beziehung

\begin{align}
  \label{eq:2}
  p_\mu p^\mu = \frac{E^2}{c^2} - \vec p^2 = (mc)^2
\end{align}

für ein freies Teilchen erfüllt. Die Betrachtungen im Zusammenhang mit der Klein-Gordon-Gleichung haben gezeigt, dass man dieses Problem für eine einfache skalare Wellenfunktion \(\psi\) nicht lösen kann. Diracs Idee war eine mehrdimensionale Wellenfunktion einzuführen
.
\begin{align}
  \label{eq:3}
  \psi(x) = \begin{pmatrix}\psi_1(x)\\\psi_2(x)\\\vdots\\ \psi_n(x)\\ \end{pmatrix}
\end{align}

Des weiteren muss die Gleichung \eqref{eq:1} folgende Forderungen erfüllen:

\begin{itemize}
\item Die Komponenten von \(\psi\) müssen die Klein-Gordon-Gleichung erfüllen, so dass der Hamilton-Operator die relativistische Beziehung \(E^2 = p^2c^2+m^2c^4\) erfüllt.
\item Es existiert ein erhaltener Viererstrom, dessen nullte Komponente eine positive Dichte ist.
\item Die Gleichung muss Lorentz-kovariant sein. Das bedeutet, dass sie unter Transformation ihre Form behält. Bezugssystem unabhängig. Damit dies erfüllt ist muss gelten: Da die zeitliche Ableitung nur in 1-Ordnung auftritt, muss auch die räumliche Ableitung in 1-Ordnung auftreten. 
\end{itemize}

Der Ansatz für den Hamiltonoperator \(H\) sollte so sein dass

\begin{align}
  \label{eq:4}
  H^2 = E^2 = p^2c^2+ m^2c^4
\end{align}

das Quadrat der relativistischen Energie im Quadrat gleich ist. Folgender allgemeiner Ansatz erfüllt diese Bedingung

\begin{align}
  \label{eq:5}
  H = c (\alpha_x p_x + \alpha_y p_y + \alpha_z p_z) + \beta m^2c^4 = c\sum_{i=1}^3\alpha_i p_i   + \beta m^2c^4
\end{align}
Die unbekannten Koeffizienten \(\alpha_i,\beta\) können nicht einfach Zahlen sein, da sonst die Gleichung nicht einmal forminvariant gegenüber räumlichen Drehungen ist (D.h. die Form der Gleichung ändert sich je nachdem wie man das Koordinatensystem wählt). \(\alpha_i,\beta\) müssen hermitische Matritzen sein damit \(H\) hermitesch ist. Daraus folgt  \(\alpha_i,\beta\) müssen \(N\times N\) Matrizen sein.

Um die unbekannten Matritzen  \(\alpha_i,\beta\) zu bestimmen gehen wir von der Klein-Gordon Gleichung aus

\begin{align}
  \label{eq:6}
  -\hbar^2 \pdiff^2_t \psi(x) &= \underbrace{c^2(\vec p^2 + m^2c^2)}_{H^2}\psi(x)\\
 &\stackrel{!}= \left(  c\sum_{i=1}^3\alpha_i p_i   + \beta m^2c^4 \right)^2\psi(x)\\
&=[c \sum_{j=1}^3 \alpha_jp_j+\beta mc^2] \cdot [c \sum_{i=1}^3 \alpha_ip_i+\beta mc^2]\psi(p) \\
&=c^2 [ \sum_{j=1}^3 \alpha_jp_j+\beta mc] \cdot [ \sum_{i=1}^3 \alpha_ip_i+\beta mc]\psi(p) \\
&=c^2 \left( \sum_{j=1}^3 \alpha_jp_j\sum_{i=1}^3 \alpha_ip_i  +   \sum_{j=1}^3 \alpha_jp_j \beta mc  + \beta mc \sum_{i=1}^3 \alpha_ip_i+ \beta^2 m^2c^2\right)\psi(p) \\
&= c^2\left(\sum_{i,j=1}^3\alpha_i\alpha_j p_ip_j+\sum_{i=1}^3(\alpha_i\beta+\beta\alpha_i)p_i mc +  \beta^2m^2c^2\right)\psi(p)
\end{align}

Der Koeffizientenverleich zwischen \(c^2(\vec p^2 + m^2c^2)\) und \( c^2\left(\sum_{i,j=1}^3\alpha_i\alpha_j p_ip_j+\sum_{i=1}^3(\alpha_i\beta+\beta\alpha_i)p_i mc +  \beta^2m^2c^2\right)\) liefert

\begin{itemize}
\item  \(\boxed{\beta^2=1}\)
\item Antikommutator: 
  \begin{equation}
    \label{eq:8}
    \boxed{\{\alpha_i,\beta\}=0}
  \end{equation}
Damit der Mischterm \((\alpha_i\beta+\beta\alpha_i)p_i mc\)  verschwindet 

\item \(i\neq j\): z.B:  \(\{\alpha_i,\alpha_j\}=0\) damit die unterschiedlichen Terme \(p_ip_j = \delta_{ij}\) verschwinden
\item \(i=j\): \(\alpha_x^2p_x^2+\alpha_y^2p_y^2+\alpha_z^2p_z^2=\vec p^2 \Rightarrow \alpha_i^2 = 1\)

\begin{equation}
  \label{eq:7}
  \Rightarrow \boxed{\{\alpha_i,\alpha_j\}=2\delta_{ij}}
\end{equation}

\item \(\hat p_i,\hat H\) hermitesch \(\Rightarrow \vec\alpha,\beta\) hermitesch
\item \(\alpha_i^2=1,\beta^2=1 \Rightarrow \) Eigenwerte von \(\alpha_i,\beta\) sind \(\pm 1\)
\item \(\alpha_i\beta + \beta\alpha_i=0\qquad |\cdot \beta\)
\[\Rightarrow \alpha_i=-\beta\alpha_i\beta \Rightarrow Tr[\alpha_i] = -Tr[\beta\alpha_i\beta]=-Tr[\alpha_i\beta^2]=-Tr[\alpha_i]\]
\end{itemize}


Die Eigenwerte von \(\alpha_i\) und \(\beta\) wie oben schon erwähnt \(\pm 1\). D.h. wir haben für \(+1\) genau so viele Eigenwerte wie für \(-1\). Es kommen also nur Matritzen in den Dimensionen \(N=2,4,6...\) in Frage. Für den Fall \(N=2\) können die Paulimatritzen \(\sigma_i\) mit den Eigenwerten \(\pm 1\) benutzt werden. Sie erlauben jedoch nur 6 Antikommutator Beziehungen zu beschreiben. Benötigt werden aber 9, nämlich 6x~(\ref{eq:7}) und 3x (\ref{eq:8}). Welche durch 4 dimensionale Matritzen (4x4=16) ausreichend beschrieben werden.


\(\beta\) diagonal mit den Eigenwerten \(\pm 1\) also wähle 
\begin{equation}
  \label{eq:9}
  \beta= \begin{pmatrix}\mathbb 1&0\\ 0&-\mathbb 1\end{pmatrix} = \begin{pmatrix}1&0&0&0\\ 0&1&0&0\\ 0&0&-1&0\\0&0&0&-1\end{pmatrix}
\end{equation}

Um die \(\alpha_i\) zu bestimmen, nutzen wir, dass \(\alpha_i\) hermitesch ist und der Antikommutator (\ref{eq:8}) zwischen \(\alpha_i\) und \(\beta\) gleich Null ist. Allgemeiner Ansatz für \(\alpha\)


\begin{equation}
  \label{eq:10}
  \alpha_i =\begin{pmatrix}A&B\\ C&D\end{pmatrix}
\end{equation}

Mit der Beziehung dass die Spur von \(\alpha_i\) gleich Null seien muss

\begin{align}
   \label{eq:11}
   &\{\alpha_i,\beta\} =  \alpha_i\beta + \beta\alpha_i=0 \notag\\
   \Leftrightarrow &\alpha_i\beta =-\beta\alpha_i \quad |\cdot \beta\notag\\
   &\alpha_i\underbrace{\beta^2}_{=1}=-\beta\alpha_i\beta
\end{align}

Die Spur von (\ref{eq:11}) ergibt

\begin{align}
  \label{eq:12}
  &\text{Tr}(\alpha_i)=-\text{Tr}(\beta\alpha_i\beta)=-\text{Tr}(\alpha_i\beta^2)
  =-\text{Tr}(\alpha_i)\notag\\
  \Leftrightarrow&\text{Tr}\;\alpha_i = -\text{Tr}\;\alpha_i\notag\\ 
  \Rightarrow &\text{Tr}\;\alpha_i=0
\end{align}

Für hermitesche 2D Matrix gilt \(C=B^\dagger\)  und aus Spurenfreiheit folgt \(A=D=0\). 
\begin{equation}
  \label{eq:13}
  \Rightarrow \alpha_i =\begin{pmatrix}0&B_i\\ B^\dagger_i&0\end{pmatrix}
\end{equation}


Setze für \(B=\tau_i\). Aus (\ref{eq:7})


\begin{align}
  \label{eq:18}
  \{\alpha_i,\alpha_j\}&=\begin{pmatrix}0&B_i\\ B^\dagger_i&0\end{pmatrix}\begin{pmatrix}0&B_j\\ B^\dagger_j&0\end{pmatrix} + \begin{pmatrix}0&B_j\\ B^\dagger_j&0\end{pmatrix}\begin{pmatrix}0&B_i\\ B^\dagger_i&0\end{pmatrix}\\
&=  \begin{pmatrix}B_iB_j^\dagger&0\\ 0&B^\dagger_iB_j\end{pmatrix}+\begin{pmatrix}B_j B_i^\dagger &0\\ 0&B^\dagger_j B_i\end{pmatrix}   \\  
&=  \begin{pmatrix}B_iB_j^\dagger +B_j B_i^\dagger &0\\ 0&B^\dagger_iB_j+B^\dagger_j B_i\end{pmatrix}\\
\end{align}

Versuchen wir für \(B_i\) die Paulimatritzen einzusetzen


\begin{align}
  \label{eq:19}
  \{\alpha_i,\alpha_j\} &= \begin{pmatrix}\sigma_i\sigma_j^\dagger +\sigma_j \sigma_i^\dagger &0\\ 0&\sigma^\dagger_i\sigma_j+\sigma^\dagger_j \sigma_i\end{pmatrix} \qquad \text{ Paulimatritzen hermitesch } \sigma_i^\dagger = \sigma_i \\
&= \begin{pmatrix}\sigma_i\sigma_j +\sigma_j \sigma_i &0\\ 0&\sigma_i\sigma_j+\sigma_j \sigma_i\end{pmatrix} \qquad \text{mit } \{\sigma_i,\sigma_j\}=2\delta_{ij}\mathds 1 \\ 
&=  \begin{pmatrix}2\delta_{ij}\mathds 1 &0\\ 0&2\delta_{ij}\mathds 1 \end{pmatrix}\\
&=2\delta_{ij}\mathds 1
\end{align}

Damit erfüllen die Paulimatrizen die Antikommutator-Beziehung  \(\{\alpha_i,\alpha_j\} =2\delta_{ij} \). Somit können wir für \(B_i\) die Paulimatrizen nutzen. Zusammenfassend lässt sich schreiben

\begin{equation}
  \label{eq:15}
  \Rightarrow \boxed{ \beta= \begin{pmatrix}\mathbb 1&0\\ 0&-\mathbb 1\end{pmatrix};\qquad \alpha_i=\begin{pmatrix} 0&\sigma_i\\ \sigma_i&0\end{pmatrix} }
\end{equation}


Die Dirac Gleichung (\ref{eq:1}) lässt sich mit den Hamiltonoperator (\ref{eq:5}) und den gefundenen Matritzen \(\alpha_i\) und \(\beta\) nun wie folgt schreiben

\begin{equation}
  \label{eq:16}
  \boxed{i\hbar \frac{\partial}{\partial t}\psi(x) = c\left(\vec \alpha\cdot\frac{\hbar}{i}\vec \nabla + \beta mc\right)\psi(x)}
\end{equation}


Alternativ gibt es die Dirac-Gleichung in kovarianter Form, dazu bringen wir alle Terme auf eine Seite


\begin{align}
  \label{eq:20}
  &\left( i\hbar \frac{\partial}{\partial t} - c \vec \alpha\cdot\frac{\hbar}{i}\vec \nabla - \beta mc^2 \right)\psi(x) = 0 \qquad |\cdot \frac{\beta}{\hbar c} \notag \\
 &\left( \beta\frac{i}{c} \frac{\partial}{\partial t} - \vec\alpha\cdot\beta\frac{1}{i}\vec \nabla  - \underbr{\beta^2}_{\mathds 1} \frac{mc}{\hbar } \right)\psi(x) = 0 \notag\\
 &\left( i \underbr{ \beta}_{\gamma^0}  \underbr{ \frac{1}{c} \frac{\partial}{\partial t}}_{\partial_0} + i\underbr{ \beta \vec \alpha }_{\vec \gamma}\vec \nabla -  \frac{mc}{\hbar } \right)\psi(x) = 0 \notag\\
 &\left( i \underbr{(\gamma^0  \partial_0 + \vec \gamma \vec \nabla)}_{\gamma^\mu\partial_\mu} -  \frac{mc}{\hbar } \right)\psi(x) = 0 \notag\\
\end{align}
Damit erhalten wir die Dirac-Gleichung in kovarianter Form

\begin{equation}
  \label{eq:17}
  \boxed{\left(i\gamma^\mu\partial_\mu - \frac{mc}{\hbar}\right)\psi=0}
\end{equation}


Wobei die \(\gamma^\mu\) die Gamma-Matrizen sind für die gilt \(\gamma^0 = \beta\); \(\gamma^i = \beta\alpha_i\)




\subsection*{Referenzen}
\begin{itemize}
\item Wachter Relativistische Quantenmechanik
\item Schwabl Quantenmechanik für Fortgeschrittene
\item Rollnik Quantentheorie 2
\end{itemize}

\end{document}
