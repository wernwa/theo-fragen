\documentclass[10pt,a4paper,oneside,fleqn]{article}
\usepackage{geometry}
\geometry{a4paper,left=20mm,right=20mm,top=1cm,bottom=2cm}
\usepackage[utf8]{inputenc}
%\usepackage{ngerman}
\usepackage{amsmath}                % brauche ich um dir Formel zu umrahmen.
\usepackage{amsfonts}                % brauche ich für die Mengensymbole
\usepackage{graphicx}
\setlength{\parindent}{0px}
\setlength{\mathindent}{10mm}
\usepackage{bbold}                    %brauche ich für die doppel Zahlen Darstellung (Einheitsmatrix z.B)



\usepackage{color}
\usepackage{titlesec} %sudo apt-get install texlive-latex-extra

\definecolor{darkblue}{rgb}{0.1,0.1,0.55}
\definecolor{verydarkblue}{rgb}{0.1,0.1,0.35}
\definecolor{darkred}{rgb}{0.55,0.2,0.2}

%hyperref Link color
\usepackage[colorlinks=true,
        linkcolor=darkblue,
        citecolor=darkblue,
        filecolor=darkblue,
        pagecolor=darkblue,
        urlcolor=darkblue,
        bookmarks=true,
        bookmarksopen=true,
        bookmarksopenlevel=3,
        plainpages=false,
        pdfpagelabels=true]{hyperref}

\titleformat{\chapter}[display]{\color{darkred}\normalfont\huge\bfseries}{\chaptertitlename\
\thechapter}{20pt}{\Huge}

\titleformat{\section}{\color{darkblue}\normalfont\Large\bfseries}{\thesection}{1em}{}
\titleformat{\subsection}{\color{verydarkblue}\normalfont\large\bfseries}{\thesubsection}{1em}{}

% Notiz Box
\usepackage{fancybox}
\newcommand{\notiz}[1]{\vspace{5mm}\ovalbox{\begin{minipage}{1\textwidth}#1\end{minipage}}\vspace{5mm}}

\usepackage{cancel}
\setcounter{secnumdepth}{3}
\setcounter{tocdepth}{3}





%-------------------------------------------------------------------------------
%Diff-Makro:
%Das Diff-Makro stellt einen Differentialoperator da.
%
%Benutzung:
% \diff  ->  d
% \diff f  ->  df
% \diff^2 f  ->  d^2 f
% \diff_x  ->  d/dx
% \diff^2_x  ->  d^2/dx^2
% \diff f_x  ->  df/dx
% \diff^2 f_x  ->  d^2f/dx^2
% \diff^2{f(x^5)}_x  ->  d^2(f(x^5))/dx^2
%
%Ersetzt man \diff durch \pdiff, so wird der partieller
%Differentialoperator dargestellt.
%
\makeatletter
\def\diff@n^#1{\@ifnextchar{_}{\diff@n@d^#1}{\diff@n@fun^#1}}
\def\diff@n@d^#1_#2{\frac{\textrm{d}^#1}{\textrm{d}#2^#1}}
\def\diff@n@fun^#1#2{\@ifnextchar{_}{\diff@n@fun@d^#1#2}{\textrm{d}^#1#2}}
\def\diff@n@fun@d^#1#2_#3{\frac{\textrm{d}^#1 #2}{\textrm{d}#3^#1}}
\def\diff@one@d_#1{\frac{\textrm{d}}{\textrm{d}#1}}
\def\diff@one@fun#1{\@ifnextchar{_}{\diff@one@fun@d #1}{\textrm{d}#1}}
\def\diff@one@fun@d#1_#2{\frac{\textrm{d}#1}{\textrm{d}#2}}
\newcommand*{\diff}{\@ifnextchar{^}{\diff@n}
  {\@ifnextchar{_}{\diff@one@d}{\diff@one@fun}}}
%
%Partieller Diff-Operator.
\def\pdiff@n^#1{\@ifnextchar{_}{\pdiff@n@d^#1}{\pdiff@n@fun^#1}}
\def\pdiff@n@d^#1_#2{\frac{\partial^#1}{\partial#2^#1}}
\def\pdiff@n@fun^#1#2{\@ifnextchar{_}{\pdiff@n@fun@d^#1#2}{\partial^#1#2}}
\def\pdiff@n@fun@d^#1#2_#3{\frac{\partial^#1 #2}{\partial#3^#1}}
\def\pdiff@one@d_#1{\frac{\partial}{\partial #1}}
\def\pdiff@one@fun#1{\@ifnextchar{_}{\pdiff@one@fun@d #1}{\partial#1}}
\def\pdiff@one@fun@d#1_#2{\frac{\partial#1}{\partial#2}}
\newcommand*{\pdiff}{\@ifnextchar{^}{\pdiff@n}
  {\@ifnextchar{_}{\pdiff@one@d}{\pdiff@one@fun}}}
\makeatother
%
%Das gleich nur mit etwas andere Syntax. Die Potenz der Differentiation wird erst
%zum Schluss angegeben. Somit lautet die Syntax:
%
% \diff_x^2  ->  d^2/dx^2
% \diff f_x^2  ->  d^2f/dx^2
% \diff{f(x^5)}_x^2  ->  d^2(f(x^5))/dx^2
% Ansonsten wie Oben.
%
%Ersetzt man \diff durch \pdiff, so wird der partieller
%Differentialoperator dargestellt.
%
%\makeatletter
%\def\diff@#1{\@ifnextchar{_}{\diff@fun#1}{\textrm{d} #1}}
%\def\diff@one_#1{\@ifnextchar{^}{\diff@n{#1}}%
%  {\frac{\textrm d}{\textrm{d} #1}}}
%\def\diff@fun#1_#2{\@ifnextchar{^}{\diff@fun@n#1_#2}%
%  {\frac{\textrm d #1}{\textrm{d} #2}}}
%\def\diff@n#1^#2{\frac{\textrm d^#2}{\textrm{d}#1^#2}}
%\def\diff@fun@n#1_#2^#3{\frac{\textrm d^#3 #1}%
%  {\textrm{d}#2^#3}}
%\def\diff{\@ifnextchar{_}{\diff@one}{\diff@}}
%\newcommand*{\diff}{\@ifnextchar{_}{\diff@one}{\diff@}}
%
%Partieller Diff-Operator.
%\def\pdiff@#1{\@ifnextchar{_}{\pdiff@fun#1}{\partial #1}}
%\def\pdiff@one_#1{\@ifnextchar{^}{\pdiff@n{#1}}%
%  {\frac{\partial}{\partial #1}}}
%\def\pdiff@fun#1_#2{\@ifnextchar{^}{\pdiff@fun@n#1_#2}%
%  {\frac{\partial #1}{\partial #2}}}
%\def\pdiff@n#1^#2{\frac{\partial^#2}{\partial #1^#2}}
%\def\pdiff@fun@n#1_#2^#3{\frac{\partial^#3 #1}%
%  {\partial #2^#3}}
%\newcommand*{\pdiff}{\@ifnextchar{_}{\pdiff@one}{\pdiff@}}
%\makeatother

%-------------------------------------------------------------------------------
%%Nützliche Makros um in der Quantenmechanik Bras, Kets und das Skalarprodukt
%%zwischen den beiden darzustellen.
%%Benutzung:
%% \bra{x}  ->    < x |
%% \ket{x}  ->    | x >
%% \braket{x}{y} ->   < x | y >

\newcommand\bra[1]{\left\langle #1 \right|}
\newcommand\ket[1]{\left| #1 \right\rangle}
\newcommand\braket[2]{%
  \left\langle #1\vphantom{#2} \right.%
  \left|\vphantom{#1#2}\right.%
  \left. \vphantom{#1}#2 \right\rangle}%

%-------------------------------------------------------------------------------
%%Aus dem Buch:
%%Titel:  Latex in Naturwissenschaften und Mathematik
%%Autor:  Herbert Voß
%%Verlag: Franzis Verlag, 2006
%%ISBN:   3772374190, 9783772374197
%%
%%Hier werden drei Makros definiert:\mathllap, \mathclap und \mathrlap, welche
%%analog zu den aus Latex bekannten \rlap und \llap arbeiten, d.h. selbst
%%keinerlei horizontalen Platz benötigen, aber dennoch zentriert zum aktuellen
%%Punkt erscheinen.

\newcommand*\mathllap{\mathstrut\mathpalette\mathllapinternal}
\newcommand*\mathllapinternal[2]{\llap{$\mathsurround=0pt#1{#2}$}}
\newcommand*\clap[1]{\hbox to 0pt{\hss#1\hss}}
\newcommand*\mathclap{\mathpalette\mathclapinternal}
\newcommand*\mathclapinternal[2]{\clap{$\mathsurround=0pt#1{#2}$}}
\newcommand*\mathrlap{\mathpalette\mathrlapinternal}
\newcommand*\mathrlapinternal[2]{\rlap{$\mathsurround=0pt#1{#2}$}}

%%Das Gleiche nur mit \def statt \newcommand.
%\def\mathllap{\mathpalette\mathllapinternal}
%\def\mathllapinternal#1#2{%
%  \llap{$\mathsurround=0pt#1{#2}$}% $
%}
%\def\clap#1{\hbox to 0pt{\hss#1\hss}}
%\def\mathclap{\mathpalette\mathclapinternal}
%\def\mathclapinternal#1#2{%
%  \clap{$\mathsurround=0pt#1{#2}$}%
%}
%\def\mathrlap{\mathpalette\mathrlapinternal}
%\def\mathrlapinternal#1#2{%
%  \rlap{$\mathsurround=0pt#1{#2}$}% $
%}

%-------------------------------------------------------------------------------
%%Hier werden zwei neue Makros definiert \overbr und \underbr welche analog zu
%%\overbrace und \underbrace funktionieren jedoch die Gleichung nicht
%%'zerreißen'. Dies wird ermöglicht durch das \mathclap Makro.

\def\overbr#1^#2{\overbrace{#1}^{\mathclap{#2}}}
\def\underbr#1_#2{\underbrace{#1}_{\mathclap{#2}}}
 



\begin{document}

\textit{29. März 2012}
\input{../headers/authors.tex}

\section*{Die Dirac Gleichung}

 Die Möglichkeit \(\rho=\psi^*\psi\) für die Wahrscheinlichkeitsdichte zu schreiben folgt aus der Tatsache, dass die Zeitableitung in der nicht relativistische Schrödinger-Gleichung nur in 1-Ordnung auftritt. Im Vergleich nimmt \(\rho\) in der Klein-Gordon Gleichung auch negative Werte an, dort ist die Zeitlichte Ableitung von 2-Ordnung.\\
\\
Es gilt nun eine Differential Gleichung 1-Ordung in der Zeit der Form

\begin{align}
  \label{eq:1}
  i\hbar \pdiff_t \psi = H \psi
\end{align}

die relativistische Enegie-Impuls-Beziehung

\begin{align}
  \label{eq:2}
  p_\mu p^\mu = \frac{E^2}{c^2} - \vec p^2 = (mc)^2
\end{align}

für ein freies Teilchen erfüllt. Die Betrachtungen im Zusammenhang mit der Klein-Gordon-Gleichung haben gezeigt, dass man dieses Problem für eine einfache skalare Wellenfunktion \(\psi\) nicht lösen kann. Diracs Idee war eine mehrdimensionale Wellenfunktion einzuführen
.
\begin{align}
  \label{eq:3}
  \psi(x) = \begin{pmatrix}\psi_1(x)\\\psi_2(x)\\\vdots\\ \psi_n(x)\\ \end{pmatrix}
\end{align}

Des weiteren muss die Gleichung \eqref{eq:1} folgende Forderungen erfüllen:

\begin{itemize}
\item Die Komponenten von \(\psi\) müssen die Klein-Gordon-Gleichung erfüllen, so dass der Hamilton-Operator die relativistische Beziehung \(E^2 = p^2c^2+m^2c^4\) erfüllt.
\item Es existiert ein erhaltener Viererstrom, dessen nullte Komponente eine positive Dichte ist.
\item Die Gleichung muss Lorentz-kovariant sein. Das bedeutet, dass sie unter Transformation ihre Form behält. Bezugssystem unabhängig. Damit dies erfüllt ist muss gelten: Da die zeitliche Ableitung nur in 1-Ordnung auftritt, muss auch die räumliche Ableitung in 1-Ordnung auftreten. 
\end{itemize}

Der Ansatz für den Hamiltonoperator \(H\) sollte so sein dass

\begin{align}
  \label{eq:4}
  H^2 = E^2 = p^2c^2+ m^2c^4
\end{align}

das Quadrat der relativistischen Energie im Quadrat gleich ist. Folgender allgemeiner Ansatz erfüllt diese Bedingung

\begin{align}
  \label{eq:5}
  H = c (\alpha_x p_x + \alpha_y p_y + \alpha_z p_z) + \beta mc^2 = c\sum_{i=1}^3\alpha_i p_i   + \beta mc^2
\end{align}
Die unbekannten Koeffizienten \(\alpha_i,\beta\) können nicht einfach Zahlen sein, da sonst die Gleichung nicht einmal forminvariant gegenüber räumlichen Drehungen ist (D.h. die Form der Gleichung ändert sich je nachdem wie man das Koordinatensystem wählt). \(\alpha_i,\beta\) müssen hermitische Matritzen sein damit \(H\) hermitesch ist. Daraus folgt  \(\alpha_i,\beta\) müssen \(N\times N\) Matrizen sein.

Um die unbekannten Matritzen  \(\alpha_i,\beta\) zu bestimmen gehen wir von der Klein-Gordon Gleichung aus

\begin{align}
  \label{eq:6}
  -\hbar^2 \pdiff^2_t \psi(x) &= \underbrace{c^2(\vec p^2 + m^2c^2)}_{H^2}\psi(x)\\
 &\stackrel{!}= \left(  c\sum_{i=1}^3\alpha_i p_i   + \beta mc^2 \right)^2\psi(x)\\
&=[c \sum_{j=1}^3 \alpha_jp_j+\beta mc] \cdot [c \sum_{i=1}^3 \alpha_ip_i+\beta mc]\psi(p) \\
&=c^2 [ \sum_{j=1}^3 \alpha_jp_j+\beta mc] \cdot [ \sum_{i=1}^3 \alpha_ip_i+\beta mc]\psi(p) \\
&=c^2 \left( \sum_{j=1}^3 \alpha_jp_j\sum_{i=1}^3 \alpha_ip_i  +   \sum_{j=1}^3 \alpha_jp_j \beta mc  + \beta mc \sum_{i=1}^3 \alpha_ip_i+ \beta^2 m^2c^2\right)\psi(p) \\
&= c^2\left(\sum_{i,j=1}^3\alpha_i\alpha_j p_ip_j+\sum_{i=1}^3(\alpha_i\beta+\beta\alpha_i)p_i mc +  \beta^2m^2c^2\right)\psi(p)
\end{align}

Der Koeffizientenverleich zwischen \(c^2(\vec p^2 + m^2c^2)\) und \( c^2\left(\sum_{i,j=1}^3\alpha_i\alpha_j p_ip_j+\sum_{i=1}^3(\alpha_i\beta+\beta\alpha_i)p_i mc +  \beta^2m^2c^2\right)\) liefert


\begin{itemize}
\item  \(\boxed{\beta^2=1}\)
\item Antikommutator: 
  \begin{equation}
    \label{eq:8}
    \boxed{\{\alpha_i,\beta\}=0}
  \end{equation}
Damit der Mischterm \((\alpha_i\beta+\beta\alpha_i)p_i mc\)  verschwindet 

\item \(i\neq j\): z.B:  \(\{\alpha_i,\alpha_j\}=0\) damit die unterschiedlichen Terme \(p_ip_j = \delta_{ij}\) verschwinden
\item \(i=j\): \(\alpha_x^2p_x^2+\alpha_y^2p_y^2+\alpha_z^2p_z^2=\vec p^2 \Rightarrow \alpha_i^2 = 1\)

\begin{equation}
  \label{eq:7}
  \Rightarrow \boxed{\{\alpha_i,\alpha_j\}=2\delta_{ij}}
\end{equation}

\item \(\hat p_i,\hat H\) hermitesch \(\Rightarrow \vec\alpha,\beta\) hermitesch
\item \(\alpha_i^2=1,\beta^2=1 \Rightarrow \) Eigenwerte von \(\alpha_i,\beta\) sind \(\pm 1\)
\item \(\alpha_i\beta + \beta\alpha_i=0\qquad |\cdot \beta\)
\[\Rightarrow \alpha_i=-\beta\alpha_i\beta \Rightarrow Tr[\alpha_i] = -Tr[\beta\alpha_i\beta]=-Tr[\alpha_i\beta^2]=-Tr[\alpha_i]\]
\end{itemize}


Die Eigenwerte von \(\alpha_i\) und \(\beta\) wie oben schon erwähnt \(\pm 1\). D.h. wir haben für \(+1\) genau so viele Eigenwerte wie für \(-1\). Es kommen also nur Matritzen in den Dimensionen \(N=2,4,6...\) in Frage. Für den Fall \(N=2\) können die Paulimatritzen \(\sigma_i\) mit den Eigenwerten \(\pm 1\) benutzt werden. Sie erlauben jedoch nur 6 Antikommutator Beziehungen zu beschreiben. Benötigt werden aber 9, nämlich 6x~(\ref{eq:7}) und 3x (\ref{eq:8}). Welche durch 4 dimensionale Matritzen (4x4=16) ausreichend beschrieben werden.


\(\beta\) diagonal mit den Eigenwerten \(\pm 1\) also wähle 
\begin{equation}
  \label{eq:9}
  \beta= \begin{pmatrix}\mathbb 1_2&0\\ 0&-\mathbb 1_2\end{pmatrix} = \begin{pmatrix}1&0&0&0\\ 0&1&0&0\\ 0&0&-1&0\\0&0&0&-1\end{pmatrix}
\end{equation}

Um die \(\alpha_i\) zu bestimmen, nutzen wir, dass \(\alpha_i\) hermitesch ist und der Antikommutator (\ref{eq:8}) zwischen \(\alpha_i\) und \(\beta\) gleich Null ist. Allgemeiner Ansatz für \(\alpha\)


\begin{equation}
  \label{eq:10}
  \alpha_i =\begin{pmatrix}A&B\\ C&D\end{pmatrix}
\end{equation}

Mit der Beziehung dass die Spur von \(\alpha_i\) gleich Null seien muss

\begin{align}
   \label{eq:11}
   &\{\alpha_i,\beta\} =  \alpha_i\beta + \beta\alpha_i=0 \notag\\
   \Leftrightarrow &\alpha_i\beta =-\beta\alpha_i \quad |\cdot \beta\notag\\
   &\alpha_i\underbrace{\beta^2}_{=1}=-\beta\alpha_i\beta
\end{align}

Die Spur von (\ref{eq:11}) ergibt

\begin{align}
  \label{eq:12}
  &\text{Tr}(\alpha_i)=-\text{Tr}(\beta\alpha_i\beta)=-\text{Tr}(\alpha_i\beta^2)
  =-\text{Tr}(\alpha_i)\notag\\
  \Leftrightarrow&\text{Tr}\;\alpha_i = -\text{Tr}\;\alpha_i\notag\\ 
  \Rightarrow &\text{Tr}\;\alpha_i=0
\end{align}

Für hermitesche 2D Matrix gilt \(C=B^\dagger\)  und aus Spurenfreiheit folgt \(A=D=0\). 
\begin{equation}
  \label{eq:13}
  \Rightarrow \alpha_i =\begin{pmatrix}0&B_i\\ B^\dagger_i&0\end{pmatrix}
\end{equation}


Setze für \(B=\tau_i\). Aus (\ref{eq:7})


\begin{align}
  \label{eq:18}
  \{\alpha_i,\alpha_j\}&=\begin{pmatrix}0&B_i\\ B^\dagger_i&0\end{pmatrix}\begin{pmatrix}0&B_j\\ B^\dagger_j&0\end{pmatrix} + \begin{pmatrix}0&B_j\\ B^\dagger_j&0\end{pmatrix}\begin{pmatrix}0&B_i\\ B^\dagger_i&0\end{pmatrix}\\
&=  \begin{pmatrix}B_iB_j^\dagger&0\\ 0&B^\dagger_iB_j\end{pmatrix}+\begin{pmatrix}B_j B_i^\dagger &0\\ 0&B^\dagger_j B_i\end{pmatrix}   \\  
&=  \begin{pmatrix}B_iB_j^\dagger +B_j B_i^\dagger &0\\ 0&B^\dagger_iB_j+B^\dagger_j B_i\end{pmatrix}\\
\end{align}

Versuchen wir für \(B_i\) die Paulimatritzen einzusetzen


\begin{align}
  \label{eq:19}
  \{\alpha_i,\alpha_j\} &= \begin{pmatrix}\sigma_i\sigma_j^\dagger +\sigma_j \sigma_i^\dagger &0\\ 0&\sigma^\dagger_i\sigma_j+\sigma^\dagger_j \sigma_i\end{pmatrix} \qquad \text{ Paulimatritzen hermitesch } \sigma_i^\dagger = \sigma_i \\
&= \begin{pmatrix}\sigma_i\sigma_j +\sigma_j \sigma_i &0\\ 0&\sigma_i\sigma_j+\sigma_j \sigma_i\end{pmatrix} \qquad \text{mit } \{\sigma_i,\sigma_j\}=2\delta_{ij}\mathds 1_2 \\ 
&=  \begin{pmatrix}2\delta_{ij}\mathds 1_2 &0\\ 0&2\delta_{ij}\mathds 1_2 \end{pmatrix}\\
&=2\delta_{ij}\mathds 1_4
\end{align}

Damit erfüllen die Paulimatrizen die Antikommutator-Beziehung  \(\{\alpha_i,\alpha_j\} =2\delta_{ij} \). Somit können wir für \(B_i\) die Paulimatrizen nutzen. Zusammenfassend lässt sich schreiben

\begin{equation}
  \label{eq:15}
  \Rightarrow \boxed{ \beta= \begin{pmatrix}\mathbb 1_2&0\\ 0&-\mathbb 1_2\end{pmatrix};\qquad \alpha_i=\begin{pmatrix} 0&\sigma_i\\ \sigma_i&0\end{pmatrix} }
\end{equation}


Die Dirac Gleichung (\ref{eq:1}) lässt sich mit den Hamiltonoperator (\ref{eq:5}) und den gefundenen Matritzen \(\alpha_i\) und \(\beta\) nun wie folgt schreiben

\begin{equation}
  \label{eq:16}
  \boxed{i\hbar \frac{\partial}{\partial t}\psi(x) = c\left(\vec \alpha\cdot\frac{\hbar}{i}\vec \nabla + \beta mc\right)\psi(x)}
\end{equation}


Alternativ gibt es die Dirac-Gleichung in kovarianter Form, dazu bringen wir alle Terme auf eine Seite


\begin{align}
  \label{eq:20}
 \frac{\beta}{\hbar c} \cdot| \qquad  &\left( i\hbar \frac{\partial}{\partial t} - c \vec \alpha\cdot\frac{\hbar}{i}\vec \nabla - \beta mc^2 \right)\psi(x) = 0  \notag \\
 &\left( \beta\frac{i}{c} \frac{\partial}{\partial t} - \beta\vec\alpha\frac{1}{i}\vec \nabla  - \underbr{\beta^2}_{\mathds 1} \frac{mc}{\hbar } \right)\psi(x) = 0 \notag\\
 &\left( i \underbr{ \beta}_{\gamma^0}  \underbr{ \frac{1}{c} \frac{\partial}{\partial t}}_{\partial_0} + i\underbr{ \beta \vec \alpha }_{\vec \gamma}\vec \nabla -  \frac{mc}{\hbar } \right)\psi(x) = 0 \notag\\
 &\left( i \underbr{(\gamma^0  \partial_0 + \vec \gamma \vec \nabla)}_{\gamma^\mu\partial_\mu} -  \frac{mc}{\hbar } \right)\psi(x) = 0 \notag\\
\end{align}
Damit erhalten wir die Dirac-Gleichung in kovarianter Form

\begin{equation}
  \label{eq:17}
  \boxed{\left(i\gamma^\mu\partial_\mu - \frac{mc}{\hbar}\right)\psi=0}
\end{equation}


Wobei die \(\gamma^\mu\) die Gamma-Matrizen sind für die gilt

\begin{equation}
  \label{eq:24}
  \boxed{
  \begin{aligned}
    \gamma^0 &= \beta \\
    \gamma^i &= \beta\alpha_i \\
    \{\gamma^\mu, \gamma^\nu\}&=2g^{\mu\nu}\\
    \{\gamma^i, \gamma^j\}&=-2\delta_{ij}
  \end{aligned}}
\end{equation}

Mit
\begin{equation}
  \label{eq:25}
  \{\gamma^i, \gamma^j\}=\beta\underbr{\alpha_i\beta}_{-\beta\alpha_i}\alpha_j
  +\beta\underbr{\alpha_j\beta}_{-\beta\alpha_j}\alpha_i=-\{\alpha_i, \alpha_j\}=-2\delta_{ij}
\end{equation}




\subsection*{Kontinuitätsgleichung aus der Dirac-Gleichung}

Das Problem der Klein-Gordon Gleichung war, dass die Wahrscheinlichkeitsdichte negativ werden konnte. Das hat Dirac dazu veranlasst eine Gleichung zu finden, die einen Positiven Wahrscheinlichkeitsstrom liefert. Dies wollen wir näher untersuchen, indem wir die Kontinuitätsgleichung aufstellen. Wir gehen von der Gleichung (\ref{eq:16}) aus und multiplizieren sie von der linken Seite mit \(\psi^\dagger\)


\begin{equation}
  \label{eq:26}
  \psi^\dagger\cdot|\qquad i\hbar \frac{\partial \psi}{\partial t} = \frac{\hbar c}{i}\vec \alpha\cdot\vec\nabla\psi+\beta mc^2\psi
\end{equation}

Und analog mit \(\psi\) die adjungierte Dirac Gleichung von der rechten Seite

\begin{equation}
  \label{eq:14}
  -i\hbar \frac{\partial \psi^\dagger}{\partial t} = -\frac{\hbar c}{i}(\vec\nabla\psi^\dagger)\vec \alpha+\beta mc^2\psi^\dagger \qquad |\cdot \psi
\end{equation}


Differenz der beiden Gleichungen \(\ref{eq:26}-\ref{eq:14}\):

\begin{equation}
  \label{eq:21}
  \underbrace{ i\hbar(\frac{\partial}{\partial t} \psi^\dagger)\psi+i\hbar\psi^\dagger (\frac{\partial}{\partial t} \psi)}_{\text{Produktregel} \quad = i\hbar\frac{\partial}{\partial t} (\psi^\dagger\psi) } =\underbrace{\frac{\hbar c}{i}(\psi^\dagger \vec \alpha\cdot\vec\nabla\psi+(\vec\nabla\psi^\dagger)\vec\alpha \psi)}_{\text{Produktregel} \quad = -c\vec\nabla(\psi^\dagger\vec\alpha\psi)}
\end{equation}


\[\Rightarrow \frac{\partial}{\partial t}(\psi^\dagger\psi) = -c\vec\nabla(\psi^\dagger\vec\alpha\psi)\]

Damit sieht die Kontinuitätsgleichung wie folgt aus


\begin{equation}
  \label{eq:22}
  \boxed{ \frac{\partial}{\partial t}\underbrace{(\psi^\dagger\psi) }_{\rho}+\vec\nabla\cdot(\underbrace{c\psi^\dagger\vec\alpha\psi }_{\vec j})=0 }
\end{equation}

Mit der positiven Wahrscheinlichkeitsdichte


\begin{equation}
  \label{eq:23}
  \rho =\psi^\dagger\psi = \sum_i|\psi_i|^2\geq 0
\end{equation}

Und dem Warhscheinlichkeits-Strom in kovarianter Form

\begin{align}
j^\mu &= (c\rho ,\vec j )\\
&= (c\underbrace{\psi^\dagger\psi}_{\equiv \rho} ,c\psi^\dagger\beta\beta\vec\alpha\psi )\\
&= (c\psi^\dagger\underbrace{\beta\gamma^0}_{\mathbb 1}\psi ,c\psi^\dagger\beta\vec\gamma\psi )\\
&= c\psi^\dagger \beta\gamma^\mu \psi\\
&= c\overline \psi\gamma^\mu \psi
\end{align}

wobei \(\overline \psi = \psi^\dagger\beta=\psi^\dagger\gamma^0\) der Pauli adungierte Spinor ist.


\subsection*{16 Bilinearformen}

Die vier Matirzen \(\gamma^\nu\) sind linear unabhängige Matzritzen, bilden aber keine vollständige Basis in dem 4x4=16 dimensionalen Raum. Durch einfache Matrizxmultiplikation der \(\gamma\) Matritzen  ist es möglich alle linear unabhängige Basis Elemente des zugehörigen 16-dimensionalen Raumes zu finden so dass man folgende Basis erhält \(\Gamma = \{\Gamma^S, \Gamma^V, \Gamma^T, \Gamma^A, \Gamma^P\}\) (mit S:Skalar, V:Vektor, T:Tensor, A:Axialvektor, P:Pseudoskalar, wobei die Bezeichnung vom Transformationsverhalten unter Lorenz-Transformation herrührt). 

\begin{table}[h]
  \centering
 
\begin{tabular}{ccccc}
 \(\Gamma^{(n)}\)        &Bezeichnung& Anzahl    & \((\Gamma^{(n)})^2=+ 1\)&\((\Gamma^{(n)})^2=- 1\)\\
\hline 
 \(\Gamma^{S}\)&\(\mathds 1_4\)&1          &\(\mathds 1_4\)& \\
  \(\Gamma^{V}\)&\(\gamma^\mu\) &4          & \(\gamma^0\)& \(\gamma^1,\gamma^2,\gamma^3\) \\
 \(\Gamma^{T}\)&\(\gamma^\mu\gamma^\nu\) mit \(\mu<\nu\)  &6          & \(\gamma^0\gamma^1,\gamma^0\gamma^2,\gamma^0\gamma^3\)& \(\gamma^1\gamma^2,\gamma^1\gamma^3,\gamma^2\gamma^3\) \\
 \(\Gamma^{A}\)&\(\gamma^\mu\gamma^5\)   &4          & \(i\gamma^0\gamma^2\gamma^3,i\gamma^0,\gamma^3\gamma^1,i\gamma^0,\gamma^1\gamma^2 \)& \(i\gamma^1\gamma^2\gamma^3\) \\
\(\Gamma^{P}\)&\(\gamma^5\)   &1          & \(i\gamma^0\gamma^1\gamma^2\gamma^3 \)&  \\
\end{tabular}
\caption{Basiselemente des 16-dimensionalen Raumes}
\end{table}

\textbf{Bilineare Form} Als Bilinearform bezeichnet man in der linearen Algebra eine Funktion, welche zwei Vektoren einen Skalarwert zuordnet und die linear in ihren beiden Argumenten ist.

\begin{itemize}
\item Skalare Bilinearform, ist gegeben durch \(\bar \psi\mathds 1\psi\) (mit \(\bar \psi = \psi^\dagger\beta = \psi^\dagger\gamma^0\)) und transformiert sich wie ein Lorenz-Skalar.
\item Vektorielle Bilinearform, ist durch \(\bar \psi\gamma^\mu\psi \) gegeben, und transformiert sich wie ein kontravarianter Vektor.
\item Tensorielle Bilinearform, hat die Form \(\bar \psi\gamma^\mu\gamma^\nu \psi \), und transformiert sich wie ein kontravarianter Tensor 2-er Stufe.
\item Pseudoskalare Bilinearform, kommt als \(\bar \psi\gamma^5 \psi \) daher, und transformiert sich wie ein skalar mit dem zusätzlichen Faktor (siehe wiki Pseudoskalar).
\item Axialvektorielle Bilinearform, ist gegeben durch \( \bar \psi\gamma^\mu\gamma^5\psi \), und transformiert sich wie ein Vektor bis auf ein zusätzlichen Faktor (Pseudoskalar).
\end{itemize}



\subsection*{Referenzen}
\begin{itemize}
\item Wachter Relativistische Quantenmechanik
\item Schwabl Quantenmechanik für Fortgeschrittene
\item Rollnik Quantentheorie 2
\item http://de.wikipedia.org/wiki/Bilinearform
\end{itemize}


\end{document}
