\input{../headers/header_script.tex}


\usepackage{amsmath} 



\begin{document}

\section*{Streuung}

Wir betrachten die einfallende Stromdichte \(J_{\text{ein}}\) (Anzahl der Teilchen pro Zeiteinheit und Fläche), das an einem Punktteilchen mit einem Zentralpotential, gestreut wird. Ein Detektor misst eine Telmenge des gestreuten Teilchenstroms \(J_{\text{gestr}}\) im dem Raumwinkel \(\theta\) und Abstand \(r\). Die Ströme sind offenbar proportional zu einander:


\begin{equation}
  \label{eq:1}
  J_{\text{gestr}} \propto J_{\text{ein}}
\end{equation}


und die Proportionalitätskonstante \(\frac{d\sigma}{dA}\) bestimmt die Menge der Teilchen pro Zeit die am Detektor ankommen. 

\begin{align}
  \label{eq:2}
  J_{\text{gestr}} &= \frac{d\sigma}{dA} J_{\text{ein}}\\
\Leftrightarrow \frac{d\sigma}{dA} &= \frac{ J_{\text{gestr}} }{J_{\text{ein}}} \\
\frac{d\sigma}{d\Omega r^2} &= \frac{ J_{\text{gestr}} }{J_{\text{ein}}} \\
\end{align}
Diese Größe ist noch vom Abstand \(r^2\) Abhängig. Multipliziert man die Gleichung (\ref{eq:2}) mit \(r^2\) so erhält man den differenziellen Wirkungsquerschnitt:
\begin{equation}
  \label{eq:3}
  \boxed{ \frac{d\sigma}{d\Omega} = \frac{ J_{\text{gestr}}r^2 }{J_{\text{ein}}} }
\end{equation}

Um den differenziellen Wirkungsquerschnitt bestimmen zu können müssen wir als nächstes die Zustandsfunktion der einlafenden und  gestreuten Teilchen herleiten. Betrachte das Einfallende Teilchen als ebene Wellen Pakete. 




\subsection*{Referenzen}
\begin{itemize}
\item Claude Cohen-Tannoudji Quantenmechanik Band 2
\item Zettili Quanten Mehanics
\item Rollnik Quantentheorie 2
\end{itemize}

\end{document}
