\input{../headers/header_script.tex}
\usepackage{amsmath} 



\begin{document}

\section*{Loretz-Transformation von Vierervektoren}


Wir Betrachten einen linearen vierdimensionalen Vektoren, den sogenannten  \textbf{Minkowski-Raum}. Er besteht aus 4 komponentigen Koordinatenvektoren bzw. Vierervektoren

\begin{align}
  \label{eq:1}
  x^\mu = \begin{pmatrix}x^0\\x^1\\ x^2\\x^3 \end{pmatrix} = \begin{pmatrix}ct \\\vec x \end{pmatrix}, \qquad x^0=ct
\end{align}

bzw. in kovarianter Schreibweise

\begin{align}
  \label{eq:2}
  x_\mu = (x^0,-x^1,-x^2,-x^3) = (ct,-\vec x)
\end{align}

Um zwischen den ko- und kontra-varianten Vektoren zu wechseln, benötigt man den metrischen Tensor

\begin{align}
  \label{eq:3}
  g^{\mu\nu} = g_{\mu\nu} = \begin{pmatrix} 1&0&0&0 \\  0&-1&0&0 \\ 0&0&-1&0 \\ 0&0&0&-1  \end{pmatrix}
\end{align}


Damit gilt

\begin{align}
  \label{eq:4}
  x_{\mu} = g_{\mu\nu}x^\nu, \qquad x^\nu = g^{\nu\mu}x_\mu
\end{align}

weitere wichtige Relation

\begin{align}
  \label{eq:5}
  x^\nu = g^{\nu\mu}\underbr{x_\mu}_{\eqref{eq:4}} = g^{\nu\mu}  g_{\mu\alpha}x^\alpha = g^\nu_{\,\,\alpha}x^{\alpha}
\end{align}

Aus der Gleichung (\ref{eq:5}) folgt

\begin{align}
  \label{eq:6}
  g^\nu_{\,\,\alpha} = \begin{Bmatrix} 1,& \nu = \alpha\\ 0,&\nu\neq\alpha \end{Bmatrix} = \delta^\nu_{\alpha}
\end{align}




\(x' = \Lambda x\)  mit \(x^{'\mu}=\Lambda^\mu_{\,\, \nu}x^\nu\)

Bsp: Boost in z-Richtung: \(z' = \gamma(z-vt)\), \(t' = \gamma(t-\frac{v}{c^2}z)\), \(x' = x\), \(y'=y\) mit \(\gamma = \frac{1}{\sqrt{1-\frac{v^2}{c^2}}}\)

Lorenztranformation erhält relative Länge:

\[x'\cdot x' = g_{\mu\nu}x^{'\mu}x^{'\nu} = \underbrace{\Lambda^\mu_{\,\, \rho}  \Lambda^\nu_{\,\, \sigma}g_{\mu\nu}}_{g_{\rho\sigma}}x^\rho x^\sigma = x\cdot x = x^\rho x^\sigma g_{\sigma\rho}\]

Def. Eigenschaft einer Lorenztransformation

\[\Lambda_\mu^{\,\, \rho}\Lambda^\mu_{\,\,\sigma} = g^\rho_{\,\,\sigma} = \delta^\rho_{\,\, \sigma}  \]

oder \((\Lambda^{-1})^\rho_{\,\,\mu} = \Lambda_\mu^{\,\,\rho}\)

\(\Rightarrow det\Lambda = \pm 1\) (Verallgemeinerung von orthogonalen Transformation)








\subsection*{Referenzen}
\begin{itemize}
\item Claude Cohen-Tannoudji Quantenmechanik Band 2
\item Zettili Quanten Mehanics
\item Rollnik Quantentheorie 2
\end{itemize}

\end{document}
