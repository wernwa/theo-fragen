\input{../headers/header_script.tex}
\usepackage{amsmath} 



\begin{document}

\section*{Loretz-Transformation von Vierervektoren}


Wir Betrachten einen linearen vierdimensionalen Vektoren, den sogenannten  \textbf{Minkowski-Raum}. Er besteht aus 4 komponentigen Koordinatenvektoren bzw. Vierervektoren

\begin{align}
  \label{eq:1}
  x^\mu = \begin{pmatrix}x^0\\x^1\\ x^2\\x^3 \end{pmatrix} = \begin{pmatrix}ct \\\vec x \end{pmatrix}, \qquad x^0=ct
\end{align}

bzw. in kovarianter Schreibweise

\begin{align}
  \label{eq:2}
  x_\mu = (x^0,-x^1,-x^2,-x^3) = (ct,-\vec x)
\end{align}

Aus dem Viererortsvektor lässt sich mit Hilfe des Eigenzeitdifferentials

\begin{align}
  \label{eq:7}
  d\tau = dt\sqrt{1-\frac{1}{c^2}\left(\frac{dx}{dt}\right)^2}
\end{align}

herleiten. Die Vierergeschwindigkeit \(u^\mu\) als Ableitung vom Ort nach der Eigenzeit

\begin{align}
  \label{eq:8}
  u^\mu = \frac{dx^\mu}{d\tau} = \frac{dt}{d\tau}\frac{dx^\mu}{dt}=\frac{1}{\sqrt{1-\frac{v^2}{c^2}}}\begin{pmatrix}c\\\vec v \end{pmatrix}
\end{align}

der Viererimpuls \(p^\mu\) aus dem Produkt aus der Ruhemasse \(m_0\) und der Vierergeschwindigkeit

\begin{align}
  \label{eq:9}
  p^\mu = m_0u^\mu = \frac{m_0}{\sqrt{1-\frac{v^2}{c^2}}}\begin{pmatrix}c\\\vec v \end{pmatrix} =\begin{pmatrix}mc\\\vec p \end{pmatrix}\qquad \text{mit }m= \frac{m_0}{\sqrt{1-\frac{v^2}{c^2}}}
\end{align}

Sowie der Viererkraft \(F^\mu\) als Ableitung des Viererimpuls nach der Eigenzeit

\begin{align}
  \label{eq:10}
  F^\mu = \frac{dp^\mu}{d\tau} = \frac{dt}{d\tau}\frac{dp^\mu}{dt} = \frac{1}{\sqrt{1-\frac{v^2}{c^2}}}\frac{dp^\mu}{dt} = \begin{pmatrix}c \frac{dm}{dt}\\\vec F \end{pmatrix}.
\end{align}


Um zwischen den ko- und kontra-varianten Vektoren zu wechseln, benötigt man den metrischen Tensor

\begin{align}
  \label{eq:3}
  g^{\mu\nu} = g_{\mu\nu} = \begin{pmatrix} 1&0&0&0 \\  0&-1&0&0 \\ 0&0&-1&0 \\ 0&0&0&-1  \end{pmatrix}
\end{align}


Damit gilt

\begin{align}
  \label{eq:4}
  x_{\mu} = g_{\mu\nu}x^\nu, \qquad x^\nu = g^{\nu\mu}x_\mu
\end{align}

weitere wichtige Relation

\begin{align}
  \label{eq:5}
  x^\nu = g^{\nu\mu}\underbr{x_\mu}_{\eqref{eq:4}} = g^{\nu\mu}  g_{\mu\alpha}x^\alpha = g^\nu_{\,\,\alpha}x^{\alpha}
\end{align}

Aus der Gleichung (\ref{eq:5}) folgt

\begin{align}
  \label{eq:6}
  g^\nu_{\,\,\alpha} = \begin{Bmatrix} 1,& \nu = \alpha\\ 0,&\nu\neq\alpha \end{Bmatrix} = \delta^\nu_{\alpha}
\end{align}


Wir betrachten zwei Inertialsysteme. Um von einem Intertialsystem IE in ein anderes IE' zu wechseln benötigt man die Loretz-Transfomation mit

\begin{align}
  \label{eq:11}
  x^{'\mu}=\Lambda^\mu_{\,\, \nu}x^\nu
\end{align}

Hier bezeichnet \(x'^\mu\) ein Vierervektor im IE' Intertialsystem und \(x^\mu\) im IE Intertialsystem . Die \(\Lambda^\mu_{\,\, \nu}\) sind 4x4 Matrizen die den Zusammenhang zwischen den Inertialsystemen herstellen. Betrachten wir eine Drehung um die z-Achse, d.h IE' ist um ein Winkel \(\theta\) zu IE gedreht, dann gilt

\begin{align}
  \label{eq:12}
  \Lambda^\mu_{\,\, \nu} =
  \begin{pmatrix}
    1&0&0&0\\
0&\cos\theta&\sin\theta&0\\
0&-\sin\theta&\cos\theta&0\\
    0&0&0&1
  \end{pmatrix}
\end{align}

Ein anderer Fall wäre wenn die zwei Intertialsysteme mit einer Geschwindigkeit \(v\) zu einander bewegen. Dies bezeichnet man als \textbf{Boosts}. Eine Transformation für ein Boost in die z-Richtung

\begin{align}
  \label{eq:13}
  \Lambda^\mu_{\,\, \nu} =
  \begin{pmatrix}
    \gamma&0&0&-\beta\gamma\\
0&1&0&0\\
0&0&1&0\\
    -\beta\gamma&0&0&\gamma
  \end{pmatrix}
\end{align}

mit den Abkürzungen

\begin{align}
  \label{eq:14}
  \beta = \frac{v}{c} ,\qquad \gamma = \frac{1}{\sqrt{1-\frac{v^2}{c^2}}}
\end{align}


Für dieses Beispiel angewandt ergibt sich im IE' für die einzelnen Komponenten \(x'^\mu\)

\begin{align}
  \label{eq:15}
  x'^\mu =
  \begin{pmatrix}
     \gamma(x^0-\beta x^3)\\
     x^1\\
     x^2\\
 \gamma(x^3-vt)
  \end{pmatrix}
\end{align}

Eine Lorentztransformation ist eine Transformation die die Länge \(x^2\)  des Vektors \(x^\mu\) unverändert lässt. Zum Beweis

\begin{align}
  \label{eq:16}
  (x')^2 &= x'_\nu x'^\nu = \underbr{ g_{\mu\nu}x^{'\mu}}_{~(\ref{eq:4})}x^{'\nu} \notag \\
&=  g_{\mu\nu}\underbr{ x^{'\mu}}_{\Lambda^\mu_{\,\, \rho}x^\rho}\cdot \underbr{ x^{'\nu}}_{\Lambda^\nu_{\,\, \sigma}x^\sigma} \notag \\
 &= \underbrace{\Lambda^\mu_{\,\, \rho}  \Lambda^\nu_{\,\, \sigma}g_{\mu\nu}}_{g_{\rho\sigma}}x^\rho x^\sigma\notag \\
&= g_{\sigma\rho} x^\rho x^\sigma  = x_\sigma x^\sigma = x^2
\end{align}

Aus der Gleichung (\ref{eq:16})  folgt  \((x')^2 = x^2\) und d.h. dass die relative Länge in beiden Inertialsystemen erhalten ist. \(x^2\) ist Lorentz-Invariant. 


Da die Lorenztransformation eine unitäre Transformation ist, gilt die allgemeine Eigenschaft \(\Lambda^{-1}\Lambda=\mathds 1_4\)  . In der Tensorschreibweise sieht das folgendermaßen aus

\begin{align}
  \label{eq:17}
  \Lambda_\mu^{\,\, \rho}\Lambda^\mu_{\,\,\sigma} = g^\rho_{\,\,\sigma} = \delta^\rho_{\,\, \sigma} 
\end{align}

mit


\begin{align}
  \label{eq:18}
 \boxed{(\Lambda^{-1})^\rho_{\,\,\mu} = \Lambda_\mu^{\,\,\rho}}
\end{align}

ergibt sich

\begin{align}
  \label{eq:19}
  &(\Lambda^{-1})^\rho_{\,\,\mu}\Lambda^\mu_{\,\,\sigma} =  \delta^\rho_{\,\, \sigma}  
\end{align}



\(\Rightarrow det\Lambda = \pm 1\) (Verallgemeinerung von orthogonalen Transformation)


TODO: qms9 S.481






\subsection*{Referenzen}
\begin{itemize}
\item Wachter: Relativistische Quantenmechanik
\item www.tphys.physik.uni-tuebingen.de/muether/quanten/qms9.ps
\end{itemize}

\end{document}
