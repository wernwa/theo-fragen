\documentclass[10pt,a4paper,oneside,fleqn]{article}
\usepackage{geometry}
\geometry{a4paper,left=20mm,right=20mm,top=1cm,bottom=2cm}
\usepackage[utf8]{inputenc}
%\usepackage{ngerman}
\usepackage{amsmath}                % brauche ich um dir Formel zu umrahmen.
\usepackage{amsfonts}                % brauche ich für die Mengensymbole
\usepackage{graphicx}
\setlength{\parindent}{0px}
\setlength{\mathindent}{10mm}
\usepackage{bbold}                    %brauche ich für die doppel Zahlen Darstellung (Einheitsmatrix z.B)



\usepackage{color}
\usepackage{titlesec} %sudo apt-get install texlive-latex-extra

\definecolor{darkblue}{rgb}{0.1,0.1,0.55}
\definecolor{verydarkblue}{rgb}{0.1,0.1,0.35}
\definecolor{darkred}{rgb}{0.55,0.2,0.2}

%hyperref Link color
\usepackage[colorlinks=true,
        linkcolor=darkblue,
        citecolor=darkblue,
        filecolor=darkblue,
        pagecolor=darkblue,
        urlcolor=darkblue,
        bookmarks=true,
        bookmarksopen=true,
        bookmarksopenlevel=3,
        plainpages=false,
        pdfpagelabels=true]{hyperref}

\titleformat{\chapter}[display]{\color{darkred}\normalfont\huge\bfseries}{\chaptertitlename\
\thechapter}{20pt}{\Huge}

\titleformat{\section}{\color{darkblue}\normalfont\Large\bfseries}{\thesection}{1em}{}
\titleformat{\subsection}{\color{verydarkblue}\normalfont\large\bfseries}{\thesubsection}{1em}{}

% Notiz Box
\usepackage{fancybox}
\newcommand{\notiz}[1]{\vspace{5mm}\ovalbox{\begin{minipage}{1\textwidth}#1\end{minipage}}\vspace{5mm}}

\usepackage{cancel}
\setcounter{secnumdepth}{3}
\setcounter{tocdepth}{3}





%-------------------------------------------------------------------------------
%Diff-Makro:
%Das Diff-Makro stellt einen Differentialoperator da.
%
%Benutzung:
% \diff  ->  d
% \diff f  ->  df
% \diff^2 f  ->  d^2 f
% \diff_x  ->  d/dx
% \diff^2_x  ->  d^2/dx^2
% \diff f_x  ->  df/dx
% \diff^2 f_x  ->  d^2f/dx^2
% \diff^2{f(x^5)}_x  ->  d^2(f(x^5))/dx^2
%
%Ersetzt man \diff durch \pdiff, so wird der partieller
%Differentialoperator dargestellt.
%
\makeatletter
\def\diff@n^#1{\@ifnextchar{_}{\diff@n@d^#1}{\diff@n@fun^#1}}
\def\diff@n@d^#1_#2{\frac{\textrm{d}^#1}{\textrm{d}#2^#1}}
\def\diff@n@fun^#1#2{\@ifnextchar{_}{\diff@n@fun@d^#1#2}{\textrm{d}^#1#2}}
\def\diff@n@fun@d^#1#2_#3{\frac{\textrm{d}^#1 #2}{\textrm{d}#3^#1}}
\def\diff@one@d_#1{\frac{\textrm{d}}{\textrm{d}#1}}
\def\diff@one@fun#1{\@ifnextchar{_}{\diff@one@fun@d #1}{\textrm{d}#1}}
\def\diff@one@fun@d#1_#2{\frac{\textrm{d}#1}{\textrm{d}#2}}
\newcommand*{\diff}{\@ifnextchar{^}{\diff@n}
  {\@ifnextchar{_}{\diff@one@d}{\diff@one@fun}}}
%
%Partieller Diff-Operator.
\def\pdiff@n^#1{\@ifnextchar{_}{\pdiff@n@d^#1}{\pdiff@n@fun^#1}}
\def\pdiff@n@d^#1_#2{\frac{\partial^#1}{\partial#2^#1}}
\def\pdiff@n@fun^#1#2{\@ifnextchar{_}{\pdiff@n@fun@d^#1#2}{\partial^#1#2}}
\def\pdiff@n@fun@d^#1#2_#3{\frac{\partial^#1 #2}{\partial#3^#1}}
\def\pdiff@one@d_#1{\frac{\partial}{\partial #1}}
\def\pdiff@one@fun#1{\@ifnextchar{_}{\pdiff@one@fun@d #1}{\partial#1}}
\def\pdiff@one@fun@d#1_#2{\frac{\partial#1}{\partial#2}}
\newcommand*{\pdiff}{\@ifnextchar{^}{\pdiff@n}
  {\@ifnextchar{_}{\pdiff@one@d}{\pdiff@one@fun}}}
\makeatother
%
%Das gleich nur mit etwas andere Syntax. Die Potenz der Differentiation wird erst
%zum Schluss angegeben. Somit lautet die Syntax:
%
% \diff_x^2  ->  d^2/dx^2
% \diff f_x^2  ->  d^2f/dx^2
% \diff{f(x^5)}_x^2  ->  d^2(f(x^5))/dx^2
% Ansonsten wie Oben.
%
%Ersetzt man \diff durch \pdiff, so wird der partieller
%Differentialoperator dargestellt.
%
%\makeatletter
%\def\diff@#1{\@ifnextchar{_}{\diff@fun#1}{\textrm{d} #1}}
%\def\diff@one_#1{\@ifnextchar{^}{\diff@n{#1}}%
%  {\frac{\textrm d}{\textrm{d} #1}}}
%\def\diff@fun#1_#2{\@ifnextchar{^}{\diff@fun@n#1_#2}%
%  {\frac{\textrm d #1}{\textrm{d} #2}}}
%\def\diff@n#1^#2{\frac{\textrm d^#2}{\textrm{d}#1^#2}}
%\def\diff@fun@n#1_#2^#3{\frac{\textrm d^#3 #1}%
%  {\textrm{d}#2^#3}}
%\def\diff{\@ifnextchar{_}{\diff@one}{\diff@}}
%\newcommand*{\diff}{\@ifnextchar{_}{\diff@one}{\diff@}}
%
%Partieller Diff-Operator.
%\def\pdiff@#1{\@ifnextchar{_}{\pdiff@fun#1}{\partial #1}}
%\def\pdiff@one_#1{\@ifnextchar{^}{\pdiff@n{#1}}%
%  {\frac{\partial}{\partial #1}}}
%\def\pdiff@fun#1_#2{\@ifnextchar{^}{\pdiff@fun@n#1_#2}%
%  {\frac{\partial #1}{\partial #2}}}
%\def\pdiff@n#1^#2{\frac{\partial^#2}{\partial #1^#2}}
%\def\pdiff@fun@n#1_#2^#3{\frac{\partial^#3 #1}%
%  {\partial #2^#3}}
%\newcommand*{\pdiff}{\@ifnextchar{_}{\pdiff@one}{\pdiff@}}
%\makeatother

%-------------------------------------------------------------------------------
%%Nützliche Makros um in der Quantenmechanik Bras, Kets und das Skalarprodukt
%%zwischen den beiden darzustellen.
%%Benutzung:
%% \bra{x}  ->    < x |
%% \ket{x}  ->    | x >
%% \braket{x}{y} ->   < x | y >

\newcommand\bra[1]{\left\langle #1 \right|}
\newcommand\ket[1]{\left| #1 \right\rangle}
\newcommand\braket[2]{%
  \left\langle #1\vphantom{#2} \right.%
  \left|\vphantom{#1#2}\right.%
  \left. \vphantom{#1}#2 \right\rangle}%

%-------------------------------------------------------------------------------
%%Aus dem Buch:
%%Titel:  Latex in Naturwissenschaften und Mathematik
%%Autor:  Herbert Voß
%%Verlag: Franzis Verlag, 2006
%%ISBN:   3772374190, 9783772374197
%%
%%Hier werden drei Makros definiert:\mathllap, \mathclap und \mathrlap, welche
%%analog zu den aus Latex bekannten \rlap und \llap arbeiten, d.h. selbst
%%keinerlei horizontalen Platz benötigen, aber dennoch zentriert zum aktuellen
%%Punkt erscheinen.

\newcommand*\mathllap{\mathstrut\mathpalette\mathllapinternal}
\newcommand*\mathllapinternal[2]{\llap{$\mathsurround=0pt#1{#2}$}}
\newcommand*\clap[1]{\hbox to 0pt{\hss#1\hss}}
\newcommand*\mathclap{\mathpalette\mathclapinternal}
\newcommand*\mathclapinternal[2]{\clap{$\mathsurround=0pt#1{#2}$}}
\newcommand*\mathrlap{\mathpalette\mathrlapinternal}
\newcommand*\mathrlapinternal[2]{\rlap{$\mathsurround=0pt#1{#2}$}}

%%Das Gleiche nur mit \def statt \newcommand.
%\def\mathllap{\mathpalette\mathllapinternal}
%\def\mathllapinternal#1#2{%
%  \llap{$\mathsurround=0pt#1{#2}$}% $
%}
%\def\clap#1{\hbox to 0pt{\hss#1\hss}}
%\def\mathclap{\mathpalette\mathclapinternal}
%\def\mathclapinternal#1#2{%
%  \clap{$\mathsurround=0pt#1{#2}$}%
%}
%\def\mathrlap{\mathpalette\mathrlapinternal}
%\def\mathrlapinternal#1#2{%
%  \rlap{$\mathsurround=0pt#1{#2}$}% $
%}

%-------------------------------------------------------------------------------
%%Hier werden zwei neue Makros definiert \overbr und \underbr welche analog zu
%%\overbrace und \underbrace funktionieren jedoch die Gleichung nicht
%%'zerreißen'. Dies wird ermöglicht durch das \mathclap Makro.

\def\overbr#1^#2{\overbrace{#1}^{\mathclap{#2}}}
\def\underbr#1_#2{\underbrace{#1}_{\mathclap{#2}}}
 

\begin{document}

\section*{Klein-Gordon-Gleichung}


Die Klein-Gordon-Gleichung ist eine relativistische Gleichung und beschreibt Teilchen mit Spin 0 z.B: \(\pi\)-Mesonen und \(K^0\)-Mesonen. Sie folgt aus dem Korrespondenzprinzip (klassisch)
\begin{equation}
  \label{eq:1}
  \vec p \rightarrow \frac{\hbar}{i}\vec\nabla 
\end{equation}

bzw. kovarianter Schreibweise (relativistisch)
\begin{equation}
  \label{eq:2}
  p_\mu \rightarrow i\hbar \partial_\mu
\end{equation}

Betrachten wir die Länge eines Impulses im Laborsystem IS

\[p^2 = p_\mu p^\mu = (p_0,-\vec p) (p_0,\vec p) =  p_0^2 -\vec p^2\]

mit \(p_0 = \frac{E}{c}\) 
\begin{equation}
  \label{eq:3}
  p_\mu p^\mu =  \left( \frac{E}{c}\right)^2 -\vec p^2
\end{equation}


Die Energie für ein Teilchen in seinem Ruhesystem IS' ist \(E=mc^2\) (mit \(m\) für Ruhe-Energie) und \(\vec p' = 0\) setzen wir die Gleichung in (\ref{eq:3}) ein so erhalten wir
\begin{equation}
  \label{eq:5}
   p'_\mu p^{'\mu} = m^2c^2
\end{equation}

Setzt man nun die beiden Gleichungen (\ref{eq:3}) und (\ref{eq:4}) gleich (Erhaltung der Länge des Impulses in allen Inertialsystemen) IS=IS'  und stellt sie nach \(E\) um somit folgt
\begin{align}
  \label{eq:4}
  \left( \frac{E}{c}\right)^2 -\vec p^2  = m^2c^2 \notag \\
\Leftrightarrow E = \pm \sqrt{\vec p^2c^2 + m^2c^4}
\end{align}

Die negative Energien in Gleichung (\ref{eq:4}) führt zu der Annahme von Antiteilchen.

Zurück zu Gleichung (\ref{eq:5}) dem Teilchen in seinem Ruhesystem. Eine Umformung und einsetzen des Korrespondenzprinzips für den Impuls (\ref{eq:2}) liefert

\begin{align}
  \label{eq:6}
   p'_\mu p^{'\mu} - m^2c^2 &= 0 \notag \\
(i\hbar)^2\partial_\mu\partial^\mu -  m^2c^2 &= 0  \notag \\
-\hbar^2\left[\partial_\mu\partial^\mu + \left(\frac{mc}{\hbar}\right)^2  \right] &= 0 \notag \\
\partial_\mu\partial^\mu + \left(\frac{mc}{\hbar}\right)^2  &= 0
\end{align}

Führen wir den \textbf{d'Alembert-Operator} ein \(\square \equiv \partial_\mu\partial^\mu = \frac{1}{c^2}\pdiff^2_t - \triangle \) und multiplizieren wir die Gleichung (\ref{eq:6}) mit der Wellenfunktion \(\psi\) von rechts, so ergibt das die \textbf{Klein-Gordon-Gleichung}

\begin{equation}
  \label{eq:7}
 \boxed{\left[  \square + \left(\frac{mc}{\hbar}\right)^2\right]\psi  = 0 }
\end{equation}


Die Klein-Gordon-Gleichung erfüllt die Gesetze der speziellen Relativitätstheorie, erhält aber zwei fundamentale Probleme. Ohne deren Bewältigung die Gleichung physikalisch unhaltbar ist. \\
\\
Das \textbf{erste Problem} ist, dass die Lösung der Klein-Gordon-Gleichung auch negative Energieen zulässt. Dies wollen wir näher untersuchen.\\
\\
Die Lösungen der  Klein-Gordon-Gleichung sind ebene Wellen mit der Form
\begin{equation}
  \label{eq:8}
\psi(x) = Ne^{-ipx/\hbar}  
\end{equation}

mit \(p\cdot x = p^\mu x_\mu = Et - \vec p\vec x\), eingesetzt in (\ref{eq:8})

\begin{equation}
  \label{eq:9}
  \psi(x) = Ne^{\frac{i}{\hbar}(Et - \vec p\vec x)}
\end{equation}

Setzen wir nun die ebene Welle in die Klein-Gordon-Gleichung (\ref{eq:7})  ein

\begin{align}
  \label{eq:10}
  \left[  \square + \left(\frac{mc}{\hbar}\right)^2\right] Ne^{\frac{i}{\hbar}(Et - \vec p\vec x)}   &= 0 \\
  \left(\frac{1}{c^2}\pdiff^2_t - \triangle \right) e^{\frac{i}{\hbar}(Et - \vec p\vec x)}  + \left(\frac{mc}{\hbar}\right)^2e^{\frac{i}{\hbar}(Et - \vec p\vec x)}   &= 0 \\
 \frac{1}{c^2}\pdiff^2_t e^{\frac{i}{\hbar}(Et - \vec p\vec x)}  - \nabla^2  e^{\frac{i}{\hbar}(Et - \vec p\vec x)}  + \left(\frac{mc}{\hbar}\right)^2e^{\frac{i}{\hbar}(Et - \vec p\vec x)}   &= 0 \\
 -\frac{E^2}{c^2\hbar^2} \cancel{ e^{\frac{i}{\hbar}(Et - \vec p\vec x)}}  +\frac{1}{\hbar^2}  p^2\cancel{ e^{\frac{i}{\hbar}(Et - \vec p\vec x)}}  + \left(\frac{mc}{\hbar}\right)^2 \cancel{ e^{\frac{i}{\hbar}(Et - \vec p\vec x)}}   &= 0 \\
 -\frac{E^2}{c^2} +  p^2  + m^2c^2 = 0 \\
\Leftrightarrow  E^2 =  p^2c^2  + m^2c^4
\end{align}

Durch ziehen der Wurzel auf beiden Seiten erhält man 
\begin{equation}
  \label{eq:11}
  E =\pm \sqrt{ p^2c^2  + m^2c^4 }
\end{equation}

Die selbe Energie wie Gleichung (\ref{eq:4}). Lösungen der negativer Energie des Energiespektrums ist nach unten nicht beschränkt. Formal liegt das darin begründet, dass die Klein-Gordon-Gleichung eine DGL 2-Ordnung nach der Zeit ist. Es stellt sich also das Problem der Intepretation der negativen Energie. Was später mit Antiteilchen begründet wird.

Das \textbf{zweite Problem} der Klein-Gordon-Gleichung ist dass sie negative Wahrscheinlichkeitsdichten hervorbringt. Dies wollen wir nun näher untersuchen.

Zur Herleitung einer Kontinuitätsgleichung multipliziert man die Klein-Gordon-Gleichung von links mit \(\psi^*\)

\begin{equation}
  \label{eq:12}
  \psi^*\left[  \square + \left(\frac{mc}{\hbar}\right)^2\right]\psi  = 0
\end{equation}

und zieht davon die komplex konjugierte Gleichung

\begin{equation}
  \label{eq:13}
  \psi \left[  \square + \left(\frac{mc}{\hbar}\right)^2\right]\psi^*  = 0
\end{equation}

ab. Somit folgt

\begin{align}
  \label{eq:14}
   \psi^*\left[  \square + \left(\frac{mc}{\hbar}\right)^2\right]\psi - \psi \left[  \square + \left(\frac{mc}{\hbar}\right)^2\right]\psi^* &= 0 \\
   \psi^* \square \psi  +  \cancel{\left(\frac{mc}{\hbar}\right)^2 |\psi|^2} - \psi \square\psi^* - \cancel{ \left(\frac{mc}{\hbar}\right)^2|\psi|^2 }  &= 0 \\
   \psi^* \square \psi  - \psi \square\psi^*  &= 0 \\
\end{align}
Mit \(\square \equiv \partial_\mu\partial^\mu \) eingesetzt mit 


\begin{align}
  \label{eq:15}
     \psi^* \partial_\mu\partial^\mu  \psi  - \psi \partial_\mu\partial^\mu\psi^*  &= 0 \notag \\
   \partial_\mu(  \psi^* \partial^\mu  \psi  - \psi \partial^\mu\psi^* ) &= 0 \qquad \text{Produktregel ?}
\end{align}

Setzt man in die Gleichung (\ref{eq:15}) die Entsprechung für die partielle Ableitungen ein

\begin{equation}
  \label{eq:17}
  \partial_\mu = \left(\frac{1}{c}\pdiff_t,\vec \nabla\right), \qquad  \partial^\mu = \left(\frac{1}{c}\pdiff_t,-\vec \nabla\right)
\end{equation}

so ergibt das

\begin{align}
  \label{eq:16}
   \partial_0(  \psi^* \partial^0  \psi  - \psi \partial^0\psi^* ) + \vec \nabla ( - \psi^* \vec \nabla  \psi  + \psi \vec \nabla \psi^* )  &= 0 \notag \\
 \frac{1}{c^2}\pdiff_t (  \psi^* \pdiff_t \psi  - \psi \pdiff_t\psi^* ) + \vec \nabla ( - \psi^* \vec \nabla  \psi  + \psi \vec \nabla \psi^* )  &= 0 \quad |\cdot -1 \notag \\
- \frac{1}{c^2}\pdiff_t (  \psi^* \pdiff_t \psi  - \psi \pdiff_t\psi^* ) + \vec \nabla ( \psi^* \vec \nabla  \psi  - \psi \vec \nabla \psi^* )  &= 0 
\end{align}
Multipliziert man die Gleichung (\ref{eq:16}) mit \(\frac{\hbar}{2mi}\) somit folgt eine ähnliche Form für die Kontinuitätsgleichung in nicht relativistischen Fall

\begin{equation}
  \label{eq:18}
  \pdiff_t \underbr{ \left[ \frac{i\hbar}{2mc^2}\left( \psi^* \pdiff_t \psi  -  \psi \pdiff_t\psi^*\right) \right]}_{\rho} + \vec \nabla \underbr{\frac{\hbar}{2mi} \left( \psi^* \vec \nabla  \psi  - \psi \vec \nabla \psi^* \right)}_{\vec j}  = 0 \\
\end{equation}
Vergleichen wir die Gleichung (\ref{eq:18}) mit der Kontinuitätsgleichung 

\begin{equation}
  \label{eq:19}
  \pdiff_t \rho + \vec\nabla\vec j = 0
\end{equation}


So erhalten wir für die Wahrscheinlichkeitsdichte \(\rho\)

\begin{equation}
  \label{eq:20}
  \rho = \left[ \frac{i\hbar}{2mc^2}\left( \psi^* \pdiff_t \psi  -  \psi \pdiff_t\psi^*\right) \right]
\end{equation}

Und für die Wahrscheinlichkeitsstromdichte

\begin{equation}
  \label{eq:21}
  \vec j = \frac{\hbar}{2mi} \left( \psi^* \vec \nabla  \psi  - \psi \vec \nabla \psi^* \right)
\end{equation}


Um zu sehen dass die Wahrscheinlichkeitsdichte \(\rho\) auch negativ werden kann, betrachte zunächst die Schrödinger Gleichung sowie die komlexkonjugierte davon

\begin{align}
  \label{eq:22}
  \pdiff_t \psi &= - \frac{i}{\hbar} H\psi = -  \frac{i}{\hbar} E\psi\\
\pdiff_t \psi^* &= \frac{i}{\hbar} E\psi^*
\end{align}


Eingesetzt in (\ref{eq:20})

\begin{align}
  \label{eq:23}
   \rho &=  \frac{i\hbar}{2mc^2}\left( - \frac{i}{\hbar} E \psi^*\psi   -\frac{i}{\hbar} E  \psi\psi^*   \right) \\
&= \frac{i\hbar}{2mc^2}\left( - \frac{i2}{\hbar} E |\psi|^2   \right)  \\
&= \frac{E}{mc^2}|\psi|^2    \\
\end{align}
Da wir wissen das \(E<0\) werden kann folgt daraus dass die Wahrscheinlichkeitsstromdichte \(\rho\) ebenfalls kleiner Null wird. Daraus folgt \(\rho\) kann \textbf{nicht} die Bedeutung einer Wahrscheinlichkeitsdichte haben, sondern eventuell einer Ladungsdichte. Zur Begründung dass \(\rho\) eine mögliche Ladungsdichte ist betrachten wir die Zustande mit \(E>0\) z.B. \(\pi^+\) und \(E<0\) z.B. \(\pi^-\) (Antiteilchen von \(\pi^+\). Im Fall \(\rho>0\) dominieren die \(\pi^+\)Teilchen. Also ist die Ladungsdichte positiv. Im Fall \(\rho<0\) dominieren \(\pi^-\)Teilchen, also wird die Ladungsdichte negativ. Somit ist \(\rho\) proportional zu elektrischen Ladungsdichte.

Die klein-Gordon-Gleichung ist eine Differentialgleichung zweiter Ordnung in t, deshalb können die Anfangswerte von \(\psi\) und \(\pdiff_t\psi\) unabhängig vorgegeben werden, so dass \(\rho\) als Funktion von \(\vec x\) sowohl positiv wie auch negativ sein kann. 




\subsection*{Referenzen}
\begin{itemize}
\item Schwabl QMII
\item Rollnik Quantentheorie 2
\end{itemize}

\end{document}
