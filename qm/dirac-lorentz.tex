\documentclass[10pt,a4paper,oneside,fleqn]{article}
\usepackage{geometry}
\geometry{a4paper,left=20mm,right=20mm,top=1cm,bottom=2cm}
\usepackage[utf8]{inputenc}
%\usepackage{ngerman}
\usepackage{amsmath}                % brauche ich um dir Formel zu umrahmen.
\usepackage{amsfonts}                % brauche ich für die Mengensymbole
\usepackage{graphicx}
\setlength{\parindent}{0px}
\setlength{\mathindent}{10mm}
\usepackage{bbold}                    %brauche ich für die doppel Zahlen Darstellung (Einheitsmatrix z.B)



\usepackage{color}
\usepackage{titlesec} %sudo apt-get install texlive-latex-extra

\definecolor{darkblue}{rgb}{0.1,0.1,0.55}
\definecolor{verydarkblue}{rgb}{0.1,0.1,0.35}
\definecolor{darkred}{rgb}{0.55,0.2,0.2}

%hyperref Link color
\usepackage[colorlinks=true,
        linkcolor=darkblue,
        citecolor=darkblue,
        filecolor=darkblue,
        pagecolor=darkblue,
        urlcolor=darkblue,
        bookmarks=true,
        bookmarksopen=true,
        bookmarksopenlevel=3,
        plainpages=false,
        pdfpagelabels=true]{hyperref}

\titleformat{\chapter}[display]{\color{darkred}\normalfont\huge\bfseries}{\chaptertitlename\
\thechapter}{20pt}{\Huge}

\titleformat{\section}{\color{darkblue}\normalfont\Large\bfseries}{\thesection}{1em}{}
\titleformat{\subsection}{\color{verydarkblue}\normalfont\large\bfseries}{\thesubsection}{1em}{}

% Notiz Box
\usepackage{fancybox}
\newcommand{\notiz}[1]{\vspace{5mm}\ovalbox{\begin{minipage}{1\textwidth}#1\end{minipage}}\vspace{5mm}}

\usepackage{cancel}
\setcounter{secnumdepth}{3}
\setcounter{tocdepth}{3}





%-------------------------------------------------------------------------------
%Diff-Makro:
%Das Diff-Makro stellt einen Differentialoperator da.
%
%Benutzung:
% \diff  ->  d
% \diff f  ->  df
% \diff^2 f  ->  d^2 f
% \diff_x  ->  d/dx
% \diff^2_x  ->  d^2/dx^2
% \diff f_x  ->  df/dx
% \diff^2 f_x  ->  d^2f/dx^2
% \diff^2{f(x^5)}_x  ->  d^2(f(x^5))/dx^2
%
%Ersetzt man \diff durch \pdiff, so wird der partieller
%Differentialoperator dargestellt.
%
\makeatletter
\def\diff@n^#1{\@ifnextchar{_}{\diff@n@d^#1}{\diff@n@fun^#1}}
\def\diff@n@d^#1_#2{\frac{\textrm{d}^#1}{\textrm{d}#2^#1}}
\def\diff@n@fun^#1#2{\@ifnextchar{_}{\diff@n@fun@d^#1#2}{\textrm{d}^#1#2}}
\def\diff@n@fun@d^#1#2_#3{\frac{\textrm{d}^#1 #2}{\textrm{d}#3^#1}}
\def\diff@one@d_#1{\frac{\textrm{d}}{\textrm{d}#1}}
\def\diff@one@fun#1{\@ifnextchar{_}{\diff@one@fun@d #1}{\textrm{d}#1}}
\def\diff@one@fun@d#1_#2{\frac{\textrm{d}#1}{\textrm{d}#2}}
\newcommand*{\diff}{\@ifnextchar{^}{\diff@n}
  {\@ifnextchar{_}{\diff@one@d}{\diff@one@fun}}}
%
%Partieller Diff-Operator.
\def\pdiff@n^#1{\@ifnextchar{_}{\pdiff@n@d^#1}{\pdiff@n@fun^#1}}
\def\pdiff@n@d^#1_#2{\frac{\partial^#1}{\partial#2^#1}}
\def\pdiff@n@fun^#1#2{\@ifnextchar{_}{\pdiff@n@fun@d^#1#2}{\partial^#1#2}}
\def\pdiff@n@fun@d^#1#2_#3{\frac{\partial^#1 #2}{\partial#3^#1}}
\def\pdiff@one@d_#1{\frac{\partial}{\partial #1}}
\def\pdiff@one@fun#1{\@ifnextchar{_}{\pdiff@one@fun@d #1}{\partial#1}}
\def\pdiff@one@fun@d#1_#2{\frac{\partial#1}{\partial#2}}
\newcommand*{\pdiff}{\@ifnextchar{^}{\pdiff@n}
  {\@ifnextchar{_}{\pdiff@one@d}{\pdiff@one@fun}}}
\makeatother
%
%Das gleich nur mit etwas andere Syntax. Die Potenz der Differentiation wird erst
%zum Schluss angegeben. Somit lautet die Syntax:
%
% \diff_x^2  ->  d^2/dx^2
% \diff f_x^2  ->  d^2f/dx^2
% \diff{f(x^5)}_x^2  ->  d^2(f(x^5))/dx^2
% Ansonsten wie Oben.
%
%Ersetzt man \diff durch \pdiff, so wird der partieller
%Differentialoperator dargestellt.
%
%\makeatletter
%\def\diff@#1{\@ifnextchar{_}{\diff@fun#1}{\textrm{d} #1}}
%\def\diff@one_#1{\@ifnextchar{^}{\diff@n{#1}}%
%  {\frac{\textrm d}{\textrm{d} #1}}}
%\def\diff@fun#1_#2{\@ifnextchar{^}{\diff@fun@n#1_#2}%
%  {\frac{\textrm d #1}{\textrm{d} #2}}}
%\def\diff@n#1^#2{\frac{\textrm d^#2}{\textrm{d}#1^#2}}
%\def\diff@fun@n#1_#2^#3{\frac{\textrm d^#3 #1}%
%  {\textrm{d}#2^#3}}
%\def\diff{\@ifnextchar{_}{\diff@one}{\diff@}}
%\newcommand*{\diff}{\@ifnextchar{_}{\diff@one}{\diff@}}
%
%Partieller Diff-Operator.
%\def\pdiff@#1{\@ifnextchar{_}{\pdiff@fun#1}{\partial #1}}
%\def\pdiff@one_#1{\@ifnextchar{^}{\pdiff@n{#1}}%
%  {\frac{\partial}{\partial #1}}}
%\def\pdiff@fun#1_#2{\@ifnextchar{^}{\pdiff@fun@n#1_#2}%
%  {\frac{\partial #1}{\partial #2}}}
%\def\pdiff@n#1^#2{\frac{\partial^#2}{\partial #1^#2}}
%\def\pdiff@fun@n#1_#2^#3{\frac{\partial^#3 #1}%
%  {\partial #2^#3}}
%\newcommand*{\pdiff}{\@ifnextchar{_}{\pdiff@one}{\pdiff@}}
%\makeatother

%-------------------------------------------------------------------------------
%%Nützliche Makros um in der Quantenmechanik Bras, Kets und das Skalarprodukt
%%zwischen den beiden darzustellen.
%%Benutzung:
%% \bra{x}  ->    < x |
%% \ket{x}  ->    | x >
%% \braket{x}{y} ->   < x | y >

\newcommand\bra[1]{\left\langle #1 \right|}
\newcommand\ket[1]{\left| #1 \right\rangle}
\newcommand\braket[2]{%
  \left\langle #1\vphantom{#2} \right.%
  \left|\vphantom{#1#2}\right.%
  \left. \vphantom{#1}#2 \right\rangle}%

%-------------------------------------------------------------------------------
%%Aus dem Buch:
%%Titel:  Latex in Naturwissenschaften und Mathematik
%%Autor:  Herbert Voß
%%Verlag: Franzis Verlag, 2006
%%ISBN:   3772374190, 9783772374197
%%
%%Hier werden drei Makros definiert:\mathllap, \mathclap und \mathrlap, welche
%%analog zu den aus Latex bekannten \rlap und \llap arbeiten, d.h. selbst
%%keinerlei horizontalen Platz benötigen, aber dennoch zentriert zum aktuellen
%%Punkt erscheinen.

\newcommand*\mathllap{\mathstrut\mathpalette\mathllapinternal}
\newcommand*\mathllapinternal[2]{\llap{$\mathsurround=0pt#1{#2}$}}
\newcommand*\clap[1]{\hbox to 0pt{\hss#1\hss}}
\newcommand*\mathclap{\mathpalette\mathclapinternal}
\newcommand*\mathclapinternal[2]{\clap{$\mathsurround=0pt#1{#2}$}}
\newcommand*\mathrlap{\mathpalette\mathrlapinternal}
\newcommand*\mathrlapinternal[2]{\rlap{$\mathsurround=0pt#1{#2}$}}

%%Das Gleiche nur mit \def statt \newcommand.
%\def\mathllap{\mathpalette\mathllapinternal}
%\def\mathllapinternal#1#2{%
%  \llap{$\mathsurround=0pt#1{#2}$}% $
%}
%\def\clap#1{\hbox to 0pt{\hss#1\hss}}
%\def\mathclap{\mathpalette\mathclapinternal}
%\def\mathclapinternal#1#2{%
%  \clap{$\mathsurround=0pt#1{#2}$}%
%}
%\def\mathrlap{\mathpalette\mathrlapinternal}
%\def\mathrlapinternal#1#2{%
%  \rlap{$\mathsurround=0pt#1{#2}$}% $
%}

%-------------------------------------------------------------------------------
%%Hier werden zwei neue Makros definiert \overbr und \underbr welche analog zu
%%\overbrace und \underbrace funktionieren jedoch die Gleichung nicht
%%'zerreißen'. Dies wird ermöglicht durch das \mathclap Makro.

\def\overbr#1^#2{\overbrace{#1}^{\mathclap{#2}}}
\def\underbr#1_#2{\underbrace{#1}_{\mathclap{#2}}}
\usepackage{amsmath} 



\begin{document}

\section*{Lorentz-Transformation der Dirac-Gleichung}

Wir wollen nun die Lorantz-Transformation der Dirac-Gleichung betrachten. Dabei errinern wir uns dass die 4-Dimensionale Vektoren sich wie folgt transformieren


\begin{align}
  \label{eq:1}
  x'^{\mu} = \Lambda^{\mu}_{\hphantom\mu \nu}x^{\nu}
\end{align}

Und die vierer-Ableitung transfomiert sich wie ein kovarianter Vektor. Beweis

\begin{align}
  \label{eq:2}
 \partial'_\mu =  \frac{\partial}{\partial x'^\mu} = \frac{\partial x^\nu}{\partial x^\nu } \frac{\partial}{\partial x'^\mu} = \frac{\partial x^\nu}{\partial x'^\mu } \frac{\partial}{\partial x^\nu} = \frac{\partial x^\nu}{\partial x'^\mu }\partial_\nu
\end{align}

Wir multiplizieren Gleichung (\ref{eq:1}) mit \(\Lambda^{-1}\)

\begin{align}
  \label{eq:3}
  x^{\nu} = \Lambda_{\mu}^{\hphantom\mu \nu}x'^\mu
\end{align}
Diese Gleichung (\ref{eq:3}) in Gleichung (\ref{eq:2}) eingesetzt

\begin{align}
  \label{eq:4}
   \partial'_\mu &= \frac{\partial x^\nu}{\partial x'^\mu } \partial_\nu \notag \\
&= \frac{\partial (\Lambda_{\mu}^{\hphantom\mu \nu}x'^\mu) }{\partial x'^\mu } \partial_\nu \notag \\
&= \Lambda_{\mu}^{\hphantom\mu \nu}\frac{\partial (x'^\mu) }{\partial x'^\mu } \partial_\nu  \notag\\
&= \Lambda_{\mu}^{\hphantom\mu \nu}\partial_\nu
\end{align}

Wie man in der Gleichung (\ref{eq:4}) sieht transformiert sich die vierer-Ableitung kovariant. Als nebenprodukt dieser Rechung ergibt sich folgende nützliche Relation

\begin{align}
  \label{eq:5}
  \Lambda_{\mu}^{\hphantom\mu \nu} = \frac{\partial x^\nu}{\partial x'^\mu }
\end{align}


Betrachten wir die zwei Intertialsysteme IS und IS' so gilt für die einzelten komponenten

\begin{table}[h]
  \centering

\begin{tabular}{c|c}
  IS&IS'\\
\(x^\mu\)& \(x^{'\mu}=\Lambda^\mu_{\hphantom\mu \nu} x^\nu\)\\
\(\partial_\mu \)&\(\partial'_\mu = \Lambda_{\mu}^{\hphantom\mu\nu}\partial_\nu \)\\
\(\psi(x)\)& ???\\
\((i\gamma^\mu\frac{\partial}{\partial x^\mu}-\frac{mc}{\hbar})\psi(x) = 0\)&\((i\gamma^\mu\frac{\partial}{\partial x^{'\mu}}-\frac{mc}{\hbar})\psi'(x') = 0\)
\end{tabular}

  \caption{Lorentz-Transformation der einzelnen Komponenten der Dirac-Gleichung}
  \label{tab:1}
\end{table}

Wie man ersehen kann, fehlt die Transformation für die vierkomponentiger Dirac-Spinor Funktion \(\psi(x)\) (die vier Komponenten nicht mit den 4 Dimensionen der Raumzeit zu verwechseln!)  Dies wollen wir nun näher ergründen indem wir versuchen in der Tabelle \ref{tab:1} aus der linken Dirac-Gleichung auf die rechte zu kommen.

Das wollen wir zeigen dadurch dass es zu jeder Lorentz-Transformation eine lineare Abbildung \(S(\Lambda)\) des Spinoren gibt, so das gilt

\begin{align}
  \label{eq:6}
  \psi'(x) = S(\Lambda)\psi(x) = S(\Lambda)\psi(\Lambda^{-1} x')
\end{align}

Die Menge \(\{S(\Lambda)\}\) bilden die Darstellung der Lorentz-Gruppe mit der allgemeinen Gruppeneigenschaften

\begin{align}
  \label{eq:7}
  \boxed{S(\Lambda_1\Lambda_2) = S(\Lambda_1)S(\Lambda_2) \Rightarrow S(\mathbb 1) = \mathbb 1, \quad S(\Lambda^{-1})=(S(\Lambda))^{-1}}
\end{align}

Ersetze \(\psi\) mit \(\psi'\) aus (\ref{eq:6}) und multipliziere die Dirac-Gleichung mit \(S(\Lambda)\) von links so ergibt sich

\begin{align}
  \label{eq:8}
  S(\Lambda)  \left(i\gamma^\mu\frac{\partial}{\partial x^\mu}-\frac{mc}{\hbar}\right)S(\Lambda^{-1}) \psi'(x') &= 0 \notag \\
\left(iS(\Lambda) \gamma^\mu  \frac{\partial}{\partial x^\mu}S(\Lambda^{-1}) - \frac{mc}{\hbar}\underbr{S(\Lambda)S(\Lambda^{-1})}_{\mathds 1}\right)\psi'(x') &= 0 \notag \\
\left(iS(\Lambda) \gamma^\mu S(\Lambda^{-1}) \underbrace{\frac{\partial}{\partial x^\mu}}_{\Lambda^\nu_{\,\,\mu}\frac{\partial}{\partial x^{'\nu}}} - \frac{mc}{\hbar}\right)\psi'(x') &= 0 \notag \\
\left(i\underbr{S(\Lambda) \gamma^\mu S(\Lambda^{-1}) \Lambda^\nu_{\,\,\mu}}_{\gamma^\mu}\frac{\partial}{\partial x^{'\nu}} - \frac{mc}{\hbar}\right)\psi'(x') &= 0 \\
\end{align}
\(S(\Lambda^{-1})\) vertauscht offensichtlich mit \(\frac{\partial}{\partial x^\mu}\) (wieso?). Vergleicht man nun aus Tabelle \ref{tab:1} die Gestrichelte Dirac-Funktion, so stellt man fest, dass die Größe \( S(\Lambda) \gamma^\mu S(\Lambda^{-1}) \Lambda^\nu_{\,\,\mu}\) die Matritze \(\gamma^\mu\) ergeben muss, also

\begin{align}
  \label{eq:9}
  \gamma^\mu &= S(\Lambda) \gamma^\mu S(\Lambda^{-1}) \Lambda^\nu_{\,\,\mu} 
\end{align}

Multipliziere die Gleichung (\ref{eq:9}) mit \(S(\Lambda^{-1})\)  von links  und mit \(S(\Lambda)\) von rechts, so ergibt sich


\begin{align}
  \label{eq:10}
 S(\Lambda^{-1})\cdot|\qquad \qquad  \gamma^\mu &= S(\Lambda) \gamma^\mu S(\Lambda^{-1}) \Lambda^\nu_{\,\,\mu} \qquad \qquad |\cdot S(\Lambda) \notag \\
 S(\Lambda^{-1}) \gamma^\mu S(\Lambda) &=\underbr{S(\Lambda^{-1}) S(\Lambda)}_{\mathds 1} \gamma^\mu S(\Lambda^{-1}) \Lambda^\nu_{\,\,\mu} S(\Lambda) \notag \\
 S(\Lambda^{-1}) \gamma^\mu S(\Lambda) &= \gamma^\mu\underbr{ S(\Lambda^{-1})S(\Lambda)}_{\mathds 1} \Lambda^\nu_{\,\,\mu}  \notag \\
\Leftrightarrow   \Lambda^\nu_{\,\,\mu}\gamma^\mu &= S(\Lambda^{-1})\gamma^\nu S(\Lambda) 
\end{align}

die Bedingung für die Transformationsmatrix \(S(\Lambda)\). Hier wurde vorausgesetzt das \(\Lambda^\nu_{\,\,\mu}\) und \( S(\Lambda)\) vertauschen.  


\subsection*{Konstruktion der S Matrix}

Wir wollen die Transformationsmatrix \(S(\Lambda)\) bestimmen. Dazu betrachten wir die infinitesimale Lorenztransformationen. Für eine Lorenztransformation setzen wir an

\begin{align}
  \label{eq:11}
  \Lambda^{\mu}_{\hphantom\mu \nu} = e^{\omega^\mu_{\hphantom\mu \nu}}
\end{align}

Jetzt entwickeln wir die \(e\)-Funktion bis zu 1-ter Ordnung

\begin{align}
  \label{eq:12}
  \Lambda^{\mu}_{\hphantom\mu \nu} = \mathds 1 + \omega^\mu_{\hphantom\mu \nu} + \mathcal O\left((\omega^\mu_{\hphantom\mu \nu})^2\right)
\end{align}

Betrachten wir nun die infinitesimale Transformationen, d.h \(\omega \rightarrow \delta\omega\) und vernachlässige Terme höherer Ordnung \(\delta\omega^2\dots\)

\begin{align}
  \label{eq:13}
  \Lambda^{\mu}_{\hphantom\mu \nu} = \mathds 1 + \delta\omega^\mu_{\hphantom\mu \nu}
\end{align}

Analog setzen wir für die Spinor-Transformationsmatrix \(S(\Lambda)\)

\begin{align}
  \label{eq:14}
  S(\Lambda) = e^{\tau} 
\end{align}

Die gleiche Rechnung wie (\ref{eq:11}) bis (\ref{eq:13}) führt auf

\begin{align}
  \label{eq:15}
  S(\Lambda) = \mathds 1 + \delta\tau
\end{align}

Die Kehrwertmatrix von \(S(\Lambda)\) ist

\begin{align}
  \label{eq:16}
  S(\Lambda^{-1}) = S^{-1}(\Lambda) = e^{-\tau}
\end{align}

Die infinitisimale Rechnung von (\ref{eq:16}) liefert

\begin{align}
  \label{eq:17}
   S(\Lambda^{-1}) = \mathds 1 - \delta\tau
\end{align}

Setzen wir die Gleichungen (\ref{eq:13}), (\ref{eq:15}) und (\ref{eq:17}) in (\ref{eq:10}) ein, so ergibt sich eine Beziehung für die infinitesimale Größen

\begin{align}
  \label{eq:18}
 (\mathds 1 + \delta\omega^\mu_{\hphantom\mu \nu}) \gamma^\mu &= (\mathds 1 - \delta\tau)\gamma^\nu( \mathds 1 + \delta\tau)\notag \\
 \gamma^\mu + \delta\omega^\mu_{\hphantom\mu \nu} \gamma^\mu &= (\gamma^\nu - \delta\tau\gamma^\nu)( \mathds 1 + \delta\tau)\notag \\
 \cancel{\gamma^\mu} + \delta\omega^\mu_{\hphantom\mu \nu} \gamma^\mu &= \cancel{\gamma^\nu} + \gamma^\nu\delta\tau - \delta\tau\gamma^\nu - \underbr{\delta\tau^2\gamma^\nu}_{\approx 0}\notag \\
 \delta\omega^\mu_{\hphantom\mu \nu} \gamma^\mu &= [\gamma^\nu,\delta\tau]
\end{align}


Mit der Beziehung 

\begin{align}
  \label{eq:19}
  \delta\omega^\mu_{\hphantom\mu \nu} = - \delta\omega^{\hphantom\nu \mu}_\nu
\end{align}

Läst sich die Gleichung (\ref{eq:18}) schreiben

\begin{align}
  \label{eq:20}
  \boxed{[\delta\tau,\gamma^\nu] = \gamma^\mu \delta\omega^{\hphantom\nu \mu}_\nu}
\end{align}

Desweiteren gilt dass die Norm von \(\psi\) bei Lorenz-Transformation invariant seien soll, d.h. \(\psi'=S\psi\). Das bedeutet die Länge von \(\psi'\) und \(\psi\) muss gleich sein. Das  heißt für die Transformationsmatrix

\begin{align}
  \label{eq:21}
  \det S = \pm 1
\end{align}

Wir betrachten nur die eigentliche Transformationen \(\det S = 1\) und lassen die Spiegelungen \(\det S = -1\) weg. Mit \(\det e^{A} = e^{\text{tr} A}\)

\begin{align}
  \label{eq:22}
  1 = \det S = e^{\text{tr}\,\tau} =   1 + \text{tr}\tau + \mathcal O(\tau^2)
\end{align}

Betrachte wieder den infinitesimalen Fall

\begin{align}
  \label{eq:23}
  1 =  1 + \text{tr}\,\delta\tau 
\end{align}
Aus dieser Gleichung folgt dass

\begin{align}
  \label{eq:24}
  \text{tr}\,\delta\tau = 0
\end{align}

seien muss. Die Lösung der Gleichung (\ref{eq:20}) und (\ref{eq:24}) lautet (TODO)

\begin{align}
  \label{eq:25}
  \delta\tau = -\frac{i}{4}\sigma_{\mu\nu}\delta\omega^{\mu\nu}
\end{align}

mit

\begin{align}
  \label{eq:26}
  \sigma_{\mu\nu} = \frac{i}{2}[\gamma_\mu,\gamma_\nu]
\end{align}

Für den nicht infinitesimalen Fall lautet die Gleichung (\ref{eq:25})

\begin{align}
  \label{eq:28}
  \tau = -\frac{i}{4}\sigma_{\mu\nu}\omega^{\mu\nu}
\end{align}

Setzt man diese in unseren Ansatz (\ref{eq:14}) ein so lautet die Transformations-Matrix \(S(\Lambda)\)

\begin{align}
  \label{eq:27}
  S(\Lambda) = e^{-\frac{i}{4}\sigma_{\mu\nu}\omega^{\mu\nu}}
\end{align}


\subsection*{Referenzen}
\begin{itemize}
\item Rollnik Quantentheorie 2
\end{itemize}

\end{document}
