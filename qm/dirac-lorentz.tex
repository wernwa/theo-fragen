\input{../headers/header_script.tex}
\usepackage{amsmath} 



\begin{document}

\section*{Lorentz-Transformation der Dirac-Gleichung}

Wir wollen nun die Lorantz-Transformation der Dirac-Gleichung betrachten. Dabei errinern wir uns dass die 4-Dimensionale Vektoren sich wie folgt transformieren


\begin{align}
  \label{eq:1}
  x'^{\mu} = \Lambda^{\mu}_{\hphantom\mu \nu}x^{\nu}
\end{align}

Und die vierer-Ableitung transfomiert sich wie ein kovarianter Vektor. Beweis

\begin{align}
  \label{eq:2}
 \partial'_\mu =  \frac{\partial}{\partial x'^\mu} = \frac{\partial x^\nu}{\partial x^\nu } \frac{\partial}{\partial x'^\mu} = \frac{\partial x^\nu}{\partial x'^\mu } \frac{\partial}{\partial x^\nu} = \frac{\partial x^\nu}{\partial x'^\mu }\partial_\nu
\end{align}

Wir multiplizieren Gleichung (\ref{eq:1}) mit \(\Lambda^{-1}\)

\begin{align}
  \label{eq:3}
  x^{\nu} = \Lambda_{\mu}^{\hphantom\mu \nu}x'^\mu
\end{align}
Diese Gleichung (\ref{eq:3}) in Gleichung (\ref{eq:2}) eingesetzt

\begin{align}
  \label{eq:4}
   \partial'_\mu &= \frac{\partial x^\nu}{\partial x'^\mu } \partial_\nu \notag \\
&= \frac{\partial (\Lambda_{\mu}^{\hphantom\mu \nu}x'^\mu) }{\partial x'^\mu } \partial_\nu \notag \\
&= \Lambda_{\mu}^{\hphantom\mu \nu}\frac{\partial (x'^\mu) }{\partial x'^\mu } \partial_\nu  \notag\\
&= \Lambda_{\mu}^{\hphantom\mu \nu}\partial_\nu
\end{align}

Wie man in der Gleichung (\ref{eq:4}) sieht transformiert sich die vierer-Ableitung kovariant. Als nebenprodukt dieser Rechung ergibt sich folgende nützliche Relation

\begin{align}
  \label{eq:5}
  \Lambda_{\mu}^{\hphantom\mu \nu} = \frac{\partial x^\nu}{\partial x'^\mu }
\end{align}


Betrachten wir die zwei Intertialsysteme IS und IS' so gilt für die einzelten komponenten

\begin{table}[h]
  \centering

\begin{tabular}{c|c}
  IS&IS'\\
\(x^\mu\)& \(x^{'\mu}=\Lambda^\mu_{\hphantom\mu \nu} x^\nu\)\\
\(\partial_\mu \)&\(\partial'_\mu = \Lambda_{\mu}^{\hphantom\mu\nu}\partial_\nu \)\\
\(\psi(x)\)& ???\\
\((i\gamma^\mu\frac{\partial}{\partial x^\mu}-\frac{mc}{\hbar})\psi(x) = 0\)&\((i\gamma^\mu\frac{\partial}{\partial x^{'\mu}}-\frac{mc}{\hbar})\psi'(x') = 0\)
\end{tabular}

  \caption{Lorentz-Transformation der einzelnen Komponenten der Dirac-Gleichung}
  \label{tab:1}
\end{table}

Wie man ersehen kann, fehlt die Transformation für die vierkomponentiger Dirac-Spinor Funktion \(\psi(x)\) (die vier Komponenten nicht mit den 4 Dimensionen der Raumzeit zu verwechseln!)  Dies wollen wir nun näher ergründen indem wir versuchen in der Tabelle \ref{tab:1} aus der linken Dirac-Gleichung auf die rechte zu kommen.

Das wollen wir zeigen dadurch dass es zu jeder Lorentz-Transformation eine lineare Abbildung \(S(\Lambda)\) des Spinoren gibt, so das gilt

\begin{align}
  \label{eq:6}
  \psi'(x) = S(\Lambda)\psi(x) = S(\Lambda)\psi(\Lambda^{-1} x')
\end{align}

Die Menge \(\{S(\Lambda)\}\) bilden die Darstellung der Lorentz-Gruppe mit der allgemeinen Gruppeneigenschaften

\begin{align}
  \label{eq:7}
  \boxed{S(\Lambda_1\Lambda_2) = S(\Lambda_1)S(\Lambda_2) \Rightarrow S(\mathbb 1) = \mathbb 1, \quad S(\Lambda^{-1})=(S(\Lambda))^{-1}}
\end{align}

Ersetze \(\psi\) mit \(\psi'\) aus (\ref{eq:6}) und multipliziere die Dirac-Gleichung mit \(S(\Lambda)\) von links so ergibt sich

\begin{align}
  \label{eq:8}
  S(\Lambda)  \left(i\gamma^\mu\frac{\partial}{\partial x^\mu}-\frac{mc}{\hbar}\right)S(\Lambda^{-1}) \psi'(x') &= 0 \notag \\
\left(iS(\Lambda) \gamma^\mu  \frac{\partial}{\partial x^\mu}S(\Lambda^{-1}) - \frac{mc}{\hbar}\underbr{S(\Lambda)S(\Lambda^{-1})}_{\mathds 1}\right)\psi'(x') &= 0 \notag \\
\left(iS(\Lambda) \gamma^\mu S(\Lambda^{-1}) \underbrace{\frac{\partial}{\partial x^\mu}}_{\Lambda^\nu_{\,\,\mu}\frac{\partial}{\partial x^{'\nu}}} - \frac{mc}{\hbar}\right)\psi'(x') &= 0 \notag \\
\left(i\underbr{S(\Lambda) \gamma^\mu S(\Lambda^{-1}) \Lambda^\nu_{\,\,\mu}}_{\gamma^\mu}\frac{\partial}{\partial x^{'\nu}} - \frac{mc}{\hbar}\right)\psi'(x') &= 0 \\
\end{align}
\(S(\Lambda^{-1})\) vertauscht offensichtlich mit \(\frac{\partial}{\partial x^\mu}\) (wieso?). Vergleicht man nun aus Tabelle \ref{tab:1} die Gestrichelte Dirac-Funktion, so stellt man fest, dass die Größe \( S(\Lambda) \gamma^\mu S(\Lambda^{-1}) \Lambda^\nu_{\,\,\mu}\) die Matritze \(\gamma^\mu\) ergeben muss, also

\begin{align}
  \label{eq:9}
  \gamma^\mu &= S(\Lambda) \gamma^\mu S(\Lambda^{-1}) \Lambda^\nu_{\,\,\mu} 
\end{align}

Multipliziere die Gleichung (\ref{eq:9}) mit \(S(\Lambda^{-1})\)  von links  und mit \(S(\Lambda)\) von rechts, so ergibt sich


\begin{align}
  \label{eq:10}
 S(\Lambda^{-1})\cdot|\qquad \qquad  \gamma^\mu &= S(\Lambda) \gamma^\mu S(\Lambda^{-1}) \Lambda^\nu_{\,\,\mu} \qquad \qquad |\cdot S(\Lambda) \notag \\
 S(\Lambda^{-1}) \gamma^\mu S(\Lambda) &=\underbr{S(\Lambda^{-1}) S(\Lambda)}_{\mathds 1} \gamma^\mu S(\Lambda^{-1}) \Lambda^\nu_{\,\,\mu} S(\Lambda) \notag \\
 S(\Lambda^{-1}) \gamma^\mu S(\Lambda) &= \gamma^\mu\underbr{ S(\Lambda^{-1})S(\Lambda)}_{\mathds 1} \Lambda^\nu_{\,\,\mu}  \notag \\
\Leftrightarrow   \Lambda^\nu_{\,\,\mu}\gamma^\mu &= S(\Lambda^{-1})\gamma^\nu S(\Lambda) 
\end{align}

die Bedingung für die Transformationsmatrix \(S(\Lambda)\). Hier wurde vorausgesetzt das \(\Lambda^\nu_{\,\,\mu}\) und \( S(\Lambda)\) vertauschen.  



\subsection*{Referenzen}
\begin{itemize}
\item Rollnik Quantentheorie 2
\end{itemize}

\end{document}
