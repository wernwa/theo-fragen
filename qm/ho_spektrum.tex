\documentclass[10pt,a4paper,oneside,fleqn]{article}
\usepackage{geometry}
\geometry{a4paper,left=20mm,right=20mm,top=1cm,bottom=2cm}
\usepackage[utf8]{inputenc}
%\usepackage{ngerman}
\usepackage{amsmath}                % brauche ich um dir Formel zu umrahmen.
\usepackage{amsfonts}                % brauche ich für die Mengensymbole
\usepackage{graphicx}
\setlength{\parindent}{0px}
\setlength{\mathindent}{10mm}
\usepackage{bbold}                    %brauche ich für die doppel Zahlen Darstellung (Einheitsmatrix z.B)



\usepackage{color}
\usepackage{titlesec} %sudo apt-get install texlive-latex-extra

\definecolor{darkblue}{rgb}{0.1,0.1,0.55}
\definecolor{verydarkblue}{rgb}{0.1,0.1,0.35}
\definecolor{darkred}{rgb}{0.55,0.2,0.2}

%hyperref Link color
\usepackage[colorlinks=true,
        linkcolor=darkblue,
        citecolor=darkblue,
        filecolor=darkblue,
        pagecolor=darkblue,
        urlcolor=darkblue,
        bookmarks=true,
        bookmarksopen=true,
        bookmarksopenlevel=3,
        plainpages=false,
        pdfpagelabels=true]{hyperref}

\titleformat{\chapter}[display]{\color{darkred}\normalfont\huge\bfseries}{\chaptertitlename\
\thechapter}{20pt}{\Huge}

\titleformat{\section}{\color{darkblue}\normalfont\Large\bfseries}{\thesection}{1em}{}
\titleformat{\subsection}{\color{verydarkblue}\normalfont\large\bfseries}{\thesubsection}{1em}{}

% Notiz Box
\usepackage{fancybox}
\newcommand{\notiz}[1]{\vspace{5mm}\ovalbox{\begin{minipage}{1\textwidth}#1\end{minipage}}\vspace{5mm}}

\usepackage{cancel}
\setcounter{secnumdepth}{3}
\setcounter{tocdepth}{3}





%-------------------------------------------------------------------------------
%Diff-Makro:
%Das Diff-Makro stellt einen Differentialoperator da.
%
%Benutzung:
% \diff  ->  d
% \diff f  ->  df
% \diff^2 f  ->  d^2 f
% \diff_x  ->  d/dx
% \diff^2_x  ->  d^2/dx^2
% \diff f_x  ->  df/dx
% \diff^2 f_x  ->  d^2f/dx^2
% \diff^2{f(x^5)}_x  ->  d^2(f(x^5))/dx^2
%
%Ersetzt man \diff durch \pdiff, so wird der partieller
%Differentialoperator dargestellt.
%
\makeatletter
\def\diff@n^#1{\@ifnextchar{_}{\diff@n@d^#1}{\diff@n@fun^#1}}
\def\diff@n@d^#1_#2{\frac{\textrm{d}^#1}{\textrm{d}#2^#1}}
\def\diff@n@fun^#1#2{\@ifnextchar{_}{\diff@n@fun@d^#1#2}{\textrm{d}^#1#2}}
\def\diff@n@fun@d^#1#2_#3{\frac{\textrm{d}^#1 #2}{\textrm{d}#3^#1}}
\def\diff@one@d_#1{\frac{\textrm{d}}{\textrm{d}#1}}
\def\diff@one@fun#1{\@ifnextchar{_}{\diff@one@fun@d #1}{\textrm{d}#1}}
\def\diff@one@fun@d#1_#2{\frac{\textrm{d}#1}{\textrm{d}#2}}
\newcommand*{\diff}{\@ifnextchar{^}{\diff@n}
  {\@ifnextchar{_}{\diff@one@d}{\diff@one@fun}}}
%
%Partieller Diff-Operator.
\def\pdiff@n^#1{\@ifnextchar{_}{\pdiff@n@d^#1}{\pdiff@n@fun^#1}}
\def\pdiff@n@d^#1_#2{\frac{\partial^#1}{\partial#2^#1}}
\def\pdiff@n@fun^#1#2{\@ifnextchar{_}{\pdiff@n@fun@d^#1#2}{\partial^#1#2}}
\def\pdiff@n@fun@d^#1#2_#3{\frac{\partial^#1 #2}{\partial#3^#1}}
\def\pdiff@one@d_#1{\frac{\partial}{\partial #1}}
\def\pdiff@one@fun#1{\@ifnextchar{_}{\pdiff@one@fun@d #1}{\partial#1}}
\def\pdiff@one@fun@d#1_#2{\frac{\partial#1}{\partial#2}}
\newcommand*{\pdiff}{\@ifnextchar{^}{\pdiff@n}
  {\@ifnextchar{_}{\pdiff@one@d}{\pdiff@one@fun}}}
\makeatother
%
%Das gleich nur mit etwas andere Syntax. Die Potenz der Differentiation wird erst
%zum Schluss angegeben. Somit lautet die Syntax:
%
% \diff_x^2  ->  d^2/dx^2
% \diff f_x^2  ->  d^2f/dx^2
% \diff{f(x^5)}_x^2  ->  d^2(f(x^5))/dx^2
% Ansonsten wie Oben.
%
%Ersetzt man \diff durch \pdiff, so wird der partieller
%Differentialoperator dargestellt.
%
%\makeatletter
%\def\diff@#1{\@ifnextchar{_}{\diff@fun#1}{\textrm{d} #1}}
%\def\diff@one_#1{\@ifnextchar{^}{\diff@n{#1}}%
%  {\frac{\textrm d}{\textrm{d} #1}}}
%\def\diff@fun#1_#2{\@ifnextchar{^}{\diff@fun@n#1_#2}%
%  {\frac{\textrm d #1}{\textrm{d} #2}}}
%\def\diff@n#1^#2{\frac{\textrm d^#2}{\textrm{d}#1^#2}}
%\def\diff@fun@n#1_#2^#3{\frac{\textrm d^#3 #1}%
%  {\textrm{d}#2^#3}}
%\def\diff{\@ifnextchar{_}{\diff@one}{\diff@}}
%\newcommand*{\diff}{\@ifnextchar{_}{\diff@one}{\diff@}}
%
%Partieller Diff-Operator.
%\def\pdiff@#1{\@ifnextchar{_}{\pdiff@fun#1}{\partial #1}}
%\def\pdiff@one_#1{\@ifnextchar{^}{\pdiff@n{#1}}%
%  {\frac{\partial}{\partial #1}}}
%\def\pdiff@fun#1_#2{\@ifnextchar{^}{\pdiff@fun@n#1_#2}%
%  {\frac{\partial #1}{\partial #2}}}
%\def\pdiff@n#1^#2{\frac{\partial^#2}{\partial #1^#2}}
%\def\pdiff@fun@n#1_#2^#3{\frac{\partial^#3 #1}%
%  {\partial #2^#3}}
%\newcommand*{\pdiff}{\@ifnextchar{_}{\pdiff@one}{\pdiff@}}
%\makeatother

%-------------------------------------------------------------------------------
%%Nützliche Makros um in der Quantenmechanik Bras, Kets und das Skalarprodukt
%%zwischen den beiden darzustellen.
%%Benutzung:
%% \bra{x}  ->    < x |
%% \ket{x}  ->    | x >
%% \braket{x}{y} ->   < x | y >

\newcommand\bra[1]{\left\langle #1 \right|}
\newcommand\ket[1]{\left| #1 \right\rangle}
\newcommand\braket[2]{%
  \left\langle #1\vphantom{#2} \right.%
  \left|\vphantom{#1#2}\right.%
  \left. \vphantom{#1}#2 \right\rangle}%

%-------------------------------------------------------------------------------
%%Aus dem Buch:
%%Titel:  Latex in Naturwissenschaften und Mathematik
%%Autor:  Herbert Voß
%%Verlag: Franzis Verlag, 2006
%%ISBN:   3772374190, 9783772374197
%%
%%Hier werden drei Makros definiert:\mathllap, \mathclap und \mathrlap, welche
%%analog zu den aus Latex bekannten \rlap und \llap arbeiten, d.h. selbst
%%keinerlei horizontalen Platz benötigen, aber dennoch zentriert zum aktuellen
%%Punkt erscheinen.

\newcommand*\mathllap{\mathstrut\mathpalette\mathllapinternal}
\newcommand*\mathllapinternal[2]{\llap{$\mathsurround=0pt#1{#2}$}}
\newcommand*\clap[1]{\hbox to 0pt{\hss#1\hss}}
\newcommand*\mathclap{\mathpalette\mathclapinternal}
\newcommand*\mathclapinternal[2]{\clap{$\mathsurround=0pt#1{#2}$}}
\newcommand*\mathrlap{\mathpalette\mathrlapinternal}
\newcommand*\mathrlapinternal[2]{\rlap{$\mathsurround=0pt#1{#2}$}}

%%Das Gleiche nur mit \def statt \newcommand.
%\def\mathllap{\mathpalette\mathllapinternal}
%\def\mathllapinternal#1#2{%
%  \llap{$\mathsurround=0pt#1{#2}$}% $
%}
%\def\clap#1{\hbox to 0pt{\hss#1\hss}}
%\def\mathclap{\mathpalette\mathclapinternal}
%\def\mathclapinternal#1#2{%
%  \clap{$\mathsurround=0pt#1{#2}$}%
%}
%\def\mathrlap{\mathpalette\mathrlapinternal}
%\def\mathrlapinternal#1#2{%
%  \rlap{$\mathsurround=0pt#1{#2}$}% $
%}

%-------------------------------------------------------------------------------
%%Hier werden zwei neue Makros definiert \overbr und \underbr welche analog zu
%%\overbrace und \underbrace funktionieren jedoch die Gleichung nicht
%%'zerreißen'. Dies wird ermöglicht durch das \mathclap Makro.

\def\overbr#1^#2{\overbrace{#1}^{\mathclap{#2}}}
\def\underbr#1_#2{\underbrace{#1}_{\mathclap{#2}}}
%\includegraphics[width=0.75\textwidth]{thepic.png}

\begin{document}

\textit{29. März 2012}
\input{../headers/authors.tex}

\section*{Quantenmechanischer Harmonischer Oszillator}


Harmonischer Oszillator ist ein Modell für gleichmäßig periodische Schwinungen. Ausgehend vom Hookschen Gesetzt der Kraft gibt es zunächst klassisch eine Hamiltonfunktion die neben der kinetischen Energie von potentieller Energie abhängt aus Masse und Winkelgeschwindigkeit des Oszillators. Um auf die quantenmechanische Form zu kommen drückt man Ort und Impuls durch Auf- und Absteige-Operatoren aus und erhält so einen quantenmechanischen Hamiltonoperator der Form
\begin{equation}
  \label{eq:49}
  H=\hbar\omega(a^\dagger a+\frac{1}{2})
\end{equation}

Angewandt auf die Eigenzustände ergeben sich die gequantelten Eigen-Energie-Werte \(E=\hbar\omega(n+\frac{1}{2})\). Wobei die Nullpunktsenergie auf Grund der Unschärferelation nicht Null wird. Die Eigenzustände des Hamiltonoperators erhält man aus dem Wissen der Auf- und Absteigeoperatoren wie diese auf die Eigenzustände wirken. Die Eigenfunktionen des Harmonischen Oszillators drückt man durch \textbf{Hermitpolynome} aus, da diese eine ähnliche Form besitzen.


\subsection*{HO Klassisch}


Hooksches Gesetz:

\begin{equation}
  \label{eq:1}
  F = -Dx
\end{equation}

Kombiniert mit dem Newtonschen Axiom:

\begin{align}
  ma &= -Dx  \\
  m\ddot x &= -Dx \\
&\Leftrightarrow \ddot x + \underbrace{\frac{D}{m}}_{\omega^2}x = 0 \\
\Rightarrow D &= m\omega^2\label{eq:2}
\end{align}

Die Hamiltonfunktion für dieses System ist die Summe aus kinetischer und potentieller Energie:

\begin{equation}
  \label{eq:3}
  H = E_{\text{kin}}+E_{\text{pot}} = \frac{p^2}{2m} + \frac{1}{2}Dx^2
\end{equation}

Mit D aus \eqref{eq:2} folgt die Hamilton-Funktion für den Harmonischen Oszillator:

\begin{equation}
  \label{eq:4}
  \boxed{ H = \frac{p^2}{2m} +  \frac{1}{2}m\omega^2x^2}
\end{equation}


Die Standardmethode der Analysis zur Lösung der Differentialgleichung unter der Nebenbedinuung, dass \(\psi(x)\) quadratintegrabel ist, führt auf Hermite-Polynome. Wir wollen hier jedoch eine algebraische Methode benutzen.

\subsection*{Algebraische Methode}

Als erstes definieren wir den nicht hermiteschen Operator \(a\):

\begin{equation}
  \label{eq:5}
  a = \frac{\omega m x+ip}{\sqrt{2\omega m\hbar}}  
\end{equation}

und den dazu komplexkonjgiergten Operator

\begin{equation}
  \label{eq:6}
  a^\dagger = \frac{\omega m x-ip}{\sqrt{2\omega m\hbar}}
\end{equation}

Durch die Addition von \eqref{eq:5} und  \eqref{eq:6} erhalten wir den Ortsoperator:

\begin{equation}
  \label{eq:7}
  x = \sqrt{\frac{\hbar}{2\omega m}}(a+a^\dagger)
\end{equation}

Und durch Subtraktion von \eqref{eq:5} minus \eqref{eq:6} erhalten wir den Impulsoperator:

\begin{equation}
  \label{eq:8}
  p = -i\sqrt{\frac{\hbar \omega m}{2}}(a-a^\dagger)
\end{equation}

Aus dem Kommutator \([x,p]=i\hbar\) erhalten wir eine wichtige Beziehung:

\begin{equation}
  \label{eq:9}
  \boxed{[a,a^\dagger]=1}
\end{equation}

Die von uns erhaltenen Orts- und Impuls-Gleichungen \eqref{eq:7} und \eqref{eq:8} eingesetzt in \eqref{eq:4}:


\begin{align}
  \label{eq:10}
   H &= \frac{1}{2m}\frac{\hbar \omega m}{2}(a-a^\dagger)^2 +  \frac{1}{2}m\omega^2\frac{\hbar}{2\omega m}(a+a^\dagger)^2 \\
&= -\frac{\hbar \omega}{4}(a-a^\dagger)^2 +  \omega\frac{\hbar}{4}(a+a^\dagger)^2 \\
&= \frac{\hbar \omega}{4}\left[-(a-a^\dagger)^2 + (a+a^\dagger)^2\right] \\
&= \frac{\hbar \omega}{4}\left[\cancel{-a^2-(a^\dagger)^2}+aa^\dagger+a^\dagger a +\cancel{a^2+(a^\dagger)^2}+aa^\dagger+a^\dagger a \right] \\
&= \frac{\hbar \omega}{4}\left[aa^\dagger+a^\dagger a +aa^\dagger+a^\dagger a \right] \\
&= \frac{\hbar \omega}{2}\left[aa^\dagger+a^\dagger a \right] \\
\end{align}
Mit der Beziehung \eqref{eq:9} \(aa^\dagger = 1 + a^\dagger a \) ergibt sich der Hamilton Operator als Ausdruck des Absolutquadtrats vom Oparator \(a\):

\begin{equation}
  \label{eq:11}
  \boxed{ H = \hbar\omega\left(a^\dagger a + \frac{1}{2}\right) }
\end{equation}

\subsection*{Spektrum der Energieeigenwerte}

Als nächstes wollen wir das Enegriespektrum des Harmonischen Oszillators bestimmen. Es reicht dazu die Eigenwerte des 'Zähloperators' zunächst zu betrachten:
\begin{equation}
  \label{eq:12}
  N = a^\dagger a
\end{equation}

Mit der Eigenwertgleichung:

\begin{equation}
  \label{eq:13}
  N|n\rangle  = n|n\rangle 
\end{equation}

Wendet man den Hamiltonoperator auf die Energieeigenzustände an so ergibt sich:

\begin{align}
  \label{eq:14}
  H|n\rangle &=E_n|n\rangle\\
 \hbar\omega\left(a^\dagger a + \frac{1}{2}\right)|n\rangle &= E_n|n\rangle 
\end{align}

Damit erhalten wir für die Energieeigenwerte:

\begin{equation}
  \label{eq:15}
  \boxed{E_n = \hbar\omega\left(n + \frac{1}{2}\right) }
\end{equation}

Wir werden später zeigen, dass n eine positive ganze Zahl ist.\\
\\
Nun wollen wir die physikalische Bedeutung des eigenführten Operators \(a\) untersuchen. Dazu benötigen wir zwei Relationen:

\begin{align}
  [a,H] &= [a,\hbar\omega(a^\dagger a +\frac{1}{2})]  \\
&= \hbar\omega[a,a^\dagger a ]  \\
&=-\hbar\omega[a^\dagger a,a] \qquad \text{mit }[AB,C]=A[B,C]+[A,C]B \\
&=-\hbar\omega(\underbrace{a^\dagger[a,a]}_{=0}+\underbrace{[a^\dagger,a]}_{=-1}a)\\
&=\hbar\omega a \label{eq:16}
\end{align}

Analog erhält man:

\begin{equation}
  \label{eq:17}
  [a^\dagger,H] = -\hbar\omega a^\dagger
\end{equation}

Nun betrachten wir folgende Eigenwertgleichungen:

\begin{align}
 \overbrace{\underline{H}(a}^{\eqref{eq:16}}|n\rangle ) &= (aH-\hbar\omega a)|n\rangle  = \underline{(E_n-\hbar\omega)}(a|n\rangle ) \label{eq:18}\\
  \overbrace{\underline{H}(a^\dagger}^{\eqref{eq:17}}|n\rangle ) &= (a^\dagger H+\hbar\omega a^\dagger)|n\rangle  = \underline{(E_n+\hbar\omega)}(a^\dagger|n\rangle )\label{eq:19}
\end{align}



Aus den Gleichungen \eqref{eq:18} und \eqref{eq:19} sieht man dass \((a|n\rangle) \) und \((a^\dagger|n\rangle)\) Eigenzustände von H mit den Eigenwerten \((E_n-\hbar\omega)\) und  \((E_n+ \hbar\omega)\) sind. Die entsprechende Wirkung der Operatoren \(a\) und \(a^\dagger\) auf die Energieeigenzustände \(|n\rangle \) erzeugt ein neuen Energiezustand der sich um \(\hbar\omega\) verringert bzw. erhöht. Deswegen werden die Operatoren \(a\) und \(a^\dagger\) als vernichtungs(absteige) bzw. erzeugungs(aufsteige) Operatoren bezeichnet. 

Als nächstes wollen wir feststellen was passiert wenn wir die Operatoren \(a\) und \(a^\dagger\) auf die Energieeigenzustände anwenden. Dazu benötigen wir zwei weitere Kommutator-Relationen:

\begin{align}
  [N,a] &= [a^\dagger a,a] \\
&= a^\dagger\underbrace{[a,a]}_{=0} + \underbrace{[a^\dagger,a]}_{-1}a \\
&= - a\label{eq:20}
\end{align}

und analog für \(a^\dagger\) folgt:

\begin{equation}
  \label{eq:21}
  [N,a^\dagger] = a^\dagger
\end{equation}

Nun ähnlich zu den Gleichungen \eqref{eq:18} und \eqref{eq:19} für den Hamiltonoperator betrachten wir  Eigenwertgleichungen für den Operator \(N\):


\begin{align}
 \overbrace{\underline{N}(a}^{\eqref{eq:20}}|n\rangle ) &= a(N-1)|n\rangle  = \underline{(n-1)}(a|n\rangle ) \label{eq:22}\\
  \overbrace{\underline{N}(a^\dagger}^{\eqref{eq:21}}|n\rangle ) &= a^\dagger(N+1)|n\rangle  = \underline{(n+1)}(a^\dagger|n\rangle )\label{eq:23}
\end{align}

Aus den beiden Gleichungen \eqref{eq:22} und \eqref{eq:23} folgt  \((a|n\rangle) \) und \((a^\dagger|n\rangle)\) Eigenzustände von \(N\) mit den Eigenwerten \((n-1)\) und  \((n+ 1)\) sind. Das bedeutet, dass \((a|n\rangle) \) und \((a^\dagger|n\rangle)\) angewandt auf die Zustände \(|n\rangle \) erniedrigen bzw. erhöhen jeweils um 1. D.h. dass \(a\) angewand auf \(|n\rangle \) erzeugt einen neuen Zustand mit \(|n-1\rangle \) und \(a^\dagger\) angewand auf \(|n\rangle \) erzeugt einen neuen Zustand mit \(|n+1\rangle \). Damit erhalten wir folgende Beziehung:

\begin{equation}
  \label{eq:24}
  a|n\rangle = c_{n}|n-1\rangle 
\end{equation}

Wobei \(c_n\) eine die noch zu bestimmende Konstante ist. Um \(c_n\) zu bestimmen nutzen wir die Normierungsbedinung für den Zustand \((a|n\rangle )\) aus:

\begin{equation}
  \label{eq:25}
  (\langle n| a^\dagger )\cdot(a|n\rangle ) = \langle n| a^\dagger a | n\rangle  = |c_n|^2 \langle n-1|n-1\rangle  = |c_n|^2
\end{equation}

Desweiteren gilt:

\begin{equation}
  \label{eq:26}
   (\langle n| a^\dagger )\cdot(a|n\rangle ) = \langle n| \underbrace{a^\dagger a}_{N} | n\rangle  = n
\end{equation}

Aus  \eqref{eq:25} und \eqref{eq:26} folgt:

\begin{equation}
  \label{eq:27}
  c_n = \sqrt{n}
\end{equation}

Eingesetzt in die Gleichung \eqref{eq:24}:

\begin{equation}
  \label{eq:28}
   \boxed{ a|n\rangle = \sqrt{n}|n-1\rangle }
\end{equation}

Eine analoge Rechnung für den Aufsteige-Operator ergibt:

\begin{equation}
  \label{eq:29}
  \boxed{ a^\dagger|n\rangle = \sqrt{n+1}|n+1\rangle }
\end{equation}

Wendet man den Absteige-Operator auf den Zustand \(|n\rangle \) sukzessive an, wird der Zustand immer um 1 kleiner bis zum Grundzustand mit \(n=0\). Weitere Anwendung des Operators auf den Null-Zustand \(a|0\rangle \) ergibt Null. Wendet man nun startend vom Grundzustand aus den Aufsteige-Operator an so wird \(n\) immer um eine ganze Zahl erhöht. Daraus folgt dass \(n\) eine positive ganze Zahl sein muss. Somit ist das Spektrum für die Energieeigenwerte gegeben mit:

\begin{equation}
  \label{eq:30}
  \boxed{E_n = \hbar\omega\left(n + \frac{1}{2}\right) \qquad n \in \mathbb N_0}
\end{equation}

Da \(n\) eine positive ganze Zahl ist, ist das Energiespektrum des harmonischen Oszillators quantisiert. Im Gegensatz zum Klassischen Fall gibt es eine nicht verschwindende Nullpunktenergie (Grundzustandsenergie) für \(n=0\):

\begin{equation}
  \label{eq:31}
  E_0 = \frac{\hbar\omega }{2}
\end{equation}

Die Nullpunktenergie darf nach dem Quantenmechanischen Prinzip auch nicht Null werden, da es die Unschärferelation verletzen würde.


\subsection*{Eigenzustände, Hermitpolynome}

Wir wollen die Eigenzustände des harmonischen Oszillators bestimmen. Der Grundzustand der Nullpunktenergie lässt sich relativ leicht aus der Bedingung,

\begin{equation}
  \label{eq:32}
  a |0 \rangle  = 0
\end{equation}

herleiten. Mit \eqref{eq:5}

\begin{align}
  \label{eq:33}
  a|0\rangle  &= \frac{\omega m x+ip}{\sqrt{2\omega m\hbar}}|0\rangle  \\
 0 &= \frac{\omega m x+ip}{\sqrt{2\omega m\hbar}}|0\rangle  \\
 0 &= \left(\omega m x+ip\right)|0\rangle  \\
 0 &= \left(\omega m x+i \frac{\hbar}{i} \frac{d}{dx}\right)|0\rangle  \\
 0 &= \left( \frac{\omega m}{\hbar} x+ \frac{d}{dx}\right)|0\rangle  \\
 0 &= \left( \frac{1}{x^2_0} x+ \frac{d}{dx}\right)|0\rangle \qquad \text{mit } x_0^2 = \frac{\hbar}{\omega m} 
\end{align}

Zur Lösung dieser Differentialgleichung:
\begin{equation}
  \label{eq:34}
 \Rightarrow  \left( \frac{1}{x^2_0} x+ \frac{d}{dx}\right)|0\rangle = 0
\end{equation}
verwenden wir den Ansatz:

\begin{equation}
  \label{eq:35}
  \langle x |0\rangle = A e^{-\frac{1}{2}\frac{x^2}{x_0^2}}
\end{equation}

Um den Normierungsfaktor \(A\) zu bestimmen benutzen wir die Normierungsbedingung:

\begin{equation}
  \label{eq:36}
\langle 0|0\rangle  = |A|^2 \int_{-\infty}^{\infty} e^{-\frac{x^2}{x_0^2}} = |A|^2 \sqrt{\pi}x_0 \stackrel{!}= 1
\end{equation}

\begin{equation}
  \label{eq:37}
  \Rightarrow A = \frac{1}{\sqrt[4]{\pi}\sqrt{x_0}}
\end{equation}

Somit erhalten wir die Grundzustandsfunktion:
\begin{equation}
  \label{eq:38}
   \langle x |0\rangle = \frac{1}{\sqrt[4]{\pi}\sqrt{x_0}} e^{-\frac{1}{2}\frac{x^2}{x_0^2}}
\end{equation}

Wenden man nun den Aufsteige-Operator n mal auf die Grundzustandsfunktion an, so erhält man den n-ten Zustand:


\begin{align}
  \label{eq:39}
  a^\dagger|0\rangle&=\sqrt 1 |1\rangle \\
 a^\dagger a^\dagger|1\rangle &= \sqrt 1 a^\dagger|1\rangle = \sqrt 1 \sqrt 2 |2\rangle\\
 a^\dagger a^\dagger a^\dagger|1\rangle &= \sqrt 1 a^\dagger a^\dagger|1\rangle = \sqrt 1 \sqrt 2  a^\dagger |2\rangle
                             =\sqrt 1 \sqrt 2  \sqrt 3 |3\rangle\\
 &\vdots\\
 (a^\dagger)^n|0\rangle  &= \sqrt{n!}|n\rangle\\
 \Leftrightarrow |n\rangle &=\frac{1}{\sqrt{n!}}(a^\dagger)^n|0\rangle
\end{align}
\begin{equation}
  \label{eq:40}
   \Rightarrow \langle x |n\rangle = \frac{1}{\sqrt{n!}\sqrt[4]{\pi}\sqrt{x_0}} (a^\dagger)^n e^{-\frac{1}{2}\frac{x^2}{x_0^2}}
\end{equation}

Setzen wir den Aufsteige-Operator \eqref{eq:5}

\begin{equation}
  \label{eq:41}
  a^\dagger = \frac{1}{\sqrt{2}x_0}\left( x - x_0^2 \frac{d}{dx}\right)
\end{equation}

 in die Gleichung \eqref{eq:40} ein, so erhalten wir:

 \begin{equation}
   \label{eq:42}
    \langle x |n\rangle = \frac{1}{\sqrt{n!}\sqrt[4]{\pi}\sqrt{x_0}} \frac{1}{\sqrt{2^n}x_0^n}\left( x - x_0^2 \frac{d}{dx}\right)^n  e^{-\frac{1}{2}\frac{x^2}{x_0^2}}
 \end{equation}

 \begin{equation}
   \label{eq:43}
  \rightarrow  \boxed{\psi_n(x) =  \langle x |n\rangle = \frac{1}{\sqrt{n!}\sqrt[4]{\pi}} \frac{1}{\sqrt{2^n}x_0^{n+\frac{1}{2}}}\left( x - x_0^2 \frac{d}{dx}\right)^n  e^{-\frac{1}{2}\frac{x^2}{x_0^2}}   }
 \end{equation}



Wir wollen die Gleichung \eqref{eq:43} mit Hilfe der Hermitpolynome Ausdrücken. Dafür benötigen wir die bekannte Relation:

\begin{equation}
  \label{eq:44}
  H_n(x) = e^{\frac{x^2}{2}} \left( x-\frac{d}{dx}\right)^n e^{-\frac{x^2}{2}}
\end{equation}

Für x den Wert \(\frac{x}{x_0}\) eingesetzt lautet die Gleichung nun:

\begin{align}
  H_n\left(\frac{x}{x_0} \right) &= e^{\frac{x^2}{2x_0^2}} \left( \frac{x}{x_0}- x_0\frac{d}{dx}\right)^n e^{-\frac{x^2}{2x_0^2}} \\
 &= e^{\frac{x^2}{2x_0^2}} \frac{1}{x_0^n} \left( x - x_0^2\frac{d}{dx}\right)^n e^{-\frac{x^2}{2x_0^2}} \\ \label{eq:45}
\end{align}
\begin{equation}
  \label{eq:46}
  \Leftrightarrow e^{-\frac{x^2}{2x_0^2}}H_n\left(\frac{x}{x_0} \right) = \frac{1}{x_0^n} \left( x - x_0^2\frac{d}{dx}\right)^n e^{-\frac{x^2}{2x_0^2}} 
\end{equation}
\\
Gleichung \eqref{eq:46} in \eqref{eq:43} eingesetzt:

\begin{equation}
  \label{eq:47}
   \boxed{\psi_n(x) =  \langle x |n\rangle = \frac{1}{\sqrt{n!}\sqrt[4]{\pi}\sqrt{2^n}\sqrt{x_0}}  e^{-\frac{x^2}{2x_0^2}}H_n\left(\frac{x}{x_0} \right)   }
\end{equation}

Wobei \(H_n\) die so genannte Hermitpolynome sind für die gilt:

\begin{equation}
  \label{eq:48}
  H_n(x) = (-1)^ne^{x^2}\frac{d^n}{dx^n}e^{-x^2}
\end{equation}



\subsection*{Referenzen}

\begin{itemize}
\item Schwabl - Quantenmechanik
\item Zettili - Quantum Mechanics
\end{itemize}


\end{document}
