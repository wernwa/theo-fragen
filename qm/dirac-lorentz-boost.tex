\input{../headers/header_script.tex}
\usepackage{amsmath} 



\begin{document}

\section*{Ebene Wellen Lösung der Dirac-Gleichung}

Für ein freies Teilchen sind Ebene Wellen die Lösung der Dirac-Gleichung. Sie haben folgende Form

\begin{align}
  \label{eq:1}
  \psi(x) = e^{-\frac{i}{\hbar}p\cdot x}w_r(\vec p) \qquad r=1,2,3,4
 \end{align}
Wobei \(x_\mu=(ct,-\vec x)\) der Vierer-Orts-Verktor und \(p^{\mu}=(\frac{E}{c},\vec p)\) der Vierer-Impuls-Vektor und \(w(p)\) die Impulsabhängige Spinor-Komponente ist. Zunäst betrachten wir ein Spezialfall indem wir das Teilchen in seinem Ruhesystem betrachten. 


Setzen wir den Ansatz \eqref{eq:1} in die Dirac-Gleichung ein

\begin{align}
  \label{eq:3}
  \left(i \gamma^\mu \partial_\mu-\frac{mc}{\hbar}  \right)\psi(x) &= 0 \qquad\text{ mit} \eqref{eq:1}   \notag \\
 i \gamma^\mu \partial_\mu  e^{-\frac{i}{\hbar}p\cdot x}w_r(\vec p) -\frac{mc}{\hbar} e^{-\frac{i}{\hbar}p\cdot x}w_r(\vec p)  &= 0  \notag \\
 i \gamma^\mu (-\frac{ip_\mu}{\hbar}) e^{-\frac{i}{\hbar}p\cdot x}w_r(\vec p) -\frac{mc}{\hbar} e^{-\frac{i}{\hbar}p\cdot x}w_r(\vec p)  &= 0  \notag \\
 \left(i \gamma^\mu (-\frac{ip_\mu}{\hbar}) -\frac{mc}{\hbar}\right) e^{-\frac{i}{\hbar}p\cdot x}w_r(\vec p)  &= 0  \notag \\
 \left(\underbr{ \gamma^\mu p_\mu}_{\cancel p} - mc\right)\underbr{ e^{-\frac{i}{\hbar}p\cdot x}w_r(\vec p)}_{\psi(x)}  &= 0 
\end{align}

Damit erhält man die Dirac-Gleichung in einer verkürtzen Schreibweise

\begin{align}
  \label{eq:4}
 \boxed{ \left(\cancel p - mc\right)\psi(x) = 0 }\qquad \text{mit der Notation: }\cancel p = \gamma^\mu p_\mu
\end{align}


Wir betrachten zuerst das Teilchen in seinem Ruhesystem. Für ein Teilchen in Ruhe gilt \(\vec p=0\). Dann Sieht die Lösung \eqref{eq:1} folgendermaßen aus

\begin{align}
  \label{eq:2}
  \psi(x) =  e^{-\frac{i}{\hbar}p^0\cdot x_0}w_r(0) = e^{-\frac{i}{\hbar}\frac{E}{c}ct  }w_r(0) =e^{-\frac{i}{\hbar}Et  }w_r(0)
\end{align}
Und die Dirac-Gleichung vereinfacht sich zu

\begin{align}
  \label{eq:5}
  \left(\gamma^0p_0 - mc\right) \cancel{e^{-\frac{i}{\hbar}Et  }} w_r(0)  &= 0 \notag\\
 \left(\gamma^0p_0 - mc\right) w_r(0)  &= 0 
\end{align}

Die Matrix \(\gamma^0\) ist

\begin{align}
  \label{eq:6}
  \gamma^0 = \beta =
  \begin{pmatrix}
    1&0&0&0\\
    0&1&0&0\\ 
    0&0&-1&0\\
    0&0&0&-1\\
  \end{pmatrix}
\end{align}

Und \(p_0=\frac{E}{c}\) eingesetzt in Gleichung \eqref{eq:5}

\begin{align}
  \label{eq:7}
   \left( 
  \begin{pmatrix}
    1&0&0&0\\
    0&1&0&0\\ 
    0&0&-1&0\\
    0&0&0&-1
  \end{pmatrix}\frac{E}{c} 
-  \begin{pmatrix}
    1&0&0&0\\
    0&1&0&0\\ 
    0&0&1&0\\
    0&0&0&1
  \end{pmatrix}
  mc\right) w_r(0)  &= 0 \notag \\
 \begin{pmatrix}
    \frac{E}{c} - mc&0&0&0\\
    0&\frac{E}{c} - mc&0&0\\ 
    0&0&-\frac{E}{c} - mc&0\\
    0&0&0&-\frac{E}{c} - mc
  \end{pmatrix} w_r(0) &=0
\end{align}

Die Gleichung hat 4 Lösungen zu 2 Eigenwerten mit \(E=\pm mc^2\). Die Lösungen für den Eigenwert \(E=+mc^2\) lauten

\begin{align}
  \label{eq:8}
  w_1(0) =
  \begin{pmatrix}
    1\\0\\0\\0
  \end{pmatrix}\quad \text{für ein \textit{Teilchen} mit Spin } \uparrow
\qquad 
 w_2(0) =
  \begin{pmatrix}
    0\\1\\0\\0
  \end{pmatrix}\quad \text{für ein \textit{Teilchen} mit Spin } \downarrow
\end{align}

und für den Eigenwert \(E=-mc^2\)

\begin{align}
  \label{eq:9}
   w_3(0) =
  \begin{pmatrix}
    0\\0\\1\\0
  \end{pmatrix}\quad \text{für ein \textit{\underline{Anti}-Teilchen} mit Spin } \uparrow
\qquad 
 w_4(0) =
  \begin{pmatrix}
    0\\0\\0\\1
  \end{pmatrix}\quad \text{für ein \textit{\underline{Anti}-Teilchen} mit Spin } \downarrow
\end{align}
Die Lösung für negative Energien spricht für die Existenz von Antiteilchen. Eine Allgemeine Lösung für ein Elektron mit Spin \(\uparrow\) in seinem Ruhesystem lautet beispielsweise

\begin{align}
  \label{eq:10}
  \psi(x) = e^{-\frac{i}{\hbar}Et  }w_1(0)
\end{align}

Nun möchten wir die Wellenfunktion in das Intertialsystem mit \(\vec p\ne 0\) transformieren. Dazu benötigen wir die Lorenz-Dirac-Spinor-Transformation 

\begin{align}
  \label{eq:11}
  \psi'(x) = S(\Lambda)\psi(x)
\end{align}

Es gilt also die \(S(\Lambda)\)-Matrix zu bestimmen. Die \(S(\Lambda)\)-Matrix ist allgemein wie folgt definiert

\begin{align}
  \label{eq:12}
  S(\Lambda) = e^{-\frac{i}{4}\omega^{\mu\nu}\sigma_{\mu\nu}}
\end{align}
Die \(\omega\)-Matix für ein Boost in eine beliebige Richtung wie folgt aussieht

\begin{align}
  \label{eq:13}
  \omega^{\mu\nu} = \omega
  \begin{pmatrix}
    0&n_1&n_2&n_3\\
    -n_1&0&0&0\\ 
    -n_2&0&0&0\\
    -n_3&0&0&0
  \end{pmatrix} \qquad \text{mit }\omega=|\vec v| ?
\end{align}
Wobei \(\vec n = (n_1,n_2,n_3)\) ein Einheitsvektor in Richtung des Boosts ist mit \(n^2=1\). Machen wir nun eine Nebenrechnung

\begin{align}
  \label{eq:14}
  \omega^{\mu\nu}\sigma_{\mu\nu} =& \omega^{0\mu}\sigma_{0\mu} + \omega^{1\mu}\sigma_{1\mu}+\omega^{2\mu}\sigma_{2\mu}+\omega^{3\mu}\sigma_{3\mu}\notag \\
=&\underbr{\omega^{00}\sigma_{00}}_{=0} +  \omega^{01}\sigma_{01} +  \omega^{02}\sigma_{02} +  \omega^{03}\sigma_{03} \notag\\
+& \omega^{10}\sigma_{10}+\underbr{ \omega^{11}\sigma_{11}+ \omega^{12}\sigma_{12}+ \omega^{13}\sigma_{13}}_{=0}\notag\\
+&\omega^{20}\sigma_{20}+\underbr{\omega^{21}\sigma_{21}+\omega^{22}\sigma_{22}+\omega^{23}\sigma_{23}}_{=0}\notag\\
+&\omega^{30}\sigma_{30}+\underbr{\omega^{31}\sigma_{31}+\omega^{32}\sigma_{32}+\omega^{33}\sigma_{33}}_{=0}
\end{align}
Nach Anwenden der einsteinischen Summenkonvention sieht man in der Gleichung (\ref{eq:14}) die Matrixstruktur aus der Gleichung (\ref{eq:13}). Lässt man die Null-Elemente weg verkürtzt sich die Gleichung auf

\begin{align}
  \label{eq:15}
  \omega^{\mu\nu}\sigma_{\mu\nu} &= \omega^{01}\sigma_{01} +  \omega^{02}\sigma_{02} +  \omega^{03}\sigma_{03}+\omega^{10}\sigma_{10}+\omega^{20}\sigma_{20}+\omega^{30}\sigma_{30} \notag\\
&= \sum_{i=1}^3 \omega^{0i}\sigma_{0i} + \sum_{j=1}^3 \omega^{j0}\sigma_{j0} \qquad \text{mit }\omega^{j0}=-\omega^{0j} \notag\\
&= \sum_{i=1}^3 \omega^{0i}\sigma_{0i} - \sum_{j=1}^3 \omega^{0j}\sigma_{j0} \qquad \text{mit }\sigma_{j0}=-\sigma_{0j} \notag\\
&= \sum_{i=1}^3 \omega^{0i}\sigma_{0i} + \sum_{j=1}^3 \omega^{0j}\sigma_{0j} \notag\\
&= 2\sum_{i=1}^3 \omega^{0i}\sigma_{0i}
\end{align}

Als eine weitere Nebenrechnung wollen wir \(\sigma_{0i}\) bestimmen. Allgemein gilt

\begin{align}
  \label{eq:16}
  \sigma_{\mu\nu}=\frac{i}{2}[\gamma_\mu,\gamma_\nu] 
\end{align}

In unserem Fall benötigen wir für \(\mu\) nur die Nullte Komponente und für \(\nu\) zählen wir nur von 1 bis 3. D.h. wir können schreiben

\begin{align}
  \label{eq:17}
  \sigma_{0i}= \frac{i}{2}[\gamma_0,\gamma_i]
\end{align}

Die Gamma Matrix in kovarianter Form lautet

\begin{align}
  \label{eq:18}
  \gamma_\mu = g_{\mu\nu}\gamma^{\nu} = (\beta,-\beta\vec \alpha)
\end{align}
Dies eingesetzt in (\ref{eq:17}) ergibt

\begin{align}
  \label{eq:19}
   \sigma_{0i} &= \frac{i}{2}[\beta,-\beta\alpha_i] \notag\\
&= -\frac{i}{2} ( \beta\beta\alpha_i -\beta\alpha_i\beta  ) \qquad \text{mit }\{\beta,\alpha_i\}=0\rightarrow \alpha_i\beta=-\beta\alpha_i  \notag\\
&= -\frac{i}{2} ( \underbr{\beta\beta}_{\mathds 1}\alpha_i + \underbr{\beta\beta}_{\mathds 1}\alpha_i) \qquad \text{mit }\beta^2 = \mathds 1_4 \notag\\
&= - i \alpha_i
\end{align}

Setzen wir diese Gleichung (\ref{eq:19}) in die Gleichung (\ref{eq:15}) ein und ersetzen \(\omega^{0i}\) mit \(\omega n_i\) so können wir die Summe \(\sum_{i=1}^3 \omega^{0i}\sigma_{0i}\) als ein Skalarprodukt schreiben

\begin{align}
  \label{eq:21}
  \omega^{\mu\nu}\sigma_{\mu\nu} &= -i2\omega\,\vec n\cdot\vec \alpha
\end{align}

Nun können wir endlich die \(S(\Lambda)\)-Matrix berechnen indem wir die Gleichung (\ref{eq:21}) in (\ref{eq:12}) einsetzen

\begin{align}
  \label{eq:22}
  S(\Lambda) &= e^{-\frac{i}{4}\omega^{\mu\nu}\sigma_{\mu\nu}} \notag \\
&=e^{-\frac{1}{2}\omega\,\vec n\cdot\vec \alpha  }
\end{align}
Das Teilchen bewegt sich mit dem Impuls \(\vec p\) in einem Inertialsystem. Der Boost zeigt dabei in die entgegengesetze Richtung \(-\vec p\). Desweiteren gilt \(\vec n = - \hat v = - \hat p \). Diese Bedinung eigesetzt in die Gleichung (\ref{eq:22})

\begin{align}
  \label{eq:23}
  S(\Lambda) &=e^{\frac{1}{2}\omega\,\hat p\cdot\vec \alpha  } \notag \\
&= \cosh\left(\frac{1}{2}\omega\,\hat p\cdot\vec \alpha\right) + \sinh\left(\frac{1}{2}\omega\,\hat p\cdot\vec \alpha \right)
\end{align}
Mit der Entwicklung für \(\cosh(x) = \sum_{n=0}^{\infty}\frac{x^{2n}}{(2n)!}\) und  \(\sinh(x) = \sum_{n=0}^{\infty}\frac{x^{2n+1}}{(2n+1)!}\) lautet die Gleichung weiterhin

\begin{align}
  \label{eq:24}
  S(\Lambda) &=\sum_{n=0}^{\infty}\frac{\left(\frac{1}{2}\omega\,\hat p\cdot\vec \alpha\right)^{2n}}{(2n)!} + \sum_{n=0}^{\infty}\frac{\left(\frac{1}{2}\omega\,\hat p\cdot\vec \alpha\right)^{2n+1}}{(2n+1)!}
\end{align}

Nun wollen wir herausfinden wie sich das Skalarprodukt von \(\hat p\cdot\vec \alpha\) bei verschiedenen Potenzen verhält. Für ungerade Potenzen gilt

\begin{align}
  \label{eq:25}
  \hat p\cdot\vec \alpha =
  \begin{pmatrix}
    0&\hat p\cdot\vec \sigma\\
    \hat p\cdot\vec \sigma&0
  \end{pmatrix} = (\hat p\cdot\vec \alpha)^3 =(\hat p\cdot\vec \alpha)^5 = (\hat p\cdot\vec \alpha)^{2n+1}
\end{align}
Für gerade Potenzen

\begin{align}
  \label{eq:26}
  (\hat p\cdot\vec \alpha)^2 =\hat p^2\cdot\vec \alpha^2 = \underbr{\hat p^2}_{=1}\cdot
  \begin{pmatrix}
    \vec \sigma^2&0\\
    0&\vec \sigma^2
  \end{pmatrix} = \mathds 1 = (\hat p\cdot\vec \alpha)^4 =(\hat p\cdot\vec \alpha)^6 = (\hat p\cdot\vec \alpha)^{2n}
\end{align}
Damit können wir die Matritzen aus den Entwicklungen von \(\cosh\) und \(\sinh\) ausklammern

\begin{align}
  \label{eq:27}
   S(\Lambda)  &=\sum_{n=0}^{\infty}\frac{(\frac{\omega}{2})^{2n}}{(2n)!}\underbr{(\hat p\cdot\vec \alpha)^{2n}}_{\mathds 1} +
   \sum_{n=0}^{\infty}\frac{(\frac{\omega}{2})^{2n+1}}{(2n+1)!}\underbr{(\hat p\cdot\vec \alpha)^{2n+1}}_{\hat p\cdot\vec\alpha}\notag\\
 &=\mathds 1\cdot \sum_{n=0}^{\infty}\frac{ (\frac{\omega}{2})^{2n}}{(2n)!} +
  \hat p\cdot\vec\alpha \sum_{n=0}^{\infty}\frac{(\frac{\omega}{2})^{2n+1} }{(2n+1)!}\notag\\
&= \mathds 1 \cosh\left(\frac{\omega}{2}\right) + \hat p\cdot\vec \alpha \sinh\left(\frac{\omega}{2} \right)
\end{align}

Um \(\cosh\) und \(\sinh\) durch messbare Größen auszudrücken, führen wir den Begriff der \textbf{Rapidität} ein. Sie ist Definiert

\begin{align}
  \label{eq:20}
  \boxed{ \omega = \text{arctanh}\, \frac{v}{c} = \text{arctanh}\,\beta }
\end{align}
Die Rapidität ist ein alternatives Maß für Geschwindigkeit und hat den Vorteil, dass zwei Rapiditäten einfach addiert werden können, während man bei Geschwindigkeiten das relativistische Additionstheorem verwenden muss. Damit können wir schreiben

\begin{align}
  \label{eq:28}
  \tanh\omega = \beta
\end{align}
Desweiteren gilt

\begin{align}
  \label{eq:29}
  \cosh^2 \omega - \sinh^2\omega = 1
\end{align}

Mit \(\tanh\omega = \frac{\sinh\omega}{\cosh\omega}\) eingesetzt in \eqref{eq:29} folgt

\begin{align}
  \label{eq:30}
  \cosh^2\omega - \tanh^2\omega\cosh^2\omega = 1 \notag\\
  \cosh^2\omega(1- \tanh^2\omega) = 1\notag\\
\Leftrightarrow \cosh\omega = \frac{1}{\sqrt{1-\tanh^2\omega}}
\end{align}
Setze die Gleichung \eqref{eq:28} in \eqref{eq:30} folgt

\begin{align}
  \label{eq:31}
   \boxed{\cosh\omega = \frac{1}{\sqrt{1-\beta^2}} = \gamma }
\end{align}
Aus der Beziehung

\begin{align}
  \label{eq:32}
  \sinh\omega = \tanh\omega\cosh\omega
\end{align}
Die Gleichung \eqref{eq:28} und \eqref{eq:31} in \eqref{eq:32} eingesetzt ergibt

\begin{align}
  \label{eq:33}
  \boxed{\sinh\omega = \beta\gamma}
\end{align}

Nun möchten wir zunächst den \(\cosh\) in der Gleichung \eqref{eq:27} durch die Energie ausdrücken

\begin{align}
  \label{eq:34}
  \cosh\frac{\omega}{2} = \sqrt{\frac{1+\cosh\omega}{2}} = \sqrt{\frac{1+\gamma}{2}} 
\end{align}

Aus der Beziehung

\begin{align}
  \label{eq:35}
  E=\gamma mc^2 \Leftrightarrow \gamma = \frac{E}{mc^2}
\end{align}

in Gleichung \eqref{eq:34} eingesetzt

\begin{align}
  \label{eq:36}
   \cosh\frac{\omega}{2} = \sqrt{\frac{E + mc^2}{2mc^2}} 
\end{align}
Für den \(\sinh\)-Term gilt

\begin{align}
  \label{eq:37}
  \sinh\frac{\omega}{2} &=  \sqrt{\underbr{ \cosh^2\frac{\omega}{2}}_{~\eqref{eq:36}} - 1} =  \sqrt{\frac{E + mc^2}{2mc^2}  -1} =   \sqrt{\frac{E + mc^2}{2mc^2}  -\frac{2mc^2}{2mc^2}} =  \sqrt{\frac{E - mc^2}{2mc^2} } \notag \\
&=  \sqrt{\frac{E - mc^2}{2mc^2} \frac{E+mc^2}{E+mc^2}} =  \sqrt{\frac{E^2 - m^2c^4}{2mc^2(E+mc^2)} }
\end{align}
Mit \(E^2=p^2c^2+m^2c^4 \leftrightarrow E^2-m^2c^4 = p^2c^2\) in \eqref{eq:37} eingesetzt

\begin{align}
  \label{eq:38}
   \sinh\frac{\omega}{2} &=  \sqrt{\frac{p^2c^2}{2mc^2(E+mc^2)} \frac{E+mc^2}{E+mc^2} } = \sqrt{\frac{E+mc^2}{2mc^2}}\cdot \frac{pc}{E+mc^2}
\end{align}

Nun können wir endlich die \(S(\Lambda)\)-Matrix hinschreiben, indem wir die Gleichungen \eqref{eq:36} und \eqref{eq:38} in \eqref{eq:27} einsetzen

\begin{align}
  \label{eq:39}
   S(\Lambda) &= \mathds 1 \cosh\left(\frac{\omega}{2}\right) + \hat p\cdot\vec \alpha \sinh\left(\frac{\omega}{2} \right) \notag \\
&= \mathds 1 \sqrt{\frac{E + mc^2}{2mc^2}} +  \hat p\cdot\vec \alpha \sqrt{\frac{E+mc^2}{2mc^2}}\cdot \frac{pc}{E+mc^2} \notag \\
&=  \sqrt{\frac{E + mc^2}{2mc^2}}\left(\mathds 1 +  \frac{c}{E+mc^2}\vec p\cdot\vec\alpha\right)  \notag \\
&=  \sqrt{\frac{E + mc^2}{2mc^2}}\left(\mathds 1 +  \frac{c}{E+mc^2}( p_x\alpha_1+p_y\alpha_2+p_z\alpha_3) \right)  \notag \\
\end{align}
Mit den \(\alpha\)-Matritzen

\begin{align}
  \label{eq:40}
  \alpha_1 = \begin{pmatrix}
   0&0&0&1\\
   0&0&1&0\\
 0&1&0&0\\
 1&0&0&0
  \end{pmatrix}\quad
 \alpha_2 = \begin{pmatrix}
   0&0&0&-i\\
   0&0&i&0\\
 0&-i&0&0\\
 i&0&0&0
  \end{pmatrix}\quad
 \alpha_3 = \begin{pmatrix}
   0&0&1&0\\
   0&0&0&-1\\
 1&0&0&0\\
 0&-1&0&0
  \end{pmatrix}
\end{align}

folgt für den Ausdruck

\begin{align}
  \label{eq:41}
  p_x\alpha_1+p_y\alpha_2+p_z\alpha_3 &=
  \begin{pmatrix}
     0&0&0&p_x\\
   0&0&p_x&0\\
 0&p_x&0&0\\
 p_x&0&0&0
  \end{pmatrix}+ \begin{pmatrix}
   0&0&0&-ip_y\\
   0&0&ip_y&0\\
 0&-ip_y&0&0\\
 ip_y&0&0&0
  \end{pmatrix}+
 \begin{pmatrix}
   0&0&p_z&0\\
   0&0&0&-p_z\\
 p_z&0&0&0\\
 0&-p_z&0&0
  \end{pmatrix}\notag\\
&=  \begin{pmatrix}
     0&0&p_z&p_x-ip_y\\
   0&0&p_x+ip_y&-p_z\\
 p_z&p_x-ip_y&0&0\\
 p_x+ip_y&-p_z&0&0
  \end{pmatrix}
\end{align}

Wieder eingesetzt in \eqref{eq:39} folgt nun schlußendlich Transformationsmatrix

\begin{align}
  \label{eq:42}
\boxed{
  S(\Lambda)=\sqrt{\frac{E + mc^2}{2mc^2}}
  \begin{pmatrix}
    1&0&\frac{cp_z}{E+mc^2}&\frac{c(p_x-ip_y)}{E+mc^2}\\
    0&1&\frac{c(p_x+ip_y)}{E+mc^2}&\frac{-cp_z}{E+mc^2}\\
    \frac{cp_z}{E+mc^2}&\frac{c(p_x-ip_y)}{E+mc^2}&1&0\\
    \frac{c(p_x+ip_y)}{E+mc^2}&\frac{-cp_z}{E+mc^2}&0&1
  \end{pmatrix}}
\end{align}
\subsection*{Referenzen}
\begin{itemize}
\item TODO
\end{itemize}
  
\end{document}
