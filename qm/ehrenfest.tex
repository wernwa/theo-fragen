\input{../headers/header_script.tex}
%\includegraphics[width=0.75\textwidth]{thepic.png}

\begin{document}

\textit{29. März 2012}
\input{../headers/authors.tex}

\section{Ehrenfest-Theorem}

Das Ehrenfest-Theorem besagt, dass unter bestimmten Bedingungen die klassischen Bewegungsgleichungen für die Mittelwerte der Quantenmechanik gelten.

\subsection{Herleitung}

Betrachte die Heisenberg-Bewegungsgleichung:

\[ \frac{d}{dt} O = \frac{i}{\hbar} [H,O] +\frac{\partial}{\partial t} O \]

Die Erwartungswerte der einzelnen Operatoren durch Mittlung der Gleichung:

\begin{equation} 
\label{eq:41}
\boxed{\frac{d}{dt} \langle O\rangle  = \frac{i}{\hbar} \langle [H,O]\rangle  + \langle  \frac{\partial}{\partial t}  O\rangle } 
\end{equation}

Alternative Herleitung der Gleichung \eqref{eq:41} :

\begin{align}
\frac{d}{dt}\langle  O\rangle  &=  \frac{d}{dt} \int \psi^* O \psi dx \\
&=  \int \left[\left(\frac{\partial}{\partial t} \psi^*\right) O\psi +  \psi^*  \left(\frac{\partial}{\partial t} O\right) \psi + \psi^* O\left(\frac{\partial}{\partial t} \psi\right)\right]dx\\
&=  \int \left[\left(\frac{\partial}{\partial t} \psi^*\right) O\psi  + \psi^* O\left(\frac{\partial}{\partial t} \psi\right)\right]dx + \int  \psi^*  \left(\frac{\partial}{\partial t} O\right) \psi dx \\
&=  \int \left[\left(\frac{\partial}{\partial t} \psi^*\right) O\psi  + \psi^* O\left(\frac{\partial}{\partial t} \psi\right)\right]dx +  \left\langle  \frac{\partial}{\partial t} O\right\rangle \label{eq:40}
\end{align}

Nun betrachtet man die Schrödinger und die konjugierte Gleichung :

\[ \frac{\partial \psi}{\partial t} = -\frac{i}{\hbar}H\psi \qquad \frac{\partial \psi^*}{\partial t} = \frac{i}{\hbar}\psi^* H   \]

und eingesetzt in die Gleichung \eqref{eq:40} ergibt:

\begin{align}
\frac{d}{dt}\langle  O\rangle  &=  \int \left[\left(\frac{i}{\hbar} \psi^* H\right) O\psi  + \psi^* O\left(-\frac{i}{\hbar} H\psi\right)\right]dx +  \left\langle  \frac{\partial}{\partial t} O\right\rangle\\
&=  \frac{i}{\hbar} \int ( \psi^* H O\psi  - \psi^* O H\psi)dx +  \left\langle  \frac{\partial}{\partial t} O\right\rangle\\
&=  \frac{i}{\hbar} \int \psi^* (  H O  -  O H )\psi dx +  \left\langle  \frac{\partial}{\partial t} O\right\rangle\\
&=  \frac{i}{\hbar} \int \psi^* [H, O]\psi dx +  \left\langle  \frac{\partial}{\partial t} O\right\rangle\\
&=  \frac{i}{\hbar} \left\langle [H, O]\right\rangle +  \left\langle  \frac{\partial}{\partial t} O\right\rangle
\end{align}






Nun können wir die Zeitlichen Ableitungen vom Impuls und Ort Erwartungswert berechnen mit der Annhame dass gilt \(\frac{\partial p}{\partial t} = \frac{\partial x}{\partial t} =  0 \)


\begin{align}
 \frac{d}{dt} \langle p \rangle  &= \frac{i}{\hbar} \langle [H,p]\rangle \\
&= \frac{i}{\hbar} \langle [\frac{p^2}{2m}+V(x),p]\rangle   \\
&= \frac{i}{\hbar} \langle [V(x),p]\rangle   \\
&= \frac{i}{\hbar} \langle [V(x),\frac{\hbar}{i}\frac{d}{dx} ]\rangle   \\
\end{align}
NR:
 \begin{align}
[V(x),\frac{\hbar}{i}\frac{d}{dx} ]\psi &=\frac{\hbar}{i} V(x)\frac{d}{dx} \psi - \frac{\hbar}{i}\frac{d}{dx}  (V(x)\psi)\\
&= \frac{\hbar}{i} V(x)\frac{d}{dx} \psi -\frac{\hbar}{i}(\psi(x) (\frac{d}{dx}  V(x)) + V(x) (\frac{d}{dx} \psi(x))  )  \\
&= \cancel{\frac{\hbar}{i} V(x)\frac{d}{dx} \psi} -\frac{\hbar}{i}\psi(x) (\frac{d}{dx}  V(x))\cancel{ -\frac{\hbar}{i} V(x) (\frac{d}{dx} \psi(x))}   \\
&=-\frac{\hbar}{i}\psi(x) (\frac{d}{dx}  V(x))
\end{align}
\[\Rightarrow [V(x),\frac{\hbar}{i}\frac{d}{dx} ] = -\frac{\hbar}{i} \frac{d}{dx}  V(x) \]


\begin{align}
\rightarrow \frac{d}{dt} \langle p \rangle  &=  -\langle  \frac{d}{dx}  V(x) \rangle = \langle F(x) \rangle  
\end{align}

Und nun für den Ort:

\begin{align}
 \frac{d}{dt} \langle x \rangle  &= \frac{i}{\hbar} \langle [H,x]\rangle \\
&= \frac{i}{\hbar} \langle [\frac{p^2}{2m}+V(x),x]\rangle   \\
&= \frac{i}{\hbar}\frac{1}{2m} \langle [p^2,x]\rangle   \\
&= \frac{i}{\hbar}\frac{1}{2m} \langle p\underbrace{ [p,x]}_{-i\hbar}+\underbrace{ [p,x]}_{-i\hbar}p\rangle   \\
&= \frac{i}{\hbar}\frac{1}{2m} \langle -2i\hbar p\rangle   \\
&= \frac{1}{m} \langle p \rangle   \\
\end{align}

Zusammengefasst:

\[ m \frac{d}{dt} \langle x \rangle  = \langle p \rangle \qquad \frac{d}{dt} \langle p \rangle  =  -\langle  \frac{d}{dx}  V(x) \rangle \]


\begin{equation} 
\label{eq:42}
\Rightarrow m \frac{d^2}{dt^2} \langle x \rangle  =  -\langle  \frac{d}{dx}  V(x) \rangle =  \langle F(x) \rangle 
\end{equation}

Entwickle den Erwartungswert der Kraft \(\langle F(x)\rangle \) nach Taylor um dem Erwartungswert des Ortes \(\langle x\rangle \):

\begin{align}
 \langle F(x) \rangle = \left\langle  F(\langle x\rangle ) + (x-\langle x\rangle) \frac{\partial }{\partial x} F(\langle x\rangle )+(x-\langle x\rangle)^2 \frac{\partial^2 }{\partial x^2} F(\langle x\rangle )\frac{1}{2!}+\dots \right\rangle 
\end{align}


NR: 
\begin{align}
\left\langle(x-\langle x\rangle) \right\rangle = \langle x\rangle -\langle \langle x\rangle \rangle  = \langle x\rangle -\langle x\rangle  = 0
\end{align}
Und \( \langle F(\langle x \rangle )\rangle  = F(\langle x \rangle )  \) da \(F(\langle x\rangle )\) eine Zahl ist und Erwartungswert einer Zahl ist wieder die Zahl selbst.

\begin{align}
 \langle F(x) \rangle =   F(\langle x\rangle ) + \left\langle (x-\langle x\rangle)^2 \frac{\partial^2 }{\partial x^2} F(\langle x\rangle )\frac{1}{2!}+\dots \right\rangle 
\end{align}

Wir sehen: \(\langle F(x)\rangle = F(\langle x\rangle )\) genau dann, wenn die 2. und höhere Ableitungen der Kraft nach dem Ort verschwinden. Dies ist für Systeme wie dem freien Teilchen, da hier die Kraft überall Null ist und dem Harmonischen Oszillator \(F=-Dx\) wo die Kraft nur linear abhängt, der Fall. In diesen beiden Fällen können wir die Gleichung \eqref{eq:42} auch wie folgt schreiben:

\[ \boxed{m\frac{d^2}{dt^2} \langle x \rangle  = \langle F(x) \rangle = F(\langle x\rangle )  } \]

Was der quasi klassischen Näherung entspricht. Vergleiche mit dem Newtonschen Axiom \(m\frac{d^2}{dt^2} x  = F(x)  \)





\end{document}
