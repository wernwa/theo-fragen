\documentclass[10pt,a4paper,oneside,fleqn]{article}
\usepackage{geometry}
\geometry{a4paper,left=20mm,right=20mm,top=1cm,bottom=2cm}
\usepackage[utf8]{inputenc}
%\usepackage{ngerman}
\usepackage{amsmath}                % brauche ich um dir Formel zu umrahmen.
\usepackage{amsfonts}                % brauche ich für die Mengensymbole
\usepackage{graphicx}
\setlength{\parindent}{0px}
\setlength{\mathindent}{10mm}
\usepackage{bbold}                    %brauche ich für die doppel Zahlen Darstellung (Einheitsmatrix z.B)



\usepackage{color}
\usepackage{titlesec} %sudo apt-get install texlive-latex-extra

\definecolor{darkblue}{rgb}{0.1,0.1,0.55}
\definecolor{verydarkblue}{rgb}{0.1,0.1,0.35}
\definecolor{darkred}{rgb}{0.55,0.2,0.2}

%hyperref Link color
\usepackage[colorlinks=true,
        linkcolor=darkblue,
        citecolor=darkblue,
        filecolor=darkblue,
        pagecolor=darkblue,
        urlcolor=darkblue,
        bookmarks=true,
        bookmarksopen=true,
        bookmarksopenlevel=3,
        plainpages=false,
        pdfpagelabels=true]{hyperref}

\titleformat{\chapter}[display]{\color{darkred}\normalfont\huge\bfseries}{\chaptertitlename\
\thechapter}{20pt}{\Huge}

\titleformat{\section}{\color{darkblue}\normalfont\Large\bfseries}{\thesection}{1em}{}
\titleformat{\subsection}{\color{verydarkblue}\normalfont\large\bfseries}{\thesubsection}{1em}{}

% Notiz Box
\usepackage{fancybox}
\newcommand{\notiz}[1]{\vspace{5mm}\ovalbox{\begin{minipage}{1\textwidth}#1\end{minipage}}\vspace{5mm}}

\usepackage{cancel}
\setcounter{secnumdepth}{3}
\setcounter{tocdepth}{3}





%-------------------------------------------------------------------------------
%Diff-Makro:
%Das Diff-Makro stellt einen Differentialoperator da.
%
%Benutzung:
% \diff  ->  d
% \diff f  ->  df
% \diff^2 f  ->  d^2 f
% \diff_x  ->  d/dx
% \diff^2_x  ->  d^2/dx^2
% \diff f_x  ->  df/dx
% \diff^2 f_x  ->  d^2f/dx^2
% \diff^2{f(x^5)}_x  ->  d^2(f(x^5))/dx^2
%
%Ersetzt man \diff durch \pdiff, so wird der partieller
%Differentialoperator dargestellt.
%
\makeatletter
\def\diff@n^#1{\@ifnextchar{_}{\diff@n@d^#1}{\diff@n@fun^#1}}
\def\diff@n@d^#1_#2{\frac{\textrm{d}^#1}{\textrm{d}#2^#1}}
\def\diff@n@fun^#1#2{\@ifnextchar{_}{\diff@n@fun@d^#1#2}{\textrm{d}^#1#2}}
\def\diff@n@fun@d^#1#2_#3{\frac{\textrm{d}^#1 #2}{\textrm{d}#3^#1}}
\def\diff@one@d_#1{\frac{\textrm{d}}{\textrm{d}#1}}
\def\diff@one@fun#1{\@ifnextchar{_}{\diff@one@fun@d #1}{\textrm{d}#1}}
\def\diff@one@fun@d#1_#2{\frac{\textrm{d}#1}{\textrm{d}#2}}
\newcommand*{\diff}{\@ifnextchar{^}{\diff@n}
  {\@ifnextchar{_}{\diff@one@d}{\diff@one@fun}}}
%
%Partieller Diff-Operator.
\def\pdiff@n^#1{\@ifnextchar{_}{\pdiff@n@d^#1}{\pdiff@n@fun^#1}}
\def\pdiff@n@d^#1_#2{\frac{\partial^#1}{\partial#2^#1}}
\def\pdiff@n@fun^#1#2{\@ifnextchar{_}{\pdiff@n@fun@d^#1#2}{\partial^#1#2}}
\def\pdiff@n@fun@d^#1#2_#3{\frac{\partial^#1 #2}{\partial#3^#1}}
\def\pdiff@one@d_#1{\frac{\partial}{\partial #1}}
\def\pdiff@one@fun#1{\@ifnextchar{_}{\pdiff@one@fun@d #1}{\partial#1}}
\def\pdiff@one@fun@d#1_#2{\frac{\partial#1}{\partial#2}}
\newcommand*{\pdiff}{\@ifnextchar{^}{\pdiff@n}
  {\@ifnextchar{_}{\pdiff@one@d}{\pdiff@one@fun}}}
\makeatother
%
%Das gleich nur mit etwas andere Syntax. Die Potenz der Differentiation wird erst
%zum Schluss angegeben. Somit lautet die Syntax:
%
% \diff_x^2  ->  d^2/dx^2
% \diff f_x^2  ->  d^2f/dx^2
% \diff{f(x^5)}_x^2  ->  d^2(f(x^5))/dx^2
% Ansonsten wie Oben.
%
%Ersetzt man \diff durch \pdiff, so wird der partieller
%Differentialoperator dargestellt.
%
%\makeatletter
%\def\diff@#1{\@ifnextchar{_}{\diff@fun#1}{\textrm{d} #1}}
%\def\diff@one_#1{\@ifnextchar{^}{\diff@n{#1}}%
%  {\frac{\textrm d}{\textrm{d} #1}}}
%\def\diff@fun#1_#2{\@ifnextchar{^}{\diff@fun@n#1_#2}%
%  {\frac{\textrm d #1}{\textrm{d} #2}}}
%\def\diff@n#1^#2{\frac{\textrm d^#2}{\textrm{d}#1^#2}}
%\def\diff@fun@n#1_#2^#3{\frac{\textrm d^#3 #1}%
%  {\textrm{d}#2^#3}}
%\def\diff{\@ifnextchar{_}{\diff@one}{\diff@}}
%\newcommand*{\diff}{\@ifnextchar{_}{\diff@one}{\diff@}}
%
%Partieller Diff-Operator.
%\def\pdiff@#1{\@ifnextchar{_}{\pdiff@fun#1}{\partial #1}}
%\def\pdiff@one_#1{\@ifnextchar{^}{\pdiff@n{#1}}%
%  {\frac{\partial}{\partial #1}}}
%\def\pdiff@fun#1_#2{\@ifnextchar{^}{\pdiff@fun@n#1_#2}%
%  {\frac{\partial #1}{\partial #2}}}
%\def\pdiff@n#1^#2{\frac{\partial^#2}{\partial #1^#2}}
%\def\pdiff@fun@n#1_#2^#3{\frac{\partial^#3 #1}%
%  {\partial #2^#3}}
%\newcommand*{\pdiff}{\@ifnextchar{_}{\pdiff@one}{\pdiff@}}
%\makeatother

%-------------------------------------------------------------------------------
%%Nützliche Makros um in der Quantenmechanik Bras, Kets und das Skalarprodukt
%%zwischen den beiden darzustellen.
%%Benutzung:
%% \bra{x}  ->    < x |
%% \ket{x}  ->    | x >
%% \braket{x}{y} ->   < x | y >

\newcommand\bra[1]{\left\langle #1 \right|}
\newcommand\ket[1]{\left| #1 \right\rangle}
\newcommand\braket[2]{%
  \left\langle #1\vphantom{#2} \right.%
  \left|\vphantom{#1#2}\right.%
  \left. \vphantom{#1}#2 \right\rangle}%

%-------------------------------------------------------------------------------
%%Aus dem Buch:
%%Titel:  Latex in Naturwissenschaften und Mathematik
%%Autor:  Herbert Voß
%%Verlag: Franzis Verlag, 2006
%%ISBN:   3772374190, 9783772374197
%%
%%Hier werden drei Makros definiert:\mathllap, \mathclap und \mathrlap, welche
%%analog zu den aus Latex bekannten \rlap und \llap arbeiten, d.h. selbst
%%keinerlei horizontalen Platz benötigen, aber dennoch zentriert zum aktuellen
%%Punkt erscheinen.

\newcommand*\mathllap{\mathstrut\mathpalette\mathllapinternal}
\newcommand*\mathllapinternal[2]{\llap{$\mathsurround=0pt#1{#2}$}}
\newcommand*\clap[1]{\hbox to 0pt{\hss#1\hss}}
\newcommand*\mathclap{\mathpalette\mathclapinternal}
\newcommand*\mathclapinternal[2]{\clap{$\mathsurround=0pt#1{#2}$}}
\newcommand*\mathrlap{\mathpalette\mathrlapinternal}
\newcommand*\mathrlapinternal[2]{\rlap{$\mathsurround=0pt#1{#2}$}}

%%Das Gleiche nur mit \def statt \newcommand.
%\def\mathllap{\mathpalette\mathllapinternal}
%\def\mathllapinternal#1#2{%
%  \llap{$\mathsurround=0pt#1{#2}$}% $
%}
%\def\clap#1{\hbox to 0pt{\hss#1\hss}}
%\def\mathclap{\mathpalette\mathclapinternal}
%\def\mathclapinternal#1#2{%
%  \clap{$\mathsurround=0pt#1{#2}$}%
%}
%\def\mathrlap{\mathpalette\mathrlapinternal}
%\def\mathrlapinternal#1#2{%
%  \rlap{$\mathsurround=0pt#1{#2}$}% $
%}

%-------------------------------------------------------------------------------
%%Hier werden zwei neue Makros definiert \overbr und \underbr welche analog zu
%%\overbrace und \underbrace funktionieren jedoch die Gleichung nicht
%%'zerreißen'. Dies wird ermöglicht durch das \mathclap Makro.

\def\overbr#1^#2{\overbrace{#1}^{\mathclap{#2}}}
\def\underbr#1_#2{\underbrace{#1}_{\mathclap{#2}}}
\usepackage{amsmath}                % brauche ich um dir Formel zu umrahmen.
\usepackage{amsfonts}


\begin{document}

\section*{Quantisierung des Strahlungsfeldes}

Wir wollen das Vektorpotential \(\vec A\) als Operator ausdrücken. Dazu betrachten wir die Wellengleichung für eine Elektromagnetische Welle im Vakuum

\begin{align}
  \label{eq:1}
  \frac{1}{c^2}\pdiff^2{\vec A}_t - \vec \nabla^2\vec A = 0 \qquad \text{ mit }\square\equiv\frac{1}{c^2}\pdiff^2_t - \vec \nabla^2\quad\to\quad\square\vec A = 0
\end{align}

Eine Lösung für die Wellengleichung (\ref{eq:1}) wäre z.B. eine \textit{ebene Welle}

\begin{align}
  \label{eq:2}
  \vec A(\vec r, t) = \alpha\hat \epsilon e^{i(\vec k\vec r -\omega t)} + \alpha^*\hat \epsilon^* e^{-i(\vec k\vec r-\omega t)} 
\end{align}

Die Vektoren \(\hat \epsilon\) sind die sogenannten Polarisationsvektoren der Welle. Sie stehen senkrecht zur Ausbreitungsrichtung \(\vec k\). Deswegen bleiben nur noch zwei Raumrichtungen in die sie zeigen können übrig. Betrachte z.B. eine linear polarisierte ebene Welle die sich in z-Richtung ausbreitet. Damit lautet der Wellevektor \(\vec k = (0,0,k)^T\). Für den Polarisationsvektor bleiben zwei Möglichkeiten übrig. Entweder zeigt er in x-Richtung mit \(\hat \epsilon=(1,0,0)^T\) oder in y-Richtung \(\hat \epsilon=(0,1,0)^T\).

Wir betrachten ein Strahlungsfeld mit allen möglichen \(\vec k\)-Werten. Das bedeutet eine Überlagerung aller \(\vec k\)-Werte. Dazu müssen wir die Gleichung (\ref{eq:2}) über \(\vec k\) integrieren. Dabei müssen wir noch über zwei Moden \(m=1,2\) Summieren. Diese Stehen für das elektrische und magnetische Feld.

\begin{align}
  \label{eq:3}
\vec A(\vec r, t) =\sum_{m=1,2}\int\frac{d^3k}{(2\pi)^3} \left[  \vec{\mathcal A}_m (\vec k) \hat \epsilon_m (\vec k) e^{i(\vec k\vec r-\omega t)} + \vec{\mathcal A}^*_m (k)\hat\epsilon^*_m(\vec k) e^{-i(\vec k\vec r-\omega t)}\right]
\end{align}

Die Integration entspricht einer Fouriertransformation aus dem Impulsraum in den Ortsraum. Die Vorfaktoren \(\vec{\mathcal A}_m(k)\) sind nichts anderes als die fouriertransformierte Vektorpotential im Impulsraum. Der Faktor \(\frac{1}{(2\pi)^2}\) ist eine Konvention und dient zur Normierung der Fouriertransformation.

Nun möchten wir das Vektorfeld \(\vec A(\vec r, t)\) quantisieren. D.h. \(\vec A(\vec r, t)\) ist nicht mehr eine kontinuierliche Größe, sondern darf nur bestimmte diskrete Werte annehmen. Dazu betrachten wir einen dreidimensionale Kasten mit der Kantenlänge \(L\), der mit stehenden Wellen gefüllt ist. Die Randbedingungen für eine stehende Welle die z.B. in \(x\)-Richtung läuft muss heißen

\begin{align}
  \label{eq:4}
  \vec A(x,y,z) = \vec A(x+L,y,z)
\end{align}

Diese Bedingung an die Gleichung (\ref{eq:3}) angewandt

\begin{align}
  \label{eq:5}
  \alpha\hat \epsilon e^{i(k_x x+k_y y+k_z z -\omega t)} + \alpha^*\hat \epsilon^* e^{-i(k_x x+k_y y+k_z z-\omega t)} &\stackrel{!}= \alpha\hat \epsilon e^{i(k_x (x+L)+k_y y+k_z z -\omega t)} + \alpha^*\hat \epsilon^* e^{-i(k_x (x+L)+k_y y+k_z z-\omega t)} \notag\\
\end{align}

Das Istgleichzeichen ist nur dann erfüllt wenn die realen Anteile der Gleichung den komplexkonjugierten Anteilen gleichen


\begin{subequations}
\begin{align}
  \alpha\hat \epsilon e^{i(k_x x+k_y y+k_z z -\omega t)} &\stackrel{!}= \alpha\hat \epsilon e^{i(k_x (x+L)+k_y y+k_z z -\omega t)}\label{eq:6a} \\
\text{und }\alpha^*\hat \epsilon^* e^{-i(k_x x+k_y y+k_z z-\omega t)} &\stackrel{!}=\alpha^*\hat \epsilon^* e^{-i(k_x (x+L)+k_y y+k_z z-\omega t)} \label{eq:6b} \\
\end{align}
\end{subequations}

Um eine Bedingung für \(\vec k\) zu bestimmen betrachte die Gleichung (\ref{eq:6a}) 

\begin{align}
  \label{eq:7}
   \cancel{\alpha\hat \epsilon} e^{i(k_x x)}\cancel{e^{i(k_y y+k_z z -\omega t)}} &= \cancel{\alpha\hat \epsilon} e^{i(k_x (x+L)}\cancel{e^{i(k_y y+k_z z -\omega t)}} \notag\\
    \cancel{e^{i(k_x x)}} &=  e^{ik_x (x+L)} = \cancel{e^{ik_x x}}e^{i k_x L}
\end{align}

Aus der Gleichung (\ref{eq:7}) geht hervor

\begin{align}
  \label{eq:6}
  e^{i k_x L} = 1
\end{align}

Dies ist nur dann erfüllt wenn gilt

\begin{align}
  \label{eq:8}
  k_x L = n_x\cdot 2\pi \qquad \text{mit }n_x\in \mathds Z
\end{align}

Analogen Rechnung lässt sich für eine stehende Welle in \(y\)-Richtung bzw. in \(z\)-Richtung durchführen. D.h. für beliebige Raumrichtung gilt

\begin{align}
  \label{eq:9}
  \vec k = \frac{2\pi}{L}
  \begin{pmatrix}
    n_x\\n_y\\n_z
  \end{pmatrix}
= \frac{2\pi}{L} \vec n
\end{align}

Damit haben wir gezeigt dass, das \(\vec k\) quantisiert ist. Somit lässt sich das Integral als eine Summe schreiben. Damit ändert sich die Gleichung (\ref{eq:3}) zu

\begin{align}
  \label{eq:10}
  \vec A(\vec r, t) =\sum_{m=1,2} \sum_{\vec k}  N_{\vec k} \left[  \vec{\mathcal A}_m (\vec k) \hat \epsilon_m (\vec k) e^{i(\vec k\vec r-\omega t)} + \vec{\mathcal A}^*_m (k)\hat\epsilon^*_m(\vec k) e^{-i(\vec k\vec r-\omega t)}\right]
\end{align}

Dabei ist \(N_{\vec k}\) eine Normierungskonstante. Man wählt für \(N_{\vec k}=\sqrt{\frac{2\pi\hbar c^2}{\omega_{\vec k}V}}\) damit \(\mathcal A\) einheitenlos wird.  Weiterhin betrachten wir linear polarisierte Wellen mit \(\hat \epsilon_m \in \mathds{R} \), daher ist \(\hat \epsilon_m^* = \hat \epsilon_m \) und wir können es ausklammern. Somit sieht die Gleichung (\ref{eq:10}) wie folgt aus

\begin{align}
  \label{eq:11}
    \vec A(\vec r, t) =  \sum_{m=1,2} \sum_{\vec k} \sqrt{\frac{2\pi\hbar c^2}{\omega_{\vec k}V}}  \hat \epsilon_m  \left[  \vec{\mathcal A}_m (\vec k) (\vec k) e^{i(\vec k\vec r-\omega t)} + \vec{\mathcal A}^*_m (k)(\vec k) e^{-i(\vec k\vec r-\omega t)}\right]
\end{align}


Um den Hamiltonoperator zu bestimmen betrachten wir die klassische Energie eines Strahlungsfeldes, die wie folgt definiert ist

\begin{align}
  \label{eq:12}
  E_{\text{klassisch}} = \frac{1}{8\pi} \int d^3r \left( \vec E^2+\vec B^2 \right) = \frac{1}{8\pi} \int d^3r \left( \left( \frac{1}{c} \pdiff{\vec A}_t \right)^2 + \left(\vec \nabla\times\vec A\right)^2 \right)
\end{align}

Nebenrechnung für das erste Integral mit (\ref{eq:11}) ergibt

\begin{align}
  \label{eq:13}
 \int d^3r  \left( \frac{1}{c} \pdiff{\vec A}_t \right)^2 &= \int d^3r  \left( \frac{1}{c} \pdiff_t    \sum_{m=1,2} \sum_{\vec k} \sqrt{\frac{2\pi\hbar c^2}{\omega_{\vec k}V}}  \hat \epsilon_m(\vec k)  \left[  \vec{\mathcal A}_m (\vec k)  e^{i(\vec k\vec r-\omega t)} + \vec{\mathcal A}^*_m (k) e^{-i(\vec k\vec r-\omega t)}\right] \right)^2 \notag\\
&= \int d^3r  \left( \frac{1}{c}  \sum_{m=1,2} \sum_{\vec k} \sqrt{\frac{2\pi\hbar c^2}{\omega_{\vec k}V}}  \hat \epsilon_m(\vec k)  \left[ -\omega \vec{\mathcal A}_m (\vec k)  e^{i(\vec k\vec r-\omega t)} + \omega \vec{\mathcal A}^*_m (k) e^{-i(\vec k\vec r-\omega t)}\right] \right)^2 \notag\\
&=  \frac{2\pi\hbar}{V} \int d^3r  \left( \sum_{m=1,2} \sum_{\vec k} \sqrt{\omega_{\vec k}} \hat \epsilon_m(\vec k)  \left[\vec{\mathcal A}_m (\vec k)  e^{i(\vec k\vec r-\omega t)} - \vec{\mathcal A}^*_m (k) e^{-i(\vec k\vec r-\omega t)}\right] \right)^2 \notag\\
&=  \frac{2\pi\hbar}{V} \int d^3r  \sum_{m=1,2} \sum_{\vec k}  \left( \sqrt{\omega_{\vec k}} \hat \epsilon_m(\vec k)  \left[\vec{\mathcal A}_m (\vec k)  e^{i(\vec k\vec r-\omega t)} - \vec{\mathcal A}^*_m (k) e^{-i(\vec k\vec r-\omega t)}\right] \right) \times   \notag\\
  &\quad \times \sum_{m'=1,2} \sum_{\vec k'} \left( \sqrt{\omega_{\vec k'}} \hat \epsilon_{m'}(\vec k')  \left[\vec{\mathcal A}_{m'} (\vec k')  e^{i(\vec k'\vec r-\omega t)} - \vec{\mathcal A}^*_{m'} (k') e^{-i(\vec k'\vec r-\omega t)}\right] \right) \notag \\
%
&= \frac{2\pi\hbar}{V} \int d^3r  \sum_{m,m'=1,2} \sum_{\vec k,\vec k'}   \sqrt{\omega_{\vec k'}\omega'_{\vec k'}} \hat \epsilon_m(\vec k) \hat \epsilon'_m(\vec k)  \left[ \vec{\mathcal A}_m (\vec k) \vec{\mathcal A}_{m'} (\vec k)e^{i(\vec k\vec r-\omega t)}e^{i(\vec k'\vec r-\omega t)} + \right. \notag\\
&\quad +  \vec{\mathcal A}^*_m (k) \vec{\mathcal A}^*_{m'} (k') e^{-i(\vec k\vec r-\omega t)}e^{-i(\vec k'\vec r-\omega t)} - \vec{\mathcal A}_{m} (\vec k)  e^{i(\vec k\vec r-\omega t)}\vec{\mathcal A}^*_{m'} (k') e^{-i(\vec k'\vec r-\omega t)} \notag\\
&\quad\left. - \vec{\mathcal A}^*_m (k) e^{-i(\vec k\vec r-\omega t)}\vec{\mathcal A}_{m'} (\vec k')  e^{i(\vec k'\vec r-\omega t)}   \right]\notag\\
%
&= \frac{2\pi\hbar}{V} \int d^3r  \sum_{m,m'=1,2} \sum_{\vec k,\vec k'}   \sqrt{\omega_{\vec k'}\omega'_{\vec k'}} \hat \epsilon_m(\vec k) \hat \epsilon'_m(\vec k)  
\left[ \vec{\mathcal A}_m (\vec k) \vec{\mathcal A}_{m'} (\vec k)e^{i(\vec k+\vec k')\vec r}e^{-2i\omega t} + \right. \notag\\
&\quad +  \vec{\mathcal A}^*_m (k) \vec{\mathcal A}^*_{m'} (k') e^{-i(\vec k+\vec k')\vec r}e^{2i\omega t} 
- \left.\vec{\mathcal A}_{m} (\vec k)  \vec{\mathcal A}^*_{m'} (k') e^{i(\vec k-\vec k')\vec r} 
 - \vec{\mathcal A}^*_m (k) \vec{\mathcal A}_{m'} (\vec k') e^{i(\vec k'-\vec k)\vec r}    \right]
\end{align}

Mit den zwei Relationen

\begin{align}
  \label{eq:14}
  \frac{1}{V}\int d^3 r e^{i(\vec k - \vec k'}\vec r = \delta_{\vec k,\vec k'} \quad\text{und }  \frac{1}{V}\int d^3 r e^{i(\vec k +\vec k'}\vec r = \delta_{\vec k, - \vec k'}\quad \text{und } \sum_{m,m'}\hat\epsilon_m(\vec k)\hat\epsilon_{m'}(\vec k') = \sum_{m}
\end{align}
Ergibt die erste Nebenrechnung (\ref{eq:13})
\begin{align}
  \label{eq:15}
  \int d^3r  \left( \frac{1}{c} \pdiff{\vec A}_t \right)^2 &=  \frac{2\pi\hbar}{V} \int d^3r  \sum_{m,m'=1,2} \sum_{\vec k,\vec k'}   \sqrt{\omega_{\vec k'}\omega'_{\vec k'}} \hat \epsilon_m(\vec k) \hat \epsilon'_m(\vec k)  
\left[ \vec{\mathcal A}_m (\vec k,t) \vec{\mathcal A}_{m'} (\vec k,t)e^{i(\vec k+\vec k')\vec r} + \right. \notag\\
&\quad +  \vec{\mathcal A}^*_m (\vec k,t) \vec{\mathcal A}^*_{m'} (\vec k',t) e^{-i(\vec k+\vec k')\vec r} 
- \left.\vec{\mathcal A}_{m} (\vec k)  \vec{\mathcal A}^*_{m'} (k') e^{i(\vec k-\vec k')\vec r} 
 - \vec{\mathcal A}^*_m (k) \vec{\mathcal A}_{m'} (\vec k') e^{i(\vec k'-\vec k)\vec r}    \right] \notag\\
%
&=  2\pi\hbar \sum_{m,m'=1,2} \sum_{\vec k,\vec k'}   \sqrt{\omega_{\vec k'}\omega'_{\vec k'}} \hat \epsilon_m(\vec k) \hat \epsilon'_m(\vec k)  
\left[ \vec{\mathcal A}_m (\vec k,t) \vec{\mathcal A}_{m'} (\vec k,t)\delta_{\vec k,-\vec k'}   + \right. \notag\\
&\quad\left. +  \vec{\mathcal A}^*_m (\vec k,t) \vec{\mathcal A}^*_{m'} (\vec k',t) \delta_{\vec k, - \vec k'} 
- \vec{\mathcal A}_{m} (\vec k)  \vec{\mathcal A}^*_{m'} (k')  \delta_{\vec k,\vec k'}
 - \vec{\mathcal A}^*_m (k) \vec{\mathcal A}_{m'} (\vec k')    \delta_{\vec k,\vec k'}  \right] \notag\\
%
&=  2\pi\hbar \sum_{m=1,2} \sum_{\vec k} \omega_{\vec k}  
\left[ \vec{\mathcal A}_m (\vec k,t) \vec{\mathcal A}_{m} (- \vec k,t)   + 
  \vec{\mathcal A}^*_m (\vec k,t) \vec{\mathcal A}^*_{m} (- \vec k,t)  \right.\notag\\
&\quad\left. - \vec{\mathcal A}_{m} (\vec k)  \vec{\mathcal A}^*_{m} (\vec k) 
 - \vec{\mathcal A}^*_m (k) \vec{\mathcal A}_{m} (\vec k)  \right] \notag\\
\end{align}




Mit Hilfe der Lagrange Indentität \((\vec a\times\vec b)\cdot(\vec c\times\vec d) = (\vec a\cdot\vec c)(\vec b\cdot\vec d) - (\vec b \cdot \vec c)(\vec a\cdot\vec d) \)

\begin{align}
  \label{eq:17}
  (\vec \vec \nabla\times\vec \vec A)\cdot(\vec \vec \nabla\times\vec A) = (\vec\nabla^2)(\vec A^2) - (\vec A\vec \nabla)\underbr{(\vec \nabla\vec A)}_{=0\text{ Coulombeichung}}
\end{align}

Ergibt die Nebenrechung für das zweite Integral von der Gleichung (\ref{eq:12})

\begin{align}
  \label{eq:16}
   &\int d^3r  \left(\vec \nabla\times\vec A\right)^2 = \int d^3r  \left( \vec \nabla \vec A \right)^2 \notag\\
&=  2\pi\hbar \sum_{m=1,2} \sum_{\vec k} \omega_{\vec k}  \left[ - \vec{\mathcal A}_m (\vec k,t) \vec{\mathcal A}_{m} (- \vec k,t)   
-  \vec{\mathcal A}^*_m (\vec k,t) \vec{\mathcal A}^*_{m} (- \vec k,t) 
- \vec{\mathcal A}_{m} (\vec k)  \vec{\mathcal A}^*_{m} (\vec k)  
- \vec{\mathcal A}^*_m (k) \vec{\mathcal A}_{m} (\vec k)  \right] \notag\\
\end{align}


Die zwei Nebenreichnungen (\ref{eq:14}) und (\ref{eq:16}) in (\ref{eq:12}) eingesetzt ergibt

\begin{align}
  \label{eq:18}
   H&=E_{\text{klassisch}} = \frac{1}{8\pi} \int d^3r \left( \left( \frac{1}{c} \pdiff{\vec A}_t \right)^2 + \left(\vec \nabla\times\vec A\right)^2 \right) = \notag\\
&=  \frac{1}{8\pi} 2\pi\hbar \sum_{m=1,2} \sum_{\vec k} \omega_{\vec k}  
\left[ \cancel{\vec{\mathcal A}_m (\vec k,t) \vec{\mathcal A}_{m} (- \vec k,t)}   + 
  \cancel{\vec{\mathcal A}^*_m (\vec k,t) \vec{\mathcal A}^*_{m} (- \vec k,t)}  
 - \vec{\mathcal A}_{m} (\vec k)  \vec{\mathcal A}^*_{m} (\vec k) 
 - \vec{\mathcal A}^*_m (k) \vec{\mathcal A}_{m} (\vec k)  \right. \notag\\
&\qquad\left. \cancel{ - \vec{\mathcal A}_m (\vec k,t) \vec{\mathcal A}_{m} (- \vec k,t)}  
\cancel{-  \vec{\mathcal A}^*_m (\vec k,t) \vec{\mathcal A}^*_{m} (- \vec k,t)}
- \vec{\mathcal A}_{m} (\vec k)  \vec{\mathcal A}^*_{m} (\vec k)  
- \vec{\mathcal A}^*_m (k) \vec{\mathcal A}_{m} (\vec k)  \right]
\end{align}


\end{document}


