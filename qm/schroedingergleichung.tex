\input{../headers/header_script.tex}


\begin{document}

\section*{Schrödingergleichung}

Die Schrödignergleichung beschreibt die zeitliche Entwicklung eines Zustands eines quantenmechanischen-Systems. Sie besagt, dass die zeitliche Veränderung des Zustands von der Energie bestimmt ist.\\
\\
Der Zutsand ist eine Funktion aus dem Hilbertraums. Und die Energie wird durch einen Quantenmechanischen Operator den Hamiltonoperator repräsentiert.\\
\\
Folgende Forderungen muss die Bewegungsgleichung erfüllen:
\begin{enumerate}
\item DGL 1. Ordnung in der Zeit, damit $\psi(\vec{r},t)$ durch die Anfangsverteilung
  $\psi(\vec{r},t=0)$bestimmt ist.
\item Sie muss linear in $\psi$ sein, damit Superpositionsprinzip gilt (d.h.
  Linearkombination von Lösungen stellen wieder Lösungen dar $\rightarrow$ deshalb
  treten Interferenzeffekte auf wie in der Optik. (Optik: Diese folgen
  aus der Linearität der Maxwellgleichungen)
\item Sie muss homogen sein, damit
  $\int_{-\infty}^{\infty}d^{3}r|\psi(\vec{r},t)|^{2}=1$ für
  alle Zeiten erfüllt bleibt.
\item Die ebenen Wellen \(\psi(\vec{r},t)=c\cdot  e^{i\vec{k}\cdot\vec{r}-i\omega t}\) sollen  Lösungen der Gleichung sein. 
\end{enumerate}

Für die Aufstellung der Gleichung spielen zwei Beziehungen eine fundamentale Rolle. Die eine ist ein Zusammenhang zwischen Impuls \(p\) und Wellenlänge \(\lambda\) eines Quantenobjekts von \textbf{Louise de Broglie}

\begin{equation}
  \label{eq:3}
  p = \frac{h}{\lambda} = \frac{h}{2\pi}\frac{2\pi}{\lambda} = \hbar k
\end{equation}

Und die andere ist eine Verknüpfung der Energie \(E\) mit der Frequenz \(\nu\) von \textbf{Max Planck}
\begin{equation}
  \label{eq:2}
  E=h\nu=h\frac{\omega}{2\pi} =\hbar \omega
\end{equation}

Ableitung der Ebenen Wellen nach der Zeit ergibt

\begin{align}
  \label{eq:4}
   \frac{\partial}{\partial t}\psi(\vec{r},t) &= -i \omega \psi(\vec{r},t) \qquad \text{mit } \omega=\frac{E}{\hbar} \qquad (\ref{eq:2}) \\
&= -i \frac{E}{\hbar} \psi(\vec{r},t) 
\end{align}

Unter Annahme für ein freies Teilchen setzen wir die Energie unter Berücksichtigung der de Broglie Impulsbeziehung (\ref{eq:3})

\begin{equation}
  \label{eq:6}
  E=\frac{p^2}{2m}=\frac{\hbar^2k^2}{2m}
\end{equation}


lautet die Gleichung (\ref{eq:4}) nun

\begin{align}
  \label{eq:7}
  \frac{\partial}{\partial t}\psi(\vec{r},t) &=-\frac{i\hbar k^2}{2m}  \psi(\vec{r},t) \notag \\
 &=-\frac{\hbar}{i2m} \underbr{i^2k^2}_{\nabla^2} \psi(\vec{r},t) \qquad |\cdot i\hbar
\end{align}

Dass \(i^2k^2\equiv \nabla^2\) wird ersichtlich, wenn man die Wellenfunktion 2 mal nach Ort Ableitet. Somit erhalten wir aus (\ref{eq:7}) die zeitabhängige Schrödingergleichung in der uns bekannten Form für ein freies Teilchen

  \begin{equation}
    \label{eq:11}
    \boxed{i\hbar\frac{\partial}{\partial t}\psi(\vec{r},t)
      =-\frac{\hbar^2}{2m}\nabla^2\psi(\vec{r},t)}
  \end{equation}


Annahme: Teilchen der Masse $m$ unterliegt einem Potential $V(\vec{r},t)$
\begin{align*}
  i\hbar\frac{\partial}{\partial t}\psi(\vec{r},t) 
  &=-\frac{\hbar^{2}}{2m}\nabla^2\psi(\vec{r},t)+V(\vec{r},t)\psi(\vec{r},t)\\
  &=\left[-\frac{\hbar^{2}}{2m}\nabla^2+V(\vec{r},t)\right]\psi(\vec{r},t)
\end{align*}
\newpage

Alternative Herleitung mit \(\vec p\)
Die ebenen Wellen \(\psi(\vec{r},t)=c\cdot  e^{\frac{i}{\hbar}\vec{p}\cdot\vec{r}-i\omega t}\) sollen  Lösungen der Gleichung sein. 

Wobei die Energie für Photonen \(E=h\nu=\frac{h}{2\pi}\omega=\hbar\omega\) und somit nach \(\omega\) umgeformt und kinetische Energie eingesetzt \(\omega = \frac{p^2}{2m\hbar}\)  \\

  \begin{equation}
   \label{eq:1}
    \psi(\vec{r},t)=c\cdot  e^{\frac{i}{\hbar}(\vec{p}\cdot\vec{r}-\frac{p^{2}}{2m}t)}
  \end{equation}

  Für diese ebenen Wellen gilt:
  \begin{align*}
   \frac{\partial}{\partial t}\psi(\vec{r},t) &=
      -\frac{i}{\hbar}\cdot\frac{p^{2}}{2m}\psi(\vec{r},t)\\
      &=-\frac{i}{\hbar}\cdot\frac{p^{2}}{2m}\psi(\vec{r},t)
      \cdot \frac{\hbar^2}{\hbar^2}\\
  &= \frac{i\hbar}{2m}\cdot  \underbrace{\frac{(ip)^2}{\hbar^2}\psi(\vec{r},t)}_{\nabla^2\psi(\vec{r},t)}\\
  &= \frac{i\hbar}{2m}\nabla^2\psi(\vec{r},t)
\end{align*}


\end{document}
