\input{../headers/header_script.tex}
%\includegraphics[width=0.75\textwidth]{thepic.png}

\begin{document}

\section*{Schrödingergleichung}


Suchen Bewegungsgleichung für $\psi(\vec{r},t)$. Forderungen:
\begin{enumerate}
\item DGL 1. Ordnung in der Zeit, damit $\psi(\vec{r},t)$ durch die Anfangsverteilung
  $\psi(\vec{r},t=0)$bestimmt ist.
\item Sie muss linear in $\psi$ sein, damit Superpositionsprinzip gilt (d.h.
  Linearkombination von Lösungen stellen wieder Lösungen dar $\rightarrow$ deshalb
  treten Interferenzeffekte auf wie in der Optik. (Optik: Diese folgen
  aus der Linearität der Maxwellgleichungen)
\item Sie muss homogen sein, damit
  $\int_{-\infty}^{\infty}d^{3}r|\psi(\vec{r},t)|^{2}=1$ für
  alle Zeiten erfüllt bleibt.
\item Die ebenen Wellen \(\psi(\vec{r},t)=c\cdot  e^{\frac{i}{\hbar}\vec{p}\cdot\vec{r}-i\omega t}\) sollen  Lösungen der Gleichung sein. Wobei die Energie für Photonen \(E=hf=\frac{h}{2\pi}\omega=\hbar\omega\) und somit nach \(\omega\) umgeformt und kinetische Energie eingesetzt \(\omega = \frac{p^2}{2m\hbar}\)  \\

  \begin{equation}
    \label{eq:1}
    \psi(\vec{r},t)=c\cdot  e^{\frac{i}{\hbar}(\vec{p}\cdot\vec{r}-\frac{p^{2}}{2m}t)}
  \end{equation}

  Für diese ebenen Wellen gilt:
  \begin{align*}
    \frac{\partial}{\partial t}\psi(\vec{r},t) &=
      -\frac{i}{\hbar}\cdot\frac{p^{2}}{2m}\psi(\vec{r},t)\\
      &=-\frac{i}{\hbar}\cdot\frac{p^{2}}{2m}\psi(\vec{r},t)
      \cdot \frac{\hbar^2}{\hbar^2}\\
  &= \frac{i\hbar}{2m}\cdot  \underbrace{\frac{(ip)^2}{\hbar^2}\psi(\vec{r},t)}_{\nabla^2\psi(\vec{r},t)}\\
  &= \frac{i\hbar}{2m}\nabla^2\psi(\vec{r},t)
\end{align*}
\end{enumerate}

Aus 1.-4. erhalten wir die zeitabhängige Schrödingergleichung für ein
freies Teilchen:

  \begin{equation}
    \label{eq:11}
    \boxed{i\hbar\frac{\partial}{\partial t}\Psi(\vec{r},t)
      =-\frac{\hbar^2}{2m}\nabla^2\Psi(\vec{r},t)}
  \end{equation}


Annahme: Teilchen der Masse $m$ unterliegt einem Potential $V(\vec{r},t)$
\begin{align*}
  i\hbar\frac{\partial}{\partial t}\Psi(\vec{r},t) 
  &=-\frac{\hbar^{2}}{2m}\nabla^2\Psi(\vec{r},t)+V(\vec{r},t)\Psi(\vec{r},t)\\
  &=\left[-\frac{\hbar^{2}}{2m}\nabla^2+V(\vec{r},t)\right]\Psi(\vec{r},t)
\end{align*}




\end{document}
