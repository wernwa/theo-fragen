\input{../headers/header_script.tex}
\usepackage{amsmath} 



\begin{document}

\section*{Clebsch-Gordan-Koeffizienten}


\subsection*{Beispiel}
Addition Zweier Drehimpulse anhand eines einfachen Beispiels von Teilchen mit Spin \(\frac{1}{2}\).


\[|S,M\rangle = \sum_{m_1}\sum_{m_2}\langle s_i,s_2;m_1,m_2|S,M\rangle |s_1,s_2;m_1,m_2\rangle \]

\[|1,-1\rangle =\underbrace{ \langle \frac{1}{2}\frac{1}{2};-\frac{1}{2}-\frac{1}{2}|11\rangle }_{=1}  | \frac{1}{2}\frac{1}{2};  -\frac{1}{2} -\frac{1}{2} \rangle   \equiv |--\rangle   \]
\[|1,0\rangle =  \underbrace{\langle \frac{1}{2}   \frac{1}{2}; \frac{1}{2} - \frac{1}{2}|10\rangle  }_{\frac{1}{\sqrt{2}}}|\frac{1}{2}   \frac{1}{2}; \frac{1}{2} - \frac{1}{2}\rangle  +  \underbrace{\langle \frac{1}{2}   \frac{1}{2}; -\frac{1}{2} \frac{1}{2}|10\rangle  }_{\frac{1}{\sqrt{2}}} |\frac{1}{2}   \frac{1}{2}; -\frac{1}{2} \frac{1}{2} \rangle \equiv  \frac{1}{\sqrt{2}} (|+-\rangle +  |-+\rangle )   \]
\[|1,1\rangle = =\underbrace{ \langle \frac{1}{2}\frac{1}{2};\frac{1}{2}\frac{1}{2}|11\rangle }_{=1}  | \frac{1}{2}\frac{1}{2};  \frac{1}{2} \frac{1}{2} \rangle   \equiv |++\rangle   \]
\[|0,0\rangle = \underbrace{\langle \frac{1}{2}   \frac{1}{2}; \frac{1}{2} - \frac{1}{2}|00\rangle  }_{\frac{1}{\sqrt{2}}}|\frac{1}{2}   \frac{1}{2}; \frac{1}{2} - \frac{1}{2}\rangle  +  \underbrace{\langle \frac{1}{2}   \frac{1}{2}; -\frac{1}{2} \frac{1}{2}|00\rangle  }_{-\frac{1}{\sqrt{2}}} |\frac{1}{2}   \frac{1}{2}; -\frac{1}{2} \frac{1}{2} \rangle \equiv  \frac{1}{\sqrt{2}} (|+-\rangle -  |-+\rangle ) \]







\subsection*{Referenzen}
\begin{itemize}
\item Claude Cohen-Tannoudji Quantenmechanik Band 2
\item Zettili Quanten Mehanics
\item Rollnik Quantentheorie 2
\end{itemize}

\end{document}
