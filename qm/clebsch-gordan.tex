\documentclass[10pt,a4paper,oneside,fleqn]{article}
\usepackage{geometry}
\geometry{a4paper,left=20mm,right=20mm,top=1cm,bottom=2cm}
\usepackage[utf8]{inputenc}
%\usepackage{ngerman}
\usepackage{amsmath}                % brauche ich um dir Formel zu umrahmen.
\usepackage{amsfonts}                % brauche ich für die Mengensymbole
\usepackage{graphicx}
\setlength{\parindent}{0px}
\setlength{\mathindent}{10mm}
\usepackage{bbold}                    %brauche ich für die doppel Zahlen Darstellung (Einheitsmatrix z.B)



\usepackage{color}
\usepackage{titlesec} %sudo apt-get install texlive-latex-extra

\definecolor{darkblue}{rgb}{0.1,0.1,0.55}
\definecolor{verydarkblue}{rgb}{0.1,0.1,0.35}
\definecolor{darkred}{rgb}{0.55,0.2,0.2}

%hyperref Link color
\usepackage[colorlinks=true,
        linkcolor=darkblue,
        citecolor=darkblue,
        filecolor=darkblue,
        pagecolor=darkblue,
        urlcolor=darkblue,
        bookmarks=true,
        bookmarksopen=true,
        bookmarksopenlevel=3,
        plainpages=false,
        pdfpagelabels=true]{hyperref}

\titleformat{\chapter}[display]{\color{darkred}\normalfont\huge\bfseries}{\chaptertitlename\
\thechapter}{20pt}{\Huge}

\titleformat{\section}{\color{darkblue}\normalfont\Large\bfseries}{\thesection}{1em}{}
\titleformat{\subsection}{\color{verydarkblue}\normalfont\large\bfseries}{\thesubsection}{1em}{}

% Notiz Box
\usepackage{fancybox}
\newcommand{\notiz}[1]{\vspace{5mm}\ovalbox{\begin{minipage}{1\textwidth}#1\end{minipage}}\vspace{5mm}}

\usepackage{cancel}
\setcounter{secnumdepth}{3}
\setcounter{tocdepth}{3}





%-------------------------------------------------------------------------------
%Diff-Makro:
%Das Diff-Makro stellt einen Differentialoperator da.
%
%Benutzung:
% \diff  ->  d
% \diff f  ->  df
% \diff^2 f  ->  d^2 f
% \diff_x  ->  d/dx
% \diff^2_x  ->  d^2/dx^2
% \diff f_x  ->  df/dx
% \diff^2 f_x  ->  d^2f/dx^2
% \diff^2{f(x^5)}_x  ->  d^2(f(x^5))/dx^2
%
%Ersetzt man \diff durch \pdiff, so wird der partieller
%Differentialoperator dargestellt.
%
\makeatletter
\def\diff@n^#1{\@ifnextchar{_}{\diff@n@d^#1}{\diff@n@fun^#1}}
\def\diff@n@d^#1_#2{\frac{\textrm{d}^#1}{\textrm{d}#2^#1}}
\def\diff@n@fun^#1#2{\@ifnextchar{_}{\diff@n@fun@d^#1#2}{\textrm{d}^#1#2}}
\def\diff@n@fun@d^#1#2_#3{\frac{\textrm{d}^#1 #2}{\textrm{d}#3^#1}}
\def\diff@one@d_#1{\frac{\textrm{d}}{\textrm{d}#1}}
\def\diff@one@fun#1{\@ifnextchar{_}{\diff@one@fun@d #1}{\textrm{d}#1}}
\def\diff@one@fun@d#1_#2{\frac{\textrm{d}#1}{\textrm{d}#2}}
\newcommand*{\diff}{\@ifnextchar{^}{\diff@n}
  {\@ifnextchar{_}{\diff@one@d}{\diff@one@fun}}}
%
%Partieller Diff-Operator.
\def\pdiff@n^#1{\@ifnextchar{_}{\pdiff@n@d^#1}{\pdiff@n@fun^#1}}
\def\pdiff@n@d^#1_#2{\frac{\partial^#1}{\partial#2^#1}}
\def\pdiff@n@fun^#1#2{\@ifnextchar{_}{\pdiff@n@fun@d^#1#2}{\partial^#1#2}}
\def\pdiff@n@fun@d^#1#2_#3{\frac{\partial^#1 #2}{\partial#3^#1}}
\def\pdiff@one@d_#1{\frac{\partial}{\partial #1}}
\def\pdiff@one@fun#1{\@ifnextchar{_}{\pdiff@one@fun@d #1}{\partial#1}}
\def\pdiff@one@fun@d#1_#2{\frac{\partial#1}{\partial#2}}
\newcommand*{\pdiff}{\@ifnextchar{^}{\pdiff@n}
  {\@ifnextchar{_}{\pdiff@one@d}{\pdiff@one@fun}}}
\makeatother
%
%Das gleich nur mit etwas andere Syntax. Die Potenz der Differentiation wird erst
%zum Schluss angegeben. Somit lautet die Syntax:
%
% \diff_x^2  ->  d^2/dx^2
% \diff f_x^2  ->  d^2f/dx^2
% \diff{f(x^5)}_x^2  ->  d^2(f(x^5))/dx^2
% Ansonsten wie Oben.
%
%Ersetzt man \diff durch \pdiff, so wird der partieller
%Differentialoperator dargestellt.
%
%\makeatletter
%\def\diff@#1{\@ifnextchar{_}{\diff@fun#1}{\textrm{d} #1}}
%\def\diff@one_#1{\@ifnextchar{^}{\diff@n{#1}}%
%  {\frac{\textrm d}{\textrm{d} #1}}}
%\def\diff@fun#1_#2{\@ifnextchar{^}{\diff@fun@n#1_#2}%
%  {\frac{\textrm d #1}{\textrm{d} #2}}}
%\def\diff@n#1^#2{\frac{\textrm d^#2}{\textrm{d}#1^#2}}
%\def\diff@fun@n#1_#2^#3{\frac{\textrm d^#3 #1}%
%  {\textrm{d}#2^#3}}
%\def\diff{\@ifnextchar{_}{\diff@one}{\diff@}}
%\newcommand*{\diff}{\@ifnextchar{_}{\diff@one}{\diff@}}
%
%Partieller Diff-Operator.
%\def\pdiff@#1{\@ifnextchar{_}{\pdiff@fun#1}{\partial #1}}
%\def\pdiff@one_#1{\@ifnextchar{^}{\pdiff@n{#1}}%
%  {\frac{\partial}{\partial #1}}}
%\def\pdiff@fun#1_#2{\@ifnextchar{^}{\pdiff@fun@n#1_#2}%
%  {\frac{\partial #1}{\partial #2}}}
%\def\pdiff@n#1^#2{\frac{\partial^#2}{\partial #1^#2}}
%\def\pdiff@fun@n#1_#2^#3{\frac{\partial^#3 #1}%
%  {\partial #2^#3}}
%\newcommand*{\pdiff}{\@ifnextchar{_}{\pdiff@one}{\pdiff@}}
%\makeatother

%-------------------------------------------------------------------------------
%%Nützliche Makros um in der Quantenmechanik Bras, Kets und das Skalarprodukt
%%zwischen den beiden darzustellen.
%%Benutzung:
%% \bra{x}  ->    < x |
%% \ket{x}  ->    | x >
%% \braket{x}{y} ->   < x | y >

\newcommand\bra[1]{\left\langle #1 \right|}
\newcommand\ket[1]{\left| #1 \right\rangle}
\newcommand\braket[2]{%
  \left\langle #1\vphantom{#2} \right.%
  \left|\vphantom{#1#2}\right.%
  \left. \vphantom{#1}#2 \right\rangle}%

%-------------------------------------------------------------------------------
%%Aus dem Buch:
%%Titel:  Latex in Naturwissenschaften und Mathematik
%%Autor:  Herbert Voß
%%Verlag: Franzis Verlag, 2006
%%ISBN:   3772374190, 9783772374197
%%
%%Hier werden drei Makros definiert:\mathllap, \mathclap und \mathrlap, welche
%%analog zu den aus Latex bekannten \rlap und \llap arbeiten, d.h. selbst
%%keinerlei horizontalen Platz benötigen, aber dennoch zentriert zum aktuellen
%%Punkt erscheinen.

\newcommand*\mathllap{\mathstrut\mathpalette\mathllapinternal}
\newcommand*\mathllapinternal[2]{\llap{$\mathsurround=0pt#1{#2}$}}
\newcommand*\clap[1]{\hbox to 0pt{\hss#1\hss}}
\newcommand*\mathclap{\mathpalette\mathclapinternal}
\newcommand*\mathclapinternal[2]{\clap{$\mathsurround=0pt#1{#2}$}}
\newcommand*\mathrlap{\mathpalette\mathrlapinternal}
\newcommand*\mathrlapinternal[2]{\rlap{$\mathsurround=0pt#1{#2}$}}

%%Das Gleiche nur mit \def statt \newcommand.
%\def\mathllap{\mathpalette\mathllapinternal}
%\def\mathllapinternal#1#2{%
%  \llap{$\mathsurround=0pt#1{#2}$}% $
%}
%\def\clap#1{\hbox to 0pt{\hss#1\hss}}
%\def\mathclap{\mathpalette\mathclapinternal}
%\def\mathclapinternal#1#2{%
%  \clap{$\mathsurround=0pt#1{#2}$}%
%}
%\def\mathrlap{\mathpalette\mathrlapinternal}
%\def\mathrlapinternal#1#2{%
%  \rlap{$\mathsurround=0pt#1{#2}$}% $
%}

%-------------------------------------------------------------------------------
%%Hier werden zwei neue Makros definiert \overbr und \underbr welche analog zu
%%\overbrace und \underbrace funktionieren jedoch die Gleichung nicht
%%'zerreißen'. Dies wird ermöglicht durch das \mathclap Makro.

\def\overbr#1^#2{\overbrace{#1}^{\mathclap{#2}}}
\def\underbr#1_#2{\underbrace{#1}_{\mathclap{#2}}}



\begin{document}



\section*{Addition von Drehimpulsen, Clebsch-Gordan Koeffizienten}
\label{sec:addit-von-dreh}

Wir betrachte zwei von einander unabhängige Drehimpulse \(\vec J_1\) und \(\vec
J_2\), die jeweils einen eigenen Hilbertraum mit der Basis \(\{\ket{j_1,m_1}\}\)
für \(\vec J_1\) und \(\{\ket{j_2,m_2}\}\) für \(\vec J_2\) bilden. Zusammen
spannen sie einen \((2j_1+1) \times (2j_2+1)\) dimensionalen Hilbertraum
\(\mathcal H = \mathcal H_1 \otimes \mathcal H_2\) auf, mit der Basis
\(\{\ket{j_1,m_1} \otimes \ket{j_2,m_2}\} \equiv \{\ket{j_1,j_2;m_1,m_2}\}\).
Nun wollen wir die einzelne Drehimpulse in eine gemeinsame Basis überführen, so
dass wir einen Gesamtdrehimpuls \(\vec J=\vec J_1+\vec J_2\) mit der neuen Basis
\(\{\ket{J,M}\}\) erhalten. Dies erreichen wir, in dem wir die Orthonormalität
der alten Basis ausnutzen und den Einheitsoperator \(\mathds 1\) vor die neue
Basis einschieben:
\begin{align}
  \label{eq:1}
  \ket{J,M}&=\underbrace{\left(
      \sum_{m_1,m_2}\ket{j_1,j_2;m_1,m_2}\bra{j_1,j_2;m_1,m_2}
    \right)}_{= \mathds 1}
  \ket{J,M}\notag\\
   \ket{J,M}&= \sum_{m_1,m_2} 
   \underbrace{\braket{j_1,j_2;m_1,m_2}{J,M}}_
   {\text{\textbf{Clebsch-Gordan Koef.}}}
   \ket{j_1,j_2;m_1,m_2}
\end{align}

Da sowohl die alte Basis \( \{\ket{j_1,j_2;m_1,m_2}\}\) als auch die neue Basis
\(\{\ket{J,M}\}\) orthonormiert sind, handelt es sich bei (\ref{eq:1}) um eine
unitäre Transformation. Zu bemerken ist dass bei dem von uns eingefügten
Einheitsoperator nur über die Magnetquantenzahlen \(m_1\) und \(m_2\) summiert
wird. D.h. man hält die Drehimpulsquantenzahlen \(j_1\) und \(j_2\) fest und
entwickelt nur nach den Magnetquantenzahlen, für die gilt \(m_1 = -j_1,\; -j_1+1,
\;-j_1+2, \dots j_1\) und \(m_2 = -j_2, \;-j_2+1,\;-j_2+2, \dots j_2\).\\
Die Koeffizienten \( \braket{j_1,j_2;m_1,m_2}{J,M} \) die die beiden Basen verbinden
heißen \emph{Clebsch-Gordan Koeffizienten}. D.h. das Problem der Addition zweier
Drehimpulse reduziert sich auf das Bestimmen der Glebsch-Gordan Koeffizienten.

Für die Erwartungswerte \(J\) in der neuen Basis gilt:
\begin{equation}
  \label{eq:8}
  \boxed{|j_1-j_2|\leq J\leq j_1+j_2}
\end{equation}

Dies wird durch die Vektoraddition \(\vec J = \vec J_1+\vec J_2\) deutlich.

Den Erwartungswert \(M\) können wir mit Hilfe des \(J_z=J_{z_1}+J_{z_2}\)
Operators und folgenden Eigenwertgleichungen bestimmen:

\begin{subequations}
  \begin{align}
    J_Z\ket{J,M}&= M \ket{J,M} \label{eq:9a}\\
    J_{z_1}\ket{j_1,j_2;m_1,m_2}&=m_1\ket{j_1,j_2;m_1,m_2} \label{eq:9b}\\
    J_{z_2}\ket{j_1,j_2;m_1,m_2}&=m_2\ket{j_1,j_2;m_1,m_2} \label{eq:9c}
  \end{align}
\end{subequations}

Es gilt:
\begin{align}
  \label{eq:10}
  J_z-J_{z_1}-J_{z_2}&=0\notag\\
  \bra{j_1,j_2;m_1,m_2} J_z-J_{z_1}-J_{z_2} \ket{J,M} &=0\notag\\
  \underbrace{\bra{j_1,j_2;m_1,m_2} J_z\ket{J,M}}_{~\eqref{eq:9a}}
  -\underbrace{\bra{j_1,j_2;m_1,m_2}J_{z_1}\ket{J,M}}_{~\eqref{eq:9b}}\notag\\
  -\underbrace{\bra{j_1,j_2;m_1,m_2}J_{z_2} \ket{J,M}}_{~\eqref{eq:9c}}&=0\notag\\
  M\braket{j_1,j_2;m_1,m_2}{J,M}
  -m_1\braket{j_1,j_2;m_1,m_2}{J,M}\notag\\
  -m_2\braket{j_1,j_2;m_1,m_2}{J,M}&=0\notag\\
  (M-m_1-m_2)\braket{j_1,j_2;m_1,m_2}{J,M}&=0
\end{align}

Aus der Gleichung \eqref{eq:10} folgt das entweder
\(\braket{j_1,j_2;m_1,m_2}{J,M}=0\) oder\\ 
\((M-m_1-m_2)=0\) was zur der wichtigen Beziehung führt:
\begin{equation}
  \label{eq:9}
  \boxed{M=m_1+m_2}
\end{equation}

Dies ist deswegen so wichtig weil dadurch die Entwicklung \eqref{eq:1} nur
auf die Clebsch-Gordan Koeffizienten beschränkt wird, bei denen die Bedingung
\eqref{eq:9} zutrifft. Alle anderen sind \underline{Null}. 

Der Erwartungswert für die Gesamtmagnetquantenzahl \(M\) zwischen:
\begin{equation}
  \label{eq:11}
  -j_1-j_2\leq M \leq j_1+j_2=-(j_1+j_2)\leq M \leq j_1+j_2
  =-J\leq M \leq J
\end{equation}




\subsection*{Eigenschaften der Clebsch-Gordan Koeffizienten}
\label{sec:eigensch-der-clebsch}

Die Clebsch-Gordan-Koeffizienten sind konventionsgemäß \emph{reell}:
\begin{equation}
  \label{eq:2}
  \braket{j_1,j_2;m_1,m_2}{J,M} \in \mathbb R
\end{equation}
D.h. es gilt:
\begin{equation}
  \label{eq:3}
  \braket{j_1,j_2;m_1,m_2}{J,M} = \braket{J,M}{j_1,j_2;m_1,m_2}^*=
  \braket{J,M}{j_1,j_2;m_1,m_2}
\end{equation}

Die Clebsch-Gordan-Koeffizienten sind \emph{orthonormiert}:
\begin{align}
  \label{eq:4}
  \braket{J',M'}{J,M}&=\delta_{J',J}\delta_{J',J}\notag\\
  \bra{J',M'}\underbrace{\left(
      \sum_{m_1,m_2}\ket{j_1,j_2;m_1m_2}\bra{j_1,j_2;m_1m_2}\right)
  }_{=\mathds 1}
  \ket{J,M}&=\delta_{J',J}\delta_{J',J}\notag\\
  \Rightarrow\sum_{m_1,m_2}\braket{J',M'}{j_1,j_2;m_1m_2}
  \braket{j_1,j_2;m_1m_2}{J,M}&=\delta_{J',J}\delta_{J',J}
\end{align}

Aus Gl.~(\ref{eq:3}) und der Gl.~(\ref{eq:4}) erhalten wir eine wichtige Beziehung:
\begin{equation}
  \label{eq:5}
  \sum_{m_1,m_2}\braket{j_1,j_2;m_1m_2}{J',M'}
  \braket{j_1,j_2;m_1m_2}{J,M}=\delta_{J',J}\delta_{J',J}
\end{equation}

Bzw. mit \(J'=J\) und \(M'=M\) folgt:
\begin{equation}
  \label{eq:6}
  \boxed{\sum_{m_1,m_2}\braket{j_1,j_2;m_1m_2}{J,M}^2=1}
\end{equation}

Analog kann man auch folgende Beziehung herleiten:
\begin{equation}
  \label{eq:7}
  \boxed{\sum_{J}\sum_{M}\braket{j_1,j_2;m_1m_2}{J,M}^2=1}
\end{equation}

\subsection*{Bestimmung der Clebsch-Gordan Koeffizienten}
\label{sec:best-der-clebsch}

Wir wollen nun einige Beziehungen herleiten um die Clebsch-Gordan Koeffizienten
bestimmen zu können.

Eine wichtige Beziehung können wir direkt aus \eqref{eq:10} übernehmen:
\begin{equation}
  \label{eq:12}
  \boxed{\braket{j_1,j_2;m_1m_2}{J,M}=0 \quad \text{für} \quad m_1+m_2\neq M}
\end{equation}

Eine weitere Beziehung finden wir aus den Extremalstellen für \(J\) und \(M\)
d.h. wenn gilt:
\begin{equation}
  \label{eq:13}
  J=j_1+j_2 \quad \text{und} \quad M=J=j_1+j_2=m_1+m_2
\end{equation}

Setzen wir \eqref{eq:13} in die Gl.~\eqref{eq:1} ein, so erhalten wir:
\begin{align}
  \label{eq:14}
   \ket{J,J}&=\sum_{m_1=j_1}^{j_1}\sum_{m_2=j_2}^{j_2} 
   \braket{j_1,j_2;m_1,m_2}{J,J}
   \ket{j_1,j_2;m_1,m_2}\notag\\
   &= \braket{j_1,j_2;j_1,j_2}{J,J}
   \ket{j_1,j_2;j_1,j_2}
\end{align}

Mit der Normierungsbedingung \(\braket{J,J}{J,J}\stackrel{!}=1\) folgt:
\begin{align}
  \label{eq:15}
  &\braket{J,J}{J,J}=1\notag\\
  &\bra{j_1,j_2;j_1,j_2} \braket{J,J}{j_1,j_2;j_1,j_2}
  \braket{j_1,j_2;j_1,j_2}{J,J}\ket{j_1,j_2;j_1,j_2}=1\notag\\
  &\bra{j_1,j_2;j_1,j_2} \braket{j_1,j_2;j_1,j_2}{J,J}
  \braket{j_1,j_2;j_1,j_2}{J,J}\ket{j_1,j_2;j_1,j_2}=1\notag\\
  &\braket{j_1,j_2;j_1,j_2}{J,J}^2
  \underbrace{\braket{j_1,j_2;j_1,j_2}{j_1,j_2;j_1,j_2}}_{=1}=1\notag\\
  &\braket{j_1,j_2;j_1,j_2}{J,J}^2=1
\end{align}

Aus \eqref{eq:15} folgt also:
\begin{equation}
  \label{eq:16}
  \braket{j_1,j_2;j_1,j_2}{J,J}=\pm 1
\end{equation}

Um nun zu bestimmen ob das Ergebnis in \eqref{eq:16} positive oder negative ist
d.h. ob \(+1\) oder \(-1\), wurde die sog. \emph{Condon-Shortley
  Phasenkonvention} eingeführt. Sie besagt, dass der Clebsch-Gordan Koeffizient
von der Form:\\
\(\braket{j_1,j_2;j_1,(J-j_1)}{J,J}\) \underline{reell} und \underline{positive}
sein muss. D.h. in unserem Fall \eqref{eq:13} mit \(J=j_1+j_2 \Leftrightarrow
J-j_1=j_2\) folgt:
\begin{equation}
  \label{eq:17}
  \braket{j_1,j_2;j_1,j_2}{J,J}=\underbr{\braket{j_1,j_2;j_1,(J-j_1)}{J,J}}
  _{\text{Konvention: \textbf{positve}}}=1
\end{equation}

D.h.im Spezialfall \eqref{eq:13} ist der Clebsch-Gordan Koeffizient gleich 1. Es
gilt:
\begin{equation}
  \label{eq:18}
  \boxed{\braket{j_1,j_2;j_1,j_2}{J,J}=1 \quad \text{mit}\quad J=j_1+j_2}
\end{equation}

\section*{Addition von Drehimpulsen,\\
  Clebsch-Gordan Koeffizienten}
\label{sec:addit-von-dreh}

Wir betrachte zwei von einander unabhängige Drehimpulse \(\vec j_1\) und \(\vec
j_2\), die jeweils einen eigenen Hilbertraum mit der Basis \(\{\ket{j_1,m_1}\}\)
für \(\vec j_1\) und \(\{\ket{j_2,m_2}\}\) für \(\vec j_2\) bilden. Zusammen
spannen sie einen \((2j_1+1) \times (2j_2+1)\) dimensionalen Hilbertraum
\(\mathcal H = \mathcal H_1 \otimes \mathcal H_2\) auf, mit der Basis
\(\{\ket{j_1,m_1} \otimes \ket{j_2,m_2}\} \equiv \{\ket{j_1,j_2;m_1,m_2}\}\).
Nun wollen wir die einzelne Drehimpulse in eine gemeinsame Basis überführen, so
dass wir einen Gesamtdrehimpuls \(\vec J=\vec j_1+\vec j_2\) mit der neuen Basis
\(\{\ket{J,M}\}\) erhalten. Dies erreichen wir, in dem wir die Orthonormalität
der alten Basis ausnutzen und den Einheitsoperator \(\mathds 1\) vor die neue
Basis einschieben:
\begin{align}
  \label{eq:1}
  \ket{J,M}&=\underbrace{\left(
      \sum_{m_1,m_2}\ket{j_1,j_2;m_1,m_2}\bra{j_1,j_2;m_1,m_2}
    \right)}_{= \mathds 1}
  \ket{J,M}\notag\\
   \ket{J,M}&= \sum_{m_1,m_2} 
   \underbrace{\braket{j_1,j_2;m_1,m_2}{J,M}}_
   {\text{\textbf{Glebsch-Gordan Koef.}}}
   \ket{j_1,j_2;m_1,m_2}
\end{align}

Da sowohl die alte Basis \( \{\ket{j_1,j_2;m_1,m_2}\}\) als auch die neue Basis
\(\{\ket{J,M}\}\) orthonormiert sind, handelt es sich bei (\ref{eq:1}) um eine
unitäre Transformation. Zu bemerken ist dass bei dem von uns eingefügten
Einheitsoperator nur über die Magnetquantenzahlen \(m_1\) und \(m_2\) summiert
wird. D.h. man hält die Drehimpulsquantenzahlen \(j_1\) und \(j_2\) fest und
entwickelt nur nach den Magnetquantenzahlen, für die gilt \(m_1 = -j_1,\; -j_1+1,
\;-j_1+2, \dots j_1\) und \(m_2 = -j_2, \;-j_2+1,\;-j_2+2, \dots j_2\).\\
Die Koeffizienten \( \braket{j_1,j_2;m_1,m_2}{J,M} \) die die beiden Basen verbinden
heißen \emph{Clebsch-Gordan Koeffizienten}. D.h. das Problem der Addition zweier
Drehimpulse reduziert sich auf das Bestimmen der Glebsch-Gordan Koeffizienten.

Für die Erwartungswerte \(J\) in der neuen Basis gilt:
\begin{equation}
  \label{eq:8}
  \boxed{|j_1-j_2|\leq J\leq j_1+j_2}
\end{equation}

Dies wird durch die Vektoraddition \(\vec J = \vec j_1+\vec j_2\) deutlich.
Siehe dazu Abbildung~\ref{fig:1}
\begin{figure}[!ht]
  \begin{center}
    \input{clebsch-gordan-pics/vektoradd.pdf_t}
    \caption{Addition zweier Drehimpulse.}
    \label{fig:1}
  \end{center}
\end{figure}

Den Erwartungswert \(M\) können wir mit Hilfe des \(J_z=j_{z_1}+j_{z_2}\)
Operators und folgenden Eigenwertgleichungen bestimmen:

\begin{subequations}
  \begin{align}
    J_Z\ket{J,M}&= M \ket{J,M} \label{eq:9a}\\
    j_{z_1}\ket{j_1,j_2;m_1,m_2}&=m_1\ket{j_1,j_2;m_1,m_2} \label{eq:9b}\\
    j_{z_2}\ket{j_1,j_2;m_1,m_2}&=m_2\ket{j_1,j_2;m_1,m_2} \label{eq:9c}
  \end{align}
\end{subequations}

Es gilt:
\begin{align}
  \label{eq:10}
  J_z-j_{z_1}-j_{z_2}&=0\notag\\
  \bra{j_1,j_2;m_1,m_2} J_z-j_{z_1}-j_{z_2} \ket{J,M} &=0\notag\\
  \underbrace{\bra{j_1,j_2;m_1,m_2} J_z\ket{J,M}}_{~\eqref{eq:9a}}
  -\underbrace{\bra{j_1,j_2;m_1,m_2}j_{z_1}\ket{J,M}}_{~\eqref{eq:9b}}\notag\\
  -\underbrace{\bra{j_1,j_2;m_1,m_2}j_{z_2} \ket{J,M}}_{~\eqref{eq:9c}}&=0\notag\\
  M\braket{j_1,j_2;m_1,m_2}{J,M}
  -m_1\braket{j_1,j_2;m_1,m_2}{J,M}\notag\\
  -m_2\braket{j_1,j_2;m_1,m_2}{J,M}&=0\notag\\
  (M-m_1-m_2)\braket{j_1,j_2;m_1,m_2}{J,M}&=0
\end{align}

Aus der Gleichung \eqref{eq:10} folgt das entweder
\(\braket{j_1,j_2;m_1,m_2}{J,M}=0\) oder\\ 
\((M-m_1-m_2)=0\) was zur der wichtigen Beziehung führt:
\begin{equation}
  \label{eq:9}
  \boxed{M=m_1+m_2}
\end{equation}

Dies ist deswegen so wichtig weil dadurch die Entwicklung \eqref{eq:1} nur
auf die Clebsch-Gordan Koeffizienten beschränkt wird, bei denen die Bedingung
\eqref{eq:9} zutrifft. Alle anderen sind \underline{Null}. 

Der Erwartungswert für die Gesamtmagnetquantenzahl \(M\) liegt zwischen:
\begin{equation}
  \label{eq:11}
  -j_1-j_2\leq M \leq j_1+j_2=-(j_1+j_2)\leq M \leq j_1+j_2
  =-J\leq M \leq J
\end{equation}




\subsection*{Eigenschaften der Clebsch-Gordan Koeffizienten}
\label{sec:eigensch-der-clebsch}

Die Clebsch-Gordan-Koeffizienten sind konventionsgemäß \emph{reell}:
\begin{equation}
  \label{eq:2}
  \braket{j_1,j_2;m_1,m_2}{J,M} \in \mathbb R
\end{equation}
D.h. es gilt:
\begin{equation}
  \label{eq:3}
  \braket{j_1,j_2;m_1,m_2}{J,M} = \braket{J,M}{j_1,j_2;m_1,m_2}^*=
  \braket{J,M}{j_1,j_2;m_1,m_2}
\end{equation}

Die Clebsch-Gordan-Koeffizienten sind \emph{orthonormiert}:
\begin{align}
  \label{eq:4}
  \braket{J',M'}{J,M}&=\delta_{J',J}\delta_{J',J}\notag\\
  \bra{J',M'}\underbrace{\left(
      \sum_{m_1,m_2}\ket{j_1,j_2;m_1m_2}\bra{j_1,j_2;m_1m_2}\right)
  }_{=\mathds 1}
  \ket{J,M}&=\delta_{J',J}\delta_{J',J}\notag\\
  \Rightarrow\sum_{m_1,m_2}\braket{J',M'}{j_1,j_2;m_1m_2}
  \braket{j_1,j_2;m_1m_2}{J,M}&=\delta_{J',J}\delta_{J',J}
\end{align}

Aus Gl.~(\ref{eq:3}) und der Gl.~(\ref{eq:4}) erhalten wir eine wichtige Beziehung:
\begin{equation}
  \label{eq:5}
  \sum_{m_1,m_2}\braket{j_1,j_2;m_1m_2}{J',M'}
  \braket{j_1,j_2;m_1m_2}{J,M}=\delta_{J',J}\delta_{J',J}
\end{equation}

Bzw. mit \(J'=J\) und \(M'=M\) folgt:
\begin{equation}
  \label{eq:6}
  \boxed{\sum_{m_1,m_2}\braket{j_1,j_2;m_1m_2}{J,M}^2=1}
\end{equation}

Analog kann man auch folgende Beziehung herleiten:
\begin{equation}
  \label{eq:7}
  \boxed{\sum_{J}\sum_{M}\braket{j_1,j_2;m_1m_2}{J,M}^2=1}
\end{equation}

\subsection*{Bestimmung der Clebsch-Gordan Koeffizienten}
\label{sec:best-der-clebsch}

Wir wollen nun einige Beziehungen herleiten um die Clebsch-Gordan Koeffizienten
bestimmen zu können.

Eine wichtige Beziehung können wir direkt aus \eqref{eq:10} übernehmen:
\begin{equation}
  \label{eq:12}
  \boxed{\braket{j_1,j_2;m_1m_2}{J,M}=0 \quad \text{für} \quad m_1+m_2\neq M}
\end{equation}

Eine weitere Beziehung finden wir aus den Extremalstellen für \(J\) und \(M\)
d.h. wenn gilt:
\begin{equation}
  \label{eq:13}
  J=j_1+j_2 \quad \text{und} \quad M=J=j_1+j_2=m_1+m_2
\end{equation}

Setzen wir \eqref{eq:13} in die Gl.~\eqref{eq:1} ein, so erhalten wir:
\begin{align}
  \label{eq:14}
   \ket{J,J}&=\sum_{m_1=j_1}^{j_1}\sum_{m_2=j_2}^{j_2} 
   \braket{j_1,j_2;m_1,m_2}{J,J}
   \ket{j_1,j_2;m_1,m_2}\notag\\
   &= \braket{j_1,j_2;j_1,j_2}{J,J}
   \ket{j_1,j_2;j_1,j_2}
\end{align}

Mit der Normierungsbedingung \(\braket{J,J}{J,J}\stackrel{!}=1\) folgt:
\begin{align}
  \label{eq:15}
  &\braket{J,J}{J,J}=1\notag\\
  &\bra{j_1,j_2;j_1,j_2} \braket{J,J}{j_1,j_2;j_1,j_2}
  \braket{j_1,j_2;j_1,j_2}{J,J}\ket{j_1,j_2;j_1,j_2}=1\notag\\
  &\bra{j_1,j_2;j_1,j_2} \braket{j_1,j_2;j_1,j_2}{J,J}
  \braket{j_1,j_2;j_1,j_2}{J,J}\ket{j_1,j_2;j_1,j_2}=1\notag\\
  &\braket{j_1,j_2;j_1,j_2}{J,J}^2
  \underbrace{\braket{j_1,j_2;j_1,j_2}{j_1,j_2;j_1,j_2}}_{=1}=1\notag\\
  &\braket{j_1,j_2;j_1,j_2}{J,J}^2=1
\end{align}

Aus \eqref{eq:15} folgt also:
\begin{equation}
  \label{eq:16}
  \braket{j_1,j_2;j_1,j_2}{J,J}=\pm 1
\end{equation}

Um nun zu bestimmen ob das Ergebnis in \eqref{eq:16} positive oder negative ist
d.h. ob \(+1\) oder \(-1\), wurde die sog. \emph{Condon-Shortley
  Phasenkonvention} eingeführt. Sie besagt, dass der Clebsch-Gordan Koeffizient
von der Form:\\
\(\braket{j_1,j_2;j_1,(J-j_1)}{J,J}\) \underline{reell} und \underline{positive}
sein muss. D.h. in unserem Fall \eqref{eq:13} mit \(J=j_1+j_2 \Leftrightarrow
J-j_1=j_2\) folgt:
\begin{equation}
  \label{eq:17}
  \braket{j_1,j_2;j_1,j_2}{J,J}=\underbr{\braket{j_1,j_2;j_1,(J-j_1)}{J,J}}
  _{\text{Konvention: \textbf{positve}}}=1
\end{equation}

D.h.im Spezialfall \eqref{eq:13} ist der Clebsch-Gordan Koeffizient gleich 1. Es
gilt:
\begin{equation}
  \label{eq:18}
  \boxed{\braket{j_1,j_2;j_1,j_2}{J,J}=1 \quad \text{mit}\quad J=j_1+j_2}
\end{equation}

Eine weitere Extremalstelle ist wenn gilt:
\begin{equation}
  \label{eq:19}
    J=j_1+j_2 \quad \text{und} \quad M=-J=-j_1-j_2=m_1+m_2
\end{equation}

Setzen wir \eqref{eq:19} wieder in die Gl.~\eqref{eq:1} ein, so erhalten wir:
\begin{align}
  \label{eq:20}
     \ket{J,-J}&=\sum_{m_1=-j_1}^{-j_1}\sum_{m_2=-j_2}^{-j_2} 
   \braket{j_1,j_2;m_1,m_2}{J,-J}
   \ket{j_1,j_2;m_1,m_2}\notag\\
   &= \braket{j_1,j_2;-j_1,-j_2}{J,-J}
   \ket{j_1,j_2;j_1,j_2}
\end{align}

Die Normierungsbedingung \(\braket{J,-J}{J,-J}\stackrel{!}=1\) und analoge
Rechnung wie in \eqref{eq:15} führt zu:
\begin{equation}
  \label{eq:21}
  \braket{j_1,j_2;-j_1,-j_2}{J,-J}^2=1
  \Leftrightarrow \braket{j_1,j_2;-j_1,-j_2}{J,-J}=\pm 1
\end{equation}

Um festzustellen ob das Ergebnis in \eqref{eq:21} positiv oder negativ ist,
führen wir eine kleine Substitution durch.
\begin{align}
  \label{eq:22}
  j_1'=-j_1; \quad j_2'=-j_2
  \Rightarrow -J=-j_1-j_2=j_1'+j_2'\equiv J'
\end{align}

\eqref{eq:22} eingesetzt in \(\braket{j_1,j_2;-j_1,-j_2}{J,-J}\) führt zu:
\begin{align}
  \label{eq:23}
  \braket{j_1,j_2;j_1',j_2'}{J,J'}
  =\underbrace{\braket{j_1,j_2;j_1',(J'-j_1')}{J,J'}}
   _{\text{Konvention: \textbf{positve}}}
   \quad \text{mit }  j_2'=J'-j_1'
\end{align}

Aus \eqref{eq:22} und \eqref{eq:23} folgt also, dass das Ergebnis in
Gl.~\eqref{eq:21} positiv sein muss. Wir erhalten also:
\begin{equation}
  \label{eq:24}
  \boxed{\braket{j_1,j_2;-j_1,-j_2}{J,-J}=1 \quad \text{mit}\quad J=j_1+j_2}
\end{equation}

D.h. an den beiden Extrimalstellen, mit \(M=-J\) und \(M=J\), sind die
Clebsch-Gordan Koeffizienten gleich Eins.

Mit \eqref{eq:18} bzw. \eqref{eq:24} können wir schon zwei der Clebsch-Gordan
Koeffizienten bestimmen. Um alle weiteren ebenfalls bestimmen zu können, wollen
wir jetzt eine \textbf{Rekursionsformel} herleiten, mit deren Hilfe wir
ausgehend von einem CGK. (Clebsch-Gordan Koeffizienten) einen weiteren bestimmen
können, um so sukzessive alle weiteren berechnen zu können. 

Dazu betrachten wir das Matrixelement \(\bra{j_1,j_2;m_1,m_2}J_{\pm}\ket{J,M}\)
und lassen den Auf/Absteigeoperator \(J_{\pm}=j_{1\pm}+j_{2\pm}\) einmal auf die
rechte Seite und einmal auf die linke Seite wirken.

Mit der Eigenwertgleichung:
\begin{equation}
  \label{eq:25}
  J_{\pm}\ket{J,M}=\hbar\sqrt{(J\mp M)(J\pm M+1)}\ket{J,M\pm 1}
\end{equation}

Erhalten wir:
\begin{equation}
  \label{eq:26}
  \bra{j_1,j_2;m_1,m_2}J_{\pm}\ket{J,M}=\hbar\sqrt{(J\mp M)
    (J\pm M+1)}\braket{j_1,j_2;m_1,m_2}{J,M\pm 1}
\end{equation}

Lassen wir den Auf/Absteigeoperator auf die linke Seite wirken, so brauchen wir
folgende Eigenwertgleichung:
\begin{align}
  \label{eq:27}
   j_{1\mp}\ket{j_1,m_1}&=\hbar\sqrt{(j_1\pm m_1)(j_1\mp m_1+1)}
   \ket{j_1,m_1\mp1}\notag\\
   (j_{1\mp}\ket{j_1,m_1})^\dagger&=(\hbar\sqrt{(j_1\pm m_1)(j_1\mp m_1+1)}
   \ket{j_1,m_1\mp1})^\dagger\notag\\
   \bra{j_1,m_1}j_{1\mp}^\dagger&=\hbar\sqrt{(j_1\pm m_1)(j_1\mp m_1+1)}
   \bra{j_1,m_1\mp1}\notag\\
   \bra{j_1,m_1}j_{1\pm}&=\hbar\sqrt{(j_1\pm m_1)(j_1\mp m_1+1)}
   \bra{j_1,m_1\mp 1}
\end{align}
Bzw. analog:
\begin{equation}
  \label{eq:28}
  \bra{j_2,m_2}j_{1\pm}=\hbar\sqrt{(j_2\pm m_2)(j_2\mp m_2+1)}
   \bra{j_2,m_2\mp 1}
\end{equation}

Mit \eqref{eq:27} und \eqref{eq:28} erhalten wir für das Matrixelement:
\begin{align}
  \label{eq:29}
  &\bra{j_1,j_2;m_1,m_2}J_{\pm}\ket{J,M}=\bra{j_1,j_2;m_1,m_2}j_{1\pm}+j_{2\pm}
  \ket{J,M}\notag\\
  &\qquad
  =\bra{j_1,j_2;m_1,m_2}j_{1\pm}\ket{J,M}+\bra{j_1,j_2;m_1,m_2}j_{2\pm}
  \ket{J,M}\notag\\
  &\qquad
  =\hbar\sqrt{(j_1\pm m)(j_1\mp m+1)}
   \braket{j_1,m\mp 1}{J,M}\notag\\
   &\qquad\qquad
   +\hbar\sqrt{(j_2\pm m)(j_2\mp m+1)}
   \braket{j_2,m\mp 1}{J,M}\notag\\
\end{align}

Setzen wir nun die beiden Gleichungen \eqref{eq:26} und \eqref{eq:29} gleich und
kürzen auf beiden Seiten durch \(\hbar\), so erhalten wir folgende
\emph{rekursive} Beziehung:

\begin{equation}
  \label{eq:30}
  \boxed{
    \begin{split}
      &\sqrt{(J\mp M)
        (J\pm M+1)}\braket{j_1,j_2;m_1,m_2}{J,M\pm 1}\\
      &\qquad
      = \sqrt{(j_1\pm m_1)(j_1\mp m_1+1)}
      \braket{j_1,j_2;,m_1\mp 1,m_2}{J,M}\\
      &\qquad\qquad
      +\sqrt{(j_2\pm m_2)(j_2\mp m_2+1)}
      \braket{j_1,j_2;m_1,m_2\mp 1}{J,M}
    \end{split}}
\end{equation}





\subsection*{Beispiel}

Addition Zweier Drehimpulse anhand eines einfachen Beispiels von Teilchen mit Spin \(\frac{1}{2}\). Die Koeffizienten nehmen folgenden Werte an:

\begin{equation}
  \label{eq:31}
  j_1 = \frac{1}{2}\quad j_2 = \frac{1}{2}\qquad m_1 = -\frac{1}{2},\frac{1}{2}\quad m_2=-\frac{1}{2},\frac{1}{2}
\end{equation}

\begin{equation}
  \label{eq:32}
  0 \leq J \leq 1\qquad M = -1,0,1
\end{equation}


Allgemein die Basis \(\ket{J,M}\) in der Clebsch-Gordan Basis \(\ket{j_1,j_2,m_1,m_2}\) ausgedrückt lautet:


\begin{equation}
  \label{eq:33}
  |J,M\rangle = \sum_{m_1}\sum_{m_2}\langle j_1,j_2;m_1,m_2|J,M\rangle |j_1,j_2;m_1,m_2\rangle
\end{equation}


Betrachte das 'größste' und das 'kleinste' Element \(\ket{1,1}\) und \(\ket{1,-1}\) weil es nur ein möglichen CGK aus der Summation übrigbleibt


\begin{equation}
  \label{eq:35}
  \ket{1,1} =\underbrace{ \braket{ \frac{1}{2}\frac{1}{2};\frac{1}{2}\frac{1}{2} }{11} }_{=1} \ket{ \frac{1}{2}\frac{1}{2};  \frac{1}{2} \frac{1}{2}}
\end{equation}

\begin{equation}
  \label{eq:36}
    \ket{1,-1} =\underbrace{ \braket{ \frac{1}{2}\frac{1}{2};-\frac{1}{2}-\frac{1}{2} }{1-1} }_{=1} \ket{ \frac{1}{2}\frac{1}{2};  -\frac{1}{2} -\frac{1}{2}}
\end{equation}



Nach der Beziehung (\ref{eq:6}) muss die Quadratische Summe aller Koeffizienten 1 ergeben. Da es nur ein möglicher CGK in beiden Fällen da ist, steht das Ergebnis gleich fest. 

Es bleiben noch zwei weniger einfache Fälle \(\ket{1,0}\) und \(\ket{0,0}\). Verwende hierfür die Rekursionsformel (\ref{eq:30}). In den meisten Fällen gelingt eine Lösung in dem man die einzelnen Faktoren so wält, dass auf der linken Seite der gesuchte CGK steht. Auf der rechten Seite stehen dann entweder schon bekannte oder unbekannte CGK's. Wir versuchen das für \(\ket{1,0}\). Wähle \(m_1=\frac{1}{2}\), \(m_1=-\frac{1}{2}\), \(J=1\) und \(M=1\)

\begin{align}
  \label{eq:34}
  \sqrt{2}\braket{\frac{1}{2},\frac{1}{2};\frac{1}{2},-\frac{1}{2}}{10}= \underbr{ \braket{\frac{1}{2},\frac{1}{2};\frac{1}{2},\frac{1}{2}}{11}}_{=1 \qquad (\ref{eq:35})}  \\
\rightarrow \braket{\frac{1}{2},\frac{1}{2};\frac{1}{2},-\frac{1}{2}}{10}=\frac{1}{\sqrt{2}}
\end{align}

Wähle nun \(m_1=-\frac{1}{2}\), \(m_1=\frac{1}{2}\), \(J=1\) und \(M=1\)




\begin{align}
  \label{eq:37}
  \sqrt{2}\braket{\frac{1}{2},\frac{1}{2};-\frac{1}{2},\frac{1}{2}}{10}= \underbr{ \braket{\frac{1}{2},\frac{1}{2};\frac{1}{2},\frac{1}{2}}{11}}_{=1 \qquad (\ref{eq:35})}  \\
\rightarrow \braket{\frac{1}{2},\frac{1}{2};-\frac{1}{2},\frac{1}{2}}{10}=\frac{1}{\sqrt{2}}
\end{align}


Somit können wir für \(\ket{1,0}\) das Ergebnis schreiben

\begin{equation}
  \label{eq:38}
  \ket{1,0} = \underbr{\braket{\frac{1}{2},\frac{1}{2};\frac{1}{2}-\frac{1}{2}}{10}}_{\frac{1}{\sqrt{2}}}\ket{ \frac{1}{2},\frac{1}{2};\frac{1}{2}-\frac{1}{2} }  +\underbr{\braket{\frac{1}{2},\frac{1}{2};-\frac{1}{2}\frac{1}{2}}{10}}_{\frac{1}{\sqrt{2}}}\ket{\frac{1}{2},\frac{1}{2};-\frac{1}{2}\frac{1}{2}} 
\end{equation}


Für den Zustand \(\ket{0,0}\) erzeugen wir jetzt anstelle auf der linken Seite zuerst auf der rechten Seite den gesuchten CGK. Wähle die Werte  \(m_1=-\frac{1}{2}\), \(m_1=-\frac{1}{2}\), \(J=0\) und \(M=0\)

\begin{align}
  \label{eq:39}
  0 = \braket{\frac{1}{2},\frac{1}{2};\frac{1}{2},-\frac{1}{2}}{00} +  \braket{\frac{1}{2},\frac{1}{2};-\frac{1}{2},\frac{1}{2}}{00}\\
\Leftrightarrow \underbr{\braket{\frac{1}{2},\frac{1}{2};\frac{1}{2},-\frac{1}{2}}{00}}_{\text{positiv}} = -\braket{\frac{1}{2},\frac{1}{2};-\frac{1}{2},\frac{1}{2}}{00}
\end{align}

Der CGK \(\braket{\frac{1}{2},\frac{1}{2};\frac{1}{2},-\frac{1}{2}}{00} \) ist nach der Condon-Shortley
  Phasenkonvention \eqref{eq:17} eine positive Größe. Also ergibt sich für den übrigbleibenden CGK in der Gleichung (\ref{eq:39}) etwas negatives. Aufrund der Normierungsbedingung \(\braket{0,0}{0,0}=1\) ist es möglich beide CGK's herauszufinden. Wir nehmen an dass beide vom Betrag her gleich sind und nur vom Vorzeichen sich unterscheiden also dass gilt:
  \begin{equation}
    \label{eq:40}
    C_1 = -C_2
  \end{equation}

Aus der Normierungsbedinung ergibt sich

\begin{align}
  \label{eq:41}
  |C_1|^2 + |C_2|^2 =  |C_1|^2 + |-C_1|^2 = 1 \\
\Leftrightarrow C_1 = \frac{1}{\sqrt{2}}, \quad C_2 = -\frac{1}{\sqrt{2}}
\end{align}

Mit \(C_1 =\braket{\frac{1}{2},\frac{1}{2};\frac{1}{2},-\frac{1}{2}}{00} \) und  \(C_2 =\braket{\frac{1}{2},\frac{1}{2};-\frac{1}{2},\frac{1}{2}}{00} \) folgt für den Zustand \(\ket{0,0}\)

\begin{equation}
  \label{eq:42}
   \ket{0,0} = \underbr{\braket{\frac{1}{2},\frac{1}{2};\frac{1}{2}-\frac{1}{2}}{00}}_{\frac{1}{\sqrt{2}}}\ket{ \frac{1}{2},\frac{1}{2};\frac{1}{2}-\frac{1}{2} }  +\underbr{\braket{\frac{1}{2},\frac{1}{2};-\frac{1}{2}\frac{1}{2}}{00}}_{-\frac{1}{\sqrt{2}}}\ket{\frac{1}{2},\frac{1}{2};-\frac{1}{2}\frac{1}{2}} 
\end{equation}

\subsection{Beispiel 2}
\label{sec:beispiel-2}
In diesem Beispiel betrachten wir ein Teilchen mit dem Bahndrehimpuls \(L=1\)
und einem Spin \(S=\frac 1 2\). Auch hier wollen wir die Basis
\(\{\ket{L,M}\otimes\ket{S,M}\}\equiv\{\ket{j_1,m_1}\otimes\ket{j_2,m_2}\}
\equiv\{\ket{j_1,j_2;m_1,m_2}\}\) in eine gemeinsame Drehimpulsbasis
\(\{\ket{J,M}\}\) überführen.

Es gilt:
\begin{subequations}
  \begin{align}
    &L \equiv j_1 = 1 \Rightarrow m_1=-1,\, 0,\, 1 \label{eq:43}\\
    &S\equiv j_2 = \frac 1 2 \Rightarrow m_2=-\frac 1 2,\, 
    \frac 1 2 \label{eq:44}\\
    &|j_1-j_2|\leq J\leq j_1+j_2  \Rightarrow J= \frac 1 2,\,\frac 3 2
    \label{eq:45}\\
    &M=-J,\dots J\label{eq:46}
  \end{align}
\end{subequations}

\subsubsection{Vorgehensweise zu Bestimmung der gesuchten Zustände}
\label{sec:vorgehensweise}

\begin{itemize}
\item[\textbf{Schritt 1}:] Wähle den größten Basiszustand \(\ket{J,J}\) mit \(J=j_1+j_2\), der
  Clebsch-Gordan Koeffizient in diesem Zustand ist konventionsgemäß 1. Siehe
auch \eqref{eq:18}. In unserem Fall ist es der Zustand \(\ket{\frac 3 2, \frac 3
2}\)
\item[\textbf{Schritt 2}:] Berechne den nächst kleineren Zustand \(\ket{J, J-1}\) in
  unserem Fall ist es der \(\ket{\frac 3 2, \frac 1 2}\). Um die Clebsch-Gordan
  Koeffizienten in diesem Zustand zu bestimmen benutze die untere Zeile der
  Rekursionsformel \eqref{eq:30} die da lautet:
  \begin{equation}
    \label{eq:47}
    \boxed{
    \begin{split}
      &\sqrt{(J+ M)
        (J- M+1)}\braket{j_1,j_2;m_1,m_2}{J,M- 1}\\
      &\qquad
      = \sqrt{(j_1- m_1)(j_1+m_1+1)}
      \braket{j_1,j_2;,m_1+ 1,m_2}{J,M}\\
      &\qquad\qquad
      +\sqrt{(j_2- m_2)(j_2+ m_2+1)}
      \braket{j_1,j_2;m_1,m_2+ 1}{J,M}
    \end{split}}
  \end{equation}
Wähle dazu die Parameter \(m_1,\, m_2,\, M\) so, dass die Rekursionsformel
\eqref{eq:47} auf der linken Seite den gesuchten CGK. enthält. Auf der rechten
Seite wird dann der schon aus dem vorhergehenden Schritt bekannter CGK. stehen.
Bemerkung: die Rekursionsformel \eqref{eq:47} ist nichts Anderes als das
Anwenden des Absteigeoperators \(J_-\).
\item[\textbf{Schritt 3}:] Wiederhole \emph{Schritt 2} bis der Zustand
  \(\ket{J,-J}\) erreicht ist. In unserem Fall ist es der Zustand \(\ket{\frac 3
  2, -\frac 3 2}\) Der Clebsch-Gordan Koeffizient in diesem Zustand
  ist ebenfalls 1. Siehe dazu auch \eqref{eq:24}.
\item[\textbf{Schritt 4}:] Erniedrige \(J\) um Eins und berechne den dafür
  größtmöglichen Zustand d.h. den Zustand \(\ket{J-1,J-1}\), in unserem Fall
  \(\ket{\frac 1 2, \frac 1 2}\). Die unbekannten CGK. in diesem Zustand können
  bestimmt werden in dem man das Skalarprodukt zwischen dem schon aus der
  vorhergegangenen Rechnung bekannten Zustand \(\ket{J,J-1}\) in unserem Fall
  \(\ket{\frac 3 2, \frac 1 2}\) und dem gesuchten  Zustand \(\ket{J-1,J-1}\)
  bildet. Wegen der Orthogonalitätsbedingung muss gelten:
  \begin{equation}
    \label{eq:48}
    \braket{J,J-1}{J-1,J-1}\stackrel{!}=0
  \end{equation}
  Da die CGK. in dem Zustand \(\ket{J,J-1}\) bekannt sind kann man mit deren Hilfe,
  der Normalitätsbedingung und der \emph{Condon-Shortley Phasenkonvention} die
  unbekannten CGK. im Zustand \(\ket{J-1,J-1}\) bestimmen.
\item[\textbf{Schritt 5}:] Wiederhole die \emph{Schritte 2-4} bis alle Zustände
  bestimmt sind. D.h. in unserem Fall \(\left\{\ket{\frac 3 2, \frac 3 2},\,\ket{\frac 3 2,
    \frac 1 2},\,\ket{\frac 3 2, -\frac 1 2},\, \ket{\frac 3 2, -\frac 3
    2},\,\ket{\frac 1 2, \frac 1 2},\,\ket{\frac 1 2, -\frac 1 2} \right\}\)
\end{itemize}

\subsubsection{Berechnung der gesuchten Zustände}
\label{sec:berechn-der-gesucht}

Für den größtmöglichen Zustand gilt Gl.~\eqref{eq:1}:
\begin{align}
  \label{eq:49}
  \ket{\frac 3 2,\frac 3 2} &= \sum_{m_1=-1}^1 \sum_{m_2=-\frac 1 2}^{\frac 1 2}
  \braket{1,\frac 1 2;m_1,m_2}{\frac 3 2,\frac 3 2}\ket{1,\frac 1 2;m_1,m_2}
  \notag\\
  &= \sum_{m_1=-1}^1\braket{1,\frac 1 2;m_1,-\frac 1 2}{\frac 3 2,\frac 3 2}
  \ket{1,\frac 1 2;m_1,-\frac 1 2}\notag\\
  &\qquad\qquad\qquad
  + \braket{1,\frac 1 2;m_1,\frac 1 2}{\frac 3 2,\frac 3 2}
  \ket{1,\frac 1 2;m_1,\frac 1 2}\notag\\
   &= \underbrace{\braket{1,\frac 1 2;-1,-\frac 1 2}{\frac 3 2,\frac 3 2}}
   _{=0 \text{ da \(m_1+m_2\neq M\)}}
  \ket{1,\frac 1 2;-1,-\frac 1 2}\notag\\
  &\qquad\qquad
  + \underbrace{\braket{1,\frac 1 2;-1,\frac 1 2}{\frac 3 2,\frac 3 2}}_{=0}
  \ket{1,\frac 1 2;-1,\frac 1 2}\notag\\
   &\qquad\qquad
  +\underbrace{\braket{1,\frac 1 2;0,-\frac 1 2}{\frac 3 2,\frac 3 2}}_{=0}
  \ket{1,\frac 1 2;0,-\frac 1 2}\notag\\
  &\qquad\qquad
  + \underbrace{\braket{1,\frac 1 2;0,\frac 1 2}{\frac 3 2,\frac 3 2}}_{=0}
  \ket{1,\frac 1 2;0,\frac 1 2}\notag\\
  &\qquad\qquad
  +\underbrace{\braket{1,\frac 1 2;1,-\frac 1 2}{\frac 3 2,\frac 3 2}}_{=0}
  \ket{1,\frac 1 2;1,-\frac 1 2}\notag\\
  &\qquad\qquad\qquad
  + \braket{1,\frac 1 2;1,\frac 1 2}{\frac 3 2,\frac 3 2}
  \ket{1,\frac 1 2;1,\frac 1 2}
\end{align}

In der Gleichung \eqref{eq:49} erfüllt nur eine CGK. die Bedingung \(m_1+m_2=M\)
somit lautet der Zustand:
\begin{align}
  \label{eq:50}
  \ket{\frac 3 2,\frac 3 2} &= \underbrace{
    \braket{1,\frac 1 2;1,\frac 1 2}{\frac 3 2,\frac 3 2}}
  _{=1 \text{ Siehe Gl.~\eqref{eq:18} }}
  \ket{1,\frac 1 2;1,\frac 1 2}\notag\\
  &=1\cdot\ket{1,\frac 1 2;1,\frac 1 2}
\end{align}

\emph{Bemerkung}: Im Folgenden werde die CGK. die Bedingung \(m_1+m_1=M\) nicht erfüllen
bei der Summation gleich weggelassen.

Der nächste zu bestimmende Zustand ist:
\begin{equation}
  \label{eq:51}
  \ket{\frac 3 2, \frac 1 2}=
  \braket{1, \frac 1 2; 1, -\frac 1 2}{\frac 3 2, \frac 1 2}
  \ket{1, \frac 1 2; 1, -\frac 1 2}
  +\braket{1, \frac 1 2; 0, \frac 1 2}{\frac 3 2, \frac 1 2}
  \ket{1, \frac 1 2; 0, \frac 1 2}
\end{equation}

Um den ersten unbekannten CGK. \(\braket{1, \frac 1 2; 1, -\frac 1 2}{\frac 3 2,
  \frac 1 2}\) in der Gl.~\eqref{eq:51} zu bestimmen, setzen wir \(m_1=1,
\;m_2=-\frac 1 2\) und \( M=\frac 3 2\) in die Rekursionsgleichung \eqref{eq:47}
ein, so dass auf der linken Seite der Gleichung der unbekannte CGK. \(\braket{1,
\frac 1 2; 1, -\frac 1 2}{\frac 3 2, \frac 1 2}\) steht. Wir erhalten:
\begin{align}
  \label{eq:52}
  &\sqrt 3 \braket{1, \frac 1 2; 1, -\frac 1 2}{\frac 3 2,\frac 1 2}
  = \underbrace{
    \braket{1, \frac 1 2; 1, \frac 1 2}{\frac 3 2,\frac 3 2}}
  _{=1 \text{ Siehe Gl. \eqref{eq:18}}}\notag\\
 \Leftrightarrow &\braket{1, \frac 1 2; 1, -\frac 1 2}{\frac 3 2,\frac 1 2}
 = \frac 1 {\sqrt 3} 
\end{align}

Um den zweiten unbekannten CGK. aus der Gl.~\eqref{eq:51} zu bestimmen, setzen
wir \(m_1=0, \;m_2=\frac 1 2\) und \( M=\frac 3 2\) ebenfalls in die
Rekursionsgleichung \eqref{eq:47} ein so dass wir wieder auf der linken Seite
der Gl.~\eqref{eq:47} den gesuchten CGK. \(\braket{1, \frac 1 2; 0, \frac 1
2}{\frac 3 2,\frac 1 2}\) erhalten. Wir erhalten:
\begin{align}
  \label{eq:53}
  &\sqrt 3 \braket{1, \frac 1 2; 0, \frac 1 2}{\frac 3 2,\frac 1 2}
  = \sqrt 2 \underbrace{
    \braket{1, \frac 1 2; 1, \frac 1 2}{\frac 3 2,\frac 3 2}}
  _{=1 \text{ Siehe Gl. \eqref{eq:18}}}\notag\\
 \Leftrightarrow &\braket{1, \frac 1 2; 0, \frac 1 2}{\frac 3 2,\frac 1 2}
 = \sqrt {\frac 2  3} 
\end{align}

Zusammen mit Gl.~\eqref{eq:52} und \eqref{eq:53} lautet also der Zustand
\eqref{eq:51}:
\begin{equation}
  \label{eq:54}
  \ket{\frac 3 2, \frac 1 2}=
  \frac 1 {\sqrt 3}
  \ket{1, \frac 1 2; 1, -\frac 1 2}
  +\sqrt {\frac 2 3} 
  \ket{1, \frac 1 2; 0, \frac 1 2}
\end{equation}

Der nächste zu bestimmende Zustand lautet:
\begin{multline}
  \label{eq:55}
    \ket{\frac 3 2, -\frac 1 2}=
  \braket{1, \frac 1 2; 0, -\frac 1 2}{\frac 3 2, -\frac 1 2}
  \ket{1, \frac 1 2; 0, -\frac 1 2}\\
  +\braket{1, \frac 1 2; -1, \frac 1 2}{\frac 3 2, -\frac 1 2}
  \ket{1, \frac 1 2; -1, \frac 1 2}
\end{multline}

Setzen wir auch hier analog \(m_1=0,\;m_2=-\frac 1 2\) und \(M=\frac 1 2\) in
die Rekursionsgleichung ein, so erhalten wir:
\begin{align}
  \label{eq:56}
   &2 \braket{1, \frac 1 2; 0, -\frac 1 2}{\frac 3 2,-\frac 1 2}
  = \sqrt 2 \underbrace{
    \braket{1, \frac 1 2; 1, -\frac 1 2}{\frac 3 2,\frac 1 2}}
  _{=\frac 1 {\sqrt 3} \text{ Siehe Gl. \eqref{eq:52}}}
  +\underbrace{
    \braket{1, \frac 1 2; 0, \frac 1 2}{\frac 3 2,\frac 1 2}}
  _{=\sqrt{\frac 2 3} \text{ Siehe Gl. \eqref{eq:53}}}\notag\\
 \Leftrightarrow &\braket{1, \frac 1 2; 0, -\frac 1 2}{\frac 3 2,-\frac 1 2}
 = \sqrt {\frac 2  3} 
\end{align}

Setzen wir wiederum \(m_1=-1,\;m_2=\frac 1 2\) und \(M=\frac 1 2\) in
die Rekursionsgleichung ein, so erhalten wir:
\begin{align}
  \label{eq:57}
   &2 \braket{1, \frac 1 2; -1, \frac 1 2}{\frac 3 2,-\frac 1 2}
  = \sqrt 2 \underbrace{
    \braket{1, \frac 1 2; 0, \frac 1 2}{\frac 3 2,\frac 1 2}}
  _{=\sqrt{\frac 2  3} \text{ Siehe Gl. \eqref{eq:53}}}\notag\\
  \Leftrightarrow &\braket{1, \frac 1 2; -1, \frac 1 2}{\frac 3 2,-\frac 1 2}
 = \frac 1 {\sqrt 3} 
\end{align}

Mit Gl. \eqref{eq:56} und \eqref{eq:57} in \eqref{eq:55} folgt für den gesuchten
Zustand:
\begin{equation}
  \label{eq:58}
   \ket{\frac 3 2, -\frac 1 2}=
  \sqrt{\frac 2 3}
  \ket{1, \frac 1 2; 0, -\frac 1 2}
  +\frac 1 {\sqrt 3}
  \ket{1, \frac 1 2; -1, \frac 1 2}
\end{equation}

Als nächstes wollen wir noch den niedrigsten Zustand \(\ket{\frac 3 2, -\frac 3
  2}\) bestimmen. Für den CGK. in diesem Zustand gilt laut \eqref{eq:24} \(
 \braket{1,\frac 1 2;-1,-\frac 1 2}{\frac 3 2,-\frac 3 2}=1\). Dies wollen wir
 nun durch explizite Rechnung beweisen. Es gilt also folgenden Zustand zu
 bestimmen:
 \begin{equation}
   \label{eq:59}
   \ket{\frac 3 2, -\frac 3 2}= \braket{1,\frac 1 2;-1,-\frac 1 2}
   {\frac 3 2,-\frac 3 2}\ket{1,\frac 1 2;-1,-\frac 1 2}
 \end{equation}

Setzen wir \(m_1=-1,\;m_2=-\frac 1 2\) und \(M=-\frac 1 2\) in die
Rekursionsgleichung ein, so erhalten wir:
\begin{align}
  \label{eq:60}
     &\sqrt 3 \braket{1, \frac 1 2; -1, -\frac 1 2}{\frac 3 2,-\frac 3 2}
  = \sqrt 2 \underbrace{
    \braket{1, \frac 1 2; 0, -\frac 1 2}{\frac 3 2,-\frac 1 2}}
  _{=\sqrt{\frac 2 3} \text{ Siehe Gl. \eqref{eq:56}}}\notag\\
  &\qquad\qquad\qquad\qquad\qquad\qquad\qquad\qquad+\underbrace{
    \braket{1, \frac 1 2; -1, \frac 1 2}{\frac 3 2,-\frac 1 2}}
  _{=\frac 1 {\sqrt 3} \text{ Siehe Gl. \eqref{eq:57}}}\notag\\
 \Leftrightarrow &\braket{1, \frac 1 2; -1, -\frac 1 2}{\frac 3 2,-\frac 3 2}
 = \frac 3  3 = 1 
\end{align}

Damit wurde die Gleichung \eqref{eq:24} nochmal bestätigt. Der Zustand lautete
somit:
\begin{equation}
  \label{eq:61}
  \ket{\frac 3 2, -\frac 3 2}= 1\cdot\ket{1,\frac 1 2;-1,-\frac 1 2}
\end{equation}

Wir sind nun beim \emph{Schritt 4} im Abschnitt \ref{sec:vorgehensweise}
angelangt. D.h. wir müssen jetzt \(J\) um Eins erniedrigen und den dafür
größtmöglichen Zustand bestimmen. Der da lautet:
\begin{multline}
  \label{eq:62}
   \ket{\frac 1 2, \frac 1 2}= \braket{1,\frac 1 2;1,-\frac 1 2}
   {\frac 1 2,\frac 1 2}\ket{1,\frac 1 2;1,-\frac 1 2}\\
   +\braket{1,\frac 1 2;0,\frac 1 2}
   {\frac 1 2,\frac 1 2}\ket{1,\frac 1 2;0,\frac 1 2}
\end{multline}

Wie in der Beschreibung steht bilden wir jetzt ein Skalarprodukt mit dem schon
berechneten Zustand \(\ket{\frac 3 2, \frac 1 2}\) Gl:~\eqref{eq:54} und erhalten:
\begin{multline}
  \label{eq:63}
  \braket{\frac 3 2,\frac 1 2}{\frac 1 2, \frac 1 2}
  = \left(
    \frac 1 {\sqrt 3}\bra{1,\frac 1 2; 1,-\frac 1 2}
    +\sqrt{\frac 2 3}\bra{1, \frac 1 2; 0, \frac 1 2}\right)\\
  \times\left(
    \braket{1,\frac 1 2;1,-\frac 1 2}{\frac 1 2,\frac 1 2}
    \ket{1,\frac 1 2;1,-\frac 1 2}
    +\braket{1,\frac 1 2;0,\frac 1 2}{\frac 1 2,\frac 1 2}
    \ket{1,\frac 1 2;0,\frac 1 2}\right)
\end{multline}

Die Gl.~\eqref{eq:63} ausmultiplizieren und anwenden der Orthogonalität:

\begin{align}
  \label{eq:65}
  &\braket{1,\frac 1 2; 1,-\frac 1 2}{1,\frac 1 2; 1,-\frac 1 2}
  =\braket{1, \frac 1 2; 0, \frac 1 2}{1, \frac 1 2; 0, \frac 1 2}=1\\
  &\braket{1,\frac 1 2; 1,-\frac 1 2}{1,\frac 1 2;0,\frac 1 2}
  =\braket{1,\frac 1 2;0,\frac 1 2}{1,\frac 1 2; 1,-\frac 1 2}=0
\end{align}

Führt zu:
\begin{equation}
  \label{eq:64}
  \braket{\frac 3 2,\frac 1 2}{\frac 1 2, \frac 1 2}
  = \frac 1 {\sqrt 3}\braket{1,\frac 1 2;1,-\frac 1 2}{\frac 1 2,\frac 1 2}
  +\sqrt{\frac 2 3}\braket{1,\frac 1 2;0,\frac 1 2}{\frac 1 2,\frac 1 2}
  \stackrel{!}=0
\end{equation}

In der Gl.~\eqref{eq:64} wurde die Orthogonalitätsbedingung \(\braket{\frac 3
2,\frac 1 2}{\frac 1 2, \frac 1 2} \stackrel{!}=0\) ausgenutzt.

Die Gl.~\eqref{eq:64} umgestellt ergibt:
\begin{equation}
  \label{eq:66}
  \braket{1,\frac 1 2;0,\frac 1 2}{\frac 1 2,\frac 1 2}
  = -\frac 1 {\sqrt 2}\braket{1,\frac 1 2;1,-\frac 1 2}{\frac 1 2,\frac 1 2}
\end{equation}

Aus der Normierungsbedingung Gl.~\eqref{eq:6} folgt:
\begin{equation}
  \label{eq:67}
  \braket{1,\frac 1 2;1,-\frac 1 2}{\frac 1 2,\frac 1 2}^2+
  \underbrace{\braket{1,\frac 1 2;0,\frac 1 2}{\frac 1 2,\frac 1 2}^2}
  _{\eqref{eq:66}} =1
\end{equation}

Die Gl.~\eqref{eq:66} in \eqref{eq:67} ergibt:
\begin{align}
  \label{eq:68}
   &\braket{1,\frac 1 2;1,-\frac 1 2}{\frac 1 2,\frac 1 2}^2
   +\frac 1 2 \braket{1,\frac 1 2;1,-\frac 1 2}{\frac 1 2,\frac 1 2}^2=1
   \notag\\
   \Leftrightarrow 
   &\frac 3 2 \braket{1,\frac 1 2;1,-\frac 1 2}{\frac 1 2,\frac 1 2}^2=1
   \notag\\
   \Leftrightarrow&\underbrace{ 
    \braket{1,\frac 1 2;1,-\frac 1 2}{\frac 1 2,\frac 1 2}}
  _{\text{Konvention: \textbf{positiv}}}=\sqrt{\frac 2 3}
\end{align}

Die Gl.~\eqref{eq:68} in die Gl.~\eqref{eq:66} eingesetzt ergibt für den zweiten
gesuchten CGK.:
\begin{equation}
  \label{eq:69}
  \braket{1,\frac 1 2;0,\frac 1 2}{\frac 1 2,\frac 1 2}=-\frac 1 {\sqrt 3}
\end{equation}

Mit den Gleichungen \eqref{eq:68} und \eqref{eq:69} bekommen wir schlussendlich
für den gesuchten Zustand Gl.~\eqref{eq:62}:
\begin{equation}
  \label{eq:70}
    \ket{\frac 1 2, \frac 1 2}=\sqrt{\frac 2 3}\ket{1,\frac 1 2;1,-\frac 1 2}
   -\frac 1 {\sqrt 3}\ket{1,\frac 1 2;0,\frac 1 2}
\end{equation}

Zum Schluss wollen wir noch den letzten verbliebenen Zustand bestimmen:
\begin{equation}
  \label{eq:71}
  \ket{\frac 1 2,-\frac 1 2}=\braket{1,\frac 1 2;0,-\frac 1 2}
  {\frac 1 2,-\frac 1 2 }\ket{1,\frac 1 2;0,-\frac 1 2}
\end{equation}

Dazu benutzen wir wieder unsere Rekursionsformel Gl.~\eqref{eq:47}. Mit
\(m_1=0,\; m_2=-\frac 1 2\) und \(M=\frac 1 2\) erhalten wir:
\begin{align}
  \label{eq:72}
  &\braket{1,\frac 1 2;0,-\frac 1 2}{\frac 1 2,-\frac 1 2 }
 =\sqrt 2 \underbrace{
    \braket{1, \frac 1 2; 1, -\frac 1 2}{\frac 1 2,-\frac 1 2}}
  _{=\sqrt{\frac 2 3} \text{ Siehe Gl. \eqref{eq:68}}}\notag\\
  &\qquad\qquad\qquad\qquad\qquad\qquad\qquad\qquad+\underbrace{
    \braket{1, \frac 1 2; 0, \frac 1 2}{\frac 1 2,-\frac 1 2}}
  _{=-\frac 1 {\sqrt 3} \text{ Siehe Gl. \eqref{eq:69}}}\notag\\
 \Leftrightarrow &\braket{1, \frac 1 2; 0, -\frac 1 2}{\frac 1 2,-\frac 1 2}
 = \frac 1  {\sqrt 3} 
\end{align}

Die Gleichung \eqref{eq:72} in die Gl.~\eqref{eq:71} eingesetzt ergibt:
\begin{equation}
  \label{eq:73}
   \ket{\frac 1 2,-\frac 1 2}=\frac 1  {\sqrt 3} \ket{1,\frac 1 2;0,-\frac 1 2}
\end{equation}

Damit haben wir also alle Zustände der Gesamtdrehimpulsbasis bestimt. Hier
nochmal zusammengefasst:

\[\boxed{
\begin{aligned}
  \ket{\frac 3 2,\frac 3 2}&=1\cdot\ket{1,\frac 1 2;1,\frac 1 2}\\
  \ket{\frac 3 2, \frac 1 2}&= 
     \frac 1 {\sqrt 3}
     \ket{1, \frac 1 2; 1, -\frac 1 2}
     +\sqrt {\frac 2 3} 
     \ket{1, \frac 1 2; 0, \frac 1 2}\\
  \ket{\frac 3 2, -\frac 1 2}&=
     \sqrt{\frac 2 3}
     \ket{1, \frac 1 2; 0, -\frac 1 2}
     +\frac 1 {\sqrt 3}
     \ket{1, \frac 1 2; -1, \frac 1 2}\\
  \ket{\frac 3 2, -\frac 3 2}&= 1\cdot\ket{1,\frac 1 2;-1,-\frac 1 2}\\
  \ket{\frac 1 2, \frac 1 2}&=\sqrt{\frac 2 3}\ket{1,\frac 1 2;1,-\frac 1 2}
   -\frac 1 {\sqrt 3}\ket{1,\frac 1 2;0,\frac 1 2}\\
  \ket{\frac 1 2,-\frac 1 2}&=\frac 1  {\sqrt 3} \ket{1,\frac 1 2;0,-\frac 1 2}
\end{aligned}}\]





\subsection*{Referenzen}
\begin{itemize}
%\item Claude Cohen-Tannoudji Quantenmechanik Band 2
\item Zettili Quanten Mehanics
%\item Rollnik Quantentheorie 2
\end{itemize}

\end{document}
