\documentclass[10pt,a4paper,oneside,fleqn]{article}
\usepackage{geometry}
\geometry{a4paper,left=20mm,right=20mm,top=1cm,bottom=2cm}
\usepackage[utf8]{inputenc}
%\usepackage{ngerman}
\usepackage{amsmath}                % brauche ich um dir Formel zu umrahmen.
\usepackage{amsfonts}                % brauche ich für die Mengensymbole
\usepackage{graphicx}
\setlength{\parindent}{0px}
\setlength{\mathindent}{10mm}
\usepackage{bbold}                    %brauche ich für die doppel Zahlen Darstellung (Einheitsmatrix z.B)



\usepackage{color}
\usepackage{titlesec} %sudo apt-get install texlive-latex-extra

\definecolor{darkblue}{rgb}{0.1,0.1,0.55}
\definecolor{verydarkblue}{rgb}{0.1,0.1,0.35}
\definecolor{darkred}{rgb}{0.55,0.2,0.2}

%hyperref Link color
\usepackage[colorlinks=true,
        linkcolor=darkblue,
        citecolor=darkblue,
        filecolor=darkblue,
        pagecolor=darkblue,
        urlcolor=darkblue,
        bookmarks=true,
        bookmarksopen=true,
        bookmarksopenlevel=3,
        plainpages=false,
        pdfpagelabels=true]{hyperref}

\titleformat{\chapter}[display]{\color{darkred}\normalfont\huge\bfseries}{\chaptertitlename\
\thechapter}{20pt}{\Huge}

\titleformat{\section}{\color{darkblue}\normalfont\Large\bfseries}{\thesection}{1em}{}
\titleformat{\subsection}{\color{verydarkblue}\normalfont\large\bfseries}{\thesubsection}{1em}{}

% Notiz Box
\usepackage{fancybox}
\newcommand{\notiz}[1]{\vspace{5mm}\ovalbox{\begin{minipage}{1\textwidth}#1\end{minipage}}\vspace{5mm}}

\usepackage{cancel}
\setcounter{secnumdepth}{3}
\setcounter{tocdepth}{3}





%-------------------------------------------------------------------------------
%Diff-Makro:
%Das Diff-Makro stellt einen Differentialoperator da.
%
%Benutzung:
% \diff  ->  d
% \diff f  ->  df
% \diff^2 f  ->  d^2 f
% \diff_x  ->  d/dx
% \diff^2_x  ->  d^2/dx^2
% \diff f_x  ->  df/dx
% \diff^2 f_x  ->  d^2f/dx^2
% \diff^2{f(x^5)}_x  ->  d^2(f(x^5))/dx^2
%
%Ersetzt man \diff durch \pdiff, so wird der partieller
%Differentialoperator dargestellt.
%
\makeatletter
\def\diff@n^#1{\@ifnextchar{_}{\diff@n@d^#1}{\diff@n@fun^#1}}
\def\diff@n@d^#1_#2{\frac{\textrm{d}^#1}{\textrm{d}#2^#1}}
\def\diff@n@fun^#1#2{\@ifnextchar{_}{\diff@n@fun@d^#1#2}{\textrm{d}^#1#2}}
\def\diff@n@fun@d^#1#2_#3{\frac{\textrm{d}^#1 #2}{\textrm{d}#3^#1}}
\def\diff@one@d_#1{\frac{\textrm{d}}{\textrm{d}#1}}
\def\diff@one@fun#1{\@ifnextchar{_}{\diff@one@fun@d #1}{\textrm{d}#1}}
\def\diff@one@fun@d#1_#2{\frac{\textrm{d}#1}{\textrm{d}#2}}
\newcommand*{\diff}{\@ifnextchar{^}{\diff@n}
  {\@ifnextchar{_}{\diff@one@d}{\diff@one@fun}}}
%
%Partieller Diff-Operator.
\def\pdiff@n^#1{\@ifnextchar{_}{\pdiff@n@d^#1}{\pdiff@n@fun^#1}}
\def\pdiff@n@d^#1_#2{\frac{\partial^#1}{\partial#2^#1}}
\def\pdiff@n@fun^#1#2{\@ifnextchar{_}{\pdiff@n@fun@d^#1#2}{\partial^#1#2}}
\def\pdiff@n@fun@d^#1#2_#3{\frac{\partial^#1 #2}{\partial#3^#1}}
\def\pdiff@one@d_#1{\frac{\partial}{\partial #1}}
\def\pdiff@one@fun#1{\@ifnextchar{_}{\pdiff@one@fun@d #1}{\partial#1}}
\def\pdiff@one@fun@d#1_#2{\frac{\partial#1}{\partial#2}}
\newcommand*{\pdiff}{\@ifnextchar{^}{\pdiff@n}
  {\@ifnextchar{_}{\pdiff@one@d}{\pdiff@one@fun}}}
\makeatother
%
%Das gleich nur mit etwas andere Syntax. Die Potenz der Differentiation wird erst
%zum Schluss angegeben. Somit lautet die Syntax:
%
% \diff_x^2  ->  d^2/dx^2
% \diff f_x^2  ->  d^2f/dx^2
% \diff{f(x^5)}_x^2  ->  d^2(f(x^5))/dx^2
% Ansonsten wie Oben.
%
%Ersetzt man \diff durch \pdiff, so wird der partieller
%Differentialoperator dargestellt.
%
%\makeatletter
%\def\diff@#1{\@ifnextchar{_}{\diff@fun#1}{\textrm{d} #1}}
%\def\diff@one_#1{\@ifnextchar{^}{\diff@n{#1}}%
%  {\frac{\textrm d}{\textrm{d} #1}}}
%\def\diff@fun#1_#2{\@ifnextchar{^}{\diff@fun@n#1_#2}%
%  {\frac{\textrm d #1}{\textrm{d} #2}}}
%\def\diff@n#1^#2{\frac{\textrm d^#2}{\textrm{d}#1^#2}}
%\def\diff@fun@n#1_#2^#3{\frac{\textrm d^#3 #1}%
%  {\textrm{d}#2^#3}}
%\def\diff{\@ifnextchar{_}{\diff@one}{\diff@}}
%\newcommand*{\diff}{\@ifnextchar{_}{\diff@one}{\diff@}}
%
%Partieller Diff-Operator.
%\def\pdiff@#1{\@ifnextchar{_}{\pdiff@fun#1}{\partial #1}}
%\def\pdiff@one_#1{\@ifnextchar{^}{\pdiff@n{#1}}%
%  {\frac{\partial}{\partial #1}}}
%\def\pdiff@fun#1_#2{\@ifnextchar{^}{\pdiff@fun@n#1_#2}%
%  {\frac{\partial #1}{\partial #2}}}
%\def\pdiff@n#1^#2{\frac{\partial^#2}{\partial #1^#2}}
%\def\pdiff@fun@n#1_#2^#3{\frac{\partial^#3 #1}%
%  {\partial #2^#3}}
%\newcommand*{\pdiff}{\@ifnextchar{_}{\pdiff@one}{\pdiff@}}
%\makeatother

%-------------------------------------------------------------------------------
%%Nützliche Makros um in der Quantenmechanik Bras, Kets und das Skalarprodukt
%%zwischen den beiden darzustellen.
%%Benutzung:
%% \bra{x}  ->    < x |
%% \ket{x}  ->    | x >
%% \braket{x}{y} ->   < x | y >

\newcommand\bra[1]{\left\langle #1 \right|}
\newcommand\ket[1]{\left| #1 \right\rangle}
\newcommand\braket[2]{%
  \left\langle #1\vphantom{#2} \right.%
  \left|\vphantom{#1#2}\right.%
  \left. \vphantom{#1}#2 \right\rangle}%

%-------------------------------------------------------------------------------
%%Aus dem Buch:
%%Titel:  Latex in Naturwissenschaften und Mathematik
%%Autor:  Herbert Voß
%%Verlag: Franzis Verlag, 2006
%%ISBN:   3772374190, 9783772374197
%%
%%Hier werden drei Makros definiert:\mathllap, \mathclap und \mathrlap, welche
%%analog zu den aus Latex bekannten \rlap und \llap arbeiten, d.h. selbst
%%keinerlei horizontalen Platz benötigen, aber dennoch zentriert zum aktuellen
%%Punkt erscheinen.

\newcommand*\mathllap{\mathstrut\mathpalette\mathllapinternal}
\newcommand*\mathllapinternal[2]{\llap{$\mathsurround=0pt#1{#2}$}}
\newcommand*\clap[1]{\hbox to 0pt{\hss#1\hss}}
\newcommand*\mathclap{\mathpalette\mathclapinternal}
\newcommand*\mathclapinternal[2]{\clap{$\mathsurround=0pt#1{#2}$}}
\newcommand*\mathrlap{\mathpalette\mathrlapinternal}
\newcommand*\mathrlapinternal[2]{\rlap{$\mathsurround=0pt#1{#2}$}}

%%Das Gleiche nur mit \def statt \newcommand.
%\def\mathllap{\mathpalette\mathllapinternal}
%\def\mathllapinternal#1#2{%
%  \llap{$\mathsurround=0pt#1{#2}$}% $
%}
%\def\clap#1{\hbox to 0pt{\hss#1\hss}}
%\def\mathclap{\mathpalette\mathclapinternal}
%\def\mathclapinternal#1#2{%
%  \clap{$\mathsurround=0pt#1{#2}$}%
%}
%\def\mathrlap{\mathpalette\mathrlapinternal}
%\def\mathrlapinternal#1#2{%
%  \rlap{$\mathsurround=0pt#1{#2}$}% $
%}

%-------------------------------------------------------------------------------
%%Hier werden zwei neue Makros definiert \overbr und \underbr welche analog zu
%%\overbrace und \underbrace funktionieren jedoch die Gleichung nicht
%%'zerreißen'. Dies wird ermöglicht durch das \mathclap Makro.

\def\overbr#1^#2{\overbrace{#1}^{\mathclap{#2}}}
\def\underbr#1_#2{\underbrace{#1}_{\mathclap{#2}}}
\usepackage{amsmath} 

\begin{document}

\textit{29. März 2012}
\input{../headers/authors.tex}

\section*{H-Atom; LS-Kopplung}

Betrachte ein Elektron um ein Proton kreisend. Das Elektron bewegt sich mit der Geschwindigkeit \(\vec v=\frac{\vec p}{m_e}\) in dem vom Proton erzeugten elektrostatischen Feld \(\vec E\). Aus der speziellen Relativitätstheorie folgt dann das Auftreten eines magnetischen Feldes \(\vec B\) im Ruhesystem des Elektrons, das zur ersten Ordnung in \(\frac{v}{c}\) gegeben ist durch (Herleitung siehe Eckhard Rebhan Theoretische Physik I Abschn:12.3.2 S.450):

\begin{align}
  \vec B &= -\frac{1}{c^2}\vec v\times \vec E \qquad \text{ mit } \vec p = m_e\vec v \notag \\
&= -\frac{1}{c^2 m_e}\vec p\times \vec E \notag \\
&= \frac{1}{c^2 m_e}\vec E\times \vec p  \label{eq:1}
\end{align}

Wobei das Minuszeichen dadurch bewirkt wird, dass im Ruhesystem des Elektrons das E-Feld eine Geschwindigkeit \(-\vec v = -\frac{\vec p}{m_e}\) relativ zum Elektron hat. Das Elektrische Feld lässt sich wie folgt darstellen:

\begin{align}
  \label{eq:2}
  \vec E &= -\vec\nabla\phi(\vec r) \qquad \text{mit } V(r) = -e\phi(r)  \notag \\
&= \frac{1}{e} \nabla V(r) \qquad \text{setze }\vec\nabla\text{ in Kugelkoordinaten ein}\notag \\
 &= \frac{1}{e}\frac{\vec r}{r} \frac{dV(r)}{dr}
\end{align}

Einsetzen der Gleichung \eqref{eq:2} in \eqref{eq:1}:

\begin{align}
  \label{eq:3}
\vec B  &= \frac{1}{c^2 m_e} \frac{1}{e}\frac{1}{r} \frac{dV(r)}{dr}\underbrace{ \vec r \times \vec p}_{\equiv \vec L} \notag\\
&= \frac{1}{c^2 m_e e}\frac{1}{r} \frac{dV(r)}{dr}\vec L
\end{align}


Mit dem magnetische Moment für das Elektron:

\begin{align}
  \label{eq:4}
  \vec \mu = \frac{e\vec S}{m_e}
\end{align}

Folgt für die Wechselwirkungsterm im Hamiltonoperator:

\begin{align}
  \label{eq:5}
  H_{LS} = -\vec\mu\cdot\vec B = \frac{e}{m_e}\vec S\cdot\vec B =  \frac{1}{c^2 m_e^2}\frac{1}{r} \frac{dV(r)}{dr}\vec S\cdot \vec L
\end{align}

Dieser Term wird erst mit der Thomas Präzession \(\frac{1}{2}\) richtig der sich aus der Dirac Gleichung genau herleiten lässt. D.h. der korrekte Term lautet:

\begin{align}
  \label{eq:6}
  \boxed{H_{LS} = \frac{1}{2}\cdot\frac{1}{c^2 m_e^2}\frac{1}{r} \frac{dV(r)}{dr}\vec S\cdot \vec L}
\end{align}

Das Potential \(V(r)\) vom Wasserstoffatom ist gegeben mit \(V(r) = - \frac{e^2}{r}\) damit ergibt sich der \(H_{LS}\)-Term:

\begin{align}
  \label{eq:7}
  H_{LS} =  \frac{e^2}{2\cdot c^2 m_e^2}\frac{1}{r^3} \vec S\cdot \vec L
\end{align}

Der Hamilton Operator für das Wassersoffatom lautet nun:

\begin{align}
  \label{eq:8}
  H &= \frac{p^2}{2m_e} + V(r) + H_{LS}\notag \\
&= \underbrace{\frac{p^2}{2m_e} -\frac{e^2}{r}}_{H_0} + \underbrace{\frac{e^2}{2\cdot c^2 m_e^2}\frac{1}{r^3} \vec S\cdot \vec L}_{\text{Störungsterm}}
\end{align}
Die Lösungen des \(H_0\)-Terms sind bereits aus dem allgemeinen Wasserstoffproblem bekannt:

\begin{align}
  \label{eq:9}
  \psi_{nlm}(r,\theta,\phi) = R_{nl}(r)Y_{lm}(\theta,\phi)
\end{align}

Den \(H_{LS}\)-Term können wir als ein Störungsterm behandeln. Wähle anstelle der \(\vec L^2,\vec L_z,\vec S^2,\vec S_z\)-Basis eine zu den Eigenzuständen \(|nljm\rangle \)  passende \(\vec J^2,\vec J_z,\vec L^2,\vec S^2\)-Basis. 

\begin{align}
  \label{eq:10}
  \vec J &= \vec L + \vec S \notag \\
\vec J^2 &= (\vec L + \vec S)^2 = \vec L^2 + 2\vec L\vec S + \vec S^2\notag \\
\Leftrightarrow \vec L\vec S &= \frac{1}{2}(\vec J^2-\vec L^2-  \vec S^2)
\end{align}

Daraus folgt der \(H_{LS}\)-Term

\begin{align}
  \label{eq:11}
  H_{LS} =  \frac{e^2}{4  c^2 m_e^2}\frac{1}{r^3}(\vec J^2-\vec L^2-  \vec S^2)
\end{align}

Für die Energie-Eigenwerte in Störungsrechung bis 1-Ordnung ergibt sich:

\begin{align}
  \label{eq:12}
  E_{nlj} = E^{(0)}_n + \langle nljm|H_{LS}|nljm\rangle = -\frac{e^2}{2a_0}\frac{1}{n^2} + E_{LS}^{(1)}
\end{align}

\subsection*{Referenzen}


\begin{itemize}
\item Claude Cohen-Tannoudji Quantenmechanik Band 2
\item Zettili Quanten Mehanics
\item Rollnik Quantentheorie 2
\end{itemize}

\end{document}
