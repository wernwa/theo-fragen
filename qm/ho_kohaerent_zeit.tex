\input{../headers/header_script.tex}
%\includegraphics[width=0.75\textwidth]{thepic.png}

\begin{document}
\section*{Zeitliche Entwicklung eines kohärenten Zustandes}

Wir wollen nun die kohärenten Zustände eines harmonischen Oszillators \(|\alpha\rangle \) 

\begin{equation}
  \label{eq:1}
  |\alpha\rangle =  e^{-\frac{|\alpha|^2}{2}} \sum_{n=0}^{\infty} \frac{\alpha^n}{\sqrt{n!}} |n\rangle
\end{equation}

zeitlich entwickeln und zeigen, dass diese kohärent bleiben. 

Aus dem Separationsansatz der zeitabhängigen Schrödinger Gleichung wissen wir:

\begin{align}
  \label{eq:2}
  |\alpha (t)\rangle  &= \exp\left( -\frac{i}{\hbar}Ht\right)|\alpha\rangle \\
 &= \exp\left( -\frac{i}{\hbar}Ht\right)e^{-\frac{|\alpha|^2}{2}} \sum_{n=0}^{\infty} \frac{\alpha^n}{\sqrt{n!}} |n\rangle \\
 &=e^{-\frac{|\alpha|^2}{2}}  \sum_{n=0}^{\infty} \frac{\alpha^n}{\sqrt{n!}} \exp\left( -\frac{i}{\hbar}Ht\right) |n\rangle \\
 &=  e^{-\frac{|\alpha|^2}{2}}  \sum_{n=0}^{\infty} \frac{\alpha^n}{\sqrt{n!}} \exp\left( -\frac{i}{\hbar}E_n t\right)  |n\rangle 
\end{align}

Mit der allgemeinen Lösung für die Energieeigenwerte des harmonischen Oszillators \(E_n= \hbar\omega(n+\frac{1}{2})\) ergibt sich:


\begin{align}
  |\alpha (t)\rangle  &= e^{-\frac{|\alpha|^2}{2}}  \sum_{n=0}^{\infty} \frac{\alpha^n}{\sqrt{n!}} \exp\left( -\frac{i}{\hbar}\hbar\omega(n+\frac{1}{2})  t\right)  |n\rangle \\
&= e^{-\frac{|\alpha|^2}{2}}e^{-i\omega\frac{1}{2}t}  \sum_{n=0}^{\infty} \frac{\alpha^n}{\sqrt{n!}} e^{ -i\omega n t}  |n\rangle \\
&= e^{-\frac{|\alpha|^2}{2}}e^{-i\omega\frac{1}{2}t}  \sum_{n=0}^{\infty} \frac{\left( \alpha e^{ -i\omega  t}\right)^n }{\sqrt{n!}}   |n\rangle  \label{eq:3}
\end{align}


Da \(|e^{-i\omega t}|=1\) können wir statt \(|\alpha|^2\) auch \(|\alpha e^{-i\omega t}|^2\) schreiben. Damit sieht die Gleichung \eqref{eq:3} wie folgt aus:
\begin{align}
  \label{eq:4}
   |\alpha (t)\rangle &= e^{-\frac{|\alpha e^{-i\omega t} |^2}{2}}e^{-i\omega\frac{1}{2}t}  \sum_{n=0}^{\infty} \frac{\left( \alpha e^{ -i\omega  t}\right)^n }{\sqrt{n!}}   |n\rangle\\
&=e^{-\frac{1}{2}i\omega t} \underbrace{\left(e^{-\frac{|\alpha e^{-i\omega t} |^2}{2}}  \sum_{n=0}^{\infty} \frac{\left( \alpha e^{ -i\omega  t}\right)^n }{\sqrt{n!}}   |n\rangle\right)}_{\text{vergleiche mit \eqref{eq:1} }}\\
&=e^{-i\frac{1}{2}\omega t}|\alpha e^{-i\omega t}\rangle
\end{align}


Um zu zeigen dass der Zustand für alle Zeiten kohärent bleibt, wendet man den Absteigeoperator auf \eqref{eq:4} an und erhält den Eigenwert \(\alpha e^{-i\omega t}\):

\begin{align}
  \label{eq:5}
  a |\alpha(t)\rangle = \alpha e^{-i\omega t} |\alpha(t)\rangle \\
\end{align}

\subsection*{Zeitliche Oszillation}


\begin{align}
  \label{eq:6}
  \langle x \rangle &=  \sqrt{ \frac{\hbar}{2 m\omega}}\langle \alpha(t) |a^\dagger + a|\alpha(t)\rangle \\
&=  \sqrt{ \frac{\hbar}{2 m\omega}}(\langle \alpha(t) |a^\dagger|\alpha(t)\rangle + \langle \alpha(t)a|\alpha(t)\rangle) \qquad \text{mit }\alpha\in \mathbb R\\
&=  \sqrt{ \frac{\hbar}{2 m\omega}}(\alpha e^{i\omega t} + \alpha e^{-i\omega t} )\\
&=  \sqrt{ \frac{\hbar}{2 m\omega}}2\alpha \cos(\omega t)\\
&=  \sqrt{ \frac{2\hbar}{ m\omega}}\alpha \cos(\omega t)
\end{align}


Der Ortserwartungswert oszilliert mit der Frequenz \(\omega\) zwischen \(- \sqrt{ \frac{2\hbar}{ m\omega}}\alpha\) und \(\sqrt{ \frac{2\hbar}{ m\omega}}\alpha\) hin und her.

Analoge Rechnung für den Impulserwartungswert liefert:

\begin{equation}
  \label{eq:7}
\langle p \rangle = -\sqrt{ 2\hbar m\omega}\cdot  \alpha \sin (\omega t)
\end{equation}

Die Erwartungswerte für Ort und Impuls verhalten sich demnach genauso wie eine klassische harmonische Osillation. Man kann ebenso den Zustammenhang zwischen Orts- und Impuls-Operator wie in der klassischen Mechanik herstellen:


\begin{equation}
  \label{eq:8}
  m\frac{d\langle x\rangle }{dt} = \sqrt{\frac{2\hbar}{m\omega}} m\alpha \omega(-\sin(\omega t)) = - \sqrt{2\hbar m\omega}\cdot\alpha \sin(\omega t) \equiv \langle p \rangle 
\end{equation}


\subsection*{Referenzen}

\begin{itemize}
\item \url{www.physik.uni-regensburg.de/forschung/schwarz/QOptik/Wurm.pdf}
\item \url{http://www2009.ph.tum.de/studium/betrieb/ferienkurse/2009s/qm/diml.pdf}
\end{itemize}


\end{document}
