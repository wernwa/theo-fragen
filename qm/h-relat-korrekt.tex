\documentclass[10pt,a4paper,oneside,fleqn]{article}
\usepackage{geometry}
\geometry{a4paper,left=20mm,right=20mm,top=1cm,bottom=2cm}
\usepackage[utf8]{inputenc}
%\usepackage{ngerman}
\usepackage{amsmath}                % brauche ich um dir Formel zu umrahmen.
\usepackage{amsfonts}                % brauche ich für die Mengensymbole
\usepackage{graphicx}
\setlength{\parindent}{0px}
\setlength{\mathindent}{10mm}
\usepackage{bbold}                    %brauche ich für die doppel Zahlen Darstellung (Einheitsmatrix z.B)



\usepackage{color}
\usepackage{titlesec} %sudo apt-get install texlive-latex-extra

\definecolor{darkblue}{rgb}{0.1,0.1,0.55}
\definecolor{verydarkblue}{rgb}{0.1,0.1,0.35}
\definecolor{darkred}{rgb}{0.55,0.2,0.2}

%hyperref Link color
\usepackage[colorlinks=true,
        linkcolor=darkblue,
        citecolor=darkblue,
        filecolor=darkblue,
        pagecolor=darkblue,
        urlcolor=darkblue,
        bookmarks=true,
        bookmarksopen=true,
        bookmarksopenlevel=3,
        plainpages=false,
        pdfpagelabels=true]{hyperref}

\titleformat{\chapter}[display]{\color{darkred}\normalfont\huge\bfseries}{\chaptertitlename\
\thechapter}{20pt}{\Huge}

\titleformat{\section}{\color{darkblue}\normalfont\Large\bfseries}{\thesection}{1em}{}
\titleformat{\subsection}{\color{verydarkblue}\normalfont\large\bfseries}{\thesubsection}{1em}{}

% Notiz Box
\usepackage{fancybox}
\newcommand{\notiz}[1]{\vspace{5mm}\ovalbox{\begin{minipage}{1\textwidth}#1\end{minipage}}\vspace{5mm}}

\usepackage{cancel}
\setcounter{secnumdepth}{3}
\setcounter{tocdepth}{3}





%-------------------------------------------------------------------------------
%Diff-Makro:
%Das Diff-Makro stellt einen Differentialoperator da.
%
%Benutzung:
% \diff  ->  d
% \diff f  ->  df
% \diff^2 f  ->  d^2 f
% \diff_x  ->  d/dx
% \diff^2_x  ->  d^2/dx^2
% \diff f_x  ->  df/dx
% \diff^2 f_x  ->  d^2f/dx^2
% \diff^2{f(x^5)}_x  ->  d^2(f(x^5))/dx^2
%
%Ersetzt man \diff durch \pdiff, so wird der partieller
%Differentialoperator dargestellt.
%
\makeatletter
\def\diff@n^#1{\@ifnextchar{_}{\diff@n@d^#1}{\diff@n@fun^#1}}
\def\diff@n@d^#1_#2{\frac{\textrm{d}^#1}{\textrm{d}#2^#1}}
\def\diff@n@fun^#1#2{\@ifnextchar{_}{\diff@n@fun@d^#1#2}{\textrm{d}^#1#2}}
\def\diff@n@fun@d^#1#2_#3{\frac{\textrm{d}^#1 #2}{\textrm{d}#3^#1}}
\def\diff@one@d_#1{\frac{\textrm{d}}{\textrm{d}#1}}
\def\diff@one@fun#1{\@ifnextchar{_}{\diff@one@fun@d #1}{\textrm{d}#1}}
\def\diff@one@fun@d#1_#2{\frac{\textrm{d}#1}{\textrm{d}#2}}
\newcommand*{\diff}{\@ifnextchar{^}{\diff@n}
  {\@ifnextchar{_}{\diff@one@d}{\diff@one@fun}}}
%
%Partieller Diff-Operator.
\def\pdiff@n^#1{\@ifnextchar{_}{\pdiff@n@d^#1}{\pdiff@n@fun^#1}}
\def\pdiff@n@d^#1_#2{\frac{\partial^#1}{\partial#2^#1}}
\def\pdiff@n@fun^#1#2{\@ifnextchar{_}{\pdiff@n@fun@d^#1#2}{\partial^#1#2}}
\def\pdiff@n@fun@d^#1#2_#3{\frac{\partial^#1 #2}{\partial#3^#1}}
\def\pdiff@one@d_#1{\frac{\partial}{\partial #1}}
\def\pdiff@one@fun#1{\@ifnextchar{_}{\pdiff@one@fun@d #1}{\partial#1}}
\def\pdiff@one@fun@d#1_#2{\frac{\partial#1}{\partial#2}}
\newcommand*{\pdiff}{\@ifnextchar{^}{\pdiff@n}
  {\@ifnextchar{_}{\pdiff@one@d}{\pdiff@one@fun}}}
\makeatother
%
%Das gleich nur mit etwas andere Syntax. Die Potenz der Differentiation wird erst
%zum Schluss angegeben. Somit lautet die Syntax:
%
% \diff_x^2  ->  d^2/dx^2
% \diff f_x^2  ->  d^2f/dx^2
% \diff{f(x^5)}_x^2  ->  d^2(f(x^5))/dx^2
% Ansonsten wie Oben.
%
%Ersetzt man \diff durch \pdiff, so wird der partieller
%Differentialoperator dargestellt.
%
%\makeatletter
%\def\diff@#1{\@ifnextchar{_}{\diff@fun#1}{\textrm{d} #1}}
%\def\diff@one_#1{\@ifnextchar{^}{\diff@n{#1}}%
%  {\frac{\textrm d}{\textrm{d} #1}}}
%\def\diff@fun#1_#2{\@ifnextchar{^}{\diff@fun@n#1_#2}%
%  {\frac{\textrm d #1}{\textrm{d} #2}}}
%\def\diff@n#1^#2{\frac{\textrm d^#2}{\textrm{d}#1^#2}}
%\def\diff@fun@n#1_#2^#3{\frac{\textrm d^#3 #1}%
%  {\textrm{d}#2^#3}}
%\def\diff{\@ifnextchar{_}{\diff@one}{\diff@}}
%\newcommand*{\diff}{\@ifnextchar{_}{\diff@one}{\diff@}}
%
%Partieller Diff-Operator.
%\def\pdiff@#1{\@ifnextchar{_}{\pdiff@fun#1}{\partial #1}}
%\def\pdiff@one_#1{\@ifnextchar{^}{\pdiff@n{#1}}%
%  {\frac{\partial}{\partial #1}}}
%\def\pdiff@fun#1_#2{\@ifnextchar{^}{\pdiff@fun@n#1_#2}%
%  {\frac{\partial #1}{\partial #2}}}
%\def\pdiff@n#1^#2{\frac{\partial^#2}{\partial #1^#2}}
%\def\pdiff@fun@n#1_#2^#3{\frac{\partial^#3 #1}%
%  {\partial #2^#3}}
%\newcommand*{\pdiff}{\@ifnextchar{_}{\pdiff@one}{\pdiff@}}
%\makeatother

%-------------------------------------------------------------------------------
%%Nützliche Makros um in der Quantenmechanik Bras, Kets und das Skalarprodukt
%%zwischen den beiden darzustellen.
%%Benutzung:
%% \bra{x}  ->    < x |
%% \ket{x}  ->    | x >
%% \braket{x}{y} ->   < x | y >

\newcommand\bra[1]{\left\langle #1 \right|}
\newcommand\ket[1]{\left| #1 \right\rangle}
\newcommand\braket[2]{%
  \left\langle #1\vphantom{#2} \right.%
  \left|\vphantom{#1#2}\right.%
  \left. \vphantom{#1}#2 \right\rangle}%

%-------------------------------------------------------------------------------
%%Aus dem Buch:
%%Titel:  Latex in Naturwissenschaften und Mathematik
%%Autor:  Herbert Voß
%%Verlag: Franzis Verlag, 2006
%%ISBN:   3772374190, 9783772374197
%%
%%Hier werden drei Makros definiert:\mathllap, \mathclap und \mathrlap, welche
%%analog zu den aus Latex bekannten \rlap und \llap arbeiten, d.h. selbst
%%keinerlei horizontalen Platz benötigen, aber dennoch zentriert zum aktuellen
%%Punkt erscheinen.

\newcommand*\mathllap{\mathstrut\mathpalette\mathllapinternal}
\newcommand*\mathllapinternal[2]{\llap{$\mathsurround=0pt#1{#2}$}}
\newcommand*\clap[1]{\hbox to 0pt{\hss#1\hss}}
\newcommand*\mathclap{\mathpalette\mathclapinternal}
\newcommand*\mathclapinternal[2]{\clap{$\mathsurround=0pt#1{#2}$}}
\newcommand*\mathrlap{\mathpalette\mathrlapinternal}
\newcommand*\mathrlapinternal[2]{\rlap{$\mathsurround=0pt#1{#2}$}}

%%Das Gleiche nur mit \def statt \newcommand.
%\def\mathllap{\mathpalette\mathllapinternal}
%\def\mathllapinternal#1#2{%
%  \llap{$\mathsurround=0pt#1{#2}$}% $
%}
%\def\clap#1{\hbox to 0pt{\hss#1\hss}}
%\def\mathclap{\mathpalette\mathclapinternal}
%\def\mathclapinternal#1#2{%
%  \clap{$\mathsurround=0pt#1{#2}$}%
%}
%\def\mathrlap{\mathpalette\mathrlapinternal}
%\def\mathrlapinternal#1#2{%
%  \rlap{$\mathsurround=0pt#1{#2}$}% $
%}

%-------------------------------------------------------------------------------
%%Hier werden zwei neue Makros definiert \overbr und \underbr welche analog zu
%%\overbrace und \underbrace funktionieren jedoch die Gleichung nicht
%%'zerreißen'. Dies wird ermöglicht durch das \mathclap Makro.

\def\overbr#1^#2{\overbrace{#1}^{\mathclap{#2}}}
\def\underbr#1_#2{\underbrace{#1}_{\mathclap{#2}}}
\usepackage{amsmath}


\begin{document}

\textit{29. März 2012}
\input{../headers/authors.tex}

\section*{Relativistische Korrekturen zum Energiespektrum des Wasserstoffatoms}


Die relativistischen Korrekturen von der Bewegung des Elektrons um das Proton ist ist ein geringer Effekt. Man kann es aber trotzdem mit Hilfe der Spektroskopie sichtbar machen. Die Spektrallinien erfahren eine Aufspaltung. Wir betrachten die relativistische kinetische Energie, die sich zusammensetzt aus der Gesamt-Energie minus der Masse-Ruhe-Energie.

\begin{equation}
  \label{eq:1}
  T = \sqrt{p^2c^2+m_e^2c^4}-m_ec^2
\end{equation}


Machen wir eine Taylorentwicklung der Wurzel bis zur 4-er Ordnung, so können wir schreiben:

\begin{equation}
  \label{eq:2}
  T \approx \frac{p^2}{2m}-\frac{p^4}{8m_e^3c^2}+\cdots
\end{equation}


Setzen wir nun die Gleichung \eqref{eq:2} in den Hamilton Operator des Wasserstoffatoms ein, so erhalten wir:


\begin{equation}
  \label{eq:3}
  H = \underbrace{\frac{p^2}{2m}-\frac{e^2}{r}}_{H_0}-\underbrace{\frac{p^4}{8m_e^3c^2}}_{H_R}
\end{equation}


Dabei ist der \(H_0\) der ungestörte Hamilton Operator und \(H_R\) ist die relativistische Korrektur, die wir mit Hilfe der Störungsrechung bis 1-Ordnung behandeln.

\begin{equation}
  \label{eq:4}
  E^{(1)}_R = \langle nljm|H_R|nljm\rangle  = - \frac{1}{8m_e^3c^2}\langle nljm|p^4|nljm\rangle
\end{equation}

Im Folgenden leiten wir den Erwartungswert für \(\langle p^4 \rangle \) her. Da die Eigenfunktion von \(p\) die Radialfunktion ist, die die Quantenzahl \(n,l\), hat reduziert sich der Eigenvektor  \(|nljm\rangle \)  zu \(|nl\rangle \). Somit lautet der zu berechnende Erwartungswert:

\begin{equation}
  \label{eq:5}
  \langle nl|p^4|nl\rangle
\end{equation}


Wir versuchen nun den \(p\)Operator durch den Hamilton-Operator auszudrücken, da wir die Eigenwerte bereits kennen. Es hilft folgende Umformung:


\begin{align}
  \label{eq:6}
  H &= \frac{p^2}{2m_e}-\frac{e^2}{r} \\
\Leftrightarrow p^2 &= 2m_e\left( H + \frac{e^2}{r} \right) \qquad |^2\\
\rightarrow p^4 &= 4m_e^2\left( H + \frac{e^2}{r} \right)^2\\
&= 4m_e^2\left( H^2 + H \frac{e^2}{r}+ \frac{e^2}{r}H +\frac{e^4}{r^2}  \right)\\
\end{align}
Setzen wir die Gleichung (\ref{eq:6}) in (\ref{eq:5})

\begin{align}
  \label{eq:7}
  \langle nl|p^4|nl\rangle &= 4m_e^2 \langle nl|\left( H^2 + H \frac{e^2}{r}+ \frac{e^2}{r}H +\frac{e^4}{r^2}\right) |nl\rangle \\
 &= 4m_e^2 \left(  \langle nl| H^2|nl\rangle + \langle nl| H \frac{e^2}{r}|nl\rangle+ \langle nl| \frac{e^2}{r}H|nl\rangle + \langle nl|\frac{e^4}{r^2}|nl\rangle\right)  \\
 &= 4m_e^2 \left( E_n^2 + 2E_ne^2\langle nl| \frac{1}{r}|nl\rangle + e^4\langle nl|\frac{1}{r^2}|nl\rangle\right)  \\
\end{align}
Jetzt wollen wir die Erwartungswerte von \(\frac{1}{r}|nl\rangle\) und \(\frac{1}{r^2}|nl\rangle\) bestimmen. Dazu benötigen wir die Radialgleichung des Wasserstoffatoms:

\begin{equation}
  \label{eq:8}
  -\frac{\hbar^2}{2m_e}\frac{d^2u_{nl}}{dr^2} + \left[\frac{l(l+1)\hbar^2}{2m_e r^2}-\frac{e^2}{r}\right]u_{nl} = E_n u_{nl}
\end{equation}


Wobei wir \(\mu=m_e\) angenähert haben. Die Gleichung (\ref{eq:8}) können wir in folgende Form bringen, mit \(u_{nl}'' \equiv\frac{d^2u_{nl}}{dr^2} \):

\begin{equation}
  \label{eq:9}
  \frac{u_{nl}''}{u_{nl}} = \frac{l(l+1)}{r^2} - \frac{2m_e e^2}{\hbar^2}\frac{1}{r}+\frac{m_e^2e^4}{\hbar^4 n^2}
\end{equation}


Da wir den Erwartungswert von \(\frac{1}{r^2}\) berechnen wollen ist es günstig andere Terme in der Gleichung loszuwerden. Das funktioniert wenn man die Gleichung (\ref{eq:9}) nach \(l\) ableitet. Dazu sollte man berücksitchtigen dass \(n=N+l+1\) und die Ableitung \(\frac{\partial}{\partial l}\frac{1}{n^2} = -\frac{2}{n^3} \)

\begin{equation}
  \label{eq:10}
  \frac{\partial}{\partial l}\frac{u_{nl}''}{u_{nl}} = \frac{2l+1}{r^2} - \frac{2 m_e^2e^4}{\hbar^4 n^3}
\end{equation}


Da wir den Erwartungswert bestimmen möchten ist es günstig die Gleichung (\ref{eq:10}) mit \(u_{nl}^2\) multipliziert und im gesamten Bereich integriert. D.h. wir bestimmen den Erwartungswert der Gleichung:

\begin{align}
  \label{eq:11}
  \int_0^{\infty} u_{nl}^2 \frac{\partial}{\partial l}\frac{u_{nl}''}{u_{nl}} dr &= \int_0^{\infty} u_{nl}^2 \frac{2l+1}{r^2} dr - \int_0^{\infty} u_{nl}^2\frac{2 m_e^2e^4}{\hbar^4 n^3}dr\\
&=(2l+1) \underbrace{ \int_0^{\infty} u_{nl}^2 \frac{1}{r^2}dr}_{\langle nl|\frac{1}{r^2}|nl\rangle  } -\frac{2 m_e^2e^4}{\hbar^4 n^3}\underbrace{ \int_0^{\infty} u_{nl}^2 dr}_{\equiv\langle nl|nl\rangle = 1 } \\
&=(2l+1)\langle nl|\frac{1}{r^2}|nl\rangle   -\frac{2 m_e^2e^4}{\hbar^4 n^3}
\end{align}

Für die linke Seite der Gleichung (\ref{eq:11}) gilt:

\begin{align}
  \label{eq:12}
   \int_0^{\infty}dr u_{nl}^2 \frac{\partial}{\partial l}\frac{u_{nl}''}{u_{nl}} &= \int_0^{\infty}dr u_{nl}^2\left(u_{nl}'' \frac{\partial}{\partial l}\frac{1}{u_{nl}}+ \frac{1}{u_{nl}} \frac{\partial}{\partial l}u_{nl}''  \right) \\
&= \int_0^{\infty}dr\cdot u_{nl}^2\left(- u_{nl}'' \frac{1}{u_{nl}^2}\frac{\partial}{\partial l} u_{nl}  + \frac{1}{u_{nl}} \frac{\partial}{\partial l}u_{nl}''  \right)\\
&= \int_0^{\infty}dr \left(- u_{nl}'' \frac{\partial}{\partial l} u_{nl}  + u_{nl} \frac{\partial}{\partial l}u_{nl}''  \right)\\
&= - \int_0^{\infty}dr\, u_{nl}'' \frac{\partial}{\partial l} u_{nl}  +\int_0^{\infty}dr\,  u_{nl} \frac{\partial}{\partial l}u_{nl}''
\end{align}

Es lässt sich zeigen, dass das Integral (\ref{eq:12}) gleich Null wird durch partielle Integration des Ersten Summanden. Wir erhalten:


\begin{align}
  \label{eq:13}
   - \int_0^{\infty}dr\, u_{nl}'' \frac{\partial}{\partial l} u_{nl} &= -\left[  u_{nl}'\frac{\partial}{\partial l} u_{nl}\right]_0^\infty + \int_0^{\infty}dr\, u_{nl}' \frac{\partial}{\partial l} u_{nl}'\\
 &= \underbrace{-\left[  u_{nl}'\frac{\partial}{\partial l} u_{nl}\right]_0^\infty + \left[  u_{nl}\frac{\partial}{\partial l} u_{nl}'\right]_0^\infty}_{=0 \text{ da } u(0)=u(\infty)\equiv 0} - \int_0^{\infty}dr\, u_{nl} \frac{\partial}{\partial l} u_{nl}''
\end{align}

Setzen wir die Gleichung (\ref{eq:13}) in die Gleichung (\ref{eq:12}) ein so erhalten wir:

\begin{equation}
  \label{eq:14}
   \int_0^{\infty}dr u_{nl}^2 \frac{\partial}{\partial l}\frac{u_{nl}''}{u_{nl}}=  - \int_0^{\infty}dr\, u_{nl} \frac{\partial}{\partial l} u_{nl}''  +\int_0^{\infty}dr\,  u_{nl} \frac{\partial}{\partial l}u_{nl}'' = 0
\end{equation}



Damit können wir die Gleichung (\ref{eq:11}) schreiben:

\begin{align}
  \label{eq:15}
  0 &=(2l+1)\langle nl|\frac{1}{r^2}|nl\rangle   -\frac{2 m_e^2e^4}{\hbar^4 n^3}
\end{align}

\begin{equation}
  \label{eq:16}
  \boxed{\langle nl|\frac{1}{r^2}|nl\rangle = \frac{2 m_e^2e^4}{(2l+1)\hbar^4 n^3}=\frac{2}{n^3(2l+1)a_0^2} }
\end{equation}

Mit \(a_0=\frac{\hbar^2}{m_e e^2}\).

Analog verfahren wir für den Erwartungswert \(\langle nl| \frac{1}{r}|nl \rangle \). Da uns hier \(\frac{1}{r}\) Erwartungswert interessiert leiten wir die Gleichung (\ref{eq:9}) nach Ladung \(e\) ab und erhalten:

\begin{equation}
  \label{eq:17}
   \frac{\partial}{\partial e}\frac{u_{nl}''}{u_{nl}} = - \frac{4m_e e}{\hbar^2}\frac{1}{r}+\frac{4m_e^2e^3}{\hbar^4 n^2}
\end{equation}

Bilden wir wieder den Erwartungswert dieser Gleichung (\ref{eq:17}):

\begin{align}
  \label{eq:18}
\int_0^\infty dr\, u_{nl}^2\frac{\partial}{\partial e}\frac{u_{nl}''}{u_{nl}} &= -\int_0^\infty dr\, u_{nl}^2 \frac{4m_e e}{\hbar^2}\frac{1}{r}+ \int_0^\infty dr\, u_{nl}^2\frac{4m_e^2e^3}{\hbar^4 n^2} \\
 &= -\frac{4m_e e}{\hbar^2} \underbrace{\int_0^\infty dr\, u_{nl}^2 \frac{1}{r}}_{\langle nl|\frac{1}{r}|nl\rangle } + \frac{4m_e^2e^3}{\hbar^4 n^2}\underbrace{ \int_0^\infty dr\, u_{nl}^2}_{\langle nl|nl\rangle =1} \\
 &= -\frac{4m_e e}{\hbar^2} \langle nl|\frac{1}{r}|nl\rangle  + \frac{4m_e^2e^3}{\hbar^4 n^2}
\end{align}


Die Linke Seite wird wieder Null (siehe (\ref{eq:12}) bis (\ref{eq:14}) und wir können schreiben:

\begin{equation}
  \label{eq:19}
  \boxed{\langle nl|\frac{1}{r}|nl\rangle = \frac{m_e e^2}{ \hbar^2 n^2} = \frac{1}{a_0 n^2}}
\end{equation}

Mit diesen Ergebnissen können wir nun den Erwartungswert \(\langle nl | p^4 |nl \rangle \) berechnen. Setzen wir die Gleichungen (\ref{eq:16}) und (\ref{eq:19}) in (\ref{eq:7}) ein:

\begin{align}
  \label{eq:20}
   \langle nl|p^4|nl\rangle &= 4m_e^2 \left( E_n^2 + 2E_ne^2\langle nl| \frac{1}{r}|nl\rangle + e^4\langle nl|\frac{1}{r^2}|nl\rangle\right)  \\
&= 4m_e^2 \left( E_n^2 + \frac{1}{a_0 n^2}2E_ne^2 + e^4 \frac{2}{n^3(2l+1)a_0^2}  \right) \\
&= (2m_eE_n)^2 \left( 1 + \frac{2e^2}{a_0 n^2E_n} +  \frac{2e^4}{n^3(2l+1)a_0^2E_n^2}  \right) \qquad \text{ mit }E_n=-\frac{e^2}{2a_0n^2} \\
&= (2m_eE_n)^2 \left( 1 - 4 +  \frac{8n }{2l+1}  \right)
\end{align}


Unser Ziel relativistische Korrekturen für den Hamilton-Operator des Wasserstoffatoms zu berechnen ist damit erreicht. Zusammenfassend nimmt die Energie folgende Form an:

\begin{align}
  \label{eq:21}
  E_n^{\text{rel-ges}} &= E_n + E^{(1)}_R\\
&= E_n - \frac{1}{8m_e^3c^2}(2m_eE_n)^2 \left( 1 - 4 +  \frac{8n }{2l+1}  \right)\qquad \text{ mit }E_n=-\frac{e^2}{2a_0n^2}  \\
&= -\frac{e^2}{2a_0n^2} - \frac{e^4}{16m_e^2c^2a_0^2n^4 } \left( 1 - 4 +  \frac{8n }{2l+1}  \right) \\
\end{align}




\subsection*{Referenzen}
\begin{itemize}
\item Zettili Quanten Mehanics
%\item Rollnik Quantentheorie 2
\end{itemize}


\end{document}
