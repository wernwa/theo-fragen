\input{../headers/header_script.tex}
\usepackage{amsmath} 



\begin{document}

\section*{Dirac-Gleichung klassische Näherung}

Die nicht relativistische (also klassische) Näherung der Dirac-Gleichung ergibt die uns schon bekannte \textbf{Pauli-Gleichung}. Wir starten mit der Dirac-Gleichung in kanonischer Form

\begin{align}
  \label{eq:1}
  i\hbar\pdiff_t \psi(x) = c\left(\vec \alpha \vec p +\beta mc  \right)\psi(x)
\end{align}

Nun betrachten ein Teilchen in einem elektromagnetischen Feld. Dazu führen wir den veralgemeinerten Impuls ein

\begin{align}
  \label{eq:2}
  \vec p \rightarrow \vec p - \frac{e}{c}\vec A
\end{align}
Und das Skalarpotential \(\Phi=cA^{0}\). Somit erhalten wir die Dirac-Gleichung in einem elektromagnetischen Potential

\begin{align}
  \label{eq:3}
  i\hbar\pdiff_t \psi(x) = c\left(\vec \alpha (\vec p - \frac{e}{c}\vec A) + \frac{e}{c}\Phi   +\beta mc  \right)\psi(x)
\end{align}

Um diese Gleichung zu lösen machen wir folgenden Ansatz

\begin{align}
  \label{eq:4}
  \psi(x) = e^{-\frac{i}{\hbar}mc^2 t} \begin{pmatrix}  \phi\\\chi \end{pmatrix}
\end{align}
Eingesetzt in die Gleichung (\ref{eq:3}) mit dem verallgemeinen Impuls \(\vec\pi = \vec p - \frac{e}{c}\vec A \) ergibt

\begin{align}
  \label{eq:5}
   i\hbar\pdiff_t\left[ e^{-\frac{i}{\hbar}mc^2 t} \begin{pmatrix}  \phi\\\chi \end{pmatrix}\right] &= c\left(\vec \alpha \vec\pi + \frac{e}{c}\Phi   +\beta mc  \right)e^{-\frac{i}{\hbar}mc^2 t} \begin{pmatrix}  \phi\\\chi \end{pmatrix} \notag\\
%
 mc^2  \cancel{e^{-\frac{i}{\hbar}mc^2 t}} \begin{pmatrix}  \phi\\\chi \end{pmatrix} + \cancel{e^{-\frac{i}{\hbar}mc^2 t}} i\hbar\begin{pmatrix}  \dot \phi\\ \dot \chi \end{pmatrix} &= c\left(
   \begin{pmatrix}
     0&\vec \sigma\vec\pi\\
     \vec \sigma\vec\pi&0
   \end{pmatrix}
 + \frac{e}{c}\Phi    +
 \begin{pmatrix}
   \mathds 1_2&0\\
   0&-\mathds 1_2
 \end{pmatrix}
mc  \right)\cancel{e^{-\frac{i}{\hbar}mc^2 t}} \begin{pmatrix}  \phi\\\chi \end{pmatrix} \notag \\
%
 mc^2   \begin{pmatrix}  \phi\\\chi \end{pmatrix} +  i\hbar\begin{pmatrix}  \dot \phi\\ \dot \chi \end{pmatrix} &= c\left(
   \begin{pmatrix}
     0&\vec \sigma\vec\pi\\
     \vec \sigma\vec\pi&0
   \end{pmatrix}
 + \frac{e}{c}\Phi    +
 \begin{pmatrix}
   \mathds 1_2&0\\
   0&-\mathds 1_2
 \end{pmatrix}
mc  \right)\begin{pmatrix}  \phi\\\chi \end{pmatrix}  \notag\\
%
 mc^2   \begin{pmatrix}  \phi\\\chi \end{pmatrix} +  i\hbar\begin{pmatrix}  \dot \phi\\ \dot \chi \end{pmatrix} &= c
   \begin{pmatrix}
     0&\vec \sigma\vec\pi\\
     \vec \sigma\vec\pi&0
   \end{pmatrix}\begin{pmatrix}  \phi\\\chi \end{pmatrix} 
 + e\Phi\begin{pmatrix}  \phi\\\chi \end{pmatrix}     +
 \begin{pmatrix}
   \mathds 1_2&0\\
   0&-\mathds 1_2
 \end{pmatrix}
mc^2  \begin{pmatrix}  \phi\\\chi \end{pmatrix}  \notag\\
%
 mc^2   \begin{pmatrix}  \phi\\\chi \end{pmatrix} +  i\hbar\begin{pmatrix}  \dot \phi\\ \dot \chi \end{pmatrix} &= c
 \begin{pmatrix} \vec \sigma\vec\pi \chi\\\vec \sigma\vec\pi\phi \end{pmatrix} 
 + e\Phi\begin{pmatrix}  \phi\\\chi \end{pmatrix}     +
mc^2  \begin{pmatrix}  \phi\\-\chi \end{pmatrix} \qquad |-mc^2   \begin{pmatrix}  \phi\\\chi \end{pmatrix} \notag\\
%
 i\hbar\begin{pmatrix}  \dot \phi\\ \dot \chi \end{pmatrix} &= c
 \begin{pmatrix} \vec \sigma\vec\pi \chi\\\vec \sigma\vec\pi\phi \end{pmatrix} 
 + e\Phi\begin{pmatrix}  \phi\\\chi \end{pmatrix}     +
2mc^2  \begin{pmatrix}  0\\-\chi \end{pmatrix} 
\end{align}

Hieraus ergeben sich zwei gekoppelte Differentialgleichungen

\begin{align}
  i\hbar\dot\phi &= c\vec\sigma\vec\pi\chi + e\Phi\phi  \label{eq:6.1}\\
i\hbar\dot\chi &= c\vec\sigma\vec\pi\phi +  e\Phi\chi - 2mc^2\chi  \label{eq:6.2}
\end{align}

Nun wollen wir die Gleichung (\ref{eq:6.2}) untersuchen

\begin{align}
  \label{eq:6}
 \underbr{ i\hbar\pdiff_t}_{E_s}\chi &= c\vec\sigma\vec\pi\phi + e\Phi \chi - 2mc^2\chi \notag\\
 (2mc^2+E_s-e\Phi )\chi &= c\vec\sigma\vec\pi\phi  \notag\\
\Leftrightarrow \chi&= \frac{c\vec\sigma\vec\pi\phi}{2mc^2+E_s-e\Phi}
\end{align}

Bei nicht relativistischen Grenzfall ist die Ruhe-Energie \(mc^2\) die Größte Energie im Vergleich zu \(E_s\) und \(-e\Phi\). Zum Beispiel für ein Elektron gilt \(2mc^2 \approx 10MeV\) und Schrödigner-Energie \(E_s\approx 13eV \) und für die potentielle Energie \(e\Phi=\frac{e^3}{a_0} \approx 1\cdot10^{-27}eV\). Also können wir die zwei Energieen \(E_s\) und \(e\Phi\)  in Gleichung (\ref{eq:6})  vernachlässigen und erhalten folgenden Näherung

\begin{align}
  \label{eq:7}
  \chi \approx \frac{c\vec\sigma\vec\pi}{2mc^2}\phi = \frac{\vec\sigma\vec\pi}{2mc}\phi
\end{align}

Die  Spinor-Komponente \(\chi\) nennt man auch die \textbf{kleine} und \(\phi\) als \textbf{große} Komponente des Dirac-Spinors. Zum Beweis machen wir folgende Abschätzung

\begin{align}
  \label{eq:9}
  \chi = \frac{\vec\sigma\vec\pi}{2mc}\phi \approx \frac{\vec p}{2mc}\phi = \frac{m \vec v}{2mc}\phi =  \frac{ \vec v}{2c}\phi
\end{align}
D.h. \(\chi\) ist um den Proportionalitätsfaktor \(\frac{v}{c}\) kleiner als \(\phi\). Mit Sicherheit ist \(|\frac{v}{c}| \ll 1\). 

Die Näherung (\ref{eq:7}) für die kleine Komponente setzen wir in die erste Differentialgleichung (\ref{eq:6.1})  ein und erhalten eine Differentialgleichung die nur noch von \(\phi\) abhängig ist

\begin{align}
  \label{eq:8}
   i\hbar\dot\phi &= c\vec\sigma\vec\pi \frac{\vec\sigma\vec\pi}{2mc}\phi  + e\Phi\phi  \notag \\
 &= \frac{(\vec\sigma\vec\pi)^2}{2m}\phi  + e\Phi\phi
\end{align}

Wir möchten nun den Term \((\vec\sigma\vec\pi)^2\) berechnen

\begin{align}
  \label{eq:10}
  (\vec\sigma\vec\pi)^2 = \sum_{i,j}\sigma_i\sigma_j\pi_i\pi_j
\end{align}
Mit der Relation der Pauli-Matrizen

\begin{align}
  \label{eq:11}
  \sigma_i\sigma_j &= \frac{1}{2}\sigma_i\sigma_j + \frac{1}{2}\sigma_i\sigma_j \notag\\
 &= \frac{1}{2}\sigma_i\sigma_j - \frac{1}{2}\sigma_j\sigma_i  + \frac{1}{2}\sigma_i\sigma_j + \frac{1}{2}\sigma_j\sigma_i  \notag\\
&= \frac{1}{2}\underbr{[\sigma_i,\sigma_j]}_{2i\epsilon_{ijk}\sigma_k} + \frac{1}{2}\underbr{\{\sigma_i,\sigma_j\}}_{2\delta_{ij}} \notag\\
&= \delta_{ij} + i\epsilon_{ijk}\sigma_k 
\end{align}
eingesetzt in die Gleichung (\ref{eq:10})

\begin{align}
  \label{eq:12}
  (\vec\sigma\vec\pi)^2 &= \sum_{i,j}(\delta_{ij} + i\epsilon_{ijk}\sigma_k )\pi_i\pi_j \notag\\
 &= \sum_{i,j}\delta_{ij}\pi_i\pi_j + i \sum_{i,j}\epsilon_{ijk}\sigma_k \pi_i\pi_j \notag\\
&= (\vec\pi)^2 + i\sigma_k \sum_{i,j}\epsilon_{ijk} \pi_i\pi_j 
\end{align}

Mit einer Nebenrechnung und der Bedienung fürs Vorzeichenwechseln beim antizyklischen Vertauscchen des Epsilontensors \( \epsilon_{ijk} \pi_i\pi_j = \epsilon_{jik} \pi_j\pi_i= - \epsilon_{ijk} \pi_j\pi_i  \) 

\begin{align}
  \label{eq:13}
  \epsilon_{ijk}\pi_i\pi_j  &= \frac{1}{2}(  \epsilon_{ijk}\pi_i\pi_j + \underbr{ \epsilon_{ijk} \pi_i\pi_j}_{-\epsilon_{ijk} \pi_j\pi_i} )\notag\\
&= \frac{1}{2} \epsilon_{ijk}[\pi_i,\pi_j] \notag\\
&= \frac{1}{2} \epsilon_{ijk}[(\frac{\hbar}{i}\nabla_i-\frac{e}{c}A_i),(\frac{\hbar}{i}\nabla_j-\frac{e}{c}A_j) ] \notag\\
&= \frac{1}{2} \epsilon_{ijk}\left(\underbr{[\frac{\hbar}{i}\nabla_i,\frac{\hbar}{i}\nabla_j]}_{=0}-[\frac{\hbar}{i}\nabla_i, \frac{e}{c}A_j]- [\frac{e}{c}A_i,\frac{\hbar}{i}\nabla_j]+ \underbr{[\frac{e}{c}A_i,\frac{e}{c}A_j]}_{=0}\right) \notag\\
&= -\frac{1}{2} \epsilon_{ijk}\frac{\hbar e}{ic} \left( [\nabla_i, A_j] + [A_i,\nabla_j]\right) \notag\\
&= -\frac{1}{2} \epsilon_{ijk}\frac{\hbar e}{ic} \left( \nabla_i A_j-A_j\nabla_i + A_i\nabla_j-\nabla_jA_i\right)
\end{align}

Um die Klammer zu vereinfachen wenden wir sie auf eine stetig differenzierbare Funktion \(\psi\) an

\begin{align}
  \label{eq:14}
  &\nabla_i (A_j\psi) -A_j\nabla_i(\psi) + A_i\nabla_j(\psi)-\nabla_j(A_i\psi)= \notag\\
  =&\nabla_i (A_j)\psi + \cancel{A_j\nabla_i(\psi)} -\cancel{A_j\nabla_i(\psi)} + \cancel{A_i\nabla_j(\psi)} - \nabla_j(A_i)\psi - \cancel{A_i\nabla_j(\psi)} \notag\\
  =&\nabla_i (A_j)\psi  - \nabla_j(A_i)\psi 
\end{align}
Eingesetzt in (\ref{eq:13})

\begin{align}
  \label{eq:15}
  \epsilon_{ijk}\pi_i\pi_j  &= -\frac{1}{2} \epsilon_{ijk}\frac{\hbar e}{ic} \left( \underbr{\nabla_i A_j-A_j\nabla_i + A_i\nabla_j-\nabla_jA_i}_{\nabla_i A_j  - \nabla_j A_i}\right) \notag\\
  \epsilon_{ijk}\pi_i\pi_j  &= -\frac{1}{2} \frac{\hbar e}{ic}  \epsilon_{ijk}\left( \nabla_i A_j  - \nabla_j A_i\right) \notag\\
\end{align}
Setzen wir die Gleichung aus der Nebenrechnung (\ref{eq:15}) in die Gleichung (\ref{eq:12}) nun ein

\begin{align}
  \label{eq:16}
   (\vec\sigma\vec\pi)^2 &= \vec\pi^2 - i\frac{1}{2} \frac{\hbar e}{ic} \sigma_k \sum_{i,j} \epsilon_{ijk}\left( \nabla_i A_j  - \nabla_j A_i\right) \notag\\
&= \vec\pi^2 - \frac{1}{2} \frac{\hbar e}{c} \sigma_k \left( \sum_{i,j} \epsilon_{ijk}\nabla_i A_j  - \sum_{i,j} \epsilon_{ijk}\nabla_j A_i\right) \qquad\text{mit }\epsilon_{ijk}\nabla_j A_i=-\epsilon_{ijk}\nabla_i A_j \notag\\
&= \vec\pi^2 -  \frac{\hbar e}{c} \sigma_k  \underbr{\sum_{i,j} \epsilon_{ijk}\nabla_i A_j}_{(\vec\nabla \times \vec A)_k }  \notag\\
&= \vec\pi^2 -  \frac{\hbar e}{c}\, \vec \sigma\cdot \underbr{ (\vec\nabla \times \vec A)}_{\vec B}   \notag\\
&= \vec\pi^2 -  \frac{\hbar e}{c}\, \vec \sigma\cdot\vec B 
\end{align}

Die Gleichung (\ref{eq:16}) setzen wir in unsere ursprüngliche erste Differetialgleichung (\ref{eq:8}) ein

\begin{align}
  \label{eq:17}
   i\hbar\pdiff_t\phi &= \left(\frac{(\vec\sigma\vec\pi)^2}{2m}  + e\Phi \right)\phi \notag \\
 &=  \left(\frac{\vec\pi^2}{2m} -  \frac{\hbar e}{2mc}\, \vec \sigma\cdot\vec B   + e\Phi\right) \phi 
\end{align}
Setzen wir für \(\vec\pi = \vec p - \frac{e}{c}\vec A\) in die Gleichung (\ref{eq:17}) ein so erhalten wir die schon aus der nicht relativistischen Quantenmechanik bekannte \textbf{Pauli-Gleichung}

\begin{align}
  \label{eq:18}
\boxed{   i\hbar\pdiff_t\phi =  \left[\frac{1}{2m}\left( \vec p - \frac{e}{c}\vec A\right)^2  -  \frac{\hbar e}{2mc}\, \vec \sigma\cdot\vec B   + e\Phi\right] \phi  }
\end{align}




\subsection*{Referenzen}
\begin{itemize}
\item Schwabl
\end{itemize}

\end{document}
