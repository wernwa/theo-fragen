\input{../headers/header_script.tex}
\usepackage{amsmath} 



\begin{document}

\section*{Dirac-Gleichung klassische Näherung}

Die nicht relativistische (also klassische) Näherung der Dirac-Gleichung ergibt die uns schon bekannte \textbf{Pauli-Gleichung}. Wir starten mit der Dirac-Gleichung in kanonischer Form

\begin{align}
  \label{eq:1}
  i\hbar\pdiff_t \psi(x) = c\left(\vec \alpha \vec p +\beta mc  \right)
\end{align}

Nun betrachten ein Teilchen in einem elektromagnetischen Feld. Dazu führen wir den veralgemeinerten Impuls ein

\begin{align}
  \label{eq:2}
  \vec p \rightarrow \vec p - \frac{e}{c}\vec A
\end{align}
Und das Skalarpotential \(\Phi=cA^{0}\). Somit erhalten wir die Dirac-Gleichung in einenm elektromagnetischen Potential

\begin{align}
  \label{eq:3}
  i\hbar\pdiff_t \psi(x) = c\left(\vec \alpha (\vec p - \frac{e}{c}\vec A) + e\Phi   +\beta mc  \right)
\end{align}






\subsection*{Referenzen}
\begin{itemize}
\item TODO
\end{itemize}

\end{document}
