\documentclass[10pt,a4paper,oneside,fleqn]{article}
\usepackage{geometry}
\geometry{a4paper,left=20mm,right=20mm,top=1cm,bottom=2cm}
\usepackage[utf8]{inputenc}
%\usepackage{ngerman}
\usepackage{amsmath}                % brauche ich um dir Formel zu umrahmen.
\usepackage{amsfonts}                % brauche ich für die Mengensymbole
\usepackage{graphicx}
\setlength{\parindent}{0px}
\setlength{\mathindent}{10mm}
\usepackage{bbold}                    %brauche ich für die doppel Zahlen Darstellung (Einheitsmatrix z.B)



\usepackage{color}
\usepackage{titlesec} %sudo apt-get install texlive-latex-extra

\definecolor{darkblue}{rgb}{0.1,0.1,0.55}
\definecolor{verydarkblue}{rgb}{0.1,0.1,0.35}
\definecolor{darkred}{rgb}{0.55,0.2,0.2}

%hyperref Link color
\usepackage[colorlinks=true,
        linkcolor=darkblue,
        citecolor=darkblue,
        filecolor=darkblue,
        pagecolor=darkblue,
        urlcolor=darkblue,
        bookmarks=true,
        bookmarksopen=true,
        bookmarksopenlevel=3,
        plainpages=false,
        pdfpagelabels=true]{hyperref}

\titleformat{\chapter}[display]{\color{darkred}\normalfont\huge\bfseries}{\chaptertitlename\
\thechapter}{20pt}{\Huge}

\titleformat{\section}{\color{darkblue}\normalfont\Large\bfseries}{\thesection}{1em}{}
\titleformat{\subsection}{\color{verydarkblue}\normalfont\large\bfseries}{\thesubsection}{1em}{}

% Notiz Box
\usepackage{fancybox}
\newcommand{\notiz}[1]{\vspace{5mm}\ovalbox{\begin{minipage}{1\textwidth}#1\end{minipage}}\vspace{5mm}}

\usepackage{cancel}
\setcounter{secnumdepth}{3}
\setcounter{tocdepth}{3}





%-------------------------------------------------------------------------------
%Diff-Makro:
%Das Diff-Makro stellt einen Differentialoperator da.
%
%Benutzung:
% \diff  ->  d
% \diff f  ->  df
% \diff^2 f  ->  d^2 f
% \diff_x  ->  d/dx
% \diff^2_x  ->  d^2/dx^2
% \diff f_x  ->  df/dx
% \diff^2 f_x  ->  d^2f/dx^2
% \diff^2{f(x^5)}_x  ->  d^2(f(x^5))/dx^2
%
%Ersetzt man \diff durch \pdiff, so wird der partieller
%Differentialoperator dargestellt.
%
\makeatletter
\def\diff@n^#1{\@ifnextchar{_}{\diff@n@d^#1}{\diff@n@fun^#1}}
\def\diff@n@d^#1_#2{\frac{\textrm{d}^#1}{\textrm{d}#2^#1}}
\def\diff@n@fun^#1#2{\@ifnextchar{_}{\diff@n@fun@d^#1#2}{\textrm{d}^#1#2}}
\def\diff@n@fun@d^#1#2_#3{\frac{\textrm{d}^#1 #2}{\textrm{d}#3^#1}}
\def\diff@one@d_#1{\frac{\textrm{d}}{\textrm{d}#1}}
\def\diff@one@fun#1{\@ifnextchar{_}{\diff@one@fun@d #1}{\textrm{d}#1}}
\def\diff@one@fun@d#1_#2{\frac{\textrm{d}#1}{\textrm{d}#2}}
\newcommand*{\diff}{\@ifnextchar{^}{\diff@n}
  {\@ifnextchar{_}{\diff@one@d}{\diff@one@fun}}}
%
%Partieller Diff-Operator.
\def\pdiff@n^#1{\@ifnextchar{_}{\pdiff@n@d^#1}{\pdiff@n@fun^#1}}
\def\pdiff@n@d^#1_#2{\frac{\partial^#1}{\partial#2^#1}}
\def\pdiff@n@fun^#1#2{\@ifnextchar{_}{\pdiff@n@fun@d^#1#2}{\partial^#1#2}}
\def\pdiff@n@fun@d^#1#2_#3{\frac{\partial^#1 #2}{\partial#3^#1}}
\def\pdiff@one@d_#1{\frac{\partial}{\partial #1}}
\def\pdiff@one@fun#1{\@ifnextchar{_}{\pdiff@one@fun@d #1}{\partial#1}}
\def\pdiff@one@fun@d#1_#2{\frac{\partial#1}{\partial#2}}
\newcommand*{\pdiff}{\@ifnextchar{^}{\pdiff@n}
  {\@ifnextchar{_}{\pdiff@one@d}{\pdiff@one@fun}}}
\makeatother
%
%Das gleich nur mit etwas andere Syntax. Die Potenz der Differentiation wird erst
%zum Schluss angegeben. Somit lautet die Syntax:
%
% \diff_x^2  ->  d^2/dx^2
% \diff f_x^2  ->  d^2f/dx^2
% \diff{f(x^5)}_x^2  ->  d^2(f(x^5))/dx^2
% Ansonsten wie Oben.
%
%Ersetzt man \diff durch \pdiff, so wird der partieller
%Differentialoperator dargestellt.
%
%\makeatletter
%\def\diff@#1{\@ifnextchar{_}{\diff@fun#1}{\textrm{d} #1}}
%\def\diff@one_#1{\@ifnextchar{^}{\diff@n{#1}}%
%  {\frac{\textrm d}{\textrm{d} #1}}}
%\def\diff@fun#1_#2{\@ifnextchar{^}{\diff@fun@n#1_#2}%
%  {\frac{\textrm d #1}{\textrm{d} #2}}}
%\def\diff@n#1^#2{\frac{\textrm d^#2}{\textrm{d}#1^#2}}
%\def\diff@fun@n#1_#2^#3{\frac{\textrm d^#3 #1}%
%  {\textrm{d}#2^#3}}
%\def\diff{\@ifnextchar{_}{\diff@one}{\diff@}}
%\newcommand*{\diff}{\@ifnextchar{_}{\diff@one}{\diff@}}
%
%Partieller Diff-Operator.
%\def\pdiff@#1{\@ifnextchar{_}{\pdiff@fun#1}{\partial #1}}
%\def\pdiff@one_#1{\@ifnextchar{^}{\pdiff@n{#1}}%
%  {\frac{\partial}{\partial #1}}}
%\def\pdiff@fun#1_#2{\@ifnextchar{^}{\pdiff@fun@n#1_#2}%
%  {\frac{\partial #1}{\partial #2}}}
%\def\pdiff@n#1^#2{\frac{\partial^#2}{\partial #1^#2}}
%\def\pdiff@fun@n#1_#2^#3{\frac{\partial^#3 #1}%
%  {\partial #2^#3}}
%\newcommand*{\pdiff}{\@ifnextchar{_}{\pdiff@one}{\pdiff@}}
%\makeatother

%-------------------------------------------------------------------------------
%%Nützliche Makros um in der Quantenmechanik Bras, Kets und das Skalarprodukt
%%zwischen den beiden darzustellen.
%%Benutzung:
%% \bra{x}  ->    < x |
%% \ket{x}  ->    | x >
%% \braket{x}{y} ->   < x | y >

\newcommand\bra[1]{\left\langle #1 \right|}
\newcommand\ket[1]{\left| #1 \right\rangle}
\newcommand\braket[2]{%
  \left\langle #1\vphantom{#2} \right.%
  \left|\vphantom{#1#2}\right.%
  \left. \vphantom{#1}#2 \right\rangle}%

%-------------------------------------------------------------------------------
%%Aus dem Buch:
%%Titel:  Latex in Naturwissenschaften und Mathematik
%%Autor:  Herbert Voß
%%Verlag: Franzis Verlag, 2006
%%ISBN:   3772374190, 9783772374197
%%
%%Hier werden drei Makros definiert:\mathllap, \mathclap und \mathrlap, welche
%%analog zu den aus Latex bekannten \rlap und \llap arbeiten, d.h. selbst
%%keinerlei horizontalen Platz benötigen, aber dennoch zentriert zum aktuellen
%%Punkt erscheinen.

\newcommand*\mathllap{\mathstrut\mathpalette\mathllapinternal}
\newcommand*\mathllapinternal[2]{\llap{$\mathsurround=0pt#1{#2}$}}
\newcommand*\clap[1]{\hbox to 0pt{\hss#1\hss}}
\newcommand*\mathclap{\mathpalette\mathclapinternal}
\newcommand*\mathclapinternal[2]{\clap{$\mathsurround=0pt#1{#2}$}}
\newcommand*\mathrlap{\mathpalette\mathrlapinternal}
\newcommand*\mathrlapinternal[2]{\rlap{$\mathsurround=0pt#1{#2}$}}

%%Das Gleiche nur mit \def statt \newcommand.
%\def\mathllap{\mathpalette\mathllapinternal}
%\def\mathllapinternal#1#2{%
%  \llap{$\mathsurround=0pt#1{#2}$}% $
%}
%\def\clap#1{\hbox to 0pt{\hss#1\hss}}
%\def\mathclap{\mathpalette\mathclapinternal}
%\def\mathclapinternal#1#2{%
%  \clap{$\mathsurround=0pt#1{#2}$}%
%}
%\def\mathrlap{\mathpalette\mathrlapinternal}
%\def\mathrlapinternal#1#2{%
%  \rlap{$\mathsurround=0pt#1{#2}$}% $
%}

%-------------------------------------------------------------------------------
%%Hier werden zwei neue Makros definiert \overbr und \underbr welche analog zu
%%\overbrace und \underbrace funktionieren jedoch die Gleichung nicht
%%'zerreißen'. Dies wird ermöglicht durch das \mathclap Makro.

\def\overbr#1^#2{\overbrace{#1}^{\mathclap{#2}}}
\def\underbr#1_#2{\underbrace{#1}_{\mathclap{#2}}}
\usepackage{amsmath} 



\begin{document}

\textit{29. März 2012}
\input{../headers/authors.tex}

\section*{Dirac-Gleichung klassische Näherung}

Die nicht relativistische (also klassische) Näherung der Dirac-Gleichung ergibt die uns schon bekannte \textbf{Pauli-Gleichung}. Wir starten mit der Dirac-Gleichung in kanonischer Form

\begin{align}
  \label{eq:1}
  i\hbar\pdiff_t \psi(x) = c\left(\vec \alpha \vec p +\beta mc  \right)\psi(x)
\end{align}

Nun betrachten ein Teilchen in einem elektromagnetischen Feld. Dazu führen wir den veralgemeinerten Impuls ein

\begin{align}
  \label{eq:2}
  \vec p \rightarrow \vec p - \frac{e}{c}\vec A
\end{align}
Und das Skalarpotential \(\Phi=cA^{0}\). Somit erhalten wir die Dirac-Gleichung in einem elektromagnetischen Potential

\begin{align}
  \label{eq:3}
  i\hbar\pdiff_t \psi(x) = c\left(\vec \alpha (\vec p - \frac{e}{c}\vec A) + \frac{e}{c}\Phi   +\beta mc  \right)\psi(x)
\end{align}

Um diese Gleichung zu lösen machen wir folgenden Ansatz

\begin{align}
  \label{eq:4}
  \psi(x) = e^{-\frac{i}{\hbar}mc^2 t} \begin{pmatrix}  \phi\\\chi \end{pmatrix}
\end{align}
Eingesetzt in die Gleichung (\ref{eq:3}) mit dem verallgemeinen Impuls \(\vec\pi = \vec p - \frac{e}{c}\vec A \) ergibt

\begin{align}
  \label{eq:5}
   i\hbar\pdiff_t\left[ e^{-\frac{i}{\hbar}mc^2 t} \begin{pmatrix}  \phi\\\chi \end{pmatrix}\right] &= c\left(\vec \alpha \vec\pi + \frac{e}{c}\Phi   +\beta mc  \right)e^{-\frac{i}{\hbar}mc^2 t} \begin{pmatrix}  \phi\\\chi \end{pmatrix} \notag\\
%
 mc^2  \cancel{e^{-\frac{i}{\hbar}mc^2 t}} \begin{pmatrix}  \phi\\\chi \end{pmatrix} + \cancel{e^{-\frac{i}{\hbar}mc^2 t}} i\hbar\begin{pmatrix}  \dot \phi\\ \dot \chi \end{pmatrix} &= c\left(
   \begin{pmatrix}
     0&\vec \sigma\vec\pi\\
     \vec \sigma\vec\pi&0
   \end{pmatrix}
 + \frac{e}{c}\Phi    +
 \begin{pmatrix}
   \mathds 1_2&0\\
   0&-\mathds 1_2
 \end{pmatrix}
mc  \right)\cancel{e^{-\frac{i}{\hbar}mc^2 t}} \begin{pmatrix}  \phi\\\chi \end{pmatrix} \notag \\
%
 mc^2   \begin{pmatrix}  \phi\\\chi \end{pmatrix} +  i\hbar\begin{pmatrix}  \dot \phi\\ \dot \chi \end{pmatrix} &= c\left(
   \begin{pmatrix}
     0&\vec \sigma\vec\pi\\
     \vec \sigma\vec\pi&0
   \end{pmatrix}
 + \frac{e}{c}\Phi    +
 \begin{pmatrix}
   \mathds 1_2&0\\
   0&-\mathds 1_2
 \end{pmatrix}
mc  \right)\begin{pmatrix}  \phi\\\chi \end{pmatrix}  \notag\\
%
 mc^2   \begin{pmatrix}  \phi\\\chi \end{pmatrix} +  i\hbar\begin{pmatrix}  \dot \phi\\ \dot \chi \end{pmatrix} &= c
   \begin{pmatrix}
     0&\vec \sigma\vec\pi\\
     \vec \sigma\vec\pi&0
   \end{pmatrix}\begin{pmatrix}  \phi\\\chi \end{pmatrix} 
 + e\Phi\begin{pmatrix}  \phi\\\chi \end{pmatrix}     +
 \begin{pmatrix}
   \mathds 1_2&0\\
   0&-\mathds 1_2
 \end{pmatrix}
mc^2  \begin{pmatrix}  \phi\\\chi \end{pmatrix}  \notag\\
%
 mc^2   \begin{pmatrix}  \phi\\\chi \end{pmatrix} +  i\hbar\begin{pmatrix}  \dot \phi\\ \dot \chi \end{pmatrix} &= c
 \begin{pmatrix} \vec \sigma\vec\pi \chi\\\vec \sigma\vec\pi\phi \end{pmatrix} 
 + e\Phi\begin{pmatrix}  \phi\\\chi \end{pmatrix}     +
mc^2  \begin{pmatrix}  \phi\\-\chi \end{pmatrix} \qquad |-mc^2   \begin{pmatrix}  \phi\\\chi \end{pmatrix} \notag\\
%
 i\hbar\begin{pmatrix}  \dot \phi\\ \dot \chi \end{pmatrix} &= c
 \begin{pmatrix} \vec \sigma\vec\pi \chi\\\vec \sigma\vec\pi\phi \end{pmatrix} 
 + e\Phi\begin{pmatrix}  \phi\\\chi \end{pmatrix}     +
2mc^2  \begin{pmatrix}  0\\-\chi \end{pmatrix} 
\end{align}

Hieraus ergeben sich zwei gekoppelte Differentialgleichungen

\begin{align}
  i\hbar\dot\phi &= c\vec\sigma\vec\pi\chi + e\Phi\phi  \label{eq:6.1}\\
i\hbar\dot\chi &= c\vec\sigma\vec\pi\phi +  e\Phi\chi - 2mc^2\chi  \label{eq:6.2}
\end{align}

Nun wollen wir die Gleichung (\ref{eq:6.2}) untersuchen

\begin{align}
  \label{eq:6}
 \underbr{ i\hbar\pdiff_t}_{E_s}\chi &= c\vec\sigma\vec\pi\phi + e\Phi \chi - 2mc^2\chi \notag\\
 (2mc^2+E_s-e\Phi )\chi &= c\vec\sigma\vec\pi\phi  \notag\\
\Leftrightarrow \chi&= \frac{c\vec\sigma\vec\pi\phi}{2mc^2+E_s-e\Phi}
\end{align}

Bei nicht relativistischen Grenzfall ist die Ruhe-Energie \(mc^2\) die Größte Energie im Vergleich zu \(E_s\) und \(-e\Phi\). Zum Beispiel für ein Elektron gilt \(2mc^2 \approx 10MeV\) und Schrödigner-Energie \(E_s\approx 13eV \) und für die potentielle Energie \(e\Phi=\frac{e^3}{a_0} \approx 1\cdot10^{-27}eV\). Also können wir die zwei Energieen \(E_s\) und \(e\Phi\)  in Gleichung (\ref{eq:6})  vernachlässigen und erhalten folgenden Näherung

\begin{align}
  \label{eq:7}
  \chi \approx \frac{c\vec\sigma\vec\pi}{2mc^2}\phi = \frac{\vec\sigma\vec\pi}{2mc}\phi
\end{align}

Die  Spinor-Komponente \(\chi\) nennt man auch die \textbf{kleine} und \(\phi\) als \textbf{große} Komponente des Dirac-Spinors. Zum Beweis machen wir folgende Abschätzung

\begin{align}
  \label{eq:9}
  \chi = \frac{\vec\sigma\vec\pi}{2mc}\phi \approx \frac{\vec p}{2mc}\phi = \frac{m \vec v}{2mc}\phi =  \frac{ \vec v}{2c}\phi
\end{align}
D.h. \(\chi\) ist um den Proportionalitätsfaktor \(\frac{v}{c}\) kleiner als \(\phi\). Mit Sicherheit ist \(|\frac{v}{c}| \ll 1\). 

Die Näherung (\ref{eq:7}) für die kleine Komponente setzen wir in die erste Differentialgleichung (\ref{eq:6.1})  ein und erhalten eine Differentialgleichung die nur noch von \(\phi\) abhängig ist

\begin{align}
  \label{eq:8}
   i\hbar\dot\phi &= c\vec\sigma\vec\pi \frac{\vec\sigma\vec\pi}{2mc}\phi  + e\Phi\phi  \notag \\
 &= \frac{(\vec\sigma\vec\pi)^2}{2m}\phi  + e\Phi\phi
\end{align}

Wir möchten nun den Term \((\vec\sigma\vec\pi)^2\) berechnen

\begin{align}
  \label{eq:10}
  (\vec\sigma\vec\pi)^2 = \sum_{i,j}\sigma_i\sigma_j\pi_i\pi_j
\end{align}
Mit der Relation der Pauli-Matrizen

\begin{align}
  \label{eq:11}
  \sigma_i\sigma_j &= \frac{1}{2}\sigma_i\sigma_j + \frac{1}{2}\sigma_i\sigma_j \notag\\
 &= \frac{1}{2}\sigma_i\sigma_j - \frac{1}{2}\sigma_j\sigma_i  + \frac{1}{2}\sigma_i\sigma_j + \frac{1}{2}\sigma_j\sigma_i  \notag\\
&= \frac{1}{2}\underbr{[\sigma_i,\sigma_j]}_{2i\epsilon_{ijk}\sigma_k} + \frac{1}{2}\underbr{\{\sigma_i,\sigma_j\}}_{2\delta_{ij}}
\end{align}
eingesetzt in die Gleichung (\ref{eq:10})

\begin{align}
  \label{eq:12}
  (\vec\sigma\vec\pi)^2 &= \sum_{i,j}(\delta_{ij} +  \frac{1}{2}[\sigma_i,\sigma_j] )\pi_i\pi_j \notag\\
 &= \sum_{i,j}\delta_{ij}\pi_i\pi_j +  \frac{1}{2}\sum_{i,j}[\sigma_i,\sigma_j] \pi_i\pi_j \notag\\
&= \vec\pi^2 + \frac{1}{2} \sum_{i,j}[\sigma_i,\sigma_j] \pi_i\pi_j 
\end{align}

Mit einer Nebenrechnung und der Bedienung fürs Vorzeichenwechseln beim antizyklischen Vertauscchen des Epsilontensors \( [\sigma_i,\sigma_j] \pi_i\pi_j \stackrel{i\leftrightarrow j}= [\sigma_j,\sigma_i] \pi_j\pi_i = - [\sigma_i,\sigma_j] \pi_j\pi_i  \) 

\begin{align}
  \label{eq:13}
  [\sigma_i,\sigma_j]\cdot \pi_i\pi_j  &= \frac{1}{2}(  [\sigma_i,\sigma_j]\cdot \pi_i\pi_j + \underbr{  [\sigma_i,\sigma_j]\cdot  \pi_i\pi_j}_{-[\sigma_i,\sigma_j]\cdot \pi_j\pi_i} )\notag\\
 &=\frac{1}{2} [\sigma_i,\sigma_j] (\pi_i\pi_j-\pi_j\pi_i)\notag\\
&= \frac{1}{2} [\sigma_i,\sigma_j]\cdot [\pi_i,\pi_j] \notag\\
&= \frac{1}{2}[\sigma_i,\sigma_j]\cdot[(\frac{\hbar}{i}\nabla_i-\frac{e}{c}A_i),(\frac{\hbar}{i}\nabla_j-\frac{e}{c}A_j) ] \notag\\
&= \frac{1}{2} [\sigma_i,\sigma_j]\cdot\left(\underbr{[\frac{\hbar}{i}\nabla_i,\frac{\hbar}{i}\nabla_j]}_{=0}-[\frac{\hbar}{i}\nabla_i, \frac{e}{c}A_j]- [\frac{e}{c}A_i,\frac{\hbar}{i}\nabla_j]+ \underbr{[\frac{e}{c}A_i,\frac{e}{c}A_j]}_{=0}\right) \notag\\
&= -\frac{1}{2} [\sigma_i,\sigma_j]\cdot\frac{\hbar e}{ic} \left( [\nabla_i, A_j] + [A_i,\nabla_j]\right) \notag\\
&= -\frac{1}{2} [\sigma_i,\sigma_j]\cdot\frac{\hbar e}{ic} \left( \nabla_i A_j-A_j\nabla_i + A_i\nabla_j-\nabla_jA_i\right)
\end{align}

Um die Klammer zu vereinfachen wenden wir sie auf eine stetig differenzierbare Funktion \(\psi\) an

\begin{align}
  \label{eq:14}
  &\nabla_i (A_j\psi) -A_j\nabla_i(\psi) + A_i\nabla_j(\psi)-\nabla_j(A_i\psi)= \notag\\
  =&\nabla_i (A_j)\psi + \cancel{A_j\nabla_i(\psi)} -\cancel{A_j\nabla_i(\psi)} + \cancel{A_i\nabla_j(\psi)} - \nabla_j(A_i)\psi - \cancel{A_i\nabla_j(\psi)} \notag\\
  =&\nabla_i (A_j)\psi  - \nabla_j(A_i)\psi 
\end{align}
Eingesetzt in (\ref{eq:13})


\begin{align}
  \label{eq:15}
 [\sigma_i,\sigma_j]\cdot\pi_i\pi_j  &= -\frac{1}{2} [\sigma_i,\sigma_j]\cdot\frac{\hbar e}{ic} \left( \underbr{\nabla_i A_j-A_j\nabla_i + A_i\nabla_j-\nabla_jA_i}_{\nabla_i A_j  - \nabla_j A_i}\right) \notag\\
  [\sigma_i,\sigma_j]\cdot\pi_i\pi_j  &= -\frac{1}{2} \frac{\hbar e}{ic} [\sigma_i,\sigma_j]\cdot\left( \nabla_i A_j  - \nabla_j A_i\right) \notag\\
\end{align}
Setzen wir die Gleichung aus der Nebenrechnung (\ref{eq:15}) in die Gleichung (\ref{eq:12}) nun ein

\begin{align}
  \label{eq:16}
   (\vec\sigma\vec\pi)^2 &= \vec\pi^2 - \frac{1}{2} \frac{\hbar e}{2ic} \sum_{i,j} [\sigma_i,\sigma_j]\cdot\left( \nabla_i A_j  - \nabla_j A_i\right) \notag\\
&= \vec\pi^2 - \frac{1}{2} \frac{\hbar e}{2ci} \left( \sum_{i,j} [\sigma_i,\sigma_j]\cdot \nabla_i A_j  - \sum_{i,j} [\sigma_i,\sigma_j]\cdot\nabla_j A_i\right) \qquad\text{mit }[\sigma_i,\sigma_j]\cdot\nabla_j A_i=-[\sigma_i,\sigma_j]\cdot\nabla_i A_j \notag\\
&= \vec\pi^2 -  \frac{\hbar e}{2ci} \sum_{i,j} [\sigma_i,\sigma_j]\cdot\nabla_i A_j
\end{align}

Mit einer weiteren Nebenreichung für die Summe

\begin{align}
  \label{eq:19}
  \sum_{i,j} [\sigma_i,\sigma_j]\cdot\nabla_i A_j &= \sum_i\left( [\sigma_i,\sigma_1]\cdot\nabla_i A_1+[\sigma_i,\sigma_2]\cdot\nabla_i A_2+[\sigma_i,\sigma_3]\cdot\nabla_i A_3\right) \notag\\  
&=  \cancel{[\sigma_1,\sigma_1]}\cdot\nabla_1 A_1+[\sigma_1,\sigma_2]\cdot\nabla_1 A_2+[\sigma_1,\sigma_3]\cdot\nabla_1 A_3 \notag\\ 
 &\quad+[\sigma_2,\sigma_1]\cdot\nabla_2 A_1+\cancel{[\sigma_2,\sigma_2]}\cdot\nabla_2 A_2+[\sigma_2,\sigma_3]\cdot\nabla_2 A_3 \notag\\ 
 &\quad+ [\sigma_3,\sigma_1]\cdot\nabla_3 A_1+[\sigma_3,\sigma_2]\cdot\nabla_3 A_2+\cancel{[\sigma_3,\sigma_3]}\cdot\nabla_3 A_3 \quad \text{mit }[\sigma_i,\sigma_j] = 2i\epsilon_{ijk}\sigma_k \notag\\
&= 2i\left( \sigma_3\cdot\nabla_1 A_2 - \sigma_2\nabla_1 A_3 -\sigma_3\cdot\nabla_2 A_1+ \sigma_1\cdot\nabla_2 A_3 + \sigma_2\cdot\nabla_3 A_1-\sigma_1\cdot\nabla_3 A_2 \right)\notag\\
&= 2i\left(\sigma_1\underbr{(\nabla_2 A_3-\nabla_3 A_2)}_{B_1} + \sigma_2\underbr{(\nabla_3 A_1- \nabla_1 A_3)}_{B_2} +  \sigma_3\underbr{(\nabla_1 A_2-\nabla_2 A_1)}_{B_3}\right)\notag\\
&= 2i\vec\sigma\cdot(\vec\nabla\times\vec A)\notag\\
&=2i\vec\sigma\cdot\vec B
\end{align}
Die Nebenrechung (\ref{eq:19}) in (\ref{eq:16})

\begin{align}
  \label{eq:16.1}
(\vec\sigma\vec\pi)^2  &= \vec\pi^2 -  \frac{\hbar e}{2ic} 2i\vec\sigma\cdot\vec B \notag\\
&= \vec\pi^2 -  \frac{\hbar e}{c}\, \vec \sigma\cdot\vec B 
\end{align}

Die Gleichung (\ref{eq:16}) setzen wir in unsere ursprüngliche erste Differetialgleichung (\ref{eq:8}) ein

\begin{align}
  \label{eq:17}
   i\hbar\pdiff_t\phi &= \left(\frac{(\vec\sigma\vec\pi)^2}{2m}  + e\Phi \right)\phi \notag \\
 &=  \left(\frac{\vec\pi^2}{2m} -  \frac{\hbar e}{2mc}\, \vec \sigma\cdot\vec B   + e\Phi\right) \phi 
\end{align}
Setzen wir für \(\vec\pi = \vec p - \frac{e}{c}\vec A\) in die Gleichung (\ref{eq:17}) ein so erhalten wir die schon aus der nicht relativistischen Quantenmechanik bekannte \textbf{Pauli-Gleichung}

\begin{align}
  \label{eq:18}
\boxed{   i\hbar\pdiff_t\phi =  \left[\frac{1}{2m}\left( \vec p - \frac{e}{c}\vec A\right)^2  -  \frac{\hbar e}{2mc}\, \vec \sigma\cdot\vec B   + e\Phi\right] \phi  }
\end{align}

Zitat wiki: Die Pauli-Gleichung geht auf den österreichischen Physiker Wolfgang Pauli zurück. Sie beschreibt die zeitliche Entwicklung eines geladenen Spin-1/2-Teilchens, etwa eines Elektrons, das sich so langsam im elektromagnetischen Feld bewegt, dass die Feldenergie und die kinetische Energie klein gegen die Ruheenergie ist.

Aus diesem Grund ist es gerechtfertigt dass wir in der Näherung (\ref{eq:7}) die Terme \(E_s\) und \(-e\Phi\) vernachlässigt haben.\\
\\
In der Gleichung ist der gyromagnetisches Verhältniss \(g=2\) für das Elektron automatisch richtig herausgekommen. Um dies zu sehen betrachte folgende Rechnung. Mit folgenden Relationen

\begin{align}
  \label{eq:20}
  \vec B = \vec \nabla\times\vec A,\qquad \vec A = \frac{1}{2}\vec B\times \vec x,\qquad L=\vec x\times\vec p,\qquad \vec S = \frac{\hbar}{2}\vec \sigma
\end{align}

Quadriere den ersten Term in der Pauli-Gleichung (\ref{eq:20})

\begin{align}
  \label{eq:21}
   i\hbar\pdiff_t\phi =  \left[\frac{1}{2m}\vec p^2 + \frac{e^2}{2mc^2}\vec A^2-\frac{e}{mc}\vec p\cdot \vec A -  \frac{\hbar e}{2mc}\, \vec \sigma\cdot\vec B   + e\Phi\right] \phi
\end{align}
Mit einer kleinen Nebenrechnung

\begin{align}
  \label{eq:22}
  \vec p\cdot\vec A = \frac{1}{2}\vec p(\vec B\times\vec x) = \frac{1}{2}\vec B\underbr{ (\vec x\times\vec p)}_{\vec L} = \frac{1}{2}\vec B\cdot\vec L
\end{align}
Wieder eingesetzt in Gleichung (\ref{eq:21})

\begin{align}
  \label{eq:23}
   i\hbar\pdiff_t\phi &=  \left[\frac{1}{2m}\vec p^2 + \frac{e^2}{2mc^2}\vec A^2-\frac{e}{2mc}\vec B\cdot\vec L   -  \frac{\hbar e}{2mc}\, \vec \sigma\cdot\vec B   + e\Phi\right] \phi \quad \text{mit }\vec \sigma = \frac{2}{\hbar}\vec S \notag\\
   i\hbar\pdiff_t\phi &=  \left[\frac{1}{2m}\vec p^2 + \frac{e^2}{2mc^2}\vec A^2-\frac{e}{2mc}\vec B\cdot\vec L   -  \frac{2e}{2mc}\, \vec S \cdot\vec B   + e\Phi\right] \phi \notag\\
   i\hbar\pdiff_t\phi &=  \left[\frac{1}{2m}\vec p^2 + \frac{e^2}{2mc^2}\vec A^2- \frac{e}{2mc}\left(\vec L   +  \underbr{2}_{g} \vec S \right)\vec B   + e\Phi\right] \phi
\end{align}

Man sieht hier deutlich dass der Spin 2-mal stärker an das Magnetfeld koppelt als das gleichgroßer Bahndrehimpuls, was dem richtigen gyromagnetischen Faktor \(g\) für ein Spin \(\frac{1}{2}\) Teilchen entspricht.


\subsection*{Taylorentwicklung des kleinen Komponente des Spinors}

In der Dirac-Gleichung steckt noch mehr Information als in der Pauli-Gleichung. Dadurch ist es möglich weitere Korektur-Terme für das Spektrum zu bestimmen. Wir entwickeln dazu die kleine Komponente des Dirac-Spinors (\ref{eq:6}) nach Taylor

\begin{align}
  \label{eq:24}
  \chi&= \frac{c\vec\sigma\vec\pi\phi}{2mc^2+E_s-e\Phi} \notag\\
&= \frac{1}{2mc^2}\left( \frac{1}{1+ \frac{E_s-e\Phi}{2mc^2}}\right)c\vec\sigma\vec\pi\phi \notag\\
&= \frac{1}{2mc}\left(1 - \frac{E_s-e\Phi}{2mc^2} + \dots\right)\vec\sigma\vec\pi\phi
\end{align}

Setzen wir die Gleichung (\ref{eq:24}) in die große Komponente des Dirac-Spinors (\ref{eq:6.1})


\begin{align}
  \label{eq:25}
  \underbr{ i\hbar\pdiff_t}_{E_s}\phi &= c\vec\sigma\vec\pi\chi + e\Phi\phi \notag\\
 E_s\phi &= \frac{c\vec\sigma\vec\pi }{2mc}\left(1 - \frac{E_s-e\Phi}{2mc^2} \right)\vec\sigma\vec\pi\phi  + e\Phi\phi \qquad |-e\Phi\phi  \notag\\
 (E_s-e\Phi)\phi &= \frac{(\vec\sigma\vec\pi)^2 }{2m}\phi - \frac{\vec\sigma^2\vec\pi }{4m^2c^2} (E_s-e\Phi) \vec\pi\phi 
\end{align}

Es geht nun darum die Gleichung (\ref{eq:25}) zu vereinfachen. Dazu muss beachtet werden dass \(\vec \pi\) nicht dem Potential \(e\Phi(x)\) vertauscht. Wir machen folgende Nebenrechnung

\begin{align}
  \label{eq:26}
  (E_s-e\Phi)\vec\pi &= (E_s-e\Phi)\vec\pi + \vec\pi(E_s-e\Phi) - \vec\pi(E_s-e\Phi) \notag\\
&= \vec\pi(E_s-e\Phi) + [(E_s-e\Phi),\vec\pi ] \notag\\
&= \vec\pi(E_s-e\Phi) + \underbr{[E_s,\vec\pi ]}_{=0}-[e\Phi,\vec\pi ] \notag\\
&=\vec\pi(E_s-e\Phi) -[e\Phi,(\vec p -\frac{e}{c}\vec A) ]  \notag\\
&=\vec\pi(E_s-e\Phi) -[e\Phi,\vec p  ]  + \underbr{[e\Phi,\frac{e}{c}\vec A ]}_{=0}  \notag\\
&=\vec\pi(E_s-e\Phi) -[e\Phi,\vec p  ] 
\end{align}
Als weitere Nebenrechnung wollen wir den Kommutator \([e\Phi,\vec p  ] \) noch bestimmen

\begin{align}
  \label{eq:27}
  [e\Phi,\frac{\hbar}{i}\vec \nabla ]\phi &= \frac{\hbar e}{i}\left(\Phi\vec\nabla\phi -\vec\nabla( \Phi \phi)   \right) \notag\\
&= \frac{\hbar e}{i}\left(\cancel{\Phi\vec\nabla\phi} - \vec\nabla( \Phi) \phi - \cancel{\Phi\vec\nabla\phi}  \right) \notag\\
&= - \frac{\hbar e}{i}\vec\nabla( \Phi) \phi
\end{align}
Eingesetzt in (\ref{eq:26})

\begin{align}
  \label{eq:28}
  (E_s-e\Phi)\vec\pi &=\vec\pi(E_s-e\Phi) + \frac{\hbar}{i}\vec\nabla(e\Phi)
\end{align}
Fahren wir fort indem wir diese Gleichung (\ref{eq:28}) nun in die Master-Gleichung (\ref{eq:25}) einsetzen

\begin{align}
  \label{eq:29}
  \underline{(E_s-e\Phi)\phi} &= \frac{(\vec\sigma\vec\pi)^2 }{2m}\phi - \frac{\vec\sigma^2\vec\pi }{4m^2c^2}\left(\vec\pi(E_s-e\Phi) + \frac{\hbar}{i}\vec\nabla(e\Phi)\right)  \phi \notag\\
&= \frac{(\vec\sigma\vec\pi)^2 }{2m}\phi - \frac{(\vec\sigma\vec\pi)^2 }{4m^2c^2}\underline{(E_s-e\Phi)\phi}- \frac{\vec\sigma^2\vec\pi }{4m^2c^2}\frac{\hbar}{i}\vec\nabla(e\Phi) \phi 
\end{align}

Die Gleichung (\ref{eq:29}) stellt eine Rekursive Gleichung dar. Setzt man für den Term \((E_s-e\Phi)\phi\) wieder die Gleichung (\ref{eq:29}) ein, so kann man den Wert für \((E_s-e\Phi)\phi\) beliebig genau bestimmen. Da aber die Nachfolgenden Terme nur eine kleine Korrektur darstellen setzen wir \((E_s-e\Phi)\phi = \frac{(\vec\sigma\vec\pi)^2 }{2m}\phi \) ein

\begin{align}
  \label{eq:30}
  (E_s-e\Phi)\phi &= \frac{(\vec\sigma\vec\pi)^2 }{2m}\phi - \frac{(\vec\sigma\vec\pi)^2 }{4m^2c^2} \frac{(\vec\sigma\vec\pi)^2 }{2m}\phi - \frac{\vec\sigma^2\vec\pi }{4m^2c^2}\frac{\hbar}{i}\vec\nabla(e\Phi) \phi  \notag\\
&= \frac{(\vec\sigma\vec\pi)^2 }{2m}\phi - \frac{(\vec\sigma\vec\pi)^4 }{8m^3c^2}\phi - \frac{\vec\sigma^2\vec\pi }{4m^2c^2}\frac{\hbar}{i}\vec\nabla(e\Phi) \phi
\end{align}

Aus dieser Gleichung folgt die erweiterte Pauligleichung

\begin{align}
  \label{eq:31}
  i\hbar \pdiff_t \phi &=  \left[\frac{(\vec\sigma\vec\pi)^2 }{2m} - \frac{(\vec\sigma\vec\pi)^4 }{8m^3c^2} - \frac{\vec\sigma^2\vec\pi }{4m^2c^2}\frac{\hbar}{i}\vec\nabla(e\Phi)  +e\Phi\right]\phi\quad\text{mit }(\vec\sigma\vec\pi)^2 = \vec \pi^2 - \frac{\hbar e}{c}\vec\sigma\cdot\vec B  \notag\\
 &=  \left[\frac{\vec \pi^2}{2m} - \frac{\hbar e}{c}\frac{\vec\sigma\cdot\vec B}{2m} - \frac{(\vec\sigma\vec\pi)^4 }{8m^3c^2} - \frac{\vec\sigma^2\vec\pi }{4m^2c^2}\frac{\hbar}{i}\vec\nabla(e\Phi)  +e\Phi\right]\phi
\end{align}

Vergleiche mit der Pauligleichung (\ref{eq:18})

\begin{align}
  \label{eq:32}
 i\hbar \pdiff_t \phi &= \left[\frac{\vec \pi^2}{2m} - \frac{\hbar e}{c}\frac{\vec\sigma\cdot\vec B}{2m} +e\Phi\right]\phi
\end{align}

so sieht man, dass die beiden Terme \( - \frac{(\vec\sigma\vec\pi)^4 }{8m^3c^2} - \frac{\vec\sigma^2\vec\pi }{4m^2c^2}\frac{\hbar}{i}\vec\nabla(e\Phi)\) hinzugekommen sind.

Mit Hilfe der erweiterten Pauli-Gleichung kann man alle Korrekturterme für ein Wasserstoff-Spektrum bestimmen. Dazu betrachten wir den Spezialfall. 

\begin{itemize}
\item das Potential \(e\Phi = V(r)\) ist sphärisch symmetrisch \(\Rightarrow \vec\nabla(e\Phi)= \vec\nabla V=\frac{\vec r}{r}\frac{d}{dr}V \)
\item \(\vec A=0 \Rightarrow \vec \pi = \vec p = \frac{\hbar}{i}\vec \nabla\) und  mit Gleichung (\ref{eq:16.1}) \((\vec\sigma\vec \pi)^2=(\vec p)^2\)
\end{itemize}

Mit diesen Bedingungen lautet die erweiterte Pauligleichung (\ref{eq:30}) wie folgt

\begin{align}
  \label{eq:33}
   i\hbar \pdiff_t \phi &= \left[\underbr{\frac{\vec p^2}{2m}  +e\Phi}_{H_0} - \frac{(\vec p)^4 }{8m^3c^2} -\frac{\hbar}{i} \frac{\vec\sigma^2\vec p }{4m^2c^2}\frac{\vec r}{r}\frac{d}{dr}V \right]\phi
\end{align}
Mit einer kleinen Nebenrechnung für \(\sigma^2\vec p\cdot\vec r\)

\begin{align}
  \label{eq:34}
  \sigma^2\vec p\cdot\vec r &= \sum_{i,j}\sigma_i\sigma_j p_i r_j  \quad\text{mit }~(\ref{eq:11})\notag\\
 &= \sum_{i,j}(\frac{1}{2}[\sigma_i,\sigma_j]+\frac{1}{2}\{\sigma_i,\sigma_j\})  p_i r_j \notag\\
 &=  \sum_{i,j} \delta_{ij} p_i r_j + \frac{1}{2} \underbr{\sum_{i,j}[\sigma_i,\sigma_j] p_i r_j}_{2i\vec\sigma\cdot(\vec p\times\vec r)}  \quad\text{vergleiche Nebenrechnung (\ref{eq:19})} \notag \\
&=\vec p\cdot\vec r + i\vec\sigma\cdot(\vec p\times\vec r) \notag \\
&=\vec p\cdot\vec r - i\vec\sigma\cdot(\vec r\times\vec p) 
\end{align}

Eingesetzt in (\ref{eq:33}) ergibt

\begin{align}
  \label{eq:35}
   i\hbar \pdiff_t \phi &= \left[\frac{\vec p^2}{2m}  +e\Phi - \frac{(\vec p)^4 }{8m^3c^2} -\frac{\hbar }{i4m^2c^2r}(\vec p\cdot\vec r - i\vec\sigma\cdot(\vec r\times\vec p)) \frac{d}{dr}V \right]\phi \notag\\
%
 i\hbar \pdiff_t \phi &= \left[\frac{\vec p^2}{2m}  +e\Phi - \frac{(\vec p)^4 }{8m^3c^2} -\frac{\hbar }{i4m^2c^2}\vec p\cdot\underbr{\frac{\vec r}{r}\frac{d}{dr}V}_{\vec\nabla V} + \frac{1 }{2m^2c^2r}\underbr{\frac{\hbar}{2}\vec\sigma}_{\vec S}\cdot\underbr{ (\vec r\times\vec p)}_{\vec L} \frac{d}{dr}V \right]\phi \notag\\
%
 i\hbar \pdiff_t \phi &= \left[\frac{\vec p^2}{2m}  +e\Phi - \frac{(\vec p)^4 }{8m^3c^2}+ \frac{1 }{2m^2c^2}\vec L\cdot\vec S \frac{1}{r}\frac{d}{dr}V -\frac{\hbar }{i4m^2c^2}\vec p\cdot\vec\nabla V \right]\phi \notag\\
%
 i\hbar \pdiff_t \phi &= \left[\underbr{\frac{\vec p^2}{2m}  +e\Phi}_{H_0}\underbr{ - \frac{(\vec p)^4 }{8m^3c^2}}_{H_r}+\underbr{ \frac{1 }{2m^2c^2}\vec L\cdot\vec S \frac{1}{r}\frac{d}{dr}V}_{H_{LS}} + \underbr{\frac{\hbar^2 }{4m^2c^2}\vec \nabla\cdot(\vec\nabla V)}_{\tilde H_D} \right]\phi 
\end{align}

In der Gleichung (\ref{eq:35}) sehen wir die schon gewohnten Korrekturterme

\begin{itemize}
\item \(H_r\) relativistische Beitrag zur Kinetischer Energie
\item \(H_{LS}\) Spin-Bahn-Kopplung inklusive den korrekten Faktor für die Thomas-Präzession \(\frac{1}{2}\)
\item \(\tilde H_D\) Soll den Darvinterm darstellen. Allerdings er nicht hermitesch, was man durch Anwenden der Produktregel ersieht. 
  \begin{align}
    \label{eq:36}
    \tilde H_D\phi &= \frac{\hbar^2 }{4m^2c^2}\vec \nabla\cdot[(\vec\nabla V)\phi] = \frac{\hbar^2 }{4m^2c^2}\left[(\vec \nabla^2V)\phi + (\vec\nabla V)(\vec\nabla\phi)  \right] \notag \\
&= \frac{\hbar^2 }{4m^2c^2}\left[(\vec \nabla^2V) + \underbrace{(\vec\nabla V)\vec\nabla}_{\text{nicht Selbst adjungierend}}  \right]\phi 
  \end{align}
\end{itemize}

Da der letzte Term \((\vec\nabla V)\vec\nabla\) nicht hermitesch ist gibt es Probleme mit der Wahrscheinlichkeitsdichte. Denn es gilt

\begin{align}
  \label{eq:37}
  \rho = \psi^\dagger\psi = |\phi|^2+|\chi|^2
\end{align}

Mit \(\chi\) aus der Näherung (\ref{eq:7}) und \(\vec A=0\)

\begin{align}
  \label{eq:38}
   \chi \approx \frac{\vec p}{2mc}\phi
\end{align}

in die Gleichung (\ref{eq:37}) eingesetzt

\begin{align}
  \label{eq:39}
  \rho &= |\phi|^2+|\chi|^2  = \phi^\dagger\phi + \chi^\dagger\chi =  \phi^\dagger\phi + \phi^\dagger \frac{\vec p^2}{4m^2c^2}\phi = \phi^\dagger\left(1 + \frac{\vec p^2}{4m^2c^2} \right) \phi\notag\\
&= \phi^\dagger\left(1 + \frac{\vec p^2}{4m^2c^2} + \left(\frac{\vec p^2}{8m^2c^2}\right)^2 -\left(\frac{\vec p^2}{8m^2c^2}\right)^2   \right) \phi\notag\\
&= \phi^\dagger\left( \left(1 + \frac{\vec p^2}{8m^2c^2}\right)^2 -\underbr{\left(\frac{\vec p^2}{8m^2c^2}\right)^2}_{\approx 0 wegen ??}   \right) \phi\notag\\
&\approx \phi^\dagger \left(1 + \frac{\vec p^2}{8m^2c^2}\right)^2 \phi\notag\\
&= \left| \underbr{\left(1 + \frac{\vec p^2}{8m^2c^2}\right) \phi}_{\varphi} \right|^2
\end{align}


Wir führen ein neues Spinor \(\varphi\) ein für den gilt

\begin{align}
  \label{eq:40}
  \varphi = \Omega\phi = \left(1 + \frac{\vec p^2}{8m^2c^2} + \dots \right) \phi \quad\text{mit }\rho = |\varphi|^2
\end{align}
Nun wollen wir den Hamilton-Operator so transformieren damit \(\varphi\) die Lösung darstellt. Dies bezeichnet man als \textbf{Foldy-Wouthuysen} Transformation. Dabei ist unser gesamt Hamilton-Operator

\begin{align}
  \label{eq:42}
  H_\phi = \underbr{\frac{\vec p^2}{2m}  +V}_{H_0}\underbr{ - \frac{(\vec p)^4 }{8m^3c^2}}_{H_r}+\underbr{ \frac{1 }{2m^2c^2}\vec L\cdot\vec S \frac{1}{r}\frac{d}{dr}V}_{H_{LS}} + \underbr{\frac{\hbar^2 }{4m^2c^2}\vec \nabla\cdot(\vec\nabla V)}_{\tilde H_D}
\end{align}

\begin{align}
  \label{eq:41}
  \Omega\cdot|\qquad E_S \phi &= H_\phi \phi \notag\\
\Omega E_S \phi &= \Omega H_\phi \mathds 1 \phi \notag\\
E_S  \underbr{\Omega  \phi}_{\varphi} &= \underbr{\Omega H_\phi \Omega^\dagger}_{H} \underbr{\Omega  \phi}_{\varphi} \notag\\
i\hbar\pdiff_t \varphi &= H\varphi
\end{align}

Um den neuen Hamilton-Operator zu bestimmen, folgt eine kleine Nebenrechnung (?betrachte? \(\Omega\) als eine infinitesimale unitäre Matrix mit \(\Omega=e^{iS}\) und \(S=\frac{\vec p^2}{8m^2c^2}\))

\begin{align}
  \label{eq:43}
  H &= \left(1 + \frac{\vec p^2}{8m^2c^2} \right) H_\phi \left(1 - \frac{\vec p^2}{8m^2c^2} \right) \notag\\
&= \left(H_\phi + \frac{\vec p^2}{8m^2c^2}H_\phi\right) \left(1 - \frac{\vec p^2}{8m^2c^2} \right) \notag\\
&=H_\phi - H_\phi\frac{\vec p^2}{8m^2c^2}+\frac{\vec p^2}{8m^2c^2}H_\phi-\frac{\vec p^2}{8m^2c^2}H_\phi\frac{\vec p^2}{8m^2c^2}  \notag\\
&=H_\phi + \frac{1}{8m^2c^2}[p^2,H_\phi] -\frac{\vec p^2}{8m^2c^2}H_\phi\frac{\vec p^2}{8m^2c^2}  \notag\\
&\approx H_\phi + \frac{1}{8m^2c^2}\frac{\hbar^2}{i^2}[\nabla^2,H_\phi]
\end{align}

Berechnen wir nun den Kommutator

\begin{align}
  \label{eq:44}
  [\nabla^2,H_\phi] &= [\nabla^2,V] = [\nabla\nabla,V] \notag\\
&=\nabla[\nabla,V]+[\nabla,V]\nabla
\end{align}
Nebenrechnung

\begin{align}
  \label{eq:45}
  [\nabla,V]\phi &= \nabla(V\phi) - V\nabla(\phi)\notag\\ 
&= \nabla(V)\phi + \cancel{V\nabla(\phi)}- \cancel{V\nabla(\phi)}\notag\\ 
&=\nabla(V)\phi 
\end{align}
Diese Nebenrechnung eingesetzt in die Kommutator-Beziehung (\ref{eq:44})

\begin{align}
  \label{eq:46}
  [\nabla^2,H_\phi]\phi &= \nabla(\nabla V\phi)+\nabla V \nabla\phi \notag\\
&= \nabla^2 (V)\phi + \nabla V\nabla\phi +\nabla V \nabla\phi \notag\\
&=\nabla^2 (V)\phi + 2\nabla V\nabla\phi
\end{align}
Den Kommutator \([\nabla^2,H_\phi]\) wieder in die Hamilton-Operator Gleichung (\ref{eq:43}) eingesetzt

\begin{align}
  \label{eq:47}
  H &= H_\phi + \frac{1}{8m^2c^2}\frac{\hbar^2}{i^2} (\nabla^2V+2\nabla V \nabla) \notag\\
&= H_\phi - \frac{1}{8m^2c^2}\hbar^2 \nabla^2V  - \frac{1}{4m^2c^2}\hbar^2\nabla V \nabla
\end{align}

Nun setzen wir \(H_\phi\) (\ref{eq:42}) in die Gleichung (\ref{eq:47}) ein

\begin{align}
  \label{eq:48}
  H &=  \frac{\vec p^2}{2m}  +V - \frac{(\vec p)^4 }{8m^3c^2}+ \frac{1 }{2m^2c^2}\vec L\cdot\vec S \frac{1}{r}\frac{d}{dr}V +  \frac{\hbar^2 }{4m^2c^2}\underbr{ \vec \nabla\cdot(\vec\nabla V)}_{(\vec \nabla^2V) + \cancel{(\vec\nabla V)\vec\nabla}} - \frac{1}{8m^2c^2}\hbar^2 \nabla^2V  - \frac{\hbar^2}{4m^2c^2}\cancel{\nabla V \nabla} \notag\\
&=  \frac{\vec p^2}{2m}  +V - \frac{(\vec p)^4 }{8m^3c^2}+ \frac{1 }{2m^2c^2}\vec L\cdot\vec S \frac{1}{r}\frac{d}{dr}V +  \frac{\hbar^2 }{4m^2c^2}\vec \nabla^2V - \frac{\hbar^2}{8m^2c^2} \nabla^2V  \notag\\
&=  \frac{\vec p^2}{2m}  +V - \frac{(\vec p)^4 }{8m^3c^2}+ \frac{1 }{2m^2c^2}\vec L\cdot\vec S \frac{1}{r}\frac{d}{dr}V + \underbr{ \frac{\hbar^2 }{8m^2c^2}\vec \nabla^2V }_{H_D} 
\end{align}

Somit haben wir einen hermiteschen Darwinterm \(H_D\) bekommen, damit ist auch der gesamte  Foldy-Wouthuysen transformierte Hamilton-Operator \(H\) ebenfalls hermitesch.

\begin{align}
  \label{eq:49}
\boxed{   H =  \frac{\vec p^2}{2m}  +V - \frac{(\vec p)^4 }{8m^3c^2}+ \frac{1 }{2m^2c^2}\vec L\cdot\vec S \frac{1}{r}\frac{d}{dr}V + \frac{\hbar^2 }{8m^2c^2}\vec \nabla^2V  }
\end{align}


Um weitere (höhere) relativistische Korrekturen zu erhalten, wendet man die allgemeine  Foldy-Wouthuysen Transformation auf die Dirac-Gleichung an.\\
\\
Zitat Wachter: Die Fouldy-Wouthuysen-Transformation liefert ein systematisches Verfahren zur Diagonalisierung des Diracschen Hamilton-Operators bis zu jeder beliebigen (endlichen) Ordnung in \(\frac{v}{c}\).\\


 

\subsection*{Referenzen}
\begin{itemize}
\item Schwabl
\item Wachter
\end{itemize}

\end{document}
