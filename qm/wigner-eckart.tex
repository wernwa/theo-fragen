\documentclass[10pt,a4paper,oneside,fleqn]{article}
\usepackage{geometry}
\geometry{a4paper,left=20mm,right=20mm,top=1cm,bottom=2cm}
\usepackage[utf8]{inputenc}
%\usepackage{ngerman}
\usepackage{amsmath}                % brauche ich um dir Formel zu umrahmen.
\usepackage{amsfonts}                % brauche ich für die Mengensymbole
\usepackage{graphicx}
\setlength{\parindent}{0px}
\setlength{\mathindent}{10mm}
\usepackage{bbold}                    %brauche ich für die doppel Zahlen Darstellung (Einheitsmatrix z.B)



\usepackage{color}
\usepackage{titlesec} %sudo apt-get install texlive-latex-extra

\definecolor{darkblue}{rgb}{0.1,0.1,0.55}
\definecolor{verydarkblue}{rgb}{0.1,0.1,0.35}
\definecolor{darkred}{rgb}{0.55,0.2,0.2}

%hyperref Link color
\usepackage[colorlinks=true,
        linkcolor=darkblue,
        citecolor=darkblue,
        filecolor=darkblue,
        pagecolor=darkblue,
        urlcolor=darkblue,
        bookmarks=true,
        bookmarksopen=true,
        bookmarksopenlevel=3,
        plainpages=false,
        pdfpagelabels=true]{hyperref}

\titleformat{\chapter}[display]{\color{darkred}\normalfont\huge\bfseries}{\chaptertitlename\
\thechapter}{20pt}{\Huge}

\titleformat{\section}{\color{darkblue}\normalfont\Large\bfseries}{\thesection}{1em}{}
\titleformat{\subsection}{\color{verydarkblue}\normalfont\large\bfseries}{\thesubsection}{1em}{}

% Notiz Box
\usepackage{fancybox}
\newcommand{\notiz}[1]{\vspace{5mm}\ovalbox{\begin{minipage}{1\textwidth}#1\end{minipage}}\vspace{5mm}}

\usepackage{cancel}
\setcounter{secnumdepth}{3}
\setcounter{tocdepth}{3}





%-------------------------------------------------------------------------------
%Diff-Makro:
%Das Diff-Makro stellt einen Differentialoperator da.
%
%Benutzung:
% \diff  ->  d
% \diff f  ->  df
% \diff^2 f  ->  d^2 f
% \diff_x  ->  d/dx
% \diff^2_x  ->  d^2/dx^2
% \diff f_x  ->  df/dx
% \diff^2 f_x  ->  d^2f/dx^2
% \diff^2{f(x^5)}_x  ->  d^2(f(x^5))/dx^2
%
%Ersetzt man \diff durch \pdiff, so wird der partieller
%Differentialoperator dargestellt.
%
\makeatletter
\def\diff@n^#1{\@ifnextchar{_}{\diff@n@d^#1}{\diff@n@fun^#1}}
\def\diff@n@d^#1_#2{\frac{\textrm{d}^#1}{\textrm{d}#2^#1}}
\def\diff@n@fun^#1#2{\@ifnextchar{_}{\diff@n@fun@d^#1#2}{\textrm{d}^#1#2}}
\def\diff@n@fun@d^#1#2_#3{\frac{\textrm{d}^#1 #2}{\textrm{d}#3^#1}}
\def\diff@one@d_#1{\frac{\textrm{d}}{\textrm{d}#1}}
\def\diff@one@fun#1{\@ifnextchar{_}{\diff@one@fun@d #1}{\textrm{d}#1}}
\def\diff@one@fun@d#1_#2{\frac{\textrm{d}#1}{\textrm{d}#2}}
\newcommand*{\diff}{\@ifnextchar{^}{\diff@n}
  {\@ifnextchar{_}{\diff@one@d}{\diff@one@fun}}}
%
%Partieller Diff-Operator.
\def\pdiff@n^#1{\@ifnextchar{_}{\pdiff@n@d^#1}{\pdiff@n@fun^#1}}
\def\pdiff@n@d^#1_#2{\frac{\partial^#1}{\partial#2^#1}}
\def\pdiff@n@fun^#1#2{\@ifnextchar{_}{\pdiff@n@fun@d^#1#2}{\partial^#1#2}}
\def\pdiff@n@fun@d^#1#2_#3{\frac{\partial^#1 #2}{\partial#3^#1}}
\def\pdiff@one@d_#1{\frac{\partial}{\partial #1}}
\def\pdiff@one@fun#1{\@ifnextchar{_}{\pdiff@one@fun@d #1}{\partial#1}}
\def\pdiff@one@fun@d#1_#2{\frac{\partial#1}{\partial#2}}
\newcommand*{\pdiff}{\@ifnextchar{^}{\pdiff@n}
  {\@ifnextchar{_}{\pdiff@one@d}{\pdiff@one@fun}}}
\makeatother
%
%Das gleich nur mit etwas andere Syntax. Die Potenz der Differentiation wird erst
%zum Schluss angegeben. Somit lautet die Syntax:
%
% \diff_x^2  ->  d^2/dx^2
% \diff f_x^2  ->  d^2f/dx^2
% \diff{f(x^5)}_x^2  ->  d^2(f(x^5))/dx^2
% Ansonsten wie Oben.
%
%Ersetzt man \diff durch \pdiff, so wird der partieller
%Differentialoperator dargestellt.
%
%\makeatletter
%\def\diff@#1{\@ifnextchar{_}{\diff@fun#1}{\textrm{d} #1}}
%\def\diff@one_#1{\@ifnextchar{^}{\diff@n{#1}}%
%  {\frac{\textrm d}{\textrm{d} #1}}}
%\def\diff@fun#1_#2{\@ifnextchar{^}{\diff@fun@n#1_#2}%
%  {\frac{\textrm d #1}{\textrm{d} #2}}}
%\def\diff@n#1^#2{\frac{\textrm d^#2}{\textrm{d}#1^#2}}
%\def\diff@fun@n#1_#2^#3{\frac{\textrm d^#3 #1}%
%  {\textrm{d}#2^#3}}
%\def\diff{\@ifnextchar{_}{\diff@one}{\diff@}}
%\newcommand*{\diff}{\@ifnextchar{_}{\diff@one}{\diff@}}
%
%Partieller Diff-Operator.
%\def\pdiff@#1{\@ifnextchar{_}{\pdiff@fun#1}{\partial #1}}
%\def\pdiff@one_#1{\@ifnextchar{^}{\pdiff@n{#1}}%
%  {\frac{\partial}{\partial #1}}}
%\def\pdiff@fun#1_#2{\@ifnextchar{^}{\pdiff@fun@n#1_#2}%
%  {\frac{\partial #1}{\partial #2}}}
%\def\pdiff@n#1^#2{\frac{\partial^#2}{\partial #1^#2}}
%\def\pdiff@fun@n#1_#2^#3{\frac{\partial^#3 #1}%
%  {\partial #2^#3}}
%\newcommand*{\pdiff}{\@ifnextchar{_}{\pdiff@one}{\pdiff@}}
%\makeatother

%-------------------------------------------------------------------------------
%%Nützliche Makros um in der Quantenmechanik Bras, Kets und das Skalarprodukt
%%zwischen den beiden darzustellen.
%%Benutzung:
%% \bra{x}  ->    < x |
%% \ket{x}  ->    | x >
%% \braket{x}{y} ->   < x | y >

\newcommand\bra[1]{\left\langle #1 \right|}
\newcommand\ket[1]{\left| #1 \right\rangle}
\newcommand\braket[2]{%
  \left\langle #1\vphantom{#2} \right.%
  \left|\vphantom{#1#2}\right.%
  \left. \vphantom{#1}#2 \right\rangle}%

%-------------------------------------------------------------------------------
%%Aus dem Buch:
%%Titel:  Latex in Naturwissenschaften und Mathematik
%%Autor:  Herbert Voß
%%Verlag: Franzis Verlag, 2006
%%ISBN:   3772374190, 9783772374197
%%
%%Hier werden drei Makros definiert:\mathllap, \mathclap und \mathrlap, welche
%%analog zu den aus Latex bekannten \rlap und \llap arbeiten, d.h. selbst
%%keinerlei horizontalen Platz benötigen, aber dennoch zentriert zum aktuellen
%%Punkt erscheinen.

\newcommand*\mathllap{\mathstrut\mathpalette\mathllapinternal}
\newcommand*\mathllapinternal[2]{\llap{$\mathsurround=0pt#1{#2}$}}
\newcommand*\clap[1]{\hbox to 0pt{\hss#1\hss}}
\newcommand*\mathclap{\mathpalette\mathclapinternal}
\newcommand*\mathclapinternal[2]{\clap{$\mathsurround=0pt#1{#2}$}}
\newcommand*\mathrlap{\mathpalette\mathrlapinternal}
\newcommand*\mathrlapinternal[2]{\rlap{$\mathsurround=0pt#1{#2}$}}

%%Das Gleiche nur mit \def statt \newcommand.
%\def\mathllap{\mathpalette\mathllapinternal}
%\def\mathllapinternal#1#2{%
%  \llap{$\mathsurround=0pt#1{#2}$}% $
%}
%\def\clap#1{\hbox to 0pt{\hss#1\hss}}
%\def\mathclap{\mathpalette\mathclapinternal}
%\def\mathclapinternal#1#2{%
%  \clap{$\mathsurround=0pt#1{#2}$}%
%}
%\def\mathrlap{\mathpalette\mathrlapinternal}
%\def\mathrlapinternal#1#2{%
%  \rlap{$\mathsurround=0pt#1{#2}$}% $
%}

%-------------------------------------------------------------------------------
%%Hier werden zwei neue Makros definiert \overbr und \underbr welche analog zu
%%\overbrace und \underbrace funktionieren jedoch die Gleichung nicht
%%'zerreißen'. Dies wird ermöglicht durch das \mathclap Makro.

\def\overbr#1^#2{\overbrace{#1}^{\mathclap{#2}}}
\def\underbr#1_#2{\underbrace{#1}_{\mathclap{#2}}}
\usepackage{amsmath} 



\begin{document}

\textit{29. März 2012}
\input{../headers/authors.tex}

\section*{Wigner-Eckart-Theorem}


Betrachte ein \(T^{(k)}_q\) einen irreduziebler Tensor \(k-\)ter Stufe. Dieser verhält sich wie ein Zustandsvektor bei einer Drehung. Dieser Tensor ist proportional zu einem Ket \(T^{(k)}_q \sim \ket{k,q} \) (Beweis siehe \ref{sec:1}). D.h wir können folgende Linearkombination

\begin{equation}
  \label{eq:1}
  \ket{JM} = \sum_{m_1,m_2}\braket{j_1j_2;m_1m_2}{jm}\ket{j_1j_2;m_1m_2} = \sum_{m_1,m_2}\braket{j_1j_2;m_1m_2}{jm}\ket{j_1m_1} \otimes \ket{j_2m_2}
\end{equation}

dem Tensor  mit Hilfe der Clebsch-Gordan-Koeffizienten als Ket-Vektor ausdrücken

\begin{equation}
  \label{eq:2}
  \ket{JM} = \sum_{m,q} \braket{jk;mq}{JM} \ket{k,q}\otimes\ket{jm} =  \sum_{m,q} \braket{jk;mq}{JM} T^{(k)}_q  \ket{jm} 
\end{equation}


Da es noch andere Quantenzahlen vorkommen können wie Enegie multiplizieren wir die Gleichung (\ref{eq:2}) mit einem Ket \(\ket{\alpha}\) der symbolisch für andere Quantenzahlen steht.

\begin{align}
  \label{eq:3}
  \ket{\alpha}\cdot\ket{JM} &=  \ket{\alpha}\cdot\sum_{m,q} \braket{jk;mq}{JM} \ket{k,q}\otimes\ket{jm}\notag \\
\Leftrightarrow \ket{\alpha; JM} &=  \sum_{m,q} \braket{jk;mq}{JM} T^{(k)}_q  \ket{\alpha;jm} 
\end{align}

Wir möchten die Summe und  Clebsch-Gordan-Koeffizienten auf die andere Seite bringen. Dazu möchten wir folgende Relation herleiten:

\begin{align*}
  \label{eq:4}
 \braket{JM}{JM} &= \bra{JM} \mathbb 1 \ket{JM} \\
&= \bra{JM} \left( \sum_{mq}\ket{jk;mq}\bra{jk;mq}  \right) \ket{JM} \\
&=  \sum_{mq}\braket{JM}{jk;mq}\braket{jk;mq}{JM} \\
&=  \sum_{mq}|\braket{jk;mq}{JM} |^2\\
&\stackrel{!}= 1
\end{align*}

Die Relation lautet nun:

\begin{equation}
  \label{eq:5}
  \sum_{mq}|\braket{jk;mq}{JM} |^2 = 1
\end{equation}

Unter Ausnutzung der Relation (\ref{eq:5}) folgt für die Gleichung (\ref{eq:3}):

\begin{align*}
  \label{eq:6}
   \mathbb 1 \ket{\alpha; JM} &=  \sum_{m,q} \braket{jk;mq}{JM} T^{(k)}_q  \ket{\alpha;jm} \\
 \sum_{JM} \ket{\alpha; JM}\underbr{\braket{\alpha; JM}{\alpha; JM}}_{=1=\sum_{mq}|\braket{jk;mq}{JM} |^2} &=  \sum_{m,q} \braket{jk;mq}{JM} T^{(k)}_q  \ket{\alpha;jm} \\
\sum_{JM}\ket{\alpha; JM} \cancel{\sum_{mq}}|\braket{jk;mq}{JM} |^{\cancel 2}   &= \cancel{\sum_{m,q} \braket{jk;mq}{JM}} T^{(k)}_q  \ket{\alpha;jm} 
\end{align*}

Damit erhalten wir:

\begin{equation}
  \label{eq:7}
   T^{(k)}_q  \ket{\alpha;jm} = \sum_{JM}\ket{\alpha; JM} \braket{jk;mq}{JM}
\end{equation}

Multiplizieren wir mit \(\bra{\alpha;j'm'}\)

\begin{equation}
  \label{eq:8}
  \bra{\alpha;j'm'} T^{(k)}_q  \ket{\alpha;jm} = \sum_{JM} \underbr{\braket{\alpha;j'm'}{\alpha; JM}}_{\delta_{j'J}\delta_{m'M}} \braket{jk;mq}{JM}
\end{equation}

Wegen der Orthogonalitätsbegingung bleibt von der Summe nur ein Summand übrig, bei dem gilt \(j'=J\) und \(m'=M\). Gleichung (\ref{eq:8}) können wir nun schreiben:

\begin{equation}
  \label{eq:9}
   \bra{\alpha;j'm'} T^{(k)}_q  \ket{\alpha;jm} = \underbr{\braket{\alpha;j'm'}{\alpha; j'm'}}_{\text{reduziertes Matrixelement}}  \braket{jk;mq}{j'm'}
\end{equation}

Dabei ist das reduzierte Matrixelement unabhängig von der Quantenzahl \(m\). Das wird im Anschluss \ref{sec:2} bewiesen. Wir erhalten das Wigner-Eckart-Theorem

\begin{equation}
  \label{eq:15}
\boxed{\bra{\alpha;j'm'} T^{(k)}_q  \ket{\alpha;jm} =  \braket{jk;mq}{j'm'} \langle \alpha;j || T^{(q)}_k || \alpha; j' \rangle }
\end{equation}


Das reduzierte Matrixelement hat den Vorteil, dass man es für ein gegebene \(\alpha,j\) nur einmal berechnen muss. Die restlichen Matrixelemente des Tensors die von \(m\) abhängig sind bekommt man von den Clebsch-Gordan-Koeffizienten, die man nachschlägt oder ebenso berechnen kann. Man findet in der Literatur das Theorem mit unterschiedlichen Vorfaktoren wie beispielsweise \(\frac{1}{\sqrt{2j+1}}\), dieser ist aber Konvenstion (Tiefgründige Bedeutung ist uns nicht bekannt).


\subsection*{Behandlung des Tensors als Zustandsvektor}
\label{sec:1}

Wir wollen zeigen dass sich ein irreduziebler Tensor \(k-\)ter Stufe wie ein Zustandsvektor unter Rotation transformiert. Für die Transformation eines Zustandsvektors \(\ket{JM}\) gilt mit \( U(\alpha, \beta,\gamma) = e^{-\frac{i}{\hbar}\alpha J_z}  e^{-\frac{i}{\hbar}\beta J_y} e^{-\frac{i}{\hbar}\gamma J_z}\)

\begin{align} 
\label{eq:13}
U(R)|J,M\rangle &= \mathbb 1 \cdot U(R)|J,M\rangle \qquad U(R)\equiv U(\alpha, \beta,\gamma)\notag \\
&= \sum_{M'=-J}^J |J,M'\rangle\langle J,M'| U(R)|J,M\rangle \notag \\
&= \sum_{M'=-J}^J \underbr{ \langle J,M'|  U(R)|J,M\rangle}_{D^{(J)}_{M'M}(R)} |J,M'\rangle \\
&= \sum_{M'=-J}^J D^{(J)}_{M'M}(R) |J,M'\rangle \notag
\end{align}

Anderrerseits gilt

\begin{equation}
  \label{eq:21}
  U(R)|J,M\rangle = \sum_{m_1 m_2} \braket{j_1j_2;m_1m_2}{JM}U\ket{j_1m_1}\ket{j_2m_2}
\end{equation}

Setzen wir die Gleichungen (\ref{eq:13})  und (\ref{eq:21}) gleich


\begin{align}
  \label{eq:22}
  &\sum_{M'=-J}^J \langle J,M'|  U|J,M\rangle |J,M'\rangle = \sum_{m_1 m_2} \braket{j_1j_2;m_1m_2}{JM}U\ket{j_1m_1}\ket{j_2m_2}\notag \\
&\sum_{M'=-J}^J \sum_{m'_1m'_2}\cancel{\sum_{m_1m_2}}\braket{JM'}{m'_2m'_1;j_2j_1}\bra{j_1j_2;m'_1m'_2}  U(R) \cancel{\braket{j_1j_2;m_1m_2}{JM}}\ket{j_1j_2;m_1m_2}  |J,M'\rangle\notag \\
& \qquad = \cancel{\sum_{m_1 m_2}}\cancel{\braket{j_1j_2;m_1m_2}{JM}}U\ket{j_1m_1}\ket{j_2m_2}\notag \\
&\sum_{M'=-J}^J \sum_{m'_1m'_2}\braket{JM'}{m'_2m'_1;j_2j_1} \underbr{ \bra{j_1j_2;m'_1m'_2}  U \ket{j_1j_2;m_1m_2} }_{  \bra{j_1m'_1}  U(R_1) \ket{j_1m_1} \cdot  \bra{j_2m'_2}  U(R_2) \ket{j_2m_2}  } |J,M'\rangle  = U\ket{j_1m_1}\ket{j_2m_2} \notag\\
&\sum_{M'=-J}^J \sum_{m'_1m'_2}\braket{JM'}{m'_2m'_1;j_2j_1} \underbr{ \bra{j_1m'_1}  U(R_1) \ket{j_1m_1}}_{D^{(j_1)}_{m'_1m_1}} \cdot \underbr{\bra{j_2m'_2}  U(R_2) \ket{j_2m_2} }_{D^{(j_2)}_{m'_2m_2} }  |J,M'\rangle  = U\ket{j_1m_1}\ket{j_2m_2}\notag \\
&\sum_{M'=-J}^J \sum_{m'_1m'_2}\braket{JM'}{m'_2m'_1;j_2j_1} D^{(j_1)}_{m'_1m_1} \cdot D^{(j_2)}_{m'_2m_2}   |J,M'\rangle  = U\ket{j_1m_1}\ket{j_2m_2}\notag \\
&\sum_{M'=-J}^J \sum_{m'_1m'_2}\braket{JM'}{m'_2m'_1;j_2j_1} D^{(j_1)}_{m'_1m_1} \cdot D^{(j_2)}_{m'_2m_2}  \sum_{m'_1,m'_2}\braket{j_1j_2;m'_1m'_2}{JM'}\ket{j_1j_2;m'_1m'_2}  = U\ket{j_1m_1}\ket{j_2m_2}\notag \\
&\sum_{M'=-J}^J \underbr{\sum_{m'_1m'_2} \sum_{m'_1,m'_2} \braket{JM'}{m'_2m'_1;j_2j_1}\braket{j_1j_2;m'_1m'_2}{JM'}}_{=1 \quad(\ref{eq:5}) \text{ eine }\sum_{m'_1m'_2}\text{ verschwindet} } D^{(j_1)}_{m'_1m_1} \cdot D^{(j_2)}_{m'_2m_2} \ket{j_1j_2;m'_1m'_2}  = U\ket{j_1m_1}\ket{j_2m_2}\notag \\
&\sum_{M'=-J}^J  D^{(j_1)}_{m'_1m_1} \cdot D^{(j_2)}_{m'_2m_2} \ket{j_1j_2;m'_1m'_2}  = U\ket{j_1m_1}\ket{j_2m_2}
\end{align}


Daraus folgt

\begin{align}
  \label{eq:23}
   U\ket{j_1m_1}\ket{j_2m_2} &= \sum_{M'=-J}^J  D^{(j_1)}_{m'_1m_1} \cdot D^{(j_2)}_{m'_2m_2} \ket{j_1m'_1}\ket{j_2m'_2} \\
&= \sum_{m'_1m_1}  D^{(j_1)}_{m'_1m_1} \cdot D^{(j_2)}_{m'_2m_2} \ket{j_1m'_1}\ket{j_2m'_2} \\
\end{align}



Für die Rotation eines Tensors gilt
\begin{equation}
  \label{eq:16}
  U^{-1}(R) T^{(k)}_q U(R) = \sum_{q'=-k}^{k}T^{(k)}_{q'} D^{(k)}_{qq'}(R^{-1}) 
\end{equation}

Ersetze \(R\) mit \(R^{-1}\) und \(R^{-1}\) mit \(R\)
\begin{equation}
  \label{eq:17}
  U^{-1}(R^{-1}) T^{(k)}_q U(R^{-1}) = \sum_{q'=-k}^{k}T^{(k)}_{q'} D^{(k)}_{qq'}(R)
\end{equation}

Es gilt \(U(R^{-1})=U^{-1}(R)\) somit erhalten wir

\begin{align}
  \label{eq:18}
  (U^{-1})^{-1}(R) T^{(k)}_q U^{-1}(R) &= \sum_{q'=-k}^{k}T^{(k)}_{q'} D^{(k)}_{qq'}(R) \notag\\
U(R) T^{(k)}_q U^{-1}(R) &= \sum_{q'=-k}^{k}T^{(k)}_{q'} D^{(k)}_{qq'}(R)
\end{align}

Nun wenden wir den Unitären Operator \(U(R)\) auf ein Produkt-Zustand aus \(T^{(k)}_q\) und \(\ket{jm}\). Verkürze \(U(R)\) auf \(U\)

\begin{align}
  \label{eq:19}
  U T^{(k)}_q\ket{jm} &= U T^{(k)}_q\mathbb 1 \ket{jm} \notag\\
&=\underbr{ U T^{(k)}_q U^{-1}}_{=~(\ref{eq:18})} \underbr{ U \ket{jm}}_{~(\ref{eq:13})}\notag \\
&= \sum_{q'=-k}^{k}T^{(k)}_{q'} D^{(k)}_{qq'} \sum_{m'=-j}^j D^{(j)}_{m'm} |j,m'\rangle \notag \\
&= \sum_{q'=-k}^{k}\sum_{m'=-j}^j D^{(k)}_{qq'}  D^{(j)}_{m'm}\,\, T^{(k)}_{q'} |j,m'\rangle  
\end{align}

Im Vergleich zur Gleichung (\ref{eq:23}) Transformiert sich der Produkt-Zustand \( T^{(k)}_q\ket{jm}\) wie ein Produkt aus zwei Zustandsvektoren \(\ket{kq}\ket{jm}\). 



\subsection*{Reduziertes Matrixelement unabhängig von m}
\label{sec:2}

Als nächstes wollen wir beweisen, dass das reduzierte Matrixelement von der Quantenzahl \(m\) unabhängig ist. Dies lässt sich duch Anwenden des Schiebeoperators \(J_{\pm}\) zeigen. Zur Errinerung die Eigenwertgleichung lautet

\begin{equation}
  \label{eq:11}
  J_{\pm}\ket{\alpha; jm} = \sqrt{j(j+1)-m(m\pm 1)} \ket{\alpha; jm\pm 1}
\end{equation}

Durch einsetzen von \(J_{\pm}\) und durch ausgleichen von einem Vorfaktor lässt sich das reduzierte Matrixelement schreiben


\begin{align}
  \label{eq:10}
  \braket{\alpha;jm}{\alpha; jm} = \frac{1}{\sqrt{j(j+1)-m(m\mp 1)}}  \bra{\alpha;jm}J_{\pm} \ket{\alpha; jm\mp 1}
\end{align}

Lässt man nun \(J_{\pm}\) einmal auf links wirken, dabei wird \((J_{\pm})^\dagger = J_{\mp}\)

\begin{align}
  \label{eq:12}
  \braket{\alpha;jm}{\alpha; jm} &= \frac{\sqrt{j(j+1)-m(m\mp 1)}}{\sqrt{j(j+1)-m(m\mp 1)}}  \braket{\alpha;jm\mp 1 }{\alpha; jm\mp 1} \notag \\
&= \braket{\alpha;jm\mp 1 }{\alpha; jm\mp 1}
\end{align}

Aus der Gleichung (\ref{eq:12}) sieht man dass das reduzierte Matrixelement nicht von \(m\) abhängig ist. Wir können Die Gleichung (\ref{eq:9}) schreiben

\begin{equation}
  \label{eq:14}
   \bra{\alpha;jm} T^{(k)}_q  \ket{\alpha;jm} =  \underbr{\braket{\alpha;j}{\alpha; j}}_{\text{reduziertes Matrixelement}}   \braket{jk;mq}{jm}
\end{equation}






\subsection*{Referenzen}
\begin{itemize}
%\item Claude Cohen-Tannoudji Quantenmechanik Band 2
\item Zettili Quanten Mehanics
\item Rollnik Quantentheorie 2
\end{itemize}

\end{document}
