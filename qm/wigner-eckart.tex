\documentclass[10pt,a4paper,oneside,fleqn]{article}
\usepackage{geometry}
\geometry{a4paper,left=20mm,right=20mm,top=1cm,bottom=2cm}
\usepackage[utf8]{inputenc}
%\usepackage{ngerman}
\usepackage{amsmath}                % brauche ich um dir Formel zu umrahmen.
\usepackage{amsfonts}                % brauche ich für die Mengensymbole
\usepackage{graphicx}
\setlength{\parindent}{0px}
\setlength{\mathindent}{10mm}
\usepackage{bbold}                    %brauche ich für die doppel Zahlen Darstellung (Einheitsmatrix z.B)



\usepackage{color}
\usepackage{titlesec} %sudo apt-get install texlive-latex-extra

\definecolor{darkblue}{rgb}{0.1,0.1,0.55}
\definecolor{verydarkblue}{rgb}{0.1,0.1,0.35}
\definecolor{darkred}{rgb}{0.55,0.2,0.2}

%hyperref Link color
\usepackage[colorlinks=true,
        linkcolor=darkblue,
        citecolor=darkblue,
        filecolor=darkblue,
        pagecolor=darkblue,
        urlcolor=darkblue,
        bookmarks=true,
        bookmarksopen=true,
        bookmarksopenlevel=3,
        plainpages=false,
        pdfpagelabels=true]{hyperref}

\titleformat{\chapter}[display]{\color{darkred}\normalfont\huge\bfseries}{\chaptertitlename\
\thechapter}{20pt}{\Huge}

\titleformat{\section}{\color{darkblue}\normalfont\Large\bfseries}{\thesection}{1em}{}
\titleformat{\subsection}{\color{verydarkblue}\normalfont\large\bfseries}{\thesubsection}{1em}{}

% Notiz Box
\usepackage{fancybox}
\newcommand{\notiz}[1]{\vspace{5mm}\ovalbox{\begin{minipage}{1\textwidth}#1\end{minipage}}\vspace{5mm}}

\usepackage{cancel}
\setcounter{secnumdepth}{3}
\setcounter{tocdepth}{3}





%-------------------------------------------------------------------------------
%Diff-Makro:
%Das Diff-Makro stellt einen Differentialoperator da.
%
%Benutzung:
% \diff  ->  d
% \diff f  ->  df
% \diff^2 f  ->  d^2 f
% \diff_x  ->  d/dx
% \diff^2_x  ->  d^2/dx^2
% \diff f_x  ->  df/dx
% \diff^2 f_x  ->  d^2f/dx^2
% \diff^2{f(x^5)}_x  ->  d^2(f(x^5))/dx^2
%
%Ersetzt man \diff durch \pdiff, so wird der partieller
%Differentialoperator dargestellt.
%
\makeatletter
\def\diff@n^#1{\@ifnextchar{_}{\diff@n@d^#1}{\diff@n@fun^#1}}
\def\diff@n@d^#1_#2{\frac{\textrm{d}^#1}{\textrm{d}#2^#1}}
\def\diff@n@fun^#1#2{\@ifnextchar{_}{\diff@n@fun@d^#1#2}{\textrm{d}^#1#2}}
\def\diff@n@fun@d^#1#2_#3{\frac{\textrm{d}^#1 #2}{\textrm{d}#3^#1}}
\def\diff@one@d_#1{\frac{\textrm{d}}{\textrm{d}#1}}
\def\diff@one@fun#1{\@ifnextchar{_}{\diff@one@fun@d #1}{\textrm{d}#1}}
\def\diff@one@fun@d#1_#2{\frac{\textrm{d}#1}{\textrm{d}#2}}
\newcommand*{\diff}{\@ifnextchar{^}{\diff@n}
  {\@ifnextchar{_}{\diff@one@d}{\diff@one@fun}}}
%
%Partieller Diff-Operator.
\def\pdiff@n^#1{\@ifnextchar{_}{\pdiff@n@d^#1}{\pdiff@n@fun^#1}}
\def\pdiff@n@d^#1_#2{\frac{\partial^#1}{\partial#2^#1}}
\def\pdiff@n@fun^#1#2{\@ifnextchar{_}{\pdiff@n@fun@d^#1#2}{\partial^#1#2}}
\def\pdiff@n@fun@d^#1#2_#3{\frac{\partial^#1 #2}{\partial#3^#1}}
\def\pdiff@one@d_#1{\frac{\partial}{\partial #1}}
\def\pdiff@one@fun#1{\@ifnextchar{_}{\pdiff@one@fun@d #1}{\partial#1}}
\def\pdiff@one@fun@d#1_#2{\frac{\partial#1}{\partial#2}}
\newcommand*{\pdiff}{\@ifnextchar{^}{\pdiff@n}
  {\@ifnextchar{_}{\pdiff@one@d}{\pdiff@one@fun}}}
\makeatother
%
%Das gleich nur mit etwas andere Syntax. Die Potenz der Differentiation wird erst
%zum Schluss angegeben. Somit lautet die Syntax:
%
% \diff_x^2  ->  d^2/dx^2
% \diff f_x^2  ->  d^2f/dx^2
% \diff{f(x^5)}_x^2  ->  d^2(f(x^5))/dx^2
% Ansonsten wie Oben.
%
%Ersetzt man \diff durch \pdiff, so wird der partieller
%Differentialoperator dargestellt.
%
%\makeatletter
%\def\diff@#1{\@ifnextchar{_}{\diff@fun#1}{\textrm{d} #1}}
%\def\diff@one_#1{\@ifnextchar{^}{\diff@n{#1}}%
%  {\frac{\textrm d}{\textrm{d} #1}}}
%\def\diff@fun#1_#2{\@ifnextchar{^}{\diff@fun@n#1_#2}%
%  {\frac{\textrm d #1}{\textrm{d} #2}}}
%\def\diff@n#1^#2{\frac{\textrm d^#2}{\textrm{d}#1^#2}}
%\def\diff@fun@n#1_#2^#3{\frac{\textrm d^#3 #1}%
%  {\textrm{d}#2^#3}}
%\def\diff{\@ifnextchar{_}{\diff@one}{\diff@}}
%\newcommand*{\diff}{\@ifnextchar{_}{\diff@one}{\diff@}}
%
%Partieller Diff-Operator.
%\def\pdiff@#1{\@ifnextchar{_}{\pdiff@fun#1}{\partial #1}}
%\def\pdiff@one_#1{\@ifnextchar{^}{\pdiff@n{#1}}%
%  {\frac{\partial}{\partial #1}}}
%\def\pdiff@fun#1_#2{\@ifnextchar{^}{\pdiff@fun@n#1_#2}%
%  {\frac{\partial #1}{\partial #2}}}
%\def\pdiff@n#1^#2{\frac{\partial^#2}{\partial #1^#2}}
%\def\pdiff@fun@n#1_#2^#3{\frac{\partial^#3 #1}%
%  {\partial #2^#3}}
%\newcommand*{\pdiff}{\@ifnextchar{_}{\pdiff@one}{\pdiff@}}
%\makeatother

%-------------------------------------------------------------------------------
%%Nützliche Makros um in der Quantenmechanik Bras, Kets und das Skalarprodukt
%%zwischen den beiden darzustellen.
%%Benutzung:
%% \bra{x}  ->    < x |
%% \ket{x}  ->    | x >
%% \braket{x}{y} ->   < x | y >

\newcommand\bra[1]{\left\langle #1 \right|}
\newcommand\ket[1]{\left| #1 \right\rangle}
\newcommand\braket[2]{%
  \left\langle #1\vphantom{#2} \right.%
  \left|\vphantom{#1#2}\right.%
  \left. \vphantom{#1}#2 \right\rangle}%

%-------------------------------------------------------------------------------
%%Aus dem Buch:
%%Titel:  Latex in Naturwissenschaften und Mathematik
%%Autor:  Herbert Voß
%%Verlag: Franzis Verlag, 2006
%%ISBN:   3772374190, 9783772374197
%%
%%Hier werden drei Makros definiert:\mathllap, \mathclap und \mathrlap, welche
%%analog zu den aus Latex bekannten \rlap und \llap arbeiten, d.h. selbst
%%keinerlei horizontalen Platz benötigen, aber dennoch zentriert zum aktuellen
%%Punkt erscheinen.

\newcommand*\mathllap{\mathstrut\mathpalette\mathllapinternal}
\newcommand*\mathllapinternal[2]{\llap{$\mathsurround=0pt#1{#2}$}}
\newcommand*\clap[1]{\hbox to 0pt{\hss#1\hss}}
\newcommand*\mathclap{\mathpalette\mathclapinternal}
\newcommand*\mathclapinternal[2]{\clap{$\mathsurround=0pt#1{#2}$}}
\newcommand*\mathrlap{\mathpalette\mathrlapinternal}
\newcommand*\mathrlapinternal[2]{\rlap{$\mathsurround=0pt#1{#2}$}}

%%Das Gleiche nur mit \def statt \newcommand.
%\def\mathllap{\mathpalette\mathllapinternal}
%\def\mathllapinternal#1#2{%
%  \llap{$\mathsurround=0pt#1{#2}$}% $
%}
%\def\clap#1{\hbox to 0pt{\hss#1\hss}}
%\def\mathclap{\mathpalette\mathclapinternal}
%\def\mathclapinternal#1#2{%
%  \clap{$\mathsurround=0pt#1{#2}$}%
%}
%\def\mathrlap{\mathpalette\mathrlapinternal}
%\def\mathrlapinternal#1#2{%
%  \rlap{$\mathsurround=0pt#1{#2}$}% $
%}

%-------------------------------------------------------------------------------
%%Hier werden zwei neue Makros definiert \overbr und \underbr welche analog zu
%%\overbrace und \underbrace funktionieren jedoch die Gleichung nicht
%%'zerreißen'. Dies wird ermöglicht durch das \mathclap Makro.

\def\overbr#1^#2{\overbrace{#1}^{\mathclap{#2}}}
\def\underbr#1_#2{\underbrace{#1}_{\mathclap{#2}}}
\usepackage{amsmath} 



\begin{document}

\section*{Wigner-Eckart-Theorem}


Betrachte ein \(T^{(k)}_q\) oder irreduziebler Tensor \(k-\)ter Stufe. Dieser verhält sich wie ein Zustandsvektor bei einer Drehung. D.h. dieser Tensor ist proportional zu einem Ket \(T^{(k)}_q \sim \ket{k,q} \). D.h wir können folgende Linearkombination:



\begin{equation}
  \label{eq:1}
  \ket{JM} = \sum_{m_1,m_2}\braket{j_1j_2;m_1m_2}{jm}\ket{j_1j_2;m_1m_2} = \sum_{m_1,m_2}\braket{j_1j_2;m_1m_2}{jm}\ket{j_1m_1} \otimes \ket{j_2m_2}
\end{equation}

mit dem Tensor  mit Hilfe der Clebsch-Gordan-Koeffizienten als Ket-Vektor ausdrücken:



\begin{equation}
  \label{eq:2}
  \ket{JM} = \sum_{m,q} \braket{jk;mq}{JM} \ket{k,q}\otimes\ket{jm} =  \sum_{m,q} \braket{jk;mq}{JM} T^{(k)}_q  \ket{jm} 
\end{equation}


Da es noch andere Quantenzahlen vorkommen können wie Enegie multiplizieren wir die Gleichung (\ref{eq:2}) mit einem Ket \(\ket{\alpha}\) der symbolisch für andere Quantenzahlen steht.

\begin{align}
  \label{eq:3}
  \ket{\alpha}\cdot\ket{JM} &=  \ket{\alpha}\cdot\sum_{m,q} \braket{jk;mq}{JM} \ket{k,q}\otimes\ket{jm} \\
\Leftrightarrow \ket{\alpha; JM} &=  \sum_{m,q} \braket{jk;mq}{JM} T^{(k)}_q  \ket{\alpha;jm} 
\end{align}

Wir möchten die Summe und  Clebsch-Gordan-Koeffizienten auf die andere Seite bringen. Dazu möchten wir folgende Relation herleiten:

\begin{align}
  \label{eq:4}
 \braket{JM}{JM} &= \bra{JM} \mathbb 1 \ket{JM} \\
&= \bra{JM} \left( \sum_{mq}\ket{jk;mq}\bra{jk;mq}  \right) \ket{JM} \\
&=  \sum_{mq}\braket{JM}{jk;mq}\braket{jk;mq}{JM} \\
&=  \sum_{mq}|\braket{jk;mq}{JM} |^2\\
&\stackrel{!}= 1
\end{align}

Die Relation lautet nun:

\begin{equation}
  \label{eq:5}
  \sum_{mq}|\braket{jk;mq}{JM} |^2 = 1
\end{equation}

Unter Ausnutzung der Relation (\ref{eq:5}) folgt für die Gleichung (\ref{eq:3}):

\begin{align}
  \label{eq:6}
   \mathbb 1 \ket{\alpha; JM} &=  \sum_{m,q} \braket{jk;mq}{JM} T^{(k)}_q  \ket{\alpha;jm} \\
 \sum_{JM} \ket{\alpha; JM}\underbr{\braket{\alpha; JM}{\alpha; JM}}_{=1=\sum_{mq}|\braket{jk;mq}{JM} |^2} &=  \sum_{m,q} \braket{jk;mq}{JM} T^{(k)}_q  \ket{\alpha;jm} \\
\sum_{JM}\ket{\alpha; JM} \cancel{\sum_{mq}}|\braket{jk;mq}{JM} |^{\cancel 2}   &= \cancel{\sum_{m,q} \braket{jk;mq}{JM}} T^{(k)}_q  \ket{\alpha;jm} 
\end{align}

Damit erhalten wir:

\begin{equation}
  \label{eq:7}
   T^{(k)}_q  \ket{\alpha;jm} = \sum_{JM}\ket{\alpha; JM} \braket{jk;mq}{JM}
\end{equation}

Multiplizieren wir mit \(\bra{\alpha;jm}\):

\begin{equation}
  \label{eq:8}
  \bra{\alpha;jm} T^{(k)}_q  \ket{\alpha;jm} = \sum_{JM} \underbr{\braket{\alpha;jm}{\alpha; JM}}_{\delta_{jJ}\delta_{mM}} \braket{jk;mq}{JM}
\end{equation}

Wegen der Orthogonalitätsbegingung bleibt von der Summe nur ein Summand übrig, bei dem gilt \(j=J\) und \(m=M\). Gleichung (\ref{eq:8}) können wir nun schreiben:

\begin{equation}
  \label{eq:9}
   \bra{\alpha;jm} T^{(k)}_q  \ket{\alpha;jm} = \underbr{\braket{\alpha;jm}{\alpha; jm}}_{\text{reduziertes Matrixelement}}  \braket{jk;mq}{jm}
\end{equation}


Als nächstes wollen wir beweisen, dass das reduzierte Matrixelement von der Quantenzahl \(m\) unabhängig ist. Dies lässt sich duch Anwenden des Schiebeoperators \(J_{\pm}\) zeigen. Zur Errinerung die Eigenwertgleichung lautet:

\begin{equation}
  \label{eq:11}
  J_{\pm}\ket{\alpha; jm} = \sqrt{j(j+1)-m(m\pm 1)} \ket{\alpha; jm\pm 1}
\end{equation}

Durch einsetzen von \(J_{\pm}\) und durch ausgleichen von einem Vorfaktor lässt sich das reduzierte Matrixelement schreiben:


\begin{align}
  \label{eq:10}
  \braket{\alpha;jm}{\alpha; jm} = \frac{1}{\sqrt{j(j+1)-m(m\mp 1)}}  \bra{\alpha;jm}J_{\pm} \ket{\alpha; jm\mp 1}
\end{align}

Lässt man nun \(J_{\pm}\) einmal auf links wirken, dabei wird \((J_{\pm})^\dagger = J_{\mp}\):

\begin{align}
  \label{eq:12}
  \braket{\alpha;jm}{\alpha; jm} &= \frac{\sqrt{j(j+1)-m(m\mp 1)}}{\sqrt{j(j+1)-m(m\mp 1)}}  \braket{\alpha;jm\mp 1 }{\alpha; jm\mp 1} \\
&= \braket{\alpha;jm\mp 1 }{\alpha; jm\mp 1}
\end{align}

Aus der Gleichung (\ref{eq:12}) sieht man dass das reduzierte Matrixelement nicht von \(m\) abhängig ist. Wir können Die Gleichung (\ref{eq:9}) schreiben

\begin{equation}
  \label{eq:14}
   \bra{\alpha;jm} T^{(k)}_q  \ket{\alpha;jm} =  \underbr{\braket{\alpha;j}{\alpha; j}}_{\text{reduziertes Matrixelement}}   \braket{jk;mq}{jm}
\end{equation}

Schlussendlich erhalten wir das Wigner-Eckart-Theorem:

\begin{equation}
  \label{eq:15}
\boxed{\bra{\alpha;jm} T^{(k)}_q  \ket{\alpha;jm} =  \braket{jk;mq}{jm} \langle \alpha;j || T^{(q)}_k || \alpha; j \rangle }
\end{equation}


\subsection*{Referenzen}
\begin{itemize}
%\item Claude Cohen-Tannoudji Quantenmechanik Band 2
\item Zettili Quanten Mehanics
\item Rollnik Quantentheorie 2
\end{itemize}

\end{document}
