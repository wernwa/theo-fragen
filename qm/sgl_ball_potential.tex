\input{../headers/header_script.tex}
%\includegraphics[width=0.75\textwidth]{thepic.png}

\begin{document}

\textit{29. März 2012}
\input{../headers/authors.tex}

\section*{Ball - Potentialbarriere}

\subsection*{Was passiert wenn ein Ball auf eine Wand trifft?}

Oder gibt es eine Transmissionswahrscheinlichkeit für den Ball?\\
\\
Im Prinzip gehört ein Ball zu einem makroskopischen Objekt für den mehr die Regeln der klassische Physik gelten als die der Quantenmechanik. Da man Quantenobjekte grundsätzlich als isoliert betrachtet, interaggieren sie nicht mit ihrer Umwelt und können sogenannte kohärente Zustände einnehmen. Das bedeutet, dass die Wellenfunktion des Quantenobjekts eine Superposition aller dem Objekt möglichen Zustände einnimmt. Sobald allerdings das Quantenobjekt mit der Umgebung koppelt, verliert es die Superposition aller Zustände und 'muss' sich für einen konkreten Zustand entscheiden (die Wellenfunktion kollabiert 'Kopenhagener Intepretation'). Experementel sieht man das an der Inteferenzfähigkeit beim Doppelspaltexperiment an Elektronen, Protonen und anderen Teilchen. Man konnte sogar ein Interferenzmuster eines Fulleronmoleküls (oder auch als Fußballmolekül genannt 60 C-Atome) nachweisen.\\

Mit der größe der Objekte verschwindet die Kohärenz. Man deutet das mit der Theorie der \textbf{Dekohärenz} der Quantenobjekte. Zurück zum Ball und dem Tunneln. Für das Tunnelverhalten bedarf ein Quantenobjekt die Superposition seiner Zustände, was man Anhand der Rechnung vom Doppelmuldenpotential erkennen kann. Der Ball müsste also nicht größer sein als ein Fulleronmolekül damit eines Tunnelverhalten aufweist. Ist der Ball größer gibt es klassisch gesehen schlicht und ergreifend eine Totalreflektion. Man muss mit ziemlicher Sicherheit davon ausgehen, dass so ein Ball wieder auf einen zukommen wird.\\
\\
Die Anstehende Betrachtung ist nicht 'ernst' gemeint. Es soll lediglich die quantenmechanische Rechnung auf ein Makroskopisches Objekt angewandt werden, nur um zu sehen was passiert.\\
\\
Die Beziehung zwischen der Wellenlänge und Impuls nach Louis De Brouglie lautet

\begin{equation} 
\label{eq:1}
\lambda = \frac{h}{p}  
\end{equation}

sowie die Wellenzahl

\begin{equation} 
\label{eq:2}
k=\frac{2\pi}{\lambda}
\end{equation}

Die Näherungsformel für die Transmissionswahrscheinlichkeit mit der Bedinung \( kl>>1 \), wobei \(l\) die Breite des Potentials ist:

\begin{equation} 
\label{eq:3}
T\propto e^{-2l\sqrt{\frac{2m}{\hbar^2}(V_0-E)}} = e^{-2lk} 
\end{equation}

Betrachten wir ein Ball mit der Masse 1kg und einer Geschwindigkeit \(v=1\frac{m}{s}\) auf die Potentialbarriere trifft. Aus der Gleichung \eqref{eq:1} erhalten wir eine Wellenlänge von:

\begin{equation}
  \label{eq:4}
  \lambda = \frac{h}{p} = \frac{h}{mv} \approx h=6,6\cdot 10^{-34} Js
\end{equation}

Damit ergibt sich die Wellenzahl \(k\) aus \eqref{eq:2} zu:

\begin{equation}
  \label{eq:5}
  k=\frac{2\pi}{h}
\end{equation}

Eingesetzt in die Gleichung \eqref{eq:3}:

\begin{equation} 
\label{eq:6}
T\propto e^{-2l\frac{2\pi}{h} }\propto e^{-10^{34}}\rightarrow 0 
\end{equation}


Somit geht die Transmissionswahrscheinlichkeit in diesem Fall gegen Null. Dieses Ergebnis war schon zu erwarten und ist mit der Dekohärenztheorie recht gut verträglich.





\end{document}
