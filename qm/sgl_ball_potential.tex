\input{../headers/header_script.tex}
%\includegraphics[width=0.75\textwidth]{thepic.png}

\begin{document}

\section*{Ball - Potentialbarriere}

Was passiert wenn ein Ball auf eine Wand trifft?

De Brougle Wellenlänge

\begin{equation} 
\label{eq:1}
\lambda = \frac{h}{p}  
\end{equation}

und die Beziehung:

\begin{equation} 
\label{eq:2}
k=\frac{2\pi}{\lambda}
\end{equation}

Die Näherungsformel für die Transmissionswahrscheinlichkeit mit der Bedinung \( kl>>1 \), wobei \(l\) die Breite des Potentials ist:

\begin{equation} 
\label{eq:3}
T\propto e^{-2l\sqrt{\frac{2m}{\hbar^2}(V_0-E)}} = e^{-2lk} 
\end{equation}

Betrachten wir ein Ball mit der Masse 1kg und einer Geschwindigkeit \(v=1\frac{m}{s}\) auf die Potentialbarriere trifft. Aus der Gleichung \eqref{eq:1} erhalten wir eine Wellenlänge von:

\begin{equation}
  \label{eq:4}
  \lambda = \frac{h}{p} = \frac{h}{mv} \approx h=6,6\cdot 10^{-34} Js
\end{equation}

Damit ergibt sich die Wellenzahl \(k\) aus \eqref{eq:2} zu:

\begin{equation}
  \label{eq:5}
  k=\frac{2\pi}{h}
\end{equation}

Eingesetzt in die Gleichung \eqref{eq:3}:

\begin{equation} 
\label{eq:6}
T\propto e^{-2l\frac{2\pi}{h} }\propto e^{-10^{34}}\rightarrow 0 
\end{equation}

Somit geht die Transmissionswahrscheinlichkeit in diesem Fall gegen Null.





\end{document}
