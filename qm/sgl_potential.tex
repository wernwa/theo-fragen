\documentclass[10pt,a4paper,oneside,fleqn]{article}
\usepackage{geometry}
\geometry{a4paper,left=20mm,right=20mm,top=1cm,bottom=2cm}
\usepackage[utf8]{inputenc}
%\usepackage{ngerman}
\usepackage{amsmath}                % brauche ich um dir Formel zu umrahmen.
\usepackage{amsfonts}                % brauche ich für die Mengensymbole
\usepackage{graphicx}
\setlength{\parindent}{0px}
\setlength{\mathindent}{10mm}
\usepackage{bbold}                    %brauche ich für die doppel Zahlen Darstellung (Einheitsmatrix z.B)



\usepackage{color}
\usepackage{titlesec} %sudo apt-get install texlive-latex-extra

\definecolor{darkblue}{rgb}{0.1,0.1,0.55}
\definecolor{verydarkblue}{rgb}{0.1,0.1,0.35}
\definecolor{darkred}{rgb}{0.55,0.2,0.2}

%hyperref Link color
\usepackage[colorlinks=true,
        linkcolor=darkblue,
        citecolor=darkblue,
        filecolor=darkblue,
        pagecolor=darkblue,
        urlcolor=darkblue,
        bookmarks=true,
        bookmarksopen=true,
        bookmarksopenlevel=3,
        plainpages=false,
        pdfpagelabels=true]{hyperref}

\titleformat{\chapter}[display]{\color{darkred}\normalfont\huge\bfseries}{\chaptertitlename\
\thechapter}{20pt}{\Huge}

\titleformat{\section}{\color{darkblue}\normalfont\Large\bfseries}{\thesection}{1em}{}
\titleformat{\subsection}{\color{verydarkblue}\normalfont\large\bfseries}{\thesubsection}{1em}{}

% Notiz Box
\usepackage{fancybox}
\newcommand{\notiz}[1]{\vspace{5mm}\ovalbox{\begin{minipage}{1\textwidth}#1\end{minipage}}\vspace{5mm}}

\usepackage{cancel}
\setcounter{secnumdepth}{3}
\setcounter{tocdepth}{3}





%-------------------------------------------------------------------------------
%Diff-Makro:
%Das Diff-Makro stellt einen Differentialoperator da.
%
%Benutzung:
% \diff  ->  d
% \diff f  ->  df
% \diff^2 f  ->  d^2 f
% \diff_x  ->  d/dx
% \diff^2_x  ->  d^2/dx^2
% \diff f_x  ->  df/dx
% \diff^2 f_x  ->  d^2f/dx^2
% \diff^2{f(x^5)}_x  ->  d^2(f(x^5))/dx^2
%
%Ersetzt man \diff durch \pdiff, so wird der partieller
%Differentialoperator dargestellt.
%
\makeatletter
\def\diff@n^#1{\@ifnextchar{_}{\diff@n@d^#1}{\diff@n@fun^#1}}
\def\diff@n@d^#1_#2{\frac{\textrm{d}^#1}{\textrm{d}#2^#1}}
\def\diff@n@fun^#1#2{\@ifnextchar{_}{\diff@n@fun@d^#1#2}{\textrm{d}^#1#2}}
\def\diff@n@fun@d^#1#2_#3{\frac{\textrm{d}^#1 #2}{\textrm{d}#3^#1}}
\def\diff@one@d_#1{\frac{\textrm{d}}{\textrm{d}#1}}
\def\diff@one@fun#1{\@ifnextchar{_}{\diff@one@fun@d #1}{\textrm{d}#1}}
\def\diff@one@fun@d#1_#2{\frac{\textrm{d}#1}{\textrm{d}#2}}
\newcommand*{\diff}{\@ifnextchar{^}{\diff@n}
  {\@ifnextchar{_}{\diff@one@d}{\diff@one@fun}}}
%
%Partieller Diff-Operator.
\def\pdiff@n^#1{\@ifnextchar{_}{\pdiff@n@d^#1}{\pdiff@n@fun^#1}}
\def\pdiff@n@d^#1_#2{\frac{\partial^#1}{\partial#2^#1}}
\def\pdiff@n@fun^#1#2{\@ifnextchar{_}{\pdiff@n@fun@d^#1#2}{\partial^#1#2}}
\def\pdiff@n@fun@d^#1#2_#3{\frac{\partial^#1 #2}{\partial#3^#1}}
\def\pdiff@one@d_#1{\frac{\partial}{\partial #1}}
\def\pdiff@one@fun#1{\@ifnextchar{_}{\pdiff@one@fun@d #1}{\partial#1}}
\def\pdiff@one@fun@d#1_#2{\frac{\partial#1}{\partial#2}}
\newcommand*{\pdiff}{\@ifnextchar{^}{\pdiff@n}
  {\@ifnextchar{_}{\pdiff@one@d}{\pdiff@one@fun}}}
\makeatother
%
%Das gleich nur mit etwas andere Syntax. Die Potenz der Differentiation wird erst
%zum Schluss angegeben. Somit lautet die Syntax:
%
% \diff_x^2  ->  d^2/dx^2
% \diff f_x^2  ->  d^2f/dx^2
% \diff{f(x^5)}_x^2  ->  d^2(f(x^5))/dx^2
% Ansonsten wie Oben.
%
%Ersetzt man \diff durch \pdiff, so wird der partieller
%Differentialoperator dargestellt.
%
%\makeatletter
%\def\diff@#1{\@ifnextchar{_}{\diff@fun#1}{\textrm{d} #1}}
%\def\diff@one_#1{\@ifnextchar{^}{\diff@n{#1}}%
%  {\frac{\textrm d}{\textrm{d} #1}}}
%\def\diff@fun#1_#2{\@ifnextchar{^}{\diff@fun@n#1_#2}%
%  {\frac{\textrm d #1}{\textrm{d} #2}}}
%\def\diff@n#1^#2{\frac{\textrm d^#2}{\textrm{d}#1^#2}}
%\def\diff@fun@n#1_#2^#3{\frac{\textrm d^#3 #1}%
%  {\textrm{d}#2^#3}}
%\def\diff{\@ifnextchar{_}{\diff@one}{\diff@}}
%\newcommand*{\diff}{\@ifnextchar{_}{\diff@one}{\diff@}}
%
%Partieller Diff-Operator.
%\def\pdiff@#1{\@ifnextchar{_}{\pdiff@fun#1}{\partial #1}}
%\def\pdiff@one_#1{\@ifnextchar{^}{\pdiff@n{#1}}%
%  {\frac{\partial}{\partial #1}}}
%\def\pdiff@fun#1_#2{\@ifnextchar{^}{\pdiff@fun@n#1_#2}%
%  {\frac{\partial #1}{\partial #2}}}
%\def\pdiff@n#1^#2{\frac{\partial^#2}{\partial #1^#2}}
%\def\pdiff@fun@n#1_#2^#3{\frac{\partial^#3 #1}%
%  {\partial #2^#3}}
%\newcommand*{\pdiff}{\@ifnextchar{_}{\pdiff@one}{\pdiff@}}
%\makeatother

%-------------------------------------------------------------------------------
%%Nützliche Makros um in der Quantenmechanik Bras, Kets und das Skalarprodukt
%%zwischen den beiden darzustellen.
%%Benutzung:
%% \bra{x}  ->    < x |
%% \ket{x}  ->    | x >
%% \braket{x}{y} ->   < x | y >

\newcommand\bra[1]{\left\langle #1 \right|}
\newcommand\ket[1]{\left| #1 \right\rangle}
\newcommand\braket[2]{%
  \left\langle #1\vphantom{#2} \right.%
  \left|\vphantom{#1#2}\right.%
  \left. \vphantom{#1}#2 \right\rangle}%

%-------------------------------------------------------------------------------
%%Aus dem Buch:
%%Titel:  Latex in Naturwissenschaften und Mathematik
%%Autor:  Herbert Voß
%%Verlag: Franzis Verlag, 2006
%%ISBN:   3772374190, 9783772374197
%%
%%Hier werden drei Makros definiert:\mathllap, \mathclap und \mathrlap, welche
%%analog zu den aus Latex bekannten \rlap und \llap arbeiten, d.h. selbst
%%keinerlei horizontalen Platz benötigen, aber dennoch zentriert zum aktuellen
%%Punkt erscheinen.

\newcommand*\mathllap{\mathstrut\mathpalette\mathllapinternal}
\newcommand*\mathllapinternal[2]{\llap{$\mathsurround=0pt#1{#2}$}}
\newcommand*\clap[1]{\hbox to 0pt{\hss#1\hss}}
\newcommand*\mathclap{\mathpalette\mathclapinternal}
\newcommand*\mathclapinternal[2]{\clap{$\mathsurround=0pt#1{#2}$}}
\newcommand*\mathrlap{\mathpalette\mathrlapinternal}
\newcommand*\mathrlapinternal[2]{\rlap{$\mathsurround=0pt#1{#2}$}}

%%Das Gleiche nur mit \def statt \newcommand.
%\def\mathllap{\mathpalette\mathllapinternal}
%\def\mathllapinternal#1#2{%
%  \llap{$\mathsurround=0pt#1{#2}$}% $
%}
%\def\clap#1{\hbox to 0pt{\hss#1\hss}}
%\def\mathclap{\mathpalette\mathclapinternal}
%\def\mathclapinternal#1#2{%
%  \clap{$\mathsurround=0pt#1{#2}$}%
%}
%\def\mathrlap{\mathpalette\mathrlapinternal}
%\def\mathrlapinternal#1#2{%
%  \rlap{$\mathsurround=0pt#1{#2}$}% $
%}

%-------------------------------------------------------------------------------
%%Hier werden zwei neue Makros definiert \overbr und \underbr welche analog zu
%%\overbrace und \underbrace funktionieren jedoch die Gleichung nicht
%%'zerreißen'. Dies wird ermöglicht durch das \mathclap Makro.

\def\overbr#1^#2{\overbrace{#1}^{\mathclap{#2}}}
\def\underbr#1_#2{\underbrace{#1}_{\mathclap{#2}}}
\graphicspath{{sgl_potential_pics/}}% Suchpfad für Graphiken

\begin{document}
\setcounter{section}{1}
\section*{Potentialtopf, Potentialbariere und Potentialwall}

\subsection*{Vorgehensweise}

Um die Lösung für ein Potentialproblem zu finden, macht man sich an einer Zeichung klar, wo und welche Form das Potential hat. Dann kann für die verschiedenen Bereiche jeweils eine Ansatzfunktion mit allgemeinen Paramtern aufgestellt werden. Diese Funktion sollte Lösung der Schrödinger-Gleichung sein. Um die Parameter zu bestimmen müssen die einzelnen Ansatzfunktionen an den Rändern Anschlussbegingungen erfüllen. Damit die Gesamtlösung \textbf{stetig} ist sollten die Ansatzfunktionen an den Anschlusspunkten gleich sein sowie deren erste Ortsableitungen. Für ein eindimensionales Beispiel, mit zwei Ansatzfunktionen \(\psi_I(x)\) und \(\psi_{II}(x)\) und den Bereichen mit unterschiedlichen Potentialen I und II, sollte gelten
\begin{equation}
  \label{eq:15}
  \psi_I(a) = \psi_{II}(a)\qquad  \pdiff_x \psi_I(a) = \pdiff_x \psi_{II}(a)
\end{equation}

Mit \(x=a\) die Stelle an dem Bereich I endet und Bereich II beginnt. Die Notwendigkeit der Gleichheit ersten Ableitungen der Funktionen an den Anschlusspunkten begründet sich aus der Wahrscheinnlichkeitsstromerhaltung \(j_I+j_{II}=j_{\text{ges}}\) für die gesamte Lösung. Damit diese erfüllt ist, verlangt die Definition des quantenmechanischen Stromes
\begin{equation}
  \label{eq:17}
  j = \frac{i\hbar}{2m}\left(\psi \frac{\partial}{\partial x} \psi^* - \psi^* \frac{\partial}{\partial x}\psi\right)
\end{equation}

nach einer Ortsableitung.

\subsection*{Verschiedene Arten von Potentialen}

Bewegt sich ein Teilchen frei im Raum und trifft auf eine Potentialstufe, so kann das Teilchen je nach Höhe und Breite tunneln, reflektieren oder streuen. Die Energiewerte des Teilchens sind dabei kontinuierlich. Man spricht i.a. von \textbf{Streuzuständen}

Ist dagegen das Teilchen von Potentialwänden umgeben und kann nicht entweichen, weil die eigene Energie kleiner ist, so ergibt sich für die Lösung der Energieeigenwerte diskrete Zustände. Hierbei handelt es sich um \textbf{gebundene Zustände}.

\subsection{Stufenpotential allgemein}

Die stationäre Schrödingergleichung für eindimensional Probleme lautet:
\begin{align} 
  &\left[-\frac{\hbar^2}{2m}\frac{d^2}{dx^2}+V(x)\right]\varphi(x)
  =E\varphi(x)\label{eq:25}\\
  \Leftrightarrow &\left[\frac{d^2}{dx^2} 
    +\frac{2m}{\hbar^2}(E-V(x))\right]\varphi(x)=0 
  \tag{\ref{eq:25}'}\label{eq:25'} 
\end{align}

Aus der Gleichung \eqref{eq:25'} folgt:
\begin{subequations}
  \label{eq:27}
  \begin{align}
    E>V: \quad &\left[\frac{d^2}{dx^2}+k^2\right]\varphi(x)=0 
    \quad \textrm{mit } k=\sqrt{\frac{2m}{\hbar^2}\underbrace{(E-V)}_{>\,0}} \notag \\
    &\varphi(x)=Ae^{ikx}+Be^{-ikx}  
    \qquad  A,B \in \mathbb{C} \label{eq:27a} \\[2ex]
    E<V: \quad &\left[\frac{d^2}{dx^2}-\kappa^2\right]\varphi(x)=0
    \quad  \textrm{mit } \kappa=\sqrt{\frac{2m}{\hbar^2}\underbrace{(V-E)}_{>\,0}} \notag \\
    &\varphi(x)=Ce^{\kappa x}+De^{-\kappa x}  
    \qquad  C,D \in \mathbb{C} \label{eq:27b}\\[2ex]
    E=V: \quad &\frac{d^2}{dx^2}\varphi(x)=0 \notag \\
    &\varphi(x)=F+G\cdot x   
    \qquad  F,G \in \mathbb{C} \label{eq:27c}
  \end{align}
\end{subequations}

\subsubsection{Anschlußbedingung.}
$V(x)$ mit Unstetigkeit bei $x=a$\\
Sei $V_\delta$ in $[a-\delta, a+\delta] \forall \delta$ beschränkt. In
(\ref{eq:27a}) eingesetzt und einmal Integriert:
\begin{equation*}
  \left. \frac{d\varphi}{dx}\right|_{a+\delta}-\left. \frac{d\varphi}{dx}\right|_{a-\delta} =
\int_{a-\delta}^{a+\delta} \frac{2m}{\hbar^2}
\left(V_\delta(x)-E\right)\varphi(x)dx
\end{equation*}
$V_\delta$ beschränkt $\rightarrow$ rechte Seite $\rightarrow$ 0 für $\delta$
$\rightarrow$ 0
\begin{equation*}
  \left. \frac{d\varphi}{dx}\right|_{a+\delta} = \left.
    \frac{d\varphi}{dx}\right|_{a-\delta} \quad
  \Rightarrow
 \begin{cases} 
   \varphi'=\frac{d\varphi}{dx}, \mbox{ ist stetig bei } x=a  \\
   \varphi \mbox{ ebenfalls stetig bei } x=a 
 \end{cases}
\end{equation*}


\subsubsection{Reflexion- und Transmissionskoeffizient.}

Trifft ein Teilchen auf ein Potential das größer ist als die Energie des Teilchens, dann kann das Teilchen entweder reflektiert oder transmittiert werden. Die Warscheinlichkeit für die Reflexion bzw. Transmission des Teilchens wird durch den Reflexions bzw. Transmissions Koeffizienten bestimmt. Diese sind folgendermaßen definiert:

\begin{equation}
  \label{eq:1}
  R = \frac{\text{Reflektierter Wahrscheinlichkeitsstrom}}{\text{Einfallender Warscheinlichkeitsstrom}}  = \frac{j_{r}}{j_{e}}\qquad \leftarrow  \text{Reflexionskoeffizient}
\end{equation}

\begin{equation}
  \label{eq:2}
  T = \frac{\text{Transmittierter Warscheinlichkeitsstrom}}{\text{Einfallender Warscheinlichkeitsstrom}}  = \frac{j_{t}}{j_{e}}\qquad \leftarrow  \text{Transmissionskoeffizient}
\end{equation}


Für den Wahrscheinlichkeitsstrom gilt (siehe Schrödingergleichung+Kontinuitätsgleichung):

\begin{equation}
  \label{eq:3}
  j = \frac{i\hbar}{2m}\left(\psi \frac{\partial}{\partial x} \psi^* - \psi^* \frac{\partial}{\partial x}\psi\right)
\end{equation}

Aus der Warscheinlichkeitsstromerhaltung folgt:

\begin{equation}
  \label{eq:4}
  j_{e} = j_{r} + j_{t}
\end{equation}

Betrachten wir eine einfallende ebene Welle für die gilt:

\begin{equation}
  \label{eq:5}
  \psi_e = Ae^{ikx}
\end{equation}

Damit erhalten wir folgenden einfallenden Warscheinlichkeitsstrom (siehe \eqref{eq:3}):

\begin{align}
  \label{eq:6}
  j_e &= \frac{i\hbar}{2m}\left(\psi_e \frac{\partial}{\partial x} \psi^*_e - \psi^*_e \frac{\partial}{\partial x}\psi_e\right)\\
&= \frac{i\hbar}{2m}\left(Ae^{ikx} \frac{\partial}{\partial x} A^*e^{-ikx} - A^*e^{-ikx}\frac{\partial}{\partial x}Ae^{ikx}\right)\\
&= \frac{i\hbar}{2m}\left(|A|^2(-ik)  -|A|^2(ik)\right)\\
&= \frac{\hbar k}{m}|A|^2
\end{align}


Für die reflektierte Welle gilt:

\begin{equation}
  \label{eq:7}
  \psi_r = Be^{-ikx}
\end{equation}

Wir erhalten wir folgenden reflektierten Warscheinlichkeitsstrom:

\begin{align}
  \label{eq:8}
  j_r &= \frac{i\hbar}{2m}\left(\psi_r \frac{\partial}{\partial x} \psi^*_r - \psi^*_r \frac{\partial}{\partial x}\psi_r\right)\\
&= \frac{i\hbar}{2m}\left(Be^{-ikx} \frac{\partial}{\partial x} B^*e^{ikx} - B^*e^{ikx}\frac{\partial}{\partial x}Be^{-ikx}\right)\\
&= \frac{i\hbar}{2m}\left(|B|^2(ik)  -|B|^2(-ik)\right)\\
&= -\frac{\hbar k}{m}|B|^2
\end{align}

Damit erhalten wir folgenden Reflektionskoeffizienten (siehe \eqref{eq:1}):

\begin{equation}
  \label{eq:9}
  R = \frac{j_r}{j_e} = \frac{-\frac{\hbar k}{m}|B|^2 }{\frac{\hbar k}{m}|A|^2} = -\frac{|B|^2}{|A|^2}
\end{equation}


Da die transmitierte Welle in einem anderen Potential ist, ändert sich der Wellenvektor des Teilchens \(k \rightarrow q\). Somit gilt für die Wellenfunktion:

\begin{equation}
  \label{eq:10}
    \psi_t = Ce^{iqx}
\end{equation}

Für den Warschenlichkeitsstrom gilt:

\begin{align}
  \label{eq:11}
  j_t &= \frac{i\hbar}{2m}\left(\psi_t \frac{\partial}{\partial x} \psi^*_t - \psi^*_t \frac{\partial}{\partial x}\psi_t\right)\\
&= \frac{i\hbar}{2m}\left(Ce^{iqx} \frac{\partial}{\partial x} C^*e^{-iqx} - C^*e^{-ikx}\frac{\partial}{\partial x}Ce^{iqx}\right)\\
&= \frac{i\hbar}{2m}\left(|C|^2(-iq)  -|C|^2(iq)\right)\\
&= \frac{\hbar q}{m}|C|^2
\end{align}


Damit können wir den Transmissionskoeffizienten bestimmen (siehe \eqref{eq:2}) :


\begin{equation}
  \label{eq:12}
  T = \frac{j_t}{j_e} = \frac{\frac{\hbar q}{m}|C|^2 }{\frac{\hbar k}{m}|A|^2} = \frac{q}{k} \frac{|C|^2}{|A|^2}
\end{equation}

Aus der Wahrscheinlichkeitsstromerhaltung \eqref{eq:4}   können wir eine wichtige Beziehung herleiten:

\begin{align}
  \label{eq:13}
  \frac{\hbar k}{m} |A|^2 &= -\frac{\hbar k}{m}|B|^2 + \frac{\hbar q}{m} |C|^2 \qquad |:\frac{\hbar k}{m}|A|^2 \\
1 &= \underbrace{-\frac{ |B|^2}{|A|^2}}_{R} +\underbrace{ \frac{q}{k} \frac{|C|^2}{|A|^2}}_{T}
\end{align}

\begin{equation}
  \label{eq:14}
  \Rightarrow \boxed{ R+T=1 }
\end{equation}




\subsection{Potentialstufe}

\begin{figure}[htbp]
  \begin{minipage}[h]{0.4\linewidth}
    \begin{align*}
      V(x)&=V_0\cdot\Theta(x) \\
      &= \begin{cases}
        V_0 & \mbox{ für }x>0\\
        0 & \mbox{ für }x\leq0
      \end{cases}
    \end{align*}
    \vspace{1cm}
  \end{minipage}
  \hfill
  \begin{minipage}[h]{0.5\linewidth}
    \input{sgl_potential_pics/Potentialstufe.pdf_t}
  \end{minipage}
  \caption{Potentialstufe.}
  \label{figu:5}
\end{figure}

Für $E>V_0$ gilt laut (\ref{eq:27a}):
\begin{align*}
    \textbf{Bereich I:} \quad &\left[\frac{d^2}{dx^2}+k_1^2\right]\varphi_{\text{I}}(x)=0 
    \quad \textrm{mit } k_1=\sqrt{\frac{2m}{\hbar^2}E} \\
    &\varphi_\textrm{I}(x)=A_1e^{ikx}+A_1'e^{-ikx} \qquad A_1,A_1' \in \mathbb{C} \\[2ex]
    \textbf{Bereich II:} \quad &\left[\frac{d^2}{dx^2}+k_2^2\right]\varphi_{\text{II}}(x)=0
    \quad  \textrm{mit } k_2=\sqrt{\frac{2m}{\hbar^2}(E-V_0)} \\
     &\varphi_\textrm{II}(x)=A_2e^{ikx}+A_2'e^{-ikx} \qquad A_2,A_2' \in \mathbb{C}
\end{align*}

Teilchen kommt von $-\infty \rightarrow A_2' = 0$\\
\textbf{Anschlußbedingung:}
%\localeqncounter
\begin{align}
  \varphi_\textrm{I}(0) 
  = \varphi_\textrm{II}(0) & \Leftrightarrow  A_1 + A_1' 
  = A_2 + \cancel{A_2'}\label{eq1:Stufe-eq1}\\
 \varphi'_\textrm{I}(0) 
 = \varphi'_\textrm{II}(0) & \Leftrightarrow  k_1(A_1 - A_1') 
 = k_2(A_2 - \cancel{A_2'}) \label{eq2:Stufe-eq2}
\end{align}
%\globaleqncounter

\begin{equation*}
  \Rightarrow \frac{A_1'}{A_1} = \frac{k_1-k_2}{k_1+k_2} \quad
  \mbox{und} \quad
  \frac{A_2}{A_1} = \frac{2k_1}{k_1+k_2}
\end{equation*}

Wähle $A_1 = 1$ somit folgt:
\begin{equation*}
  \varphi(x) =
  \begin{cases}
    e^{ik_1x}+\frac{k_1-k_2}{k_1+k_2} e^{-ik_1x} & \mbox{ für Bereich I} \\
    \frac{2k_1}{k_1+k_2} e^{ik_2x} & \mbox{ für Bereich II}
  \end{cases}
\end{equation*}

\textbf{Reflektionskoeffizient:}
%\begin{centereq}
  \begin{equation}
    \label{eq:29}
    R= \left|\frac{A_1'}{A_1}\right|^2 
    = \left|\frac{k_1-k_2}{k_1+k_2}\right|^2
  \end{equation}
%\end{centereq}
\textbf{Transmissionsskoeffizient:}
%%\begin{centereq}
  \begin{equation}
    \label{eq:30}
    T= \frac{k_2}{k_1}\left|\frac{A_1'}{A_1}\right|^2 
    = \frac{4k_1k_2}{k_1+k_2}^2
  \end{equation}
%\end{centereq}
Es gilt:
%%\begin{centereq}
  \begin{equation}
    \label{eq:31}
    \boxed{R+T=1}
  \end{equation}
%\end{centereq}

\bigskip
Für $E<V_0$ gilt laut (\ref{eq:27a}) und (\ref{eq:27b}):
\begin{align*}
    \textbf{Bereich I:} \quad &\left[\frac{d^2}{dx^2}+k_1^2\right]\varphi_\textrm{I}(x)=0 
    \quad \textrm{mit } k_1=\sqrt{\frac{2m}{\hbar^2}E} \\
    &\varphi_\textrm{I}(x)=A_1e^{ikx}+A_1'e^{-ikx} \qquad A_1,A_1' \in \mathbb{C} \\[2ex]
    \textbf{Bereich II:} \quad &\left[\frac{d^2}{dx^2}+\kappa_2^2\right]\varphi_\textrm{II}(x)=0
    \quad  \textrm{mit } \kappa_2=\sqrt{\frac{2m}{\hbar^2}(V_0-E)} \\
     &\varphi_\textrm{II}(x)=B_2e^{\kappa x}+B_2'e^{-\kappa x} \qquad B_2,B_2' \in \mathbb{C}
\end{align*}

\textbf{Randbedingung:} $\varphi_\textrm{II}(x) \rightarrow 0$ für $x \rightarrow \infty$
daraus folgt $B_2 = 0$ \\
\begin{equation*}
  \Rightarrow \frac{A_1'}{A_1} = \frac{k_1-i\kappa}{k_1+i\kappa} \quad
  \mbox{und} \quad
  \frac{B'_2}{A_1} = \frac{2k_1}{k_1+i\kappa}
\end{equation*}

Mit $A_1 = 1$ folgt:

\begin{equation*}
  \varphi(x) =
  \begin{cases}
    e^{ik_1x}+\frac{k_1-i\kappa}{k_1+i\kappa} e^{-ik_1x} & \mbox{ für Bereich I} \\
    \frac{2k_1}{k_1+i\kappa} e^{-\kappa x} & \mbox{ für Bereich II}
  \end{cases}
\end{equation*}

\textbf{Reflexionskoeffizient:}
%%\begin{centereq}
  \begin{equation*}
    R= \left|\frac{A_1'}{A_1}\right|^2 
    =  \left|\frac{k_1-i\kappa}{k_1+i\kappa}\right|
    = \left(\frac{k_1-i\kappa}{k_1+i\kappa}\right) 
    \left(\frac{k_1+i\kappa}{k_1-i\kappa}\right)
    = 1
  \end{equation*}

\textbf{Transmissionsskoeffizient:}
\begin{equation*}
  T= 1-R = 1-1=0 \Rightarrow \mbox{ Totalreflexion}
\end{equation*}
%\end{centereq}

\textbf{Bemerkung:}\\
Bei einer unendlichhoher Schwelle d.h. $V_0 \rightarrow \infty$ geht $\kappa
\rightarrow \infty$ daraus folgt $A'_1 \rightarrow -A_1$ und $B'_2 \rightarrow
0$. Damit gilt für die Wellenfunktion:
\[ \varphi(x=0)=0 \]

\subsection{Tunneleffekt, Potentialschwelle.}
\begin{figure}[htbp]
  \begin{minipage}[h]{0.45\linewidth}
    \begin{align*}
      V(x)&=V_0\cdot\Theta(a-|x|) \\
      &= \begin{cases}
        V_0 & \mbox{ für }-a<x<a\\
        0 & \mbox{ sonst}
      \end{cases}
    \end{align*}
    \vspace{1cm}
  \end{minipage}
  \hfill
  \begin{minipage}[h]{0.5\linewidth}
    \input{sgl_potential_pics/Potentialschwelle.pdf_t}
  \end{minipage}
  \caption{Potentialschwelle.}
  \label{fig:6}
\end{figure}

Für $E<V_0$ gilt Gl.~\eqref{eq:27a} und \eqref{eq:27b}:
\begin{equation*}
  \varphi =
  \begin{cases}
    Ae^{ikx}+Be^{-ikx} & \mbox{für Bereich I}\\
    Ce^{\kappa x}+De^{-\kappa x} & \mbox{für Bereich II}\\
    Fe^{ikx}+Ge^{-ikx} & \mbox{für Bereich III}
  \end{cases}
\end{equation*}
Mit \(  k = \frac {\sqrt{2 m E}}\hbar  \) und
\( \kappa=\frac{\sqrt{2 m (V_0 - E)}}\hbar \)\\

Anschlußbedingung bei $x=-a$
%\localeqncounter
%\resetlocaleqncounter
%Lege eine lokalen Zähler für linksgebundene Tags an.
%\newcounter{potschwellecounter}
%Mit Null initialisieren.
%\setcounter{potschwellecounter}{0}
%Sichere den globalen Zähler.
%\setcounter{tmpeqnc}{\value{equation}}%
%Setze den lokalen Zähler als Standardzähler.
%\setcounter{equation}{\value{potschwellecounter}}%
%\renewcommand{\theequation}{\arabic{equation}}%

\begin{equation}
  \varphi_\textrm{I} (-a) = \varphi_\textrm{II}(-a)
   \Leftrightarrow  
  Ae^{-ika}+Be^{ika} = Ce^{-\kappa a}+De^{\kappa a} 
  \label{eq2:potschw1}
\end{equation}
\begin{align}
  \left.\frac{\text{d}\varphi_\textrm{I}}{\text{d}x} \right|_{x=-a} 
      = \left.\frac{\text{d}\varphi_\textrm{II}}{\text{d}x}\right|_{x=-a}
  & \Leftrightarrow  
  ik Ae^{-ika}-ik Be^{ika} = \kappa Ce^{-\kappa a}- \kappa D e^{\kappa a}
  \label{eq2:potschw2}\\
  & \Leftrightarrow  
   Ae^{-ika}-Be^{ika} = \frac{-i\kappa}{k} Ce^{-\kappa a}
                      +\frac{i\kappa}{k} De^{\kappa a}\notag
\end{align}

\begin{align}
  \eqref{eq2:potschw1}+\eqref{eq2:potschw2}
  \Rightarrow 
  2Ae^{-ika} &= Ce^{-\kappa a}+De^{\kappa a} 
              -\frac{i\kappa}{k} Ce^{-\kappa a}
                +\frac{i\kappa}{k}De^{\kappa a} \label{eq2:potschw3} \\
             &=  \left(1-\frac{i\kappa}{k}\right)Ce^{-\kappa a}
                +\left(1+\frac{i\kappa}{k}\right)De^{\kappa a} \notag\\[2ex]
 \eqref{eq2:potschw1}-\eqref{eq2:potschw2}
  \Rightarrow 
  2Be^{ika} &= Ce^{-\kappa a}+De^{\kappa a} 
              +\frac{i\kappa}{k}Ce^{-\kappa a}
                -\frac{i\kappa}{k}De^{\kappa a} \label{eq2:potschw4} \\
             &=  \left(1+\frac{i\kappa}{k}\right)Ce^{-\kappa a}
                +\left(1-\frac{i\kappa}{k}\right)De^{\kappa a}\notag
\end{align}

Anschlußbedingung bei $x=a$
\begin{equation}
  \label{eq2:potschw5}
   \varphi_\textrm{II} (a) = \varphi_\textrm{III}(a)
   \Leftrightarrow  
  Ce^{\kappa a}+De^{-\kappa a} = Fe^{ika}+Ge^{-ika}
\end{equation}
\begin{align}
  \left.\frac{\text{d}\varphi_\textrm{II}}{\text{d}x}\right|_{x=a}
  =\left.\frac{\text{d}\varphi_\textrm{III}}{\text{d}x}\right|_{x=a}
  & \Leftrightarrow  
 \kappa Ce^{\kappa a}- \kappa D e^{-\kappa a} = ik Fe^{ika}-ik Ge^{-ika}  
 \label{eq2:potschw6}\\
  & \Leftrightarrow  
   Ce^{\kappa a}-De^{-\kappa a} = \frac{ik}{\kappa} Fe^{ika}
                      -\frac{ik}{\kappa} Ge^{-ika}\notag
\end{align}

Hier betrachten wir eine von links einlaufende Welle d.h. $G=0$ damit reduzieren
sich die Gleichungen~\eqref{eq2:potschw5} und \eqref{eq2:potschw6} auf:

\begin{align}
  &Ce^{\kappa a}+De^{-\kappa a} = Fe^{ika}
  \tag{\ref{eq2:potschw5}'}\label{eq2:potschw5'} \\[2ex]
  &Ce^{\kappa a}-De^{-\kappa a} = \frac{ik}{\kappa}
  Fe^{ika} \tag{\ref{eq2:potschw6}'}\label{eq2:potschw6'}
\end{align} 

\begin{align}
  \eqref{eq2:potschw5'} + \eqref{eq2:potschw6'}
  \Rightarrow 
   2Ce^{\kappa a} &= Fe^{ika}+ \frac{ik}{\kappa}Fe^{ika}
   \label{eq2:potschw7}\\ %\hphantom{-}
                 &= \left(1+\frac{ik}{\kappa}\right)Fe^{ika}
                      \qquad \qquad 
                      \left|\quad \cdot \frac{1}{2}e^{-2\kappa a}\right.\notag\\
  \Leftrightarrow 
  Ce^{-\kappa a} &= \frac{1}{2}e^{-2\kappa a}
                     \left(1+\frac{ik}{\kappa}\right)Fe^{ika}\notag
\end{align}
\begin{align}
  \eqref{eq2:potschw5'} - \eqref{eq2:potschw6'}
  \Rightarrow 
  2De^{-\kappa a} &= Fe^{ika}- \frac{ik}{\kappa}Fe^{ika}
  \label{eq2:potschw8} \\
                  &= \left(1-\frac{ik}{\kappa}\right)Fe^{ika}
                      \qquad \qquad 
                      \left|\quad \cdot \frac{1}{2}e^{2\kappa a}\right.\notag\\
  \Leftrightarrow 
  De^{\kappa a} &= \frac{1}{2}e^{2\kappa a}
                    \left(1-\frac{ik}{\kappa}\right)Fe^{ika}\notag
\end{align}
%\globaleqncounter
Setzen wir nun die Gleichungen~\eqref{eq2:potschw7} und \eqref{eq2:potschw8}
in die Gleichung~\eqref{eq2:potschw3} ein, so erhalten
wir:
{\allowdisplaybreaks
\begin{align}
   2Ae^{-ika} & = \left(1-\frac{i\kappa}{k}\right)
                   \frac{1}{2}e^{-2\kappa a}
                   \left(1+\frac{ik}{\kappa}\right)Fe^{ika}\notag \\
                 &\qquad \qquad \qquad \qquad
                 +\left(1+\frac{i\kappa}{k}\right)
                    \frac{1}{2}e^{2\kappa a}
                   \left(1-\frac{ik}{\kappa}\right)Fe^{ika} \notag \\
              & = \frac{1}{2}\left[\left(1-\frac{i\kappa}{k}\right)
                     \left(1+\frac{ik}{\kappa}\right)
                     e^{-2\kappa a}
                     +\left(1+\frac{i\kappa}{k}\right)
                     \left(1-\frac{ik}{\kappa}\right)
                     e^{2\kappa a}\right]
                     Fe^{ika} \notag \\
                &= \frac{1}{2}\left[
                     \left(1+\frac{ik}{\kappa}
                             -\frac{i\kappa}{k}
                             +\underbrace{\frac{\kappa k}{k \kappa}}_{=1}\right)
                     e^{-2\kappa a}\right.\notag\\
                     &\qquad \qquad \qquad \qquad
                     +\left.\left(1-\frac{ik}{\kappa}
                     +\frac{i\kappa}{k}
                     +\underbrace{\frac{\kappa k}{k \kappa}}_{=1}\right)
                     e^{2\kappa a}\right]
                     Fe^{ika} 
                   \notag \\
              & = \frac{1}{2}\left[
                     \left(2-i\underbrace{\left(\frac{\kappa}{k}
                             -\frac{k}{\kappa}\right)}_{\equiv\epsilon}\right)
                     e^{-2\kappa a}
                     +\left(2+i\underbrace{\left(\frac{\kappa}{k}
                             -\frac{k}{\kappa}\right)}_{\equiv\epsilon}\right)
                     e^{2\kappa a}\right]
                     Fe^{ika} \notag \\
              & = \frac{1}{2}\left[
                     \left(2-i\epsilon\right)
                     e^{-2\kappa a}
                     +\left(2+i\epsilon\right)
                     e^{2\kappa a}\right]
                     Fe^{ika} \notag \\
              & = \frac{1}{2}\left[
                     2 e^{-2\kappa a}-i\epsilon \, e^{-2\kappa a}
                     +2 e^{2\kappa a}+i\epsilon \, e^{2\kappa a}\right]
                     Fe^{ika} \notag \\
              & = \frac{1}{2}\left[
                     2 
                     \underbrace{\left(e^{2\kappa a}+e^{-2\kappa a}\right)}
                     _{2\cosh(2\kappa a)}
                     +i\epsilon
                     \underbrace{\left(e^{2\kappa a}-e^{-2\kappa a}\right)}
                    _{2\sinh(2\kappa a)}\right]
                     Fe^{ika} \notag \\
              & = \left[2\cosh(2\kappa a) +i\epsilon\sinh(2\kappa a)\right]
                     Fe^{ika} \qquad \qquad \left|\quad \cdot
                       \frac{1}{2}e^{ika}\right.\notag\\[2ex]
\Leftrightarrow A & = \left[\cosh(2\kappa a) 
                     +\frac{i\epsilon}{2}\sinh(2\kappa a)\right]Fe^{2ika}
\end{align}
}

Setzen wir nun analog die Gleichungen~\eqref{eq2:potschw7} und
\eqref{eq2:potschw8} in die Gleichung~\eqref{eq2:potschw4} ein, so erhalten wir:
{\allowdisplaybreaks
\begin{align}
   2Be^{ika} & = \left(1+\frac{i\kappa}{k}\right)
                   \frac{1}{2}e^{-2\kappa a}
                   \left(1+\frac{ik}{\kappa}\right)Fe^{ika}\notag \\
                 &\qquad \qquad \qquad \qquad
                 +\left(1-\frac{i\kappa}{k}\right)
                    \frac{1}{2}e^{2\kappa a}
                   \left(1-\frac{ik}{\kappa}\right)Fe^{ika} \notag \\
              & = \frac{1}{2}\left[\left(1+\frac{i\kappa}{k}\right)
                     \left(1+\frac{ik}{\kappa}\right)
                     e^{-2\kappa a}
                     +\left(1-\frac{i\kappa}{k}\right)
                     \left(1-\frac{ik}{\kappa}\right)
                     e^{2\kappa a}\right]
                     Fe^{ika} \notag \\
              & = \frac{1}{2}\left[
                     \left(1+\frac{ik}{\kappa}
                             +\frac{i\kappa}{k}
                             -\underbrace{\frac{\kappa k}{k \kappa}}_{=1}\right)
                     e^{-2\kappa a}\right.\notag\\
                   &\qquad \qquad \qquad \qquad
                     +\left.\left(1-\frac{ik}{\kappa}
                             -\frac{i\kappa}{k}
                             -\underbrace{\frac{\kappa k}{k \kappa}}_{=1}\right)
                     e^{2\kappa a}\right]
                     Fe^{ika} \notag \\
              & = \frac{1}{2}\left[
                     i\underbrace{\left(\frac{\kappa}{k}
                             +\frac{k}{\kappa}\right)}_{\equiv\eta}
                     e^{-2\kappa a}
                     -i\underbrace{\left(\frac{\kappa}{k}
                             +\frac{k}{\kappa}\right)}_{\equiv\eta}
                     e^{2\kappa a}\right]
                     Fe^{ika} \notag \\
              & = -\frac{1}{2} i \eta 
                     \underbrace {\left[
                     e^{2\kappa a} - e^{-2\kappa a}\right]}_
                     {2 \sinh(2 \kappa a)}
                     Fe^{ika} \notag \\
              & = -i \eta \sinh(2 \kappa a)
                     Fe^{ika} \qquad \qquad \left| \quad
                       \cdot \frac{1}{2}e^{-ika}\right. \notag\\[2ex]
\Leftrightarrow B & = -i \eta \sinh(2 \kappa a) F
\end{align}
}

Für die \textbf{Translationsamplitude} gilt:
\[ \tau = \frac{F}{A} = \frac{e^{-2 i k a}}
               {\cosh(2 \kappa a) + \frac{i \epsilon}{2}\sinh(2 \kappa a)} \]

Für den \textbf{Transmissionsskoeffizienten} gilt:

\begin{align*}
  T &= |\tau|^2 = \tau^*\tau \\
    &= \frac{e^{2 i k a}}
          {\cosh(2 \kappa a) - \frac{i \epsilon}{2}\sinh(2 \kappa a)} \cdot
           \frac{e^{-2 i k a}}
           {\cosh(2 \kappa a) + \frac{i \epsilon}{2}\sinh(2 \kappa a)}\\
    &= \frac{1}{\cosh(2 \kappa a)^2
         + \frac{\epsilon^2}{4}\sinh(2 \kappa a)^2}\\
    &= \frac{1}{1+\sinh(2 \kappa a)^2
         + \frac{\epsilon^2}{4}\sinh(2 \kappa a)^2}
\end{align*}

%%\begin{centereq}
  \begin{equation}
    \label{eq:28}
    \Leftrightarrow T = \frac{1}{1+(1 + 
      \frac{\epsilon^2}{4})\sinh(2 \kappa a)^2}
  \end{equation}
%\end{centereq} 

Mit \(  k = \frac {\sqrt{2 m E}}\hbar  \),
\( \kappa=\frac{\sqrt{2 m (V_0 - E)}}\hbar \) und 
\( \epsilon = \frac \kappa k - \frac k \kappa \) erhalten wir:
{\allowdisplaybreaks
\begin{align*}
 T &= \frac{1}{1+\left[1 + \frac{1}{4} \cdot
           \left(\frac \kappa k - \frac k \kappa\right)^2
           \right]\sinh(2 \kappa a)^2}\\
   &= \frac{1}{1+\left[1 + \frac{1}{4} \cdot
           \left(\frac {\kappa^2}{k^2}
             -2\cdot \frac \kappa k \frac k \kappa
             + \frac {k^2}{\kappa^2}
           \right)
           \right]\sinh(2 \kappa a)^2}\\
   &= \frac{1}{1+\left[1 + \frac{1}{4} \cdot
           \left(\frac {V_0-E}{E}
             -2
             + \frac {E}{V_0-E}
           \right)
           \right]\sinh\left(2 \frac{\sqrt{2 m (V_0 - E)}}\hbar a\right)^2}\\
   &= \frac{1}{1+\left[1 + \frac{1}{4} \cdot
           \left(\frac {(V_0-E)^2-2E(V_0-E)+E^2}{E(V_0-E)}
           \right)
           \right]\sinh(\dots)^2} 
         \qquad \qquad
           \left|\;\cdot \frac{4 E (V_0-E)}{4 E (V_0-E)} \right. \\
   &= \frac{4 E (V_0-E)}{4 E (V_0-E)+\left[4 E (V_0-E) 
           +(V_0-E)^2-2E(V_0-E)+E^2
           \right]\sinh(\dots)^2}
\end{align*}
}

%%\begin{centereq}
  \begin{equation}
    \label{eq:32}
    \Leftrightarrow T = \frac{4 E (V_0-E)}{4 E (V_0-E)+V_0^2 \sinh\left(\frac{2a}\hbar
        \sqrt{2 m (V_0 - E)} \right)^2}
  \end{equation}
%\end{centereq}

Grenzfall: \( \kappa a \gg 1 \) d.h. sehr hohe und sehr breite Schwelle.

\[ T = \exp \left(-\frac {4a}\hbar \sqrt{2m(V_0-E)} \right) \]

Für die \textbf{Reflexionsamplitude} gilt:
\[ \rho = \frac B A = \frac{-\frac i 2 \eta \sinh(2 \kappa a)
          \cdot e^{-2 i k a}}
        {\cosh(2 \kappa a)+\frac{i\epsilon}{2} \sinh(2 \kappa a)} \]

Damit können wir den Reflexionskoeffizienten bestimmen mit:
\begin{align*}
  R &= |\rho|^2 = \rho^*\rho \\
    &= \frac{\frac i 2 \eta \sinh(2 \kappa a)
          \cdot e^{2 i k a}}
        {\cosh(2 \kappa a)-\frac{i\epsilon}{2} \sinh(2 \kappa a)} \cdot
       \frac{-\frac i 2 \eta \sinh(2 \kappa a)
          \cdot e^{-2 i k a}}
        {\cosh(2 \kappa a)+\frac{i\epsilon}{2} \sinh(2 \kappa a)}
\end{align*}

%%\begin{centereq}
  \begin{equation}
    \label{eq:32Rkoef}
    \Leftrightarrow R = \frac{\frac 1 4 \eta^2 \sinh(2 \kappa a)^2}
    {1+\left(1+\frac {\epsilon^2} 4 \right) \sinh(2 \kappa a)}
  \end{equation}
%\end{centereq}


\subsubsection{Anwendung}
\begin{enumerate}
\item Kalte Emmision von Enektronen aus einem Metall im Magnetfeld (Raster Tunnel Mikroskop)
\item Radioaktiver Zerfall (\(\alpha\)-Strahlung) Ein \(\alpha\)-Teilchen im Kerninneren muss ein Coulomb-Potential überwinden (durch Tunneln).
\item Die Umkehrung des radioaktiven Zerfalls ist die Fussion ( bevorgzugt von Z=1 Kernen )
\item Josephson-Effekt, Tunneln von Cooper-Paaren in Supraleitern (Tunneln zwischen zwei durch eine Isolatorschicht getrennten Metallen)
\end{enumerate}

\subsection{Potentialtopf.}
Wir betrachten einen Potentialtopf mit einem Potential von $-V_0$ in dem
Bereich von $-a$ bis $a$ sonst Null. Siehe dazu Abb.~\ref{fig:7}
  
\begin{figure}[htbp]
  \begin{minipage}[h]{0.45\linewidth}
    \begin{align*}
      V(x)&=-V_0\cdot\Theta\left(a-|x|\right) \\
      &= \begin{cases}
        -V_0 & \mbox{ für }-a<x<a\\
        0 & \mbox{ sonst}
      \end{cases}
    \end{align*}
    \vspace{1cm}
  \end{minipage}
  \hfill
  \begin{minipage}[h]{0.5\linewidth}
    \input{sgl_potential_pics/Potentialtopf.pdf_t}
  \end{minipage}
  \caption{Potentialtopf.}
  \label{fig:7}
\end{figure}

Für die Bereiche I und III mit $E < V$ und $V=0$ gilt (\ref{eq:27b}):
\begin{equation}
  \left(\frac{d^2}{dx}-\kappa^2\right)\varphi(x)=0
   \quad\text{mit} \quad
  \kappa=\sqrt{\frac{2m}{\hbar^2}(V-E)}
      =\sqrt{\frac{2m}{\hbar^2}(-E)}\label{eq:56}
\end{equation}

Für den Bereich II mit $E > V$ und $V=-V_0$ gilt (\ref{eq:27a}):
\begin{equation}
  \left(\frac{d^2}{dx}+k^2\right)\varphi(x)=0
  \quad \text{mit} \quad
  k=\sqrt{\frac{2m}{\hbar^2}(E-V)}
   =\sqrt{\frac{2m}{\hbar^2}(E+V_0)}\label{eq:57}
\end{equation}

Damit erhalten wir folgende Lösungen:
\begin{align*}
  \varphi_\text{I}(x)&= C_1 e^{\kappa x}+D_1 e^{-\kappa x} \\
  \varphi_\text{III}(x)&= C_2 e^{\kappa x}+D_2 e^{-\kappa x}\\
   \varphi_\text{II}(x)&= A e^{i k x}+B e^{-i k x}
\end{align*}

\textbf{Randbedingungen:}
\begin{align*}
  \varphi_\text{I} &= 0 \text{ für } x \rightarrow -\infty \Rightarrow D_1=0&\\
  \varphi_\text{III} &= 0 \text{ für } x \rightarrow \infty \Rightarrow C_2=0&\\
\end{align*}

\textbf{Anschlußbedingung} bei $x=-a$ mit:
\begin{align*}
   &\varphi_\text{I}(x) = C_1 e^{\kappa x} \equiv C e^{\kappa x} \quad
   \Leftrightarrow \quad \frac {\text{d}}{\text{d}x} \varphi_\text{I}(x) 
    = \kappa C e^{\kappa x}\\
   &\varphi_\text{II}(x) = A e^{i k x}+B e^{-i k x} \quad
   \Leftrightarrow \quad  \frac {\text{d}}{\text{d}x}\varphi_\text{II}(x) 
    = ik Ae^{ikx}-ikBe^{-ikx}
\end{align*}
%Setze lokale Nummerierung der Gleichungen.
%\localeqncounter
%\resetlocaleqncounter %Der Zähler soll von 1 anfangen.
\begin{align}
  \varphi_\textrm{I} (-a) &= \varphi_\textrm{II}(-a)
   \Leftrightarrow  
  Ce^{-\kappa a} = Ae^{-ika}+Be^{ika} \label{eq3:pottopf1}\\
  \left.\frac{\text{d}\varphi_{\text{I}}}{\text{d}x}\right|_{x=-a}
  &=\left.\frac{\text{d}\varphi_{\text{II}}}{\text{d}x}\right|_{x=-a}
   \Leftrightarrow  
   \kappa Ce^{-\kappa a} = ik Ae^{-ika}-ik Be^{ika}\label{eq3:pottopf2}
\end{align}
Setzen wir nun \eqref{eq3:pottopf1'} mit:
\begin{equation}
  \label{eq3:pottopf1'}
   Be^{ika}=Ce^{-\kappa a} - Ae^{-ika}\tag{\ref{eq3:pottopf1}'}
\end{equation}
in Gleichung~ \eqref{eq3:pottopf2} ein, so erhalten wir:
\begin{align}
   \kappa Ce^{-\kappa a} 
   &= ik Ae^{-ika}
      -ikCe^{-\kappa a} - ikAe^{-ika}\label{eq3:pottopf3}\\
   &=  2ikAe^{-ika} -ikCe^{-\kappa a}\notag\\
   \Leftrightarrow  C(\kappa&+ik) e^{-\kappa a}
   = 2ikAe^{-ika}\notag\\[2ex]
   \Leftrightarrow A 
   = &\frac{C(\kappa+ik)e^{-\kappa a}}
           {2ike^{-ika}}
    =\frac{C(\kappa+ik)}{2ik}e^{(-\kappa +ik)a}\notag
\end{align}

Setzen wir nun analog \eqref{eq3:pottopf1''} mit:
\begin{equation}
  \label{eq3:pottopf1''}
   Ae^{-ika}=Ce^{-\kappa a} - Be^{ika}\tag{\ref{eq3:pottopf1}''}
\end{equation}
in Gleichung \eqref{eq3:pottopf2} ein, so erhalten wir:
\begin{align}
   \kappa Ce^{-\kappa a} 
   &= ikCe^{-\kappa a} - ikBe^{ika}-ikBe^{ika}\label{eq3:pottopf4}\\
   &= ikCe^{-\kappa a}-2ikBe^{ika}\notag\\
   \Leftrightarrow  C(\kappa&-ik) e^{-\kappa a}
   = -2ikBe^{ika}\notag\\[2ex]
   \Leftrightarrow B 
   = &\frac{-C(\kappa-ik)e^{-\kappa a}}
           {2ike^{ika}}
    =\frac{-C(\kappa-ik)}{2ik}e^{-(\kappa +ik)a}\notag
\end{align}

\textbf{Anschlußbedingung} bei $x=a$ mit:
\begin{align*}
   &\varphi_\text{II}(x) = A e^{i k x}+B e^{-i k x} \quad
   \Leftrightarrow \quad  \frac {\text{d}}{\text{d}x}\varphi_\text{II}(x) 
   = ik Ae^{ikx}-ikBe^{-ikx}\\
   &\varphi_\text{III}(x) = D_2 e^{-\kappa x} \equiv D e^{-\kappa x} \quad
   \Leftrightarrow \quad \frac {\text{d}}{\text{d}x} \varphi_\text{III}(x) 
   = -\kappa D e^{-\kappa x}
\end{align*}

\begin{align}
  \varphi_\textrm{II} (a) &= \varphi_\textrm{III}(a)
   \Leftrightarrow  
   Ae^{ika}+Be^{-ika}=De^{-\kappa a} \label{eq3:pottopf5} \\
  \left.\frac{\text{d}\varphi_{\text{II}}}{\text{d}x}\right|_{x=a}
  &=\left.\frac{\text{d}\varphi_{\text{III}}}{\text{d}x}\right|_{x=a}
   \Leftrightarrow  
   ik Ae^{ika}-ik Be^{-ika}=-\kappa D e^{-\kappa a} \label{eq3:pottopf6}
\end{align}

Setzen wir nun \eqref{eq3:pottopf5'} mit:
\begin{equation}
  \label{eq3:pottopf5'}
   Be^{-ika}=De^{-\kappa a} - Ae^{ika}
   \tag{\ref{eq3:pottopf5}'}
\end{equation}
in Gleichung~\eqref{eq3:pottopf6} ein, so erhalten wir:
\begin{align}
  &ikAe^{ika}-ikDe^{-\kappa a}+ikAe^{ika}
   =-\kappa De^{-\kappa a}\label{eq3:pottopf7}\\
  \Leftrightarrow &2ikAe^{ika}-ikDe^{-\kappa a}
   =-\kappa De^{-\kappa a}\notag\\
   \Leftrightarrow &2ikAe^{ika}
   =D(ik-\kappa)e^{-\kappa a}\notag\\[2ex]
   \Leftrightarrow  &A
   = \frac{D(ik-\kappa) e^{-\kappa a}}
          {2ike^{ika}}
   =\frac{D(ik-\kappa)}{2ik}e^{-(\kappa+ik)a}\notag
\end{align}

Setzen wir nun analog \eqref{eq3:pottopf5''} mit:
\begin{equation}
  \label{eq3:pottopf5''}
   Ae^{ika}=De^{-\kappa a} - Be^{-ika}
   \tag{\ref{eq3:pottopf5}''}
\end{equation}
in Gleichung~\eqref{eq3:pottopf6} ein, so erhalten wir:
\begin{align}
   &ikDe^{-\kappa a}-ikBe^{-ika}-ikBe^{-ika}
   =-\kappa De^{-\kappa a}\label{eq3:pottopf8}\\
  \Leftrightarrow &ikDe^{-\kappa a}-2ikBe^{-ika}
   =-\kappa De^{-\kappa a}\notag\\
   \Leftrightarrow &2ikBe^{-ika}
   =D(ik+\kappa)e^{-\kappa a}\notag\\[2ex]
   \Leftrightarrow & B
   = \frac{D(ik+\kappa) e^{-\kappa a}}
          {2ike^{-ika}}
   =\frac{D(ik+\kappa)}{2ik}e^{(-\kappa+ik)a}\notag
\end{align}

Nun dividieren wir die Gleichung~\eqref{eq3:pottopf3} durch \eqref{eq3:pottopf4}
und erhalten:
\begin{align}
    \Rightarrow
    \frac{\eqref{eq3:pottopf3}}{\eqref{eq3:pottopf4}} 
    = \frac A B
    &=\frac{C(\kappa+ik)e^{(-\kappa +ik)a}}{2ik} \cdot
    \frac{(-2)ik}{C(\kappa-ik)e^{-(\kappa+ik)a}}\label{eq3:pottopf9}\\
    &=-\frac{(\kappa+ik)e^{ika}}{(\kappa-ik)e^{-ika}}
    =-\frac{(\kappa+ik)}{(\kappa-ik)}e^{2ika}\notag
\end{align}

Analog dividieren wir die Gleichung~\eqref{eq3:pottopf7} durch
\eqref{eq3:pottopf8} und erhalten:
\begin{align}
    \Rightarrow
    \frac{\eqref{eq3:pottopf7}}{\eqref{eq3:pottopf8}} 
    = \frac A B
    &=\frac{D(ik-\kappa)e^{-(\kappa +ik)a}}{2ik} \cdot
    \frac{2ik}{D(ik+\kappa)e^{(-\kappa+ik)a}}\label{eq3:pottopf10}\\
    &=\frac{(ik-\kappa)e^{-ika}}{(ik+\kappa)e^{ika}}
    =\frac{(ik-\kappa)}{(ik+\kappa)}e^{-2ika}\notag
\end{align}
Zum Schluss setzen wir die soeben erhaltenen Gleichungen~\eqref{eq3:pottopf9}
und \eqref{eq3:pottopf10} gleich und bekommen:
\begin{align*}
  \eqref{eq3:pottopf9}=\eqref{eq3:pottopf10}
  &\Leftrightarrow \frac A B = \frac A B\\
   &\Leftrightarrow -\frac{(\kappa+ik)}{(\kappa-ik)}e^{2ika}
      = \frac{(ik-\kappa)}{(ik+\kappa)}e^{-2ika}\\
   &\Leftrightarrow \frac{(\kappa+ik)}{(\kappa-ik)}e^{2ika}
      = \frac{(\kappa-ik)}{(\kappa+ik)}e^{-2ika}\\
   &\Leftrightarrow e^{2ika}
   = \frac{(\kappa-ik)^2}{(\kappa+ik)^2}e^{-2ika}
   \qquad \left|\;\cdot e^{2ika}\right.\\
   &\Leftrightarrow e^{4ika}
   = \frac{(\kappa-ik)^2}{(\kappa+ik)^2}
\end{align*}
%Wieder auf den globalen Gleichungszähler zurücksetzen.
%\globaleqncounter
Somit erhalten wir schlussendlich folgende Beziehung:
%%\begin{centereq}
\begin{equation}
  \label{eq:33}
  \frac{(\kappa-ik)^2}{(\kappa+ik)^2}=e^{4ika}
\end{equation}
%\end{centereq}
Ziehen wir in der Gleichung (\ref{eq:33}) auf beiden Seiten die Wurzel so
erhalten wir zwei Lösungen:
%%\begin{centereq}
\begin{equation}
  \label{eq:34}
  \boxed{\frac{\kappa-ik}{\kappa+ik}=\pm e^{2ika}}
\end{equation}
%\end{centereq}

Wir wollen nun untersuchen welche Symmetrieeigenschaften diese beiden Lösungen
erfüllen. Wir werden feststellen das der Term $-e^{2ika}$ eine symmetrische Lösung
und der Term $+e^{2ika}$ eine antisymmertrische Lösung der Wellenfunktion
darstellt. 
Um dies zu demonstrieren machen wir eine kleine Rechnung.

Nehmen wir die Gl.~\eqref{eq3:pottopf7} und setzen dort die Lösung:
\[ \frac{\kappa-ik}{\kappa+ik}=- e^{2ika} \]
ein. So erhalten wir:
\begin{equation*}
  \frac A B =\frac{(ik-\kappa)}{(ik+\kappa)}e^{-2ika}
      =-\frac{(\kappa-ik)}{(ik+\kappa)}e^{-2ika}
      = e^{2ika}\cdot e^{-2ika}
      =1
\end{equation*}
\[ \Leftrightarrow A = B \] Setzen wir nun $A$ Gl.~\eqref{eq3:pottopf3} und $B$
Gl.~\eqref{eq3:pottopf8} in die soeben erhaltene Beziehung zwischen $A$ und $B$
ein, so können wir eine weitere Beziehung feststellen.
\begin{align*}
  A &= B \\
  \frac{C(\kappa+ik)}{2ik}e^{(-\kappa +ik)a} 
  &=\frac{D(ik+\kappa)}{2ik}e^{(-\kappa+ik)a}\\
  C &= D
\end{align*}

Somit erhielten wir zwei Beziehungen $A=B$ und $C=D$. Setzen wir die nun in
unsere Wellenfunktion ein, so erhalten wir:
\begin{align*}
  \phi(x) &= \begin{cases}
  Ce^{\kappa x}\\
  Ae^{ikx}+Be^{-ikx}\\
  De^{-\kappa x}
\end{cases}
= \begin{cases}
  De^{\kappa x}\\
  A\left(e^{ikx}+e^{-ikx}\right)\\
  De^{-\kappa x}
\end{cases}\\  
&=\phi(-x) = \begin{cases} 
 De^{-\kappa x}\\
  A\left(e^{-ikx}+e^{ikx}\right)\\
  De^{\kappa x} 
\end{cases}
\end{align*}
Somit gilt $\phi(x)=\phi(-x)$ und damit ist $\phi$ eine symmetrische Funktion.

Nun wollen wir noch den zweiten Fall, den antisymmetrischen Fall betrachten.
Dazu nehmen wir analog die Gl.~\eqref{eq3:pottopf7} und setzen dort die Lösung:
\[ \frac{\kappa-ik}{\kappa+ik}= +e^{2ika} \]
ein. So erhalten wir:
\begin{equation*}
  \frac A B =\frac{(ik-\kappa)}{(ik+\kappa)}e^{-2ika}
      =-\frac{(\kappa-ik)}{(ik+\kappa)}e^{-2ika}
      = -e^{2ika}\cdot e^{-2ika}
      =-1
\end{equation*}
\[ \Leftrightarrow A = -B \] Analog setzen wir nun $A$ Gl.~\eqref{eq3:pottopf3}
und $B$ Gl.~\eqref{eq3:pottopf8} in die soeben erhaltene Beziehung zwischen $A$
und $B$ ein und erhalten eine Beziehung für $C$ und $D$.
\begin{align*}
  A &= -B \\
  \frac{C(\kappa+ik)}{2ik}e^{(-\kappa +ik)a} 
  &=-\frac{D(ik+\kappa)}{2ik}e^{(-\kappa+ik)a}\\
  C &= -D
\end{align*}
Setzen wir nun die beiden Beziehungen $A=-B$ und $C=-D$ in unsere Wellenfunktion
eine so erhalten wir:
\begin{align*}
  \phi(x) &= \begin{cases}
  Ce^{\kappa x}\\
  Ae^{ikx}+Be^{-ikx}\\
  De^{-\kappa x}
\end{cases}
= \begin{cases}
  -De^{\kappa x}\\
  A\left(e^{ikx}-e^{-ikx}\right)\\
  De^{-\kappa x}
\end{cases}\\  
&=-\phi(-x) = -\begin{cases} 
 De^{-\kappa x}\\
  A\left(e^{ikx}-e^{-ikx}\right)\\
  -De^{\kappa x} 
\end{cases}
\end{align*}
Somit gilt $\phi(x)=-\phi(-x)$ und damit ist $\phi$ eine antisymmetrische
Funktion.

Nachdem wir nun die Symmetrieeigenschaften von (\ref{eq:34}) untersucht haben,
wollen wir versuchen diese Gleichung zu lösen. Leider lässt sich die Gl.
\eqref{eq:34} nur nummerisch oder graphisch lösen. Wir werden uns hier mit der
graphischen Lösung begnügen müssen. Wir betrachten nun separat die beiden
möglichen Lösungen der Gleichung.

\textbf{ Fall 1:} Die symmetrische Lösung.\\
Wie soeben besprochen liefert uns die Gl. (\ref{eq:34}) mit dem Term $-e^{2ika}$
eine symmetrische Lösung der Wellenfunktion. Somit gilt es folgende Gleichung zu
lösen:
%%\begin{centereq}
\begin{equation}
  \label{eq:35}
  \frac{\kappa-ik}{\kappa+ik}=-e^{2ika}
\end{equation}
%\end{centereq}
Zuerst versuchen wir das Minuszeichen auf der rechten Seite zu eliminieren:
\begin{equation*}
  \frac{\kappa-ik}{\kappa+ik}=-e^{2ika}
  \Leftrightarrow  
 \frac{-(-\kappa+ik)}{\kappa+ik}=-e^{2ika}
\end{equation*}
\begin{equation}
  \label{eq:36}
  \Leftrightarrow \frac{-\kappa+ik}{\kappa+ik}=e^{2ika}
\end{equation}

Als nächstes definieren wir zwei komplexe Zahlen: 
\begin{align*}
  &u=-\kappa+ik = \text{Zähler der linken Seite von (\ref{eq:36})}\\
  &v=\kappa+ik = \text{Nenner der linken Seite von (\ref{eq:36})}
\end{align*}

Nun wollen wir versuchen die beiden komplexen Zahlen $u$ und $v$ in die Eulerische
Darstellung zu überführen. So dass sie folgende Form erhalten:
\begin{subequations}
\begin{align}
  u&=|u|e^{i\varphi_u}\label{eq:37a}\\
  v&=|v|e^{i\varphi_v}\label{eq:37b}
\end{align}
\end{subequations}

Dazu betrachten wir die Zahlen in der komplexen Ebene. Siehe dazu
Abb.~\ref{fig:8}

\begin{figure}[htbp]
  \begin{center}
    \input{sgl_potential_pics/Komplexe_Ebene.pdf_t}
    \caption{$u$ und $v$ in der komplexen Ebene.}
    \label{fig:8}
  \end{center}
\end{figure}

Die Beträge der beiden Zahlen lassen sich relativ leicht bestimmen. Für sie gilt:
\begin{subequations}
  \begin{align}
    |u|&=\sqrt{\text{Re}^2+\text{Im}^2}=\sqrt{\kappa^2+k^2}\label{eq:38a}\\
    |v|&=\sqrt{\text{Re}^2+\text{Im}^2}=\sqrt{\kappa^2+k^2}\label{eq:38b}   
  \end{align}
\end{subequations}
Um nun die dazugehörende Winkel $\varphi_u$ und $\varphi_v$ zu bestimmen ist es
günstiger wenn man von der Imaginärachse ausgeht, die bei $\pi/2$ liegt. Es ist
deshalb günstiger, weil sie die Symmetrieachse zwischen den beiden Zahlen bildet.
Wir erhalten somit:
\begin{subequations}
  \begin{align}
    \varphi_u=\frac \pi 2 + \arctan \left(\frac \kappa k \right)\label{eq:39a}\\
    \varphi_v=\frac \pi 2 - \arctan \left(\frac \kappa k \right)\label{eq:39b}
  \end{align}
\end{subequations}
Setzen wir nun die Gleichungen (\ref{eq:38a}) und (\ref{eq:39a}) in
(\ref{eq:37a}) bzw. die Gleichungen (\ref{eq:38b}) und (\ref{eq:39b}) in
(\ref{eq:37b}) ein, so erhalten wir:
\begin{subequations}
\begin{align}
  u&=\sqrt{\kappa^2+k^2}\cdot e^
       {i\left(\frac \pi 2 + \arctan \left(\frac \kappa k \right)\right)}
       \label{eq:40a}\\
  v&=\sqrt{\kappa^2+k^2}\cdot e^
       {i\left(\frac \pi 2 - \arctan \left(\frac \kappa k \right)\right)}
       \label{eq:40b}
\end{align}
\end{subequations}
Setzen wir nun die beiden frischerhaltenen Gleichungen (\ref{eq:40a}) und
(\ref{eq:40b}) in die zu lösende Gleichung~(\ref{eq:36}) und formen sie etwas um
so erhalten wir:
\begin{align*}
  e^{2ika} &=\frac{-\kappa+ik}{\kappa+ik} \equiv \frac u v
  =\frac
  {\sqrt{\kappa^2+k^2}\cdot e^
       {i\left(\frac \pi 2 + \arctan \left(\frac \kappa k \right)\right)}}
  {\sqrt{\kappa^2+k^2}\cdot e^
       {i\left(\frac \pi 2 - \arctan \left(\frac \kappa k \right)\right)}}\\
  &=\frac
  {e^{i\frac \pi 2}\cdot e^{i \arctan \left(\frac \kappa k \right)}}
  {e^{i\frac \pi 2}\cdot e^{-i \arctan \left(\frac \kappa k \right)}}
   =e^{2 i \arctan \left(\frac \kappa k \right)}\\
    \Leftrightarrow &
   2 i ka= 2 i \arctan \left(\frac \kappa k \right)
    \Leftrightarrow
   ka=\arctan \left(\frac \kappa k \right)
\end{align*}
\begin{equation}
  \label{eq:41}
  \Leftrightarrow \frac \kappa k = \tan(ka)
\end{equation}
Um die Gl.~(\ref{eq:41}) grafisch darstellen zu können führen wir ein weitere
Größe $k_0$ ein, mit:
\begin{equation}
  \label{eq:42}
  k_0=\sqrt{\kappa^2+k^2}
\end{equation}
Formen wir (\ref{eq:42}) nach $\kappa$ um und setzen es in die Gl.~(\ref{eq:41})
ein so erhalten wir:
\begin{equation*}
   k_0^2=\kappa^2+k^2 \Leftrightarrow  \kappa^2=k_0^2-k^2
                      \Leftrightarrow  \kappa=\sqrt{k_0^2-k^2} \text{  in  }
                                        (\ref{eq:41}) 
\end{equation*}
\begin{equation}
  \label{eq:43}
  \Leftrightarrow \frac {\sqrt{k_0^2-k^2}} k = \tan(ka)
\end{equation}

Da $\frac {\sqrt{k_0^2-k^2}} k$ eine positive Größe, ist muss $\tan(ka)$
ebenfalls positiv sein. D.h. es gibt nur dann eine Lösung, wenn die Bedingung:
\begin{equation}
  \label{eq:44}
  \tan(ka) > 0
\end{equation}
erfüllt ist.
Formen wir die Gleichung~(\ref{eq:43}) etwas um so erhalten wir:
\begin{align*}
  \frac {\sqrt{k_0^2-k^2}} k &= \tan(ka)
   \Leftrightarrow \frac {k_0^2-k^2} {k^2} = \tan(ka)^2\\
   \Leftrightarrow \frac {k_0^2} {k^2} &=1+\tan(ka)^2
    =1+\frac {\sin(ka)^2} {\cos(ka)^2}
    =1+\frac {1-\cos(ka)^2} {\cos(ka)^2}\\
    &=1+\frac 1 {\cos(ka)^2} - 1 = \frac 1 {\cos(ka)^2}
\end{align*}
\begin{equation}
  \label{eq:45}
   \Leftrightarrow \frac {k^2} {k_0^2} = \cos(ka)^2
\end{equation}

Aus der Gleichung~(\ref{eq:45}) erhalten wir eine weitere Bedingung:
\begin{equation}
  \label{eq:46}
   \frac {k} {k_0} = |\cos(ka)|
\end{equation}

Wir definieren zwei Hilfsfunktionen:
\begin{align*}
  f(k) &= |\cos(ka)|\\
  g(k) &= \frac 1 {k_0} \cdot k
\end{align*}

Zeichnen wir nun die beide Funktionen $f(k)$ und $g(k)$ in ein Diagramm, dann
liefern uns die Schnittpunkte der beiden Funktionen mögliche Lösungen für die
Gl.~(\ref{eq:41}). Warum nur mögliche Lösungen? Weil hier noch die Bedingung
(\ref{eq:44}) beachtet werden muss. D.h. nur die Schnittpunkte bieten eine
Lösung, bei denen die Bedingung $\tan(ka)>0$ erfüllt ist. Siehe dazu Abb.~\ref{fig:9}

\begin{figure}[!ht]
%\begin{figure}[t]
%\begin{figure}[htbp]
  \begin{center}
    \input{sgl_potential_pics/Potentialtopf-lsg-cos.pdf_t}
    \caption{Graphische Lösung für symmetrische Zustände mit: 
      Lsg.1 $\approx 0,433\cdot \frac \pi a $,  
      Lsg.2 $\approx 1,3\cdot \frac \pi a $ und  
      Lsg.3 $\approx 2,1\cdot \frac \pi a $ }
    \label{fig:9}
  \end{center}
\end{figure}

 \textbf{Fall 2:} Die antisymmetrische Lösung.\\
Nachdem wir die symmetrischen Lösungen der Gl~(\ref{eq:34}) bzw. 
Gl.~(\ref{eq:33})  gefunden haben wollen wir jetzt auch noch der
Vollständigkeits halber die antisymmetrischen Lösungen d.h. Lösungen für
$+e^{2ika}$ der Gl.~(\ref{eq:34}) finden. Somit gilt es folgende Gleichung zu
lösen:
%%\begin{centereq}
\begin{equation}
  \label{eq:47}
  \frac{\kappa-ik}{\kappa+ik}=e^{2ika}
\end{equation}
%\end{centereq}

Auch hier bildet der Nenner und der Zähler der linken Seite, der
Gleichung~(\ref{eq:47}) jeweils eine komplexe Zahl. Wobei wir hier den
Spezialfall haben, dass der Zähler die konjugiertkomplexe des Nenners ist. Der
Nenner ist jedoch der gleiche denn wir schon im Fall~1 hatten, den wir mit $v$
bezeichnet haben. D.h. wir können schreiben:
\begin{align*}
  v^*&=\kappa-ik = \text{Zähler der linken Seite von (\ref{eq:47})}\\
  v&=\kappa+ik = \text{Nenner der linken Seite von (\ref{eq:47})}
\end{align*}

Nun wollen wir versuchen $v$ in die Eulerische Darstellung zu überführen. So
dass wir folgende Form erhalten:
\begin{subequations}
\begin{align}
  v^*&=|v|e^{-i\varphi_v}\label{eq:48a}\\
  v&=|v|e^{i\varphi_v}\label{eq:48b}
\end{align}
\end{subequations}

Der Betrag von $v$ bleibt der gleiche wie in Fall 1 Gl.~(\ref{eq:38b})

Für den Winkel $\varphi_v$ gilt laut Abb.~\ref{fig:8}:
\begin{equation}
  \label{eq:49}
  \varphi_v=\arctan\left(\frac k \kappa \right)
\end{equation}

Einsetzen von (\ref{eq:38b}) und (\ref{eq:49}) in (\ref{eq:48a}) bzw. in
(\ref{eq:48b}) so erhalten wir:
\begin{subequations}
\begin{align}
  v^*&=\sqrt{\kappa^2+k^2}\cdot e^{-i\arctan\left(\frac k \kappa \right)}\label{eq:50a}\\
  v&=\sqrt{\kappa^2+k^2}\cdot e^{i\arctan\left(\frac k \kappa \right)}\label{eq:50b}
\end{align}
\end{subequations}

Setzen wir nun die beiden frischerhaltenen Gleichungen (\ref{eq:50a}) und
(\ref{eq:50b}) in die zu lösende Gleichung~(\ref{eq:47}) und formen sie etwas um
so erhalten wir:
\begin{align*}
  e^{2ika} &=\frac{\kappa-ik}{\kappa+ik} \equiv \frac {v^*} v
  =\frac
  {\sqrt{\kappa^2+k^2}\cdot e^{-i\arctan \left(\frac k \kappa \right)}}
  {\sqrt{\kappa^2+k^2}\cdot e^{i\arctan \left(\frac k \kappa \right)}}\\
  &=\frac
  {e^{-i \arctan \left(\frac k \kappa \right)}}
  {e^{i \arctan \left(\frac k \kappa \right)}}
   =e^{-2 i \arctan \left(\frac k \kappa \right)}\\
    \Leftrightarrow &
   2 i ka= -2 i \arctan \left(\frac k \kappa \right)
    \Leftrightarrow
   -ka=\arctan \left(\frac k \kappa \right)
\end{align*}
\begin{equation}
  \label{eq:51}
  \Leftrightarrow \frac k \kappa = \tan(-ka)= -\tan(ka)
\end{equation}

Auch hier wollen wir die linke Seite der Gl.~(\ref{eq:51}) durch $k_0$
(\ref{eq:42}) ausdrücken. Wir erhalten:
\begin{equation*}
   k_0^2=\kappa^2+k^2 \Leftrightarrow  \kappa^2=k_0^2-k^2
                      \Leftrightarrow  \kappa=\sqrt{k_0^2-k^2} \text{  in  }
                                        (\ref{eq:51})
\end{equation*}
\begin{equation}
  \label{eq:52}
  \Leftrightarrow \frac k {\sqrt{k_0^2-k^2}} = -\tan(ka)
\end{equation}

Damit erhalten wir eine erste Bedingung. Da $\frac k {\sqrt{k_0^2-k^2}}$ eine
positive Größe ist, muss $\tan(ka)$ negativ sein. D.h. es muss gelten:
\begin{equation}
  \label{eq:53}
  \tan(ka) < 0
\end{equation}

Formen wir die Gleichung~(\ref{eq:52}) noch etwas um, so erhalten wir:
\begin{align*}
  \frac k {\sqrt{k_0^2-k^2}} &= -\tan(ka) = -\frac {\sin(ka)} {\cos(ka)}
   \Leftrightarrow \frac {k^2}{k_0^2-k^2} = \frac {\sin(ka)^2} {\cos(ka)^2}\\
   \Leftrightarrow \frac {k_0^2-k^2} {k^2} &= \frac {\cos(ka)^2}{\sin(ka)^2}
    \Leftrightarrow \frac {k_0^2}{k^2}-\frac{k^2}{k^2} = \frac {\cos(ka)^2}{\sin(ka)^2}\\
   \Leftrightarrow \frac {k_0^2} {k^2} &=1+\frac {\cos(ka)^2}{\sin(ka)^2}
    =1+\frac {1-\sin(ka)^2} {\sin(ka)^2}
    = \frac 1 {\sin(ka)^2}
\end{align*}
\begin{equation}
  \label{eq:54}
   \Leftrightarrow \frac {k^2} {k_0^2} = \sin(ka)^2
\end{equation}

Aus Gleichung~(\ref{eq:54}) erhalten wir eine weitere Bedingung:
\begin{equation}
  \label{eq:55}
  \frac {k} {k_0} = |\sin(ka)|
\end{equation}

Definieren wir analog zum Fall~1 zwei Hilfsfunktionen:
\begin{align*}
  f(k) &= |\sin(ka)|\\
  g(k) &= \frac 1 {k_0} \cdot k
\end{align*}

Zeichnen wir nun die beide Funktionen $f(k)$ und $g(k)$ in ein Diagramm, dann
liefern uns die Schnittpunkte der beiden Funktionen mögliche Lösungen für die
Gl.~(\ref{eq:55}). Warum nur mögliche Lösungen? Weil hier noch die Bedingung
(\ref{eq:53}) beachtet werden muss. D.h. nur die Schnittpunkte bieten eine
Lösung, bei denen die Bedingung $\tan(ka)<0$ erfüllt ist. Siehe dazu Abb.~\ref{fig:10}

\begin{figure}[!ht]
  \begin{center}
    \input{sgl_potential_pics/Potentialtopf-lsg-sin.pdf_t}
    \caption{Graphische Lösung für antisymmetrische Zustände mit:
      Lsg.1 $\approx 0,866\cdot \frac \pi a $ und  
      Lsg.2 $\approx 1,733\cdot \frac \pi a $}
    \label{fig:10}
  \end{center}
\end{figure}

Wie aus den Abb.~\ref{fig:9} und Abb.~\ref{fig:10} ersichtlich ist haben wir
insgesamt fünf Lösungen. Drei davon mit einer symmetrischen Zustandsfunktion und
zwei mit einer antisymmetrischen Zustandsfunktion. Da zu jeder Zustandsfunktion
ein Energieeigenwert gehört, haben wir somit auch fünf diskrete
Energieeigenwerte. Daran sieht man auch schön, dass die Energie in einem
Potentialtopf quantisiert ist. Dabei entspricht erste Energieeigenwert mit $n=0$
und somit der niedrigster Energie (Grundzustand) der ersten Lösung aus
Abb.~\ref{fig:9}, der zweiter Energieeigenwert mit $n=1$ (erster angeregter
Zustand) der ersten Lösung aus der Abb.~\ref{fig:10}, der nachfolgender
Energieeigenwert mit $n=2$ (zweiter angeregter Zustand) ist wieder ein
symmetrischer Zustand d.h. zweite Lösung aus der Abb.~\ref{fig:9} usw...

Wir wollen nun die einzelnen Energieniveaus angeben. Dazu
stellen wir die Gl.~\eqref{eq:57} nach $E$ um und erhalten:
\begin{align}
  &k=\sqrt{\frac{2m}{\hbar^2}(E+V_0)}
  \Leftrightarrow k^2=\frac{2m}{\hbar^2}(E+V_0)
  \Leftrightarrow k^2-\frac{2m}{\hbar^2}V_0 = \frac{2m}{\hbar^2} E\notag\\
  \Leftrightarrow &E=\frac{\hbar^2}{2m}\cdot k^2-V_0\label{eq:37}
\end{align}

Setzen wir jetzt die Lösungen aus den Abb.~\ref{fig:9} und Abb.~\ref{fig:10} in
unsere Gl.~\eqref{eq:37} ein, so erhalten wir folgende Energieniveaus:\\

\begin{tabular}{lll}
\textbf{Energieniveau} & \textbf{Energie} & \textbf{Zustand}\\[2ex] 
  \( n=0 \) & \(E\approx 0,433^2\cdot\frac{\hbar^2\pi^2}{2ma^2}-V_0\) &
  symmetrisch.\\[2ex]
  \( n=1 \) & \(E\approx 0,866^2\cdot\frac{\hbar^2\pi^2}{2ma^2}-V_0\) & 
  antisymmetrisch.\\[2ex]
  \( n=2 \) & \(E\approx 1,300^2\cdot\frac{\hbar^2\pi^2}{2ma^2}-V_0\) &
  symmetrisch.\\[2ex]
  \( n=3 \) & \(E\approx 1,733^2\cdot\frac{\hbar^2\pi^2}{2ma^2}-V_0\) & 
  antisymmetrisch.\\[2ex]
  \( n=4 \) & \(E\approx 2,100^2\cdot\frac{\hbar^2\pi^2}{2ma^2}-V_0\) &
  symmetrisch.
\end{tabular}
\\
\\

Wie wir ebenfalls aus den Abbildungen~\ref{fig:9} und \ref{fig:10} entnehmen
können, haben wir um so mehr gebundenen Zustände je größer $k_0$ ist. Doch was ist
$k_0$? Nehmen wir nochmal die Gl.~\eqref{eq:42} und setzen dort die
Gl.~\eqref{eq:56} und Gl.~\eqref{eq:57} ein, so erhalten wir:
\begin{align}
  k_0&=\sqrt{\kappa^2+k^2}\notag\\
  &=\sqrt{\frac{2m}{\hbar^2}(-E)+\frac{2m}{\hbar^2}(E+V_0)}
   = \sqrt{-\frac{2m}{\hbar^2}E+\frac{2m}{\hbar^2}E+\frac{2m}{\hbar^2}V_0}\notag\\
  \Rightarrow k_0&=\sqrt{\frac{2m}{\hbar^2}\cdot V_0} \label{eq:58}
\end{align}
Die Gl.~\eqref{eq:58} besagt dass $k_0$ proportional zum Potential $V_0$ ist.
D.h. wenn wir ein unendlich tiefen Potentialtopf haben mit $V_0 \rightarrow
\infty$ dann haben wir auch unendlich viele Lösungen. Dies wollen wir nun
genauer untersuchen.\\

\textbf{Grenzwertbetrachnung} für $V_0\rightarrow \infty$\\
Wie wir gerade festgestellt haben ist $k_0$ proportional zu $V_0$ d.h. es gilt:
\begin{equation}
  \label{eq:38}
  k_0 \rightarrow \infty \quad \text{für} \quad V_0 \rightarrow \infty
  \quad \text{und} \quad \frac k {k_0} \rightarrow 0
\end{equation}
Setzen wir die Beziehung \eqref{eq:38} in die Gleichung~\eqref{eq:46} und
\eqref{eq:55} ein, so erhalten wir:
\begin{subequations}
  \begin{align}
    |\cos(ka)| &= 0\label{eq:41a}\\
    |\sin(ka)| &= 0\label{eq:41b}
  \end{align}
\end{subequations}

Aus der Gleichung~\eqref{eq:41a} erhalten wir eine Beziehung für $k$ bei
symmetrischen Zuständen:
\begin{equation}
  \label{eq:39}
  ka=(2n+1)\frac \pi 2
  \Leftrightarrow k = (2n+1)\frac \pi {2a} 
  \quad \text{mit } n=0,1,2,3\dots
\end{equation}
Analog erhalten wir aus Gleichung~\eqref{eq:41b} eine Beziehung für $k$ bei
antisymmetrischen Zuständen:
\begin{equation}
  \label{eq:40}
  ka=n \cdot \pi \Leftrightarrow k = \frac {n\pi} {a} 
  \quad \text{mit } n=1,2,3,4\dots
\end{equation}
Wir können auch die beiden Folgen aus Gl.~\eqref{eq:39} und \eqref{eq:40} zu
einer einzigen Folge zusammenfassen:
\begin{subequations}
\begin{align}
  \text{Aus \eqref{eq:39} folgt: }
  k &= \frac 1 2\frac {\pi} a,\; \frac 3 2 \frac {\pi} a,\; 
  \frac 5 2\frac {\pi} a,\; \frac 7 2 \frac {\pi} a,\; \dots\label{eq:49a}\\
  \text{Aus \eqref{eq:40} folgt: }
  k &= 1 \frac {\pi} a,\; 2 \frac {\pi} a,\; 
  3\frac {\pi} a,\; 4 \frac {\pi} a,\; \dots\notag\\
  &= \frac 2 2\frac {\pi} a,\; \frac 4 2 \frac {\pi} a,\; 
  \frac 6 2\frac {\pi} a,\; \frac 8 2 \frac {\pi} a,\; \dots\label{eq:49b}
\end{align}
\end{subequations}
Die Folgen \eqref{eq:49a} und \eqref{eq:49b} zusammengefasst ergibt:
\begin{equation*}
  k =  \frac 1 2\frac {\pi} a,\;  \frac 2 2\frac {\pi} a,\; 
  \frac 3 2\frac {\pi} a,\;  \frac 4 2\frac {\pi} a,\;
  \frac 5 2\frac {\pi} a,\;  \frac 6 2\frac {\pi} a,\;
  \frac 7 2\frac {\pi} a,\;  \frac 8 2\frac {\pi} a,\; \dots
\end{equation*}
Dies können wir auch verkürzt schreiben mit:
\begin{equation}
  \label{eq:48}
  k=\frac n 2 \frac \pi a \quad \text{mit } n=1, 2, 3, 4, \dots
\end{equation}

Setzen wir nun \eqref{eq:48} in die Gl.~\eqref{eq:37} ein, so erhalten wir:
\begin{equation}
  \label{eq:50}
  E=\frac{\hbar^2 \pi^2}{8ma^2}\cdot n^2-V_0  
\end{equation}

Die erhaltene Gl.~\eqref{eq:50} hat leider einen Schönheitsfehler und zwar
wenn wir nun für $V_0=-\infty$ einsetzen, was wir ja angenommen haben, dann
erhalten wir für die Energie im Grundzustand $-\infty$. Dies ist zwar
mathematische korrekt, denn wir befinden uns ja auf Nullniveau und schauen in
einen unendlich tiefen Potentialtopf und da die Grundzustandsenergie sich auf
dem Boden des Potentialtopfs befinden, befindet sie sich bei $-\infty$. Doch
physikalisch gesehen hat dieser Umstand keine große Aussagekraft. Deswegen ist
es besser wenn wir uns auf den Boden des Potentialtopfs begeben und von dort aus
die einzelnen Energieniveaus betrachten. Es gilt siehe Abb.~\ref{fig:7}:
\begin{equation}
  \label{eq:59}
  E_n = E - (-V_0) \Leftrightarrow E_n= E+V_0
\end{equation}
Einsetzen von \eqref{eq:50} in \eqref{eq:59} ergibt:
\begin{flalign*}
  &E_n=\frac{\hbar^2 \pi^2}{8ma^2}\cdot n^2-V_0+V_0&
\end{flalign*}
%%\begin{centereq}
  \begin{equation}
    \label{eq:60}
    \Rightarrow \boxed{E_n = \frac{\hbar^2 \pi^2}{8ma^2}\cdot n^2}
  \end{equation}
%%\end{centereq}

Die Gl.~\eqref{eq:60} beschreibt das Spektrum eines eindimensionalen
Potentialtopfs mit der Breite $2a$ und unendlich hohen Wänden.



\end{document}
