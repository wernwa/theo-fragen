\documentclass[12pt,a4paper,titlepage,oneside]{article}
%\usepackage[latin1]{inputenc}
\usepackage[utf8]{inputenc}
%\usepackage[T1]{fontenc}
%\usepackage{ngerman}
\usepackage[ngerman]{babel}
%\usepackage{bbold}          %Für den Einheitsoperator \mathbb 1
\usepackage{dsfont}          %Für den Einheitsoperator \mathds 1
\usepackage[fleqn]{amsmath}  % Sehr wichtiges Paket, mit neuen Umgebungen.
\usepackage{amsfonts}        % brauche ich für die Mengensymbole
\usepackage{amssymb}
%\usepackage{MnSymbol}       % für \rhookswarrow, \lhooksearrow
\usepackage[makeroom]{cancel}% Variablen durchstreichen.
%\usepackage{lastpage}       % zeigt die Gesamtseitenzahl an
%\usepackage[showlabels,sections,floats,textmath,displaymath]{preview}
%\usepackage{fancyhdr}
\usepackage{color}	     % Für farbige Bilder.
%\usepackage{epsfig}
\usepackage{graphicx}        % um Graphiken einzubinden
\graphicspath{{zeichnungen/}}% Suchpfad für Graphiken.

\definecolor{darkblue}{rgb}{0,0,.5}
\usepackage[colorlinks=true,
        linkcolor=darkblue,
        citecolor=darkblue,
        filecolor=darkblue,
        pagecolor=darkblue,
        urlcolor=darkblue,
        bookmarks=true,
        bookmarksopen=true,
        bookmarksopenlevel=3,
        plainpages=false,
        pdfpagelabels=true]{hyperref}

%-------------------------------------------------------------------------------
%Bewirkt das die erste Zeile eines neuen Absatzes nicht eingerückt wird.
%Stattdessen werde die Absätze durch einen vertikalen Abstand voneinander
%getrennt.
\setlength{\parindent}{0pt}
\setlength{\parskip}{5pt plus 2pt minus 1pt}

%-------------------------------------------------------------------------------
%Diff-Makro:
%Das Diff-Makro stellt einen Differentialoperator da.
%
%Benutzung:
% \diff  ->  d
% \diff f  ->  df
% \diff^2 f  ->  d^2 f
% \diff_x  ->  d/dx
% \diff^2_x  ->  d^2/dx^2
% \diff f_x  ->  df/dx
% \diff^2 f_x  ->  d^2f/dx^2
% \diff^2{f(x^5)}_x  ->  d^2(f(x^5))/dx^2
%
%Ersetzt man \diff durch \pdiff, so wird der partieller
%Differentialoperator dargestellt.
%
\makeatletter
\def\diff@n^#1{\@ifnextchar{_}{\diff@n@d^#1}{\diff@n@fun^#1}}
\def\diff@n@d^#1_#2{\frac{\textrm{d}^#1}{\textrm{d}#2^#1}}
\def\diff@n@fun^#1#2{\@ifnextchar{_}{\diff@n@fun@d^#1#2}{\textrm{d}^#1#2}}
\def\diff@n@fun@d^#1#2_#3{\frac{\textrm{d}^#1 #2}{\textrm{d}#3^#1}}
\def\diff@one@d_#1{\frac{\textrm{d}}{\textrm{d}#1}}
\def\diff@one@fun#1{\@ifnextchar{_}{\diff@one@fun@d #1}{\textrm{d}#1}}
\def\diff@one@fun@d#1_#2{\frac{\textrm{d}#1}{\textrm{d}#2}}
\newcommand*{\diff}{\@ifnextchar{^}{\diff@n}
  {\@ifnextchar{_}{\diff@one@d}{\diff@one@fun}}}
%
%Partieller Diff-Operator.
\def\pdiff@n^#1{\@ifnextchar{_}{\pdiff@n@d^#1}{\pdiff@n@fun^#1}}
\def\pdiff@n@d^#1_#2{\frac{\partial^#1}{\partial#2^#1}}
\def\pdiff@n@fun^#1#2{\@ifnextchar{_}{\pdiff@n@fun@d^#1#2}{\partial^#1#2}}
\def\pdiff@n@fun@d^#1#2_#3{\frac{\partial^#1 #2}{\partial#3^#1}}
\def\pdiff@one@d_#1{\frac{\partial}{\partial #1}}
\def\pdiff@one@fun#1{\@ifnextchar{_}{\pdiff@one@fun@d #1}{\partial#1}}
\def\pdiff@one@fun@d#1_#2{\frac{\partial#1}{\partial#2}}
\newcommand*{\pdiff}{\@ifnextchar{^}{\pdiff@n}
  {\@ifnextchar{_}{\pdiff@one@d}{\pdiff@one@fun}}}
\makeatother
%
%Das gleich nur mit etwas andere Syntax. Die Potenz der Differentiation wird erst
%zum Schluss angegeben. Somit lautet die Syntax:
%
% \diff_x^2  ->  d^2/dx^2
% \diff f_x^2  ->  d^2f/dx^2
% \diff{f(x^5)}_x^2  ->  d^2(f(x^5))/dx^2
% Ansonsten wie Oben.
%
%Ersetzt man \diff durch \pdiff, so wird der partieller
%Differentialoperator dargestellt.
%
%\makeatletter
%\def\diff@#1{\@ifnextchar{_}{\diff@fun#1}{\textrm{d} #1}}
%\def\diff@one_#1{\@ifnextchar{^}{\diff@n{#1}}%
%  {\frac{\textrm d}{\textrm{d} #1}}}
%\def\diff@fun#1_#2{\@ifnextchar{^}{\diff@fun@n#1_#2}%
%  {\frac{\textrm d #1}{\textrm{d} #2}}}
%\def\diff@n#1^#2{\frac{\textrm d^#2}{\textrm{d}#1^#2}}
%\def\diff@fun@n#1_#2^#3{\frac{\textrm d^#3 #1}%
%  {\textrm{d}#2^#3}}
%\def\diff{\@ifnextchar{_}{\diff@one}{\diff@}}
%\newcommand*{\diff}{\@ifnextchar{_}{\diff@one}{\diff@}}
%
%Partieller Diff-Operator.
%\def\pdiff@#1{\@ifnextchar{_}{\pdiff@fun#1}{\partial #1}}
%\def\pdiff@one_#1{\@ifnextchar{^}{\pdiff@n{#1}}%
%  {\frac{\partial}{\partial #1}}}
%\def\pdiff@fun#1_#2{\@ifnextchar{^}{\pdiff@fun@n#1_#2}%
%  {\frac{\partial #1}{\partial #2}}}
%\def\pdiff@n#1^#2{\frac{\partial^#2}{\partial #1^#2}}
%\def\pdiff@fun@n#1_#2^#3{\frac{\partial^#3 #1}%
%  {\partial #2^#3}}
%\newcommand*{\pdiff}{\@ifnextchar{_}{\pdiff@one}{\pdiff@}}
%\makeatother

%-------------------------------------------------------------------------------
%%Nützliche Makros um in der Quantenmechanik Bras, Kets und das Skalarprodukt
%%zwischen den beiden darzustellen.
%%Benutzung:
%% \bra{x}  ->    < x |
%% \ket{x}  ->    | x >
%% \braket{x}{y} ->   < x | y >

\newcommand\bra[1]{\left\langle #1 \right|}
\newcommand\ket[1]{\left| #1 \right\rangle}
%\newcommand\braket[2]{%
%  \left\langle #1\vphantom{#2} \right.%
%  \left|\vphantom{#1#2}\right.%
%  \left. \vphantom{#1}#2 \right\rangle}%
\newcommand\braket[2]{%
  \left\langle \vphantom{#2} #1%
    \middle|%
    \vphantom{#1} #2\right\rangle}%

%-------------------------------------------------------------------------------
%%Aus dem Buch:
%%Titel:  Latex in Naturwissenschaften und Mathematik
%%Autor:  Herbert Voß
%%Verlag: Franzis Verlag, 2006
%%ISBN:   3772374190, 9783772374197
%%
%%Hier werden drei Makros definiert:\mathllap, \mathclap und \mathrlap, welche
%%analog zu den aus Latex bekannten \rlap und \llap arbeiten, d.h. selbst
%%keinerlei horizontalen Platz benötigen, aber dennoch zentriert zum aktuellen
%%Punkt erscheinen.

\newcommand*\mathllap{\mathstrut\mathpalette\mathllapinternal}
\newcommand*\mathllapinternal[2]{\llap{$\mathsurround=0pt#1{#2}$}}
\newcommand*\clap[1]{\hbox to 0pt{\hss#1\hss}}
\newcommand*\mathclap{\mathpalette\mathclapinternal}
\newcommand*\mathclapinternal[2]{\clap{$\mathsurround=0pt#1{#2}$}}
\newcommand*\mathrlap{\mathpalette\mathrlapinternal}
\newcommand*\mathrlapinternal[2]{\rlap{$\mathsurround=0pt#1{#2}$}}

%%Das Gleiche nur mit \def statt \newcommand.
%\def\mathllap{\mathpalette\mathllapinternal}
%\def\mathllapinternal#1#2{%
%  \llap{$\mathsurround=0pt#1{#2}$}% $
%}
%\def\clap#1{\hbox to 0pt{\hss#1\hss}}
%\def\mathclap{\mathpalette\mathclapinternal}
%\def\mathclapinternal#1#2{%
%  \clap{$\mathsurround=0pt#1{#2}$}%
%}
%\def\mathrlap{\mathpalette\mathrlapinternal}
%\def\mathrlapinternal#1#2{%
%  \rlap{$\mathsurround=0pt#1{#2}$}% $
%}

%-------------------------------------------------------------------------------
%%Hier werden zwei neue Makros definiert \overbr und \underbr welche analog zu
%%\overbrace und \underbrace funktionieren jedoch die Gleichung nicht
%%'zerreißen'. Dies wird ermöglicht durch das \mathclap Makro.

\def\overbr#1^#2{\overbrace{#1}^{\mathclap{#2}}}
\def\underbr#1_#2{\underbrace{#1}_{\mathclap{#2}}}

%-------------------------------------------------------------------------------
%---------------------------------End of Preamble-------------------------------
%-------------------------------------------------------------------------------


\begin{document}

\section*{Erwartungswert von \(\sin^2(x)\)}
\label{sec:erwart-von-sin2x}

Behauptung:
\begin{align}
  \label{eq:1}
  \boxed{\left\langle \sin^2(x) \right\rangle = \frac 1 2}
\end{align}

Wir wollen nun die Behauptung \eqref{eq:1} überprüfen. Dazu benutzen wir
folgende Identität:
\begin{align}
  \label{eq:2}
  \sin(x)=\frac 1 {2i}\left(e^{ix}-e^{-ix}\right)
\end{align}

Damit können wir schreiben:
 \begin{align}
   \label{eq:3}
    \langle \sin^2(x) \rangle 
    &= \left\langle -\frac 1 4 \left(e^{ix}-e^{-ix}\right)^2\right\rangle
    = \left\langle -\frac 1 4 \left(e^{2ix}+e^{-2ix}-2\cdot
        \underbrace{e^{ix}e^{-ix}}_{=1} \right)\right\rangle\notag\\ 
    &= \left\langle \frac 1 2 -\frac 1 4 \left(e^{2ix}+e^{-2ix}
      \right)\right\rangle
    = \left\langle \frac 1 2\right\rangle 
    - \left\langle\frac 1 4 \left(e^{2ix}+e^{-2ix}
      \right)\right\rangle\notag\\
    &= \frac 1 2 
    - \left\langle\frac 1 4 \left(e^{2ix}+e^{-2ix}
      \right)\right\rangle
 \end{align}

Nun wollen wir den Eigenwert des zweiten Terms in der Gl. \eqref{eq:3}
bestimmen. Für den Erwartungswert eines Operators im Ortsraum gilt allgemein:
\begin{align}
  \label{eq:4}
  \langle O \rangle = \int \psi^* O \psi \diff x
\end{align}

Betrachte \(\psi\) als ebene Welle mit:
\begin{align}
  \label{eq:5}
  \psi=Ae^{i(kx-\omega t)}
\end{align}

Mit \eqref{eq:4} und \eqref{eq:5} folgt für den Erwartungswert in Gl.
~\eqref{eq:2}:
\begin{align}
  \label{eq:6}
   \left\langle\frac 1 4 \left(e^{2ix}+e^{-2ix}\right)\right\rangle 
    &= \frac 1 4 \int_{-\infty}^\infty A^*e^{-i(kx-\omega t)}
      \left(e^{2ix}+e^{-2ix}\right) Ae^{i(kx-\omega t)}\diff x \notag\\
    &= \frac {|A|^2} 4 \int_{-\infty}^\infty
      \left(e^{2ix}+e^{-2ix}\right)\diff x
\end{align}

Um nun das Integral in Gl.~\eqref{eq:6} zu berechnen machen wir eine
Substitution:
\begin{subequations}
  \begin{align}
    z&=e^{2ix} 
    \quad \Rightarrow z(\infty)=\infty \quad z(-\infty)=0
    \qquad \text{mit } z \in \mathbb C \label{eq:7a}\\
    \Leftrightarrow\diff z_x &= 2i e^{2ix} = 2i z
    \Leftrightarrow \diff x = \frac {\diff z} {2 i z}\label{eq:7b}
  \end{align}
\end{subequations}

Mit der Substitution \eqref{eq:7a} und \eqref{eq:7b} erhalten wir für das
Integral in der Gl.~\eqref{eq:6}:
\begin{align}
  \label{eq:7}
  \int_{-\infty}^\infty
  \left(e^{2ix}+e^{-2ix}\right)\diff x &=
  \int_{0}^\infty
  \left(z+\frac 1 {z} \right) \frac 1 {2iz} \diff z
  =\frac 1 {2i} \int_{0}^\infty \left(1+\frac 1 {z^2} \right)\diff z
  \notag\\
  &=\frac 1 {2i} \int_{0}^\infty \frac {z^2+ 1}{z^2}\diff z
\end{align}

Da die Variable \(z\) im Gl.~\eqref{eq:7} nur in quadratischer Form
vorkommt, können wir die Grenzen ersetzen mit \(\int_0^\infty\rightarrow 
\frac 1 2 \int_{-\infty}^\infty\) und erhalten folgendes Integral:
\begin{align}
  \label{eq:8}
  \frac 1 {4i} \int_{-\infty}^\infty \frac {z^2+ 1}{z^2}\diff z
\end{align}

Das Integral \eqref{eq:8} lässt sich mit Hilfe des Residuensatzes:
\begin{align}
  \label{eq:9}
  \oint f(z)\diff z = 2\pi i\sum_{k=1}^n\text{Res}(f,z_k)
\end{align}
bestimmen.
Dazu berechnen wir zuerst das Residuuum unser Funktion:
\begin{align}
  \label{eq:10}
  f(z)=\frac {z^2+1} {z^2}
\end{align}

Für das Residuum einen Funktion \(f(z)\), die in \(z_0\) einen Pol n-ter Ordnung
hat, gilt:
\begin{align}
  \label{eq:11}
  \text{Res}(f, z_0)=\lim_{z \to z_0}\frac 1 {(n-1)!}\frac{\text d^{n-1}}{\text
    d z^{n-1}}\Big[(z-z_0)^nf(z)\Big]
\end{align}

Die Funktion \eqref{eq:10} hat einen Pol 2-ter Ordnung bei \(z_0=0\). D.h es
gilt laut Gl.~\eqref{eq:11}:
\begin{align}
  \label{eq:12}
   \text{Res}(f, 0)=\lim_{z \to 0}\diff_z z^2\left(\frac{z^2+1}{z^2}\right)
   =\lim_{z \to 0}\diff_z z^2+1=\lim_{z \to 0}2z=0
\end{align}

Mit \eqref{eq:9},~\eqref{eq:10} und \eqref{eq:12} folgt für unser Integral
\eqref{eq:8}:
\begin{align}
  \label{eq:13}
  \frac 1 {4i} \int_{-\infty}^\infty \frac {z^2+ 1}{z^2}\diff z
  \equiv \frac 1 {4i}\oint f(z)\diff z 
  = \frac 1 {4i} 2\pi i\,\text{Res}(f,0)
  =\frac \pi {2}\cdot 0 =0
\end{align}

Damit gilt für den Erwartungswert in Gl.~\eqref{eq:3}:
\begin{align}
  \label{eq:14}
   \left\langle\frac 1 4 \left(e^{2ix}+e^{-2ix}\right)\right\rangle 
   \equiv \frac 1 {4i} \int_{-\infty}^\infty \frac {z^2+ 1}{z^2}\diff z
   = 0
\end{align}

Somit wäre die Behauptung Gl.~\eqref{eq:1} bestätigt.
\end{document}
