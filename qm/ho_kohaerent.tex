\input{../headers/header_script.tex}
%\includegraphics[width=0.75\textwidth]{thepic.png}

\begin{document}

\textit{29. März 2012}
\input{../headers/authors.tex}

\section*{Kohärente Zustände des harmonischen Oszillators}

Für den harmonischen Oszillator existiert eine Klasse von Zuständen, die mit gewisser Berechtigung als 'Quasi-klassische' Zustände angesehen werden. Diese kohärenten Zustände sollen hier näher betrachtet werden.

Die Zustände \(|\alpha\rangle \) sind Eigenzustände des Auf- und Absteige-Operators. Es gilt:

\begin{equation}
  \label{eq:1}
  a|\alpha\rangle =\alpha|\alpha\rangle \qquad \langle \alpha | \alpha \rangle =1 \qquad \alpha \in \mathbb C
\end{equation}


Nun wollen wir herausfinden wie sich \(|\alpha\rangle\) als Liniarkombination von Energieeigenzuständen darstellen lassen. 

\begin{equation}
  \label{eq:2}
  |\alpha\rangle = \mathbb 1 |\alpha\rangle = \sum_{n=0}^{\infty}|n\rangle\underbrace{ \langle n|\alpha\rangle }_{c_n} = \sum_{n=0}^{\infty} c_n|n\rangle 
\end{equation}

Wenden wir nun den Absteige-Operator auf den Zustand an:

\begin{equation}
  \label{eq:3}
  a|\alpha\rangle = \sum_{n=0}^{\infty} c_n a|n\rangle=\sum_{n=1}^{\infty} c_n \sqrt{n}|n-1\rangle
\end{equation}

Ersetze \(n\) mit \(n+1\)

\begin{equation}
  \label{eq:4}
  a|\alpha\rangle = \sum_{n=0}^{\infty} \underline{c_{n+1} \sqrt{n+1}}|n\rangle \stackrel{!}= \underbrace{ \alpha|\alpha\rangle}_{\eqref{eq:1}}  = \underbrace{ \sum_{n=0}^{\infty} \underline{\alpha c_n} |n\rangle }_{\eqref{eq:2} }
\end{equation}

Durch Vergleich von den beiden unterstrichenen Teilchen der Formel \eqref{eq:4} erhalten wir folgende Rekusionsformel:

\begin{equation}
  \label{eq:5}
  c_{n+1}\sqrt{n+1} = \alpha c_n
\end{equation}

Ersetze \(n\) mit \(n-1\):

\begin{equation}
  \label{eq:6}
  c_n\sqrt{n} = \alpha c_{n-1}
\end{equation}

\begin{align}
  \label{eq:7}
  \Rightarrow c_n &= \frac{\alpha}{\sqrt{n}}c_{n-1} \notag\\
c_1 &=  \frac{\alpha}{\sqrt{1}}c_{0} \notag\\
c_2 &=  \frac{\alpha}{\sqrt{2}}c_{1} =\frac{\alpha}{\sqrt{2}}\frac{\alpha}{\sqrt{1}}c_{0}  \notag \\
c_3 &=  \frac{\alpha}{\sqrt{3}}c_{2} =  \frac{\alpha}{\sqrt{3}} \frac{\alpha}{\sqrt{2}}\frac{\alpha}{\sqrt{1}}c_{0}  \notag\\
\vdots \notag\\
c_n &=  \frac{\alpha^n}{\sqrt{n!}}c_{0}
\end{align}

\(c_n\) in Gleichung \eqref{eq:2} eingesetzt ergibt:

\begin{equation}
  \label{eq:8}
  |\alpha\rangle =\sum_{n=0}^{\infty} \frac{\alpha^n}{\sqrt{n!}}c_0 |n\rangle 
\end{equation}

Bestimmen des \(c_0\) durch Normierungsbedinung:

\begin{align}
  \label{eq:9}
  \langle \alpha | \alpha \rangle  &= \underbrace{\sum_{n=0}^{\infty} \frac{|\alpha|^{2n}}{n!}}_{e^{|\alpha|^2}} |c_0|^2 \underbrace{\langle n|n\rangle}_{=1} \notag\\
&= e^{|\alpha|^2}|c_0|^2 \stackrel{!}= 1 \notag \\
 &\Leftrightarrow c_0 = e^{-\frac{|\alpha|^2}{2}} 
\end{align}

Eingesetzt in \eqref{eq:8}:

\begin{equation}
  \label{eq:10}
  \boxed{ |\alpha\rangle =  e^{-\frac{|\alpha|^2}{2}} \sum_{n=0}^{\infty} \frac{\alpha^n}{\sqrt{n!}} |n\rangle }
\end{equation}

In Fock-Raum-Schreibweise ergibt sich der kohärente Zustand \(|\alpha\rangle\) als unendliche Linearkombination von Zuständen fester Teilchenzahl (Fock-Zustände) \(|n\rangle\).


\subsection*{Minimale Unschärfe der kohärenten Zustände}

Wir wollen zeigen, dass für die kohärente Zustände \(|\alpha\rangle \) minimale Unschärfe für Orts- und Impuls-Operator gilt die mit der Heisenbergschen Unschärferelation verträglich ist. Dazu benötigen wir die Darstellung des  Orts- und Impuls-Operators als Auf- und Absteigeoperatoren:

\begin{align}
  x &= \sqrt{ \frac{\hbar}{2m\omega}} (a^\dagger+a) \label{eq:11}\\
  p &= i\sqrt{ \frac{\hbar m\omega}{2}} (a^\dagger-a) \label{eq:12}
\end{align}

Zu berechnen ist die Orts- und Impuls Unschärfe:

\begin{align}
  \Delta x &= \sqrt{ \langle x^2\rangle -\langle x\rangle^2 }\label{eq:13}\\
   \Delta p &= \sqrt{ \langle p^2\rangle -\langle p\rangle^2 }\label{eq:14}
\end{align}


\begin{align}
  \label{eq:15}
  \langle x^2\rangle &= \langle \alpha|x^2|\alpha\rangle \notag\\
  &= \frac{\hbar}{2m\omega}\langle \alpha|(a^\dagger + a)^2|\alpha\rangle \notag\\
  &= \frac{\hbar}{2m\omega}\langle \alpha|(a^\dagger)^2 + a^\dagger a+aa^\dagger +a^2|\alpha\rangle \qquad \text{mit } [a,a^\dagger] = 1 \notag\\
   &= \frac{\hbar}{2m\omega}\langle \alpha|(a^\dagger)^2 + a^\dagger a+1+a^\dagger a +a^2|\alpha\rangle \notag\\
   &= \frac{\hbar}{2m\omega}\langle \alpha|(a^\dagger)^2 + 2 a^\dagger a+1 +a^2|\alpha\rangle\notag\\
   &= \frac{\hbar}{2m\omega}\left(\langle \alpha|(a^\dagger)^2|\alpha\rangle + 2\langle \alpha| a^\dagger a|\alpha\rangle+ 1 +\langle \alpha| a^2|\alpha\rangle\right)\notag \\
   &= \frac{\hbar}{2m\omega}\left((\alpha^*)^2 + 2| \alpha|^2 + 1 + \alpha^2\right) \notag\\
   &= \frac{\hbar}{2m\omega}\left((\alpha^*+\alpha)^2 + 1\right)
\end{align}

\begin{align}
  \label{eq:16}
   \langle x\rangle &= \langle \alpha|x|\alpha\rangle  \notag\\
 &= \sqrt{ \frac{\hbar}{2m\omega}}\langle \alpha|(a^\dagger + a)|\alpha\rangle \notag\\
 &= \sqrt{\frac{\hbar}{2m\omega}}(\langle \alpha|a^\dagger|\alpha\rangle +\langle \alpha| a|\alpha\rangle)\notag \\
 &= \sqrt{\frac{\hbar}{2m\omega}}(\alpha^* + \alpha)
\end{align}

Die Erwartungswerte \eqref{eq:15} und \eqref{eq:16} in \eqref{eq:13} einsetzen ergibt:

\begin{align}
  \label{eq:17}
  \Delta x &= \sqrt{ \langle x^2\rangle -\langle x\rangle^2 }  \notag\\
 &=\sqrt{\frac{\hbar}{2m\omega}[\cancel{( \alpha^*+\alpha)^2} + 1 - \cancel{ (\alpha^* + \alpha)^2}  ]} \notag\\
 &=\sqrt{\frac{\hbar}{2m\omega}}
\end{align}

Analoge Rechnung ergibt für die Impulsunschärfe:

\begin{equation}
  \label{eq:18}
  \Delta p = \sqrt{\frac{\hbar \omega m}{2}}
\end{equation}

Daraus folgt für Unschärferelation:

\begin{equation}
  \label{eq:19}
  \Delta x\cdot\Delta p = \sqrt{\frac{\hbar}{2m\omega}} \sqrt{\frac{\hbar \omega m}{2}} = \frac{\hbar}{2}
\end{equation}


\subsection*{Zusammenfassung; Eigenschaften}

\begin{itemize}
\item Eigenzustände des Absteigeoperators \(a\) (auch des Vektorpotentials \(\vec A\))
\item minimale Unschärfe
\item behalten diese Unschärfe auch unter Zeitentwicklung bei
\item Im Fokraum \(a\) und \(a^\dagger\) spielen die Rolle von Vernichtungs- bzw. Erzeuge-Operatoren für Teilchen
\end{itemize}

\end{document}
