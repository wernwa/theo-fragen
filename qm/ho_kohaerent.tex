\documentclass[10pt,a4paper,oneside,fleqn]{article}
\usepackage{geometry}
\geometry{a4paper,left=20mm,right=20mm,top=1cm,bottom=2cm}
\usepackage[utf8]{inputenc}
%\usepackage{ngerman}
\usepackage{amsmath}                % brauche ich um dir Formel zu umrahmen.
\usepackage{amsfonts}                % brauche ich für die Mengensymbole
\usepackage{graphicx}
\setlength{\parindent}{0px}
\setlength{\mathindent}{10mm}
\usepackage{bbold}                    %brauche ich für die doppel Zahlen Darstellung (Einheitsmatrix z.B)



\usepackage{color}
\usepackage{titlesec} %sudo apt-get install texlive-latex-extra

\definecolor{darkblue}{rgb}{0.1,0.1,0.55}
\definecolor{verydarkblue}{rgb}{0.1,0.1,0.35}
\definecolor{darkred}{rgb}{0.55,0.2,0.2}

%hyperref Link color
\usepackage[colorlinks=true,
        linkcolor=darkblue,
        citecolor=darkblue,
        filecolor=darkblue,
        pagecolor=darkblue,
        urlcolor=darkblue,
        bookmarks=true,
        bookmarksopen=true,
        bookmarksopenlevel=3,
        plainpages=false,
        pdfpagelabels=true]{hyperref}

\titleformat{\chapter}[display]{\color{darkred}\normalfont\huge\bfseries}{\chaptertitlename\
\thechapter}{20pt}{\Huge}

\titleformat{\section}{\color{darkblue}\normalfont\Large\bfseries}{\thesection}{1em}{}
\titleformat{\subsection}{\color{verydarkblue}\normalfont\large\bfseries}{\thesubsection}{1em}{}

% Notiz Box
\usepackage{fancybox}
\newcommand{\notiz}[1]{\vspace{5mm}\ovalbox{\begin{minipage}{1\textwidth}#1\end{minipage}}\vspace{5mm}}

\usepackage{cancel}
\setcounter{secnumdepth}{3}
\setcounter{tocdepth}{3}





%-------------------------------------------------------------------------------
%Diff-Makro:
%Das Diff-Makro stellt einen Differentialoperator da.
%
%Benutzung:
% \diff  ->  d
% \diff f  ->  df
% \diff^2 f  ->  d^2 f
% \diff_x  ->  d/dx
% \diff^2_x  ->  d^2/dx^2
% \diff f_x  ->  df/dx
% \diff^2 f_x  ->  d^2f/dx^2
% \diff^2{f(x^5)}_x  ->  d^2(f(x^5))/dx^2
%
%Ersetzt man \diff durch \pdiff, so wird der partieller
%Differentialoperator dargestellt.
%
\makeatletter
\def\diff@n^#1{\@ifnextchar{_}{\diff@n@d^#1}{\diff@n@fun^#1}}
\def\diff@n@d^#1_#2{\frac{\textrm{d}^#1}{\textrm{d}#2^#1}}
\def\diff@n@fun^#1#2{\@ifnextchar{_}{\diff@n@fun@d^#1#2}{\textrm{d}^#1#2}}
\def\diff@n@fun@d^#1#2_#3{\frac{\textrm{d}^#1 #2}{\textrm{d}#3^#1}}
\def\diff@one@d_#1{\frac{\textrm{d}}{\textrm{d}#1}}
\def\diff@one@fun#1{\@ifnextchar{_}{\diff@one@fun@d #1}{\textrm{d}#1}}
\def\diff@one@fun@d#1_#2{\frac{\textrm{d}#1}{\textrm{d}#2}}
\newcommand*{\diff}{\@ifnextchar{^}{\diff@n}
  {\@ifnextchar{_}{\diff@one@d}{\diff@one@fun}}}
%
%Partieller Diff-Operator.
\def\pdiff@n^#1{\@ifnextchar{_}{\pdiff@n@d^#1}{\pdiff@n@fun^#1}}
\def\pdiff@n@d^#1_#2{\frac{\partial^#1}{\partial#2^#1}}
\def\pdiff@n@fun^#1#2{\@ifnextchar{_}{\pdiff@n@fun@d^#1#2}{\partial^#1#2}}
\def\pdiff@n@fun@d^#1#2_#3{\frac{\partial^#1 #2}{\partial#3^#1}}
\def\pdiff@one@d_#1{\frac{\partial}{\partial #1}}
\def\pdiff@one@fun#1{\@ifnextchar{_}{\pdiff@one@fun@d #1}{\partial#1}}
\def\pdiff@one@fun@d#1_#2{\frac{\partial#1}{\partial#2}}
\newcommand*{\pdiff}{\@ifnextchar{^}{\pdiff@n}
  {\@ifnextchar{_}{\pdiff@one@d}{\pdiff@one@fun}}}
\makeatother
%
%Das gleich nur mit etwas andere Syntax. Die Potenz der Differentiation wird erst
%zum Schluss angegeben. Somit lautet die Syntax:
%
% \diff_x^2  ->  d^2/dx^2
% \diff f_x^2  ->  d^2f/dx^2
% \diff{f(x^5)}_x^2  ->  d^2(f(x^5))/dx^2
% Ansonsten wie Oben.
%
%Ersetzt man \diff durch \pdiff, so wird der partieller
%Differentialoperator dargestellt.
%
%\makeatletter
%\def\diff@#1{\@ifnextchar{_}{\diff@fun#1}{\textrm{d} #1}}
%\def\diff@one_#1{\@ifnextchar{^}{\diff@n{#1}}%
%  {\frac{\textrm d}{\textrm{d} #1}}}
%\def\diff@fun#1_#2{\@ifnextchar{^}{\diff@fun@n#1_#2}%
%  {\frac{\textrm d #1}{\textrm{d} #2}}}
%\def\diff@n#1^#2{\frac{\textrm d^#2}{\textrm{d}#1^#2}}
%\def\diff@fun@n#1_#2^#3{\frac{\textrm d^#3 #1}%
%  {\textrm{d}#2^#3}}
%\def\diff{\@ifnextchar{_}{\diff@one}{\diff@}}
%\newcommand*{\diff}{\@ifnextchar{_}{\diff@one}{\diff@}}
%
%Partieller Diff-Operator.
%\def\pdiff@#1{\@ifnextchar{_}{\pdiff@fun#1}{\partial #1}}
%\def\pdiff@one_#1{\@ifnextchar{^}{\pdiff@n{#1}}%
%  {\frac{\partial}{\partial #1}}}
%\def\pdiff@fun#1_#2{\@ifnextchar{^}{\pdiff@fun@n#1_#2}%
%  {\frac{\partial #1}{\partial #2}}}
%\def\pdiff@n#1^#2{\frac{\partial^#2}{\partial #1^#2}}
%\def\pdiff@fun@n#1_#2^#3{\frac{\partial^#3 #1}%
%  {\partial #2^#3}}
%\newcommand*{\pdiff}{\@ifnextchar{_}{\pdiff@one}{\pdiff@}}
%\makeatother

%-------------------------------------------------------------------------------
%%Nützliche Makros um in der Quantenmechanik Bras, Kets und das Skalarprodukt
%%zwischen den beiden darzustellen.
%%Benutzung:
%% \bra{x}  ->    < x |
%% \ket{x}  ->    | x >
%% \braket{x}{y} ->   < x | y >

\newcommand\bra[1]{\left\langle #1 \right|}
\newcommand\ket[1]{\left| #1 \right\rangle}
\newcommand\braket[2]{%
  \left\langle #1\vphantom{#2} \right.%
  \left|\vphantom{#1#2}\right.%
  \left. \vphantom{#1}#2 \right\rangle}%

%-------------------------------------------------------------------------------
%%Aus dem Buch:
%%Titel:  Latex in Naturwissenschaften und Mathematik
%%Autor:  Herbert Voß
%%Verlag: Franzis Verlag, 2006
%%ISBN:   3772374190, 9783772374197
%%
%%Hier werden drei Makros definiert:\mathllap, \mathclap und \mathrlap, welche
%%analog zu den aus Latex bekannten \rlap und \llap arbeiten, d.h. selbst
%%keinerlei horizontalen Platz benötigen, aber dennoch zentriert zum aktuellen
%%Punkt erscheinen.

\newcommand*\mathllap{\mathstrut\mathpalette\mathllapinternal}
\newcommand*\mathllapinternal[2]{\llap{$\mathsurround=0pt#1{#2}$}}
\newcommand*\clap[1]{\hbox to 0pt{\hss#1\hss}}
\newcommand*\mathclap{\mathpalette\mathclapinternal}
\newcommand*\mathclapinternal[2]{\clap{$\mathsurround=0pt#1{#2}$}}
\newcommand*\mathrlap{\mathpalette\mathrlapinternal}
\newcommand*\mathrlapinternal[2]{\rlap{$\mathsurround=0pt#1{#2}$}}

%%Das Gleiche nur mit \def statt \newcommand.
%\def\mathllap{\mathpalette\mathllapinternal}
%\def\mathllapinternal#1#2{%
%  \llap{$\mathsurround=0pt#1{#2}$}% $
%}
%\def\clap#1{\hbox to 0pt{\hss#1\hss}}
%\def\mathclap{\mathpalette\mathclapinternal}
%\def\mathclapinternal#1#2{%
%  \clap{$\mathsurround=0pt#1{#2}$}%
%}
%\def\mathrlap{\mathpalette\mathrlapinternal}
%\def\mathrlapinternal#1#2{%
%  \rlap{$\mathsurround=0pt#1{#2}$}% $
%}

%-------------------------------------------------------------------------------
%%Hier werden zwei neue Makros definiert \overbr und \underbr welche analog zu
%%\overbrace und \underbrace funktionieren jedoch die Gleichung nicht
%%'zerreißen'. Dies wird ermöglicht durch das \mathclap Makro.

\def\overbr#1^#2{\overbrace{#1}^{\mathclap{#2}}}
\def\underbr#1_#2{\underbrace{#1}_{\mathclap{#2}}}
%\includegraphics[width=0.75\textwidth]{thepic.png}

\begin{document}

\section*{Kohärente Zustände des harmonischen Oszillators}

Für den harmonischen Oszillator existiert eine Klasse von Zuständen, die mit gewisser Berechtigung als 'Quasi-klassische' Zustände angesehen werden. Diese kohärenten Zustände sollen hier näher betrachtet werden.

Die Zustände \(|\alpha\rangle \) sind Eigenzustände des Auf- und Absteige-Operators. Es gilt:

\begin{equation}
  \label{eq:1}
  a|\alpha\rangle =\alpha|\alpha\rangle \qquad \langle \alpha | \alpha \rangle =1 \qquad \alpha \in \mathbb C
\end{equation}


Nun wollen wir herausfinden wie sich \(|\alpha\rangle\) als Liniarkombination von Energieeigenzuständen darstellen lassen. 

\begin{equation}
  \label{eq:2}
  |\alpha\rangle = \mathbb 1 |\alpha\rangle = \sum_{n=0}^{\infty}|n\rangle\underbrace{ \langle n|\alpha\rangle }_{c_n} = \sum_{n=0}^{\infty} c_n|n\rangle 
\end{equation}

Wenden wir nun den Absteige-Operator auf den Zustand an:

\begin{equation}
  \label{eq:3}
  a|\alpha\rangle = \sum_{n=0}^{\infty} c_n a|n\rangle=\sum_{n=1}^{\infty} c_n \sqrt{n}|n-1\rangle
\end{equation}

Ersetze \(n\) mit \(n+1\)

\begin{equation}
  \label{eq:4}
  a|\alpha\rangle = \sum_{n=0}^{\infty} \underline{c_{n+1} \sqrt{n+1}}|n\rangle \stackrel{!}= \underbrace{ \alpha|\alpha\rangle}_{\eqref{eq:1}}  = \underbrace{ \sum_{n=0}^{\infty} \underline{\alpha c_n} |n\rangle }_{\eqref{eq:2} }
\end{equation}

Durch Vergleich von den beiden unterstrichenen Teilchen der Formel \eqref{eq:4} erhalten wir folgende Rekusionsformel:

\begin{equation}
  \label{eq:5}
  c_{n+1}\sqrt{n+1} = \alpha c_n
\end{equation}

Ersetze \(n\) mit \(n-1\):

\begin{equation}
  \label{eq:6}
  c_n\sqrt{n} = \alpha c_{n-1}
\end{equation}

\begin{align}
  \label{eq:7}
  \Rightarrow c_n &= \frac{\alpha}{\sqrt{n}}c_{n-1}\\
c_1 &=  \frac{\alpha}{\sqrt{1}}c_{0}\\
c_2 &=  \frac{\alpha}{\sqrt{2}}c_{1} =\frac{\alpha}{\sqrt{2}}\frac{\alpha}{\sqrt{1}}c_{0} \\
c_3 &=  \frac{\alpha}{\sqrt{3}}c_{2} =  \frac{\alpha}{\sqrt{3}} \frac{\alpha}{\sqrt{2}}\frac{\alpha}{\sqrt{1}}c_{0}\\
\vdots\\
c_n &=  \frac{\alpha^n}{\sqrt{n!}}c_{0}
\end{align}

\(c_n\) in Gleichung \eqref{eq:2} eingesetzt ergibt:

\begin{equation}
  \label{eq:8}
  |\alpha\rangle =\sum_{n=0}^{\infty} \frac{\alpha^n}{\sqrt{n!}}c_0 |n\rangle 
\end{equation}

Bestimmen des \(c_0\) durch Normierungsbedinung:

\begin{align}
  \label{eq:9}
  \langle \alpha | \alpha \rangle  &= \underbrace{\sum_{n=0}^{\infty} \frac{|\alpha|^{2n}}{n!}}_{e^{|\alpha|^2}} |c_0|^2 \underbrace{\langle n|n\rangle}_{=1} \\
&= e^{|\alpha|^2}|c_0|^2 \stackrel{!}= 1 \\
 &\Leftrightarrow c_0 = e^{-\frac{|\alpha|^2}{2}} 
\end{align}

Eingesetzt in \eqref{eq:8}:

\begin{equation}
  \label{eq:10}
  \boxed{ |\alpha\rangle =  e^{-\frac{|\alpha|^2}{2}} \sum_{n=0}^{\infty} \frac{\alpha^n}{\sqrt{n!}} |n\rangle }
\end{equation}

In Fock-Raum-Schreibweise ergibt sich der kohärente Zustand \(|\alpha\rangle\) als unendliche Linearkombination von Zuständen fester Teilchenzahl (Fock-Zustände) \(|n\rangle\).

\end{document}
