\documentclass[10pt,a4paper,oneside,fleqn]{article}
\usepackage{geometry}
\geometry{a4paper,left=20mm,right=20mm,top=1cm,bottom=2cm}
\usepackage[utf8]{inputenc}
%\usepackage{ngerman}
\usepackage{amsmath}                % brauche ich um dir Formel zu umrahmen.
\usepackage{amsfonts}                % brauche ich für die Mengensymbole
\usepackage{graphicx}
\setlength{\parindent}{0px}
\setlength{\mathindent}{10mm}
\usepackage{bbold}                    %brauche ich für die doppel Zahlen Darstellung (Einheitsmatrix z.B)



\usepackage{color}
\usepackage{titlesec} %sudo apt-get install texlive-latex-extra

\definecolor{darkblue}{rgb}{0.1,0.1,0.55}
\definecolor{verydarkblue}{rgb}{0.1,0.1,0.35}
\definecolor{darkred}{rgb}{0.55,0.2,0.2}

%hyperref Link color
\usepackage[colorlinks=true,
        linkcolor=darkblue,
        citecolor=darkblue,
        filecolor=darkblue,
        pagecolor=darkblue,
        urlcolor=darkblue,
        bookmarks=true,
        bookmarksopen=true,
        bookmarksopenlevel=3,
        plainpages=false,
        pdfpagelabels=true]{hyperref}

\titleformat{\chapter}[display]{\color{darkred}\normalfont\huge\bfseries}{\chaptertitlename\
\thechapter}{20pt}{\Huge}

\titleformat{\section}{\color{darkblue}\normalfont\Large\bfseries}{\thesection}{1em}{}
\titleformat{\subsection}{\color{verydarkblue}\normalfont\large\bfseries}{\thesubsection}{1em}{}

% Notiz Box
\usepackage{fancybox}
\newcommand{\notiz}[1]{\vspace{5mm}\ovalbox{\begin{minipage}{1\textwidth}#1\end{minipage}}\vspace{5mm}}

\usepackage{cancel}
\setcounter{secnumdepth}{3}
\setcounter{tocdepth}{3}





%-------------------------------------------------------------------------------
%Diff-Makro:
%Das Diff-Makro stellt einen Differentialoperator da.
%
%Benutzung:
% \diff  ->  d
% \diff f  ->  df
% \diff^2 f  ->  d^2 f
% \diff_x  ->  d/dx
% \diff^2_x  ->  d^2/dx^2
% \diff f_x  ->  df/dx
% \diff^2 f_x  ->  d^2f/dx^2
% \diff^2{f(x^5)}_x  ->  d^2(f(x^5))/dx^2
%
%Ersetzt man \diff durch \pdiff, so wird der partieller
%Differentialoperator dargestellt.
%
\makeatletter
\def\diff@n^#1{\@ifnextchar{_}{\diff@n@d^#1}{\diff@n@fun^#1}}
\def\diff@n@d^#1_#2{\frac{\textrm{d}^#1}{\textrm{d}#2^#1}}
\def\diff@n@fun^#1#2{\@ifnextchar{_}{\diff@n@fun@d^#1#2}{\textrm{d}^#1#2}}
\def\diff@n@fun@d^#1#2_#3{\frac{\textrm{d}^#1 #2}{\textrm{d}#3^#1}}
\def\diff@one@d_#1{\frac{\textrm{d}}{\textrm{d}#1}}
\def\diff@one@fun#1{\@ifnextchar{_}{\diff@one@fun@d #1}{\textrm{d}#1}}
\def\diff@one@fun@d#1_#2{\frac{\textrm{d}#1}{\textrm{d}#2}}
\newcommand*{\diff}{\@ifnextchar{^}{\diff@n}
  {\@ifnextchar{_}{\diff@one@d}{\diff@one@fun}}}
%
%Partieller Diff-Operator.
\def\pdiff@n^#1{\@ifnextchar{_}{\pdiff@n@d^#1}{\pdiff@n@fun^#1}}
\def\pdiff@n@d^#1_#2{\frac{\partial^#1}{\partial#2^#1}}
\def\pdiff@n@fun^#1#2{\@ifnextchar{_}{\pdiff@n@fun@d^#1#2}{\partial^#1#2}}
\def\pdiff@n@fun@d^#1#2_#3{\frac{\partial^#1 #2}{\partial#3^#1}}
\def\pdiff@one@d_#1{\frac{\partial}{\partial #1}}
\def\pdiff@one@fun#1{\@ifnextchar{_}{\pdiff@one@fun@d #1}{\partial#1}}
\def\pdiff@one@fun@d#1_#2{\frac{\partial#1}{\partial#2}}
\newcommand*{\pdiff}{\@ifnextchar{^}{\pdiff@n}
  {\@ifnextchar{_}{\pdiff@one@d}{\pdiff@one@fun}}}
\makeatother
%
%Das gleich nur mit etwas andere Syntax. Die Potenz der Differentiation wird erst
%zum Schluss angegeben. Somit lautet die Syntax:
%
% \diff_x^2  ->  d^2/dx^2
% \diff f_x^2  ->  d^2f/dx^2
% \diff{f(x^5)}_x^2  ->  d^2(f(x^5))/dx^2
% Ansonsten wie Oben.
%
%Ersetzt man \diff durch \pdiff, so wird der partieller
%Differentialoperator dargestellt.
%
%\makeatletter
%\def\diff@#1{\@ifnextchar{_}{\diff@fun#1}{\textrm{d} #1}}
%\def\diff@one_#1{\@ifnextchar{^}{\diff@n{#1}}%
%  {\frac{\textrm d}{\textrm{d} #1}}}
%\def\diff@fun#1_#2{\@ifnextchar{^}{\diff@fun@n#1_#2}%
%  {\frac{\textrm d #1}{\textrm{d} #2}}}
%\def\diff@n#1^#2{\frac{\textrm d^#2}{\textrm{d}#1^#2}}
%\def\diff@fun@n#1_#2^#3{\frac{\textrm d^#3 #1}%
%  {\textrm{d}#2^#3}}
%\def\diff{\@ifnextchar{_}{\diff@one}{\diff@}}
%\newcommand*{\diff}{\@ifnextchar{_}{\diff@one}{\diff@}}
%
%Partieller Diff-Operator.
%\def\pdiff@#1{\@ifnextchar{_}{\pdiff@fun#1}{\partial #1}}
%\def\pdiff@one_#1{\@ifnextchar{^}{\pdiff@n{#1}}%
%  {\frac{\partial}{\partial #1}}}
%\def\pdiff@fun#1_#2{\@ifnextchar{^}{\pdiff@fun@n#1_#2}%
%  {\frac{\partial #1}{\partial #2}}}
%\def\pdiff@n#1^#2{\frac{\partial^#2}{\partial #1^#2}}
%\def\pdiff@fun@n#1_#2^#3{\frac{\partial^#3 #1}%
%  {\partial #2^#3}}
%\newcommand*{\pdiff}{\@ifnextchar{_}{\pdiff@one}{\pdiff@}}
%\makeatother

%-------------------------------------------------------------------------------
%%Nützliche Makros um in der Quantenmechanik Bras, Kets und das Skalarprodukt
%%zwischen den beiden darzustellen.
%%Benutzung:
%% \bra{x}  ->    < x |
%% \ket{x}  ->    | x >
%% \braket{x}{y} ->   < x | y >

\newcommand\bra[1]{\left\langle #1 \right|}
\newcommand\ket[1]{\left| #1 \right\rangle}
\newcommand\braket[2]{%
  \left\langle #1\vphantom{#2} \right.%
  \left|\vphantom{#1#2}\right.%
  \left. \vphantom{#1}#2 \right\rangle}%

%-------------------------------------------------------------------------------
%%Aus dem Buch:
%%Titel:  Latex in Naturwissenschaften und Mathematik
%%Autor:  Herbert Voß
%%Verlag: Franzis Verlag, 2006
%%ISBN:   3772374190, 9783772374197
%%
%%Hier werden drei Makros definiert:\mathllap, \mathclap und \mathrlap, welche
%%analog zu den aus Latex bekannten \rlap und \llap arbeiten, d.h. selbst
%%keinerlei horizontalen Platz benötigen, aber dennoch zentriert zum aktuellen
%%Punkt erscheinen.

\newcommand*\mathllap{\mathstrut\mathpalette\mathllapinternal}
\newcommand*\mathllapinternal[2]{\llap{$\mathsurround=0pt#1{#2}$}}
\newcommand*\clap[1]{\hbox to 0pt{\hss#1\hss}}
\newcommand*\mathclap{\mathpalette\mathclapinternal}
\newcommand*\mathclapinternal[2]{\clap{$\mathsurround=0pt#1{#2}$}}
\newcommand*\mathrlap{\mathpalette\mathrlapinternal}
\newcommand*\mathrlapinternal[2]{\rlap{$\mathsurround=0pt#1{#2}$}}

%%Das Gleiche nur mit \def statt \newcommand.
%\def\mathllap{\mathpalette\mathllapinternal}
%\def\mathllapinternal#1#2{%
%  \llap{$\mathsurround=0pt#1{#2}$}% $
%}
%\def\clap#1{\hbox to 0pt{\hss#1\hss}}
%\def\mathclap{\mathpalette\mathclapinternal}
%\def\mathclapinternal#1#2{%
%  \clap{$\mathsurround=0pt#1{#2}$}%
%}
%\def\mathrlap{\mathpalette\mathrlapinternal}
%\def\mathrlapinternal#1#2{%
%  \rlap{$\mathsurround=0pt#1{#2}$}% $
%}

%-------------------------------------------------------------------------------
%%Hier werden zwei neue Makros definiert \overbr und \underbr welche analog zu
%%\overbrace und \underbrace funktionieren jedoch die Gleichung nicht
%%'zerreißen'. Dies wird ermöglicht durch das \mathclap Makro.

\def\overbr#1^#2{\overbrace{#1}^{\mathclap{#2}}}
\def\underbr#1_#2{\underbrace{#1}_{\mathclap{#2}}}



\begin{document}

\section*{Zeitabhängige Störungstheorie}

Wir betrachten einen Hamiltonoperator der aus einem zeitunabhänigen Teil \(H_0\) und einer zeitabhängigen Störung \(V(t)\) besteht.

\begin{align}
  \label{eq:2}
  H=H_0+V(t)
\end{align}

Die Eigenzustände von \(H_0\) sind gegeben durch

\begin{align}
  \label{eq:1}
   H_0 |n\rangle = E_n|n\rangle
\end{align}

Da der gesamte Hamiltonoperator zeitabhängig ist gibt es keine stationäre Zustände. Deswegen betrachten wir die Übergangswahrscheinlichkeiten von einem Zustand \(\ket{n}\) zu einem Zustand \(\ket{m}\). Wir definieren den Zustand \(\ket{\alpha}\) den wir dann nach den Eigenzuständen \(\ket{n}\) des \(H_0\)-Operators entwickeln

\begin{align}
  \label{eq:5}
  \ket{\alpha} = \mathds 1 \ket{\alpha} = \sum_n \ket{n}\underbr{\bra{n} \ket{\alpha}}_{c_n} = \sum_n c_n \ket{n}
\end{align}

Die Zeitenwicklung des Zustands \(\ket{\alpha}\) ist gegeben durch

\begin{align}
  \label{eq:6}
  \ket{\alpha,t} &= e^{-\frac{i}{\hbar}Ht}\ket{\alpha} =  e^{-\frac{i}{\hbar}H_0t} e^{-\frac{i}{\hbar}V(t)t}\ket{\alpha} \stackrel{\eqref{eq:5}}= e^{-\frac{i}{\hbar}H_0t} e^{-\frac{i}{\hbar}V(t)t} \sum_n c_n \ket{n} \\
&=\sum_n c_n  e^{-\frac{i}{\hbar}V(t)t} e^{-\frac{i}{\hbar}H_0t} \ket{n} \\
&=\sum_n \underbr{c_n  e^{-\frac{i}{\hbar}V(t)t} }_{c_n(t)}e^{-\frac{i}{\hbar}E_nt} \ket{n}
\end{align}

Damit lassen sich die zeitabhängigen Eigenzustände des gesamten Hamiltonoperators schreiben als

\begin{align}
  \label{eq:7}
 \ket{\alpha,t} =\sum_n c_n(t) e^{-\frac{i}{\hbar}E_nt} \ket{n} 
\end{align}

Aus der Gleichung (\ref{eq:6}) sieht man dass die Zeitabhängigkeit von \(c_n\) nur von \(V(t)\) verursacht wird. Desweiteren lässt sich die Wahrscheinlichkeit den Zustand \(\ket{n}\) zu finden mit \(|c_n(t)|^2\) berechnen.

\subsection*{Wechselwirkungsbild}

In der Zeitabhängigen Störungstheorie ist es zweckmäßig vom Schrödingerbild in Wechselwirkungsbild zu wechseln. Dabei hat das WW-Bild volgende Eigenschaften. Für ein Zustand im WW-Bild gilt

\begin{align}
  \label{eq:3}
  |\alpha,t\rangle_I = e^{iH_0t/\hbar}|\alpha,t\rangle_S
\end{align}

Für ein Operator gilt

\begin{align}
  \label{eq:4}
   A_I(t) =  e^{iH_0t/\hbar} A_S e^{-iH_0t/\hbar}
\end{align}

Wir wollen eine schrödinger-artige Gleichung im WW-Bild herleiten

\begin{align}
  \label{eq:8}
 i\hbar \frac{\partial}{\partial t}|\alpha,t_0;t\rangle_I &= i\hbar \frac{\partial}{\partial t}(e^{\frac{i}{\hbar}H_0t}|\alpha,t_0,t\rangle_S) \notag\\
&= i\hbar\left(\frac{i}{\hbar}H_0 e^{\frac{i}{\hbar}H_0 t}|\alpha,t_i,t\rangle_S +  e^{\frac{i}{\hbar}H_0t}\underbrace{\frac{\partial}{\partial t}|\alpha,t_0,t\rangle_S}_{\frac{1}{i\hbar}(H_0+V)|\alpha,t_0,t\rangle_S} \right) \quad |\text{mit SG:}\quad H|\psi(t)\rangle=i\hbar \frac{\partial}{\partial t}|\psi(t)\rangle  \notag  \\
&= -H_0 e^{\frac{i}{\hbar}H_0 t}|\alpha,t_i,t\rangle_S + e^{\frac{i}{\hbar}H_0t}(H_0+V)|\alpha,t_0;t\rangle_S \notag \\
&= e^{\frac{i}{\hbar}H_0t}V\cdot\mathbb 1\cdot|\alpha,t_0;t\rangle_S\notag\\
&= \underbrace{e^{\frac{i}{\hbar}H_0t}Ve^{-\frac{i}{\hbar}H_0t}}_{V_I}\cdot \underbrace{e^{\frac{i}{\hbar}H_0t}|\alpha,t_0;t\rangle_S}_{|\alpha,t_0;t\rangle_I}
\end{align}

Damit lautet die schrödinger-artige Gleichung im WW-Bild

\begin{align}
  \label{eq:9}
  \boxed{i\hbar \frac{\partial}{\partial t} |\alpha,t_0;t\rangle_I = V_I|\alpha,t_0;t\rangle_I}
\end{align}
Man sieht dass diese Gleichung unabhängig von dem stationäre Anteil des Hamiltonoperators \(H_0\) ist.

\subsection*{Lösung der schrödinger-artigen Gleichung}

Um die zeitabhängigen Koeffizienten \(c_n(t)\) zu bestimmen und damit auch die Wahrscheinlichkeit das System in einem bestimmen Zustand n berechnen zu können müssen die schrödinger-artigen Gleichung (\ref{eq:9}) wie folgt umschreiben

\begin{align}
  \label{eq:10}
  \bra{n}\cdot|\qquad  i\hbar \frac{\partial}{\partial t}|\alpha,t_0,t\rangle_I &= V_I|\alpha,t_0,t\rangle_I \notag \\
 i\hbar \frac{\partial}{\partial t}\braket{n}{\alpha,t_0,t}_I &= \bra{n}V_I|\mathds 1|\alpha,t_0,t\rangle_I \notag \\
i\hbar \frac{\partial}{\partial t}\underbr{\braket{n}{\alpha,t_0,t}_I}_{c_n(t)} &= \sum_m\bra{n}V_I \ket{m}\underbr{\bra{m}\alpha,t_0,t\rangle_I}_{c_m(t)} \notag \\
i\hbar \frac{\partial}{\partial t}c_n(t) &= \sum_m\bra{n}V_I \ket{m}c_m(t) 
\end{align}

Sehen uns das Matrixelement \(\bra{n}V_I \ket{m}c_m(t) \) genauer an

\begin{align}
  \label{eq:11}
  \bra{n}V_I \ket{m} &=  \underbrace{\langle n | e^{\frac{i}{\hbar}H_0t}}_{\langle n|e^{\frac{i}{\hbar}E_nt}}V(t)\underbrace{e^{-\frac{i}{\hbar}H_0t}|m\rangle}_{e^{-\frac{i}{\hbar}E_mt}|m\rangle } \notag\\
& = \langle n|V(t)|m\rangle e^{\frac{i}{\hbar}(E_n-E_m)t} \notag \\
&= V_{nm}(t) e^{i\omega_{nm} t}
\end{align}

Damit erhalten wir mit der Abkürzung \(\omega_{nm} = - \omega_{mn} = \frac{1}{\hbar}(E_n - E_m)\) ein System gekoppelter Differentialgleichungen das es zu lösen gilt

\begin{align}
  \label{eq:12}
  \boxed{i\hbar \frac{\partial}{\partial t}c_n(t) = \sum_m V_{nm}(t) e^{i\omega_{nm}t}c_m(t)}
\end{align}

In Matrixschreibwese sieht die Gleichung (\ref{eq:12}) folgendermaßen aus

\begin{align}
  \label{eq:13}
  i\hbar \begin{pmatrix}\dot c_1\\\dot c_2\\.\\.\\.\end{pmatrix} =
\begin{pmatrix}
V_{11}&V_{12}e^{i\omega_{12}t}&.&.&.\\
V_{21}e^{i\omega_{21}t}&V_{22}&.&.&.\\
  .&.&.&.&.\\
  .&.&.&.&.\\
  .&.&.&.&.
\end{pmatrix}\cdot\begin{pmatrix}c_1\\c_2\\.\\.\\.\end{pmatrix}
\end{align}

Die gekoppelte Differentialgleichung (\ref{eq:12}) ist für hinreichend einfache Systeme mit endlich vielen Zuständen eventuell exakt lösbar. Für Systeme die nicht exakt lösbar sind wendet man die Zeitabhängige Störungsrechnung an.

\subsection*{Zeitabhängige Störungsrechnung}

Wir führen den Zeitevolutionsoperator \(U(t,t_0)\) ein, der im WW-Bild eine Zeittransformation eines zeitunabhängigen Ket durchführt

\begin{align}
  \label{eq:14}
  |\alpha,t_0;t\rangle_I = U_I(t,t_0)|\alpha,t_0;t_0\rangle_I
\end{align}

Einsetzen in der Gleichung (\ref{eq:14}) in die Schrödingerartige Gleichung (\ref{eq:9}) ergibt

\begin{align}
  \label{eq:15}
   i\hbar \frac{\partial}{\partial t}|\alpha,t_0;t\rangle_I  &= V_I|\alpha,t_0;t\rangle_I \notag\\
 i\hbar \frac{\partial}{\partial t}  U_I(t,t_0)|\alpha,t_0;t_0\rangle_I &= V_I U_I(t,t_0)|\alpha,t_0;t_0\rangle_I  \notag\\
|\alpha,t_0;t_0\rangle_I \cdot i\hbar \frac{\partial}{\partial t}  U_I(t,t_0) &= V_I U_I(t,t_0)|\alpha,t_0;t_0\rangle_I \notag\\
 _I\langle \alpha,t_0;t_0| \cdot\Big|\qquad |\alpha,t_0;t_0\rangle_I \cdot i\hbar \frac{\partial}{\partial t}  U_I(t,t_0) &= V_I U_I(t,t_0)|\alpha,t_0;t_0\rangle_I
\end{align}

Damit erhalten wir eine DGL die nicht mehr vom Zustand \(|\alpha,t_0;t\rangle_I\) abhängig ist

\begin{align}
  \label{eq:16}
  \Rightarrow \boxed{i\hbar  \frac{\partial}{\partial t}U_I(t,t_0) = V_IU(t,t_0)}
\end{align}

Um diese DGL zu lösen integrieren wir die Gleichung (\ref{eq:16}) auf beiden Seiten von \(t_0\) bis \(t\) nach \(dt\) mit der Anfangbedingung \(U(t_0,t_0)=1\)

\begin{align}
  \label{eq:17}
i\hbar  \int_{t_0}^{t}dt' \frac{\partial}{\partial t}U_I(t',t_0) &= \int_{t_0}^{t}dt' V_IU(t',t_0)\notag \\
i\hbar \left(  U_I(t,t_0) -\underbrace{U_I(t_0,t_0) }_{1} \right)  &= \int_{t_0}^{t}dt' V_IU(t',t_0)
\end{align}

Damit erhalten wir eine Integralgleichung, die den Vorteil hat, da \(V_I\) klein ist, kann man sie iterativ lösen (damit kleine Glieder vernachlässigt werden können).

\begin{align}
  \label{eq:18}
  U_I^{(n)}(t,t_0) = 1 - \frac{i}{\hbar} \int_{t_0}^{t}dt' V_IU^{(n-1)}(t',t_0)
\end{align}

Damit lauten der Zeitevolutionsoperator in verschiedenen Störungsordnungen

\begin{align}
  U_I^{(0)}(t,t_0) &= U_I^{(0)}(t_0,t_0) = 1  \label{eq:19.0} \\
 U_I^{(1)}(t,t_0) &=  1 - \frac{i}{\hbar} \int_{t_0}^{t}dt' V_IU^{(0)}(t',t_0) =  1 - \frac{i}{\hbar} \int_{t_0}^{t}dt' V_I  \label{eq:19.1}\\
 U_I^{(2)}(t,t_0) &=  1 - \frac{i}{\hbar} \int_{t_0}^{t}dt' V_IU^{(1)}(t',t_0) =  1 - \frac{i}{\hbar} \int_{t_0}^{t}dt' V_I\left(  1 - \frac{i}{\hbar} \int_{t_0}^{t}dt'' V_I \right) \\
&= 1 - \frac{i}{\hbar} \int_{t_0}^{t}dt' V_I(t')  +  \frac{1}{\hbar^2} \int_{t_0}^{t}dt' V_I(t') \int_{t_0}^{t'}dt'' V_I(t'')  \label{eq:19.2}
\end{align}

Man erhält die sogenannte \textit{DYSON-Reihe} für \(U_I^{(\infty)}\)

\begin{align}
  \label{eq:20}
 \boxed{ U_I(t,t_0) = T \sum_{n=0}^\infty \frac{(-i)^n}{n! \hbar^n}\int_{t_0}^tdtV(t')\cdots\int_{t_0}^{t^n}dt^n V(t^n) = Te^{-\frac{i}{\hbar} \int_{t_0}^t dt' V(t') } }
\end{align}

Dabei ist \(T\) der Zeitordnungsoperator, der dafür sorgt, dass die späteren Zeiten nach links und die früheren nach recht kommen, d.h. er sortiert von höheren Zeiten zu kleineren Zeiten.

Wir wollen nun die Übergangswahrscheinlichkeit von einem Inertialzustand \(\ket{i}\) zu einem Endzustand \(\ket{n}\) bestimmen. Dazu betrachten wir den Inertialzustand bei \(t=t_0\) mit, den wir dann mit Hilfe des Zeitevolutionsoperators für beliebige Zeiten entwickeln (vergleiche mit Gleichung (\ref{eq:14}))

\begin{align}
  \label{eq:23}
  |i,t_0,t\rangle_I = U_I(t,t_0)|i\rangle = \mathbb 1\cdot U_I(t,t_0)|i\rangle = \sum_n |n\rangle \underbr{\langle n| U_I(t,t_0)|i\rangle}_{c_n} = \sum_n c_n(t) |n\rangle
\end{align}

Nun möchten wir die Übergangskoeffizienten \(c_n(t)\) des Zeitordnungsoperators \(U_I(t,t_0\) bestimmen.

\begin{align}
  \label{eq:24}
  c_n(t) = \langle n|U_I(t,t_0)|i\rangle = \langle n | Te^{-\frac{i}{\hbar}\int_{t_0}^tdt'V_I(t')}|i\rangle
\end{align}

Für \(U_I\) in 2 Ordnung Störungstheorie, siehe Gleichung (\ref{eq:19.2}), lautet \(c_n(t)\)

\begin{align}
  \label{eq:21}
   c_n(t) &= \langle n|i\rangle - \frac{i}{\hbar}\langle n |\int_{t_0}^t V_I(t')dt'|i\rangle + (\frac{i}{\hbar})^2\langle n |\int_{t_0}^t dt'\int_{t_0}^{t'} V_I(t')V_I(t'')dt'' |i\rangle \notag \\
 &= \langle n|i\rangle - \frac{i}{\hbar}\langle n |\int_{t_0}^t V_I(t')dt'|i\rangle + (\frac{i}{\hbar})^2\langle n |\int_{t_0}^t dt'\int_{t_0}^{t'} V_I(t')\cdot\sum_m |m\rangle \langle m|\cdot V_I(t'')dt'' |i\rangle\notag \\
&= \delta_{ni}+(\frac{-i}{\hbar})\int_{t_0}^t V_{ni}(t')e^{i\omega_{ni}t'}dt' + (\frac{-i}{\hbar})^2\sum_m\int_{t_0}^tdt'\int_{t_0}^{t'} V_{nm}(t')e^{i\omega_{ni}t'} V_{mi}(t'')e^{i\omega_{ni}t''} dt''\notag \\
&= c_n^{(0)}(t)+c_n^{(1)}(t)+c_n^{(2)}(t)
\end{align}

Damit erhalten wir eine Übergangswahrscheinlichkeit von Zustand \(\ket{i}\) zu einem beliebigen Zustand \(\ket{n}\) in 2-ter Näherung zeitabhängigen Störungstheorie

\begin{align}
  \label{eq:22}
  \boxed{P(i\rightarrow n) = |c^{(0)}_n + c_n^{(1)}(t)+c_n^{(2)}(t)|^2}
\end{align}

\end{document}
