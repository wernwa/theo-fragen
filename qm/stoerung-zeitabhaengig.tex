\documentclass[10pt,a4paper,oneside,fleqn]{article}
\usepackage{geometry}
\geometry{a4paper,left=20mm,right=20mm,top=1cm,bottom=2cm}
\usepackage[utf8]{inputenc}
%\usepackage{ngerman}
\usepackage{amsmath}                % brauche ich um dir Formel zu umrahmen.
\usepackage{amsfonts}                % brauche ich für die Mengensymbole
\usepackage{graphicx}
\setlength{\parindent}{0px}
\setlength{\mathindent}{10mm}
\usepackage{bbold}                    %brauche ich für die doppel Zahlen Darstellung (Einheitsmatrix z.B)



\usepackage{color}
\usepackage{titlesec} %sudo apt-get install texlive-latex-extra

\definecolor{darkblue}{rgb}{0.1,0.1,0.55}
\definecolor{verydarkblue}{rgb}{0.1,0.1,0.35}
\definecolor{darkred}{rgb}{0.55,0.2,0.2}

%hyperref Link color
\usepackage[colorlinks=true,
        linkcolor=darkblue,
        citecolor=darkblue,
        filecolor=darkblue,
        pagecolor=darkblue,
        urlcolor=darkblue,
        bookmarks=true,
        bookmarksopen=true,
        bookmarksopenlevel=3,
        plainpages=false,
        pdfpagelabels=true]{hyperref}

\titleformat{\chapter}[display]{\color{darkred}\normalfont\huge\bfseries}{\chaptertitlename\
\thechapter}{20pt}{\Huge}

\titleformat{\section}{\color{darkblue}\normalfont\Large\bfseries}{\thesection}{1em}{}
\titleformat{\subsection}{\color{verydarkblue}\normalfont\large\bfseries}{\thesubsection}{1em}{}

% Notiz Box
\usepackage{fancybox}
\newcommand{\notiz}[1]{\vspace{5mm}\ovalbox{\begin{minipage}{1\textwidth}#1\end{minipage}}\vspace{5mm}}

\usepackage{cancel}
\setcounter{secnumdepth}{3}
\setcounter{tocdepth}{3}





%-------------------------------------------------------------------------------
%Diff-Makro:
%Das Diff-Makro stellt einen Differentialoperator da.
%
%Benutzung:
% \diff  ->  d
% \diff f  ->  df
% \diff^2 f  ->  d^2 f
% \diff_x  ->  d/dx
% \diff^2_x  ->  d^2/dx^2
% \diff f_x  ->  df/dx
% \diff^2 f_x  ->  d^2f/dx^2
% \diff^2{f(x^5)}_x  ->  d^2(f(x^5))/dx^2
%
%Ersetzt man \diff durch \pdiff, so wird der partieller
%Differentialoperator dargestellt.
%
\makeatletter
\def\diff@n^#1{\@ifnextchar{_}{\diff@n@d^#1}{\diff@n@fun^#1}}
\def\diff@n@d^#1_#2{\frac{\textrm{d}^#1}{\textrm{d}#2^#1}}
\def\diff@n@fun^#1#2{\@ifnextchar{_}{\diff@n@fun@d^#1#2}{\textrm{d}^#1#2}}
\def\diff@n@fun@d^#1#2_#3{\frac{\textrm{d}^#1 #2}{\textrm{d}#3^#1}}
\def\diff@one@d_#1{\frac{\textrm{d}}{\textrm{d}#1}}
\def\diff@one@fun#1{\@ifnextchar{_}{\diff@one@fun@d #1}{\textrm{d}#1}}
\def\diff@one@fun@d#1_#2{\frac{\textrm{d}#1}{\textrm{d}#2}}
\newcommand*{\diff}{\@ifnextchar{^}{\diff@n}
  {\@ifnextchar{_}{\diff@one@d}{\diff@one@fun}}}
%
%Partieller Diff-Operator.
\def\pdiff@n^#1{\@ifnextchar{_}{\pdiff@n@d^#1}{\pdiff@n@fun^#1}}
\def\pdiff@n@d^#1_#2{\frac{\partial^#1}{\partial#2^#1}}
\def\pdiff@n@fun^#1#2{\@ifnextchar{_}{\pdiff@n@fun@d^#1#2}{\partial^#1#2}}
\def\pdiff@n@fun@d^#1#2_#3{\frac{\partial^#1 #2}{\partial#3^#1}}
\def\pdiff@one@d_#1{\frac{\partial}{\partial #1}}
\def\pdiff@one@fun#1{\@ifnextchar{_}{\pdiff@one@fun@d #1}{\partial#1}}
\def\pdiff@one@fun@d#1_#2{\frac{\partial#1}{\partial#2}}
\newcommand*{\pdiff}{\@ifnextchar{^}{\pdiff@n}
  {\@ifnextchar{_}{\pdiff@one@d}{\pdiff@one@fun}}}
\makeatother
%
%Das gleich nur mit etwas andere Syntax. Die Potenz der Differentiation wird erst
%zum Schluss angegeben. Somit lautet die Syntax:
%
% \diff_x^2  ->  d^2/dx^2
% \diff f_x^2  ->  d^2f/dx^2
% \diff{f(x^5)}_x^2  ->  d^2(f(x^5))/dx^2
% Ansonsten wie Oben.
%
%Ersetzt man \diff durch \pdiff, so wird der partieller
%Differentialoperator dargestellt.
%
%\makeatletter
%\def\diff@#1{\@ifnextchar{_}{\diff@fun#1}{\textrm{d} #1}}
%\def\diff@one_#1{\@ifnextchar{^}{\diff@n{#1}}%
%  {\frac{\textrm d}{\textrm{d} #1}}}
%\def\diff@fun#1_#2{\@ifnextchar{^}{\diff@fun@n#1_#2}%
%  {\frac{\textrm d #1}{\textrm{d} #2}}}
%\def\diff@n#1^#2{\frac{\textrm d^#2}{\textrm{d}#1^#2}}
%\def\diff@fun@n#1_#2^#3{\frac{\textrm d^#3 #1}%
%  {\textrm{d}#2^#3}}
%\def\diff{\@ifnextchar{_}{\diff@one}{\diff@}}
%\newcommand*{\diff}{\@ifnextchar{_}{\diff@one}{\diff@}}
%
%Partieller Diff-Operator.
%\def\pdiff@#1{\@ifnextchar{_}{\pdiff@fun#1}{\partial #1}}
%\def\pdiff@one_#1{\@ifnextchar{^}{\pdiff@n{#1}}%
%  {\frac{\partial}{\partial #1}}}
%\def\pdiff@fun#1_#2{\@ifnextchar{^}{\pdiff@fun@n#1_#2}%
%  {\frac{\partial #1}{\partial #2}}}
%\def\pdiff@n#1^#2{\frac{\partial^#2}{\partial #1^#2}}
%\def\pdiff@fun@n#1_#2^#3{\frac{\partial^#3 #1}%
%  {\partial #2^#3}}
%\newcommand*{\pdiff}{\@ifnextchar{_}{\pdiff@one}{\pdiff@}}
%\makeatother

%-------------------------------------------------------------------------------
%%Nützliche Makros um in der Quantenmechanik Bras, Kets und das Skalarprodukt
%%zwischen den beiden darzustellen.
%%Benutzung:
%% \bra{x}  ->    < x |
%% \ket{x}  ->    | x >
%% \braket{x}{y} ->   < x | y >

\newcommand\bra[1]{\left\langle #1 \right|}
\newcommand\ket[1]{\left| #1 \right\rangle}
\newcommand\braket[2]{%
  \left\langle #1\vphantom{#2} \right.%
  \left|\vphantom{#1#2}\right.%
  \left. \vphantom{#1}#2 \right\rangle}%

%-------------------------------------------------------------------------------
%%Aus dem Buch:
%%Titel:  Latex in Naturwissenschaften und Mathematik
%%Autor:  Herbert Voß
%%Verlag: Franzis Verlag, 2006
%%ISBN:   3772374190, 9783772374197
%%
%%Hier werden drei Makros definiert:\mathllap, \mathclap und \mathrlap, welche
%%analog zu den aus Latex bekannten \rlap und \llap arbeiten, d.h. selbst
%%keinerlei horizontalen Platz benötigen, aber dennoch zentriert zum aktuellen
%%Punkt erscheinen.

\newcommand*\mathllap{\mathstrut\mathpalette\mathllapinternal}
\newcommand*\mathllapinternal[2]{\llap{$\mathsurround=0pt#1{#2}$}}
\newcommand*\clap[1]{\hbox to 0pt{\hss#1\hss}}
\newcommand*\mathclap{\mathpalette\mathclapinternal}
\newcommand*\mathclapinternal[2]{\clap{$\mathsurround=0pt#1{#2}$}}
\newcommand*\mathrlap{\mathpalette\mathrlapinternal}
\newcommand*\mathrlapinternal[2]{\rlap{$\mathsurround=0pt#1{#2}$}}

%%Das Gleiche nur mit \def statt \newcommand.
%\def\mathllap{\mathpalette\mathllapinternal}
%\def\mathllapinternal#1#2{%
%  \llap{$\mathsurround=0pt#1{#2}$}% $
%}
%\def\clap#1{\hbox to 0pt{\hss#1\hss}}
%\def\mathclap{\mathpalette\mathclapinternal}
%\def\mathclapinternal#1#2{%
%  \clap{$\mathsurround=0pt#1{#2}$}%
%}
%\def\mathrlap{\mathpalette\mathrlapinternal}
%\def\mathrlapinternal#1#2{%
%  \rlap{$\mathsurround=0pt#1{#2}$}% $
%}

%-------------------------------------------------------------------------------
%%Hier werden zwei neue Makros definiert \overbr und \underbr welche analog zu
%%\overbrace und \underbrace funktionieren jedoch die Gleichung nicht
%%'zerreißen'. Dies wird ermöglicht durch das \mathclap Makro.

\def\overbr#1^#2{\overbrace{#1}^{\mathclap{#2}}}
\def\underbr#1_#2{\underbrace{#1}_{\mathclap{#2}}}
\usepackage{amsmath}                % brauche ich um dir Formel zu umrahmen.
\usepackage{amsfonts}


\begin{document}

\section*{Zeitabhängige Störungstheorie}

Wir betrachten einen Hamiltonoperator der aus einem zeitunabhänigen Teil \(H_0\) und einer zeitabhängigen Störung \(V(t)\) besteht.

\begin{align}
  \label{eq:2}
  H=H_0+V(t)
\end{align}

Die Eigenzustände von \(H_0\) sind gegeben durch

\begin{align}
  \label{eq:1}
   H_0 |n\rangle = E_n|n\rangle
\end{align}

Da der gesamte Hamiltonoperator zeitabhängig ist gibt es keine stationäre Zustände. Deswegen betrachten wir die Übergangswahrscheinlichkeiten von einem Zustand \(\ket{n}\) zu einem Zustand \(\ket{m}\). Wir definieren den Zustand \(\ket{\alpha}\) den wir dann nach den Eigenzuständen \(\ket{n}\) des \(H_0\)-Operators entwickeln

\begin{align}
  \label{eq:5}
  \ket{\alpha} = \mathds 1 \ket{\alpha} = \sum_n \ket{n}\underbr{\bra{n} \ket{\alpha}}_{c_n} = \sum_n c_n \ket{n}
\end{align}

Die Zeitenwicklung des Zustands \(\ket{\alpha}\) ist gegeben durch

\begin{align}
  \label{eq:6}
  \ket{\alpha,t} &= e^{-\frac{i}{\hbar}Ht}\ket{\alpha} =  e^{-\frac{i}{\hbar}H_0t} e^{-\frac{i}{\hbar}V(t)t}\ket{\alpha} \stackrel{\eqref{eq:5}}= e^{-\frac{i}{\hbar}H_0t} e^{-\frac{i}{\hbar}V(t)t} \sum_n c_n \ket{n} \\
&=\sum_n c_n  e^{-\frac{i}{\hbar}V(t)t} e^{-\frac{i}{\hbar}H_0t} \ket{n} \\
&=\sum_n \underbr{c_n  e^{-\frac{i}{\hbar}V(t)t} }_{c_n(t)}e^{-\frac{i}{\hbar}E_nt} \ket{n}
\end{align}

Damit lassen sich die zeitabhängigen Eigenzustände des gesamten Hamiltonoperators schreiben als

\begin{align}
  \label{eq:7}
 \ket{\alpha,t} =\sum_n c_n(t) e^{-\frac{i}{\hbar}E_nt} \ket{n} 
\end{align}

Aus der Gleichung (\ref{eq:6}) sieht man dass die Zeitabhängigkeit von \(c_n\) nur von \(V(t)\) verursacht wird. Desweiteren lässt sich die Wahrscheinlichkeit den Zustand \(\ket{n}\) zu finden mit \(|c_n(t)|^2\) berechnen.

\subsection*{Wechselwirkungsbild}

In der Zeitabhängigen Störungstheorie ist es zweckmäßig vom Schrödingerbild in Wechselwirkungsbild zu wechseln. Dabei hat das WW-Bild volgende Eigenschaften. Für ein Zustand im WW-Bild gilt

\begin{align}
  \label{eq:3}
  |\alpha,t\rangle_I = e^{iH_0t/\hbar}|\alpha,t\rangle_S
\end{align}

Für ein Operator gilt

\begin{align}
  \label{eq:4}
   A_I(t) =  e^{iH_0t/\hbar} A_S e^{-iH_0t/\hbar}
\end{align}

Wir wollen eine schrödinger-artige Gleichung im WW-Bild herleiten

\begin{align}
  \label{eq:8}
 i\hbar \frac{\partial}{\partial t}|\alpha,t_0;t\rangle_I &= i\hbar \frac{\partial}{\partial t}(e^{\frac{i}{\hbar}H_0t}|\alpha,t_0,t\rangle_S) \notag\\
&= i\hbar\left(\frac{i}{\hbar}H_0 e^{\frac{i}{\hbar}H_0 t}|\alpha,t_i,t\rangle_S +  e^{\frac{i}{\hbar}H_0t}\underbrace{\frac{\partial}{\partial t}|\alpha,t_0,t\rangle_S}_{\frac{1}{i\hbar}(H_0+V)|\alpha,t_0,t\rangle_S} \right) \quad |\text{mit SG:}\quad H|\psi(t)\rangle=i\hbar \frac{\partial}{\partial t}|\psi(t)\rangle  \notag  \\
&= -H_0 e^{\frac{i}{\hbar}H_0 t}|\alpha,t_i,t\rangle_S + e^{\frac{i}{\hbar}H_0t}(H_0+V)|\alpha,t_0;t\rangle_S \notag \\
&= e^{\frac{i}{\hbar}H_0t}V\cdot\mathbb 1\cdot|\alpha,t_0;t\rangle_S\notag\\
&= \underbrace{e^{\frac{i}{\hbar}H_0t}Ve^{-\frac{i}{\hbar}H_0t}}_{V_I}\cdot \underbrace{e^{\frac{i}{\hbar}H_0t}|\alpha,t_0;t\rangle_S}_{|\alpha,t_0;t\rangle_I}
\end{align}

Damit lautet die schrödinger-artige Gleichung im WW-Bild

\begin{align}
  \label{eq:9}
  \boxed{i\hbar \frac{\partial}{\partial t} |\alpha,t_0;t\rangle_I = V_I|\alpha,t_0;t\rangle_I}
\end{align}
Man sieht dass diese Gleichung unabhängig von dem stationäre Anteil des Hamiltonoperators \(H_0\) ist.

\subsection*{Lösung der schrödinger-artigen Gleichung}

Um die zeitabhängigen Koeffizienten \(c_n(t)\) zu bestimmen und damit auch die Wahrscheinlichkeit das System in einem bestimmen Zustand n berechnen zu können müssen die schrödinger-artigen Gleichung (\ref{eq:9}) wie folgt umschreiben

\begin{align}
  \label{eq:10}
  \bra{n}\cdot|\qquad  i\hbar \frac{\partial}{\partial t}|\alpha,t_0,t\rangle_I &= V_I|\alpha,t_0,t\rangle_I \notag \\
 i\hbar \frac{\partial}{\partial t}\braket{n}{\alpha,t_0,t}_I &= \bra{n}V_I|\mathds 1|\alpha,t_0,t\rangle_I \notag \\
i\hbar \frac{\partial}{\partial t}\underbr{\braket{n}{\alpha,t_0,t}_I}_{c_n(t)} &= \sum_m\bra{n}V_I \ket{m}\underbr{\bra{m}\alpha,t_0,t\rangle_I}_{c_m(t)} \notag \\
i\hbar \frac{\partial}{\partial t}c_n(t) &= \sum_m\bra{n}V_I \ket{m}c_m(t) 
\end{align}

Sehen uns das Matrixelement \(\bra{n}V_I \ket{m}c_m(t) \) genauer an

\begin{align}
  \label{eq:11}
  \bra{n}V_I \ket{m} &=  \underbrace{\langle n | e^{\frac{i}{\hbar}H_0t}}_{\langle n|e^{\frac{i}{\hbar}E_nt}}V(t)\underbrace{e^{-\frac{i}{\hbar}H_0t}|m\rangle}_{e^{-\frac{i}{\hbar}E_mt}|m\rangle } \notag\\
& = \langle n|V(t)|m\rangle e^{\frac{i}{\hbar}(E_n-E_m)t} \notag \\
&= V_{nm}(t) e^{i\omega_{nm} t}
\end{align}

Damit erhalten wir mit der Abkürzung \(\omega_{nm} = - \omega_{mn} = \frac{1}{\hbar}(E_n - E_m)\) ein System gekoppelter Differentialgleichungen das es zu lösen gilt

\begin{align}
  \label{eq:12}
  \boxed{i\hbar \frac{\partial}{\partial t}c_n(t) = \sum_m V_{nm}(t) e^{i\omega_{nm}t}c_m(t)}
\end{align}

In Matrixschreibwese sieht die Gleichung (\ref{eq:12}) folgendermaßen aus

\begin{align}
  \label{eq:13}
  i\hbar \begin{pmatrix}\dot c_1\\\dot c_2\\.\\.\\.\end{pmatrix} =
\begin{pmatrix}
V_{11}&V_{12}e^{i\omega_{12}t}&.&.&.\\
V_{21}e^{i\omega_{21}t}&V_{22}&.&.&.\\
  .&.&.&.&.\\
  .&.&.&.&.\\
  .&.&.&.&.
\end{pmatrix}\cdot\begin{pmatrix}c_1\\c_2\\.\\.\\.\end{pmatrix}
\end{align}

Die gekoppelte Differentialgleichung (\ref{eq:12}) ist für hinreichend einfache Systeme mit endlich vielen Zuständen eventuell exakt lösbar. Für Systeme die nicht exakt lösbar sind wendet man die Zeitabhängige Störungsrechnung an.

\section*{Zeitabhängige Störungsrechnung}

Wir führen den Zeitevolutionsoperator \(U(t,t_0)\) ein, der im WW-Bild eine Zeittransformation eines zeitunabhängigen Ket durchführt

\begin{align}
  \label{eq:14}
  |\alpha,t_0;t\rangle_I = U_I(t,t_0)|\alpha,t_0;t_0\rangle_I
\end{align}

Einsetzen in der Gleichung (\ref{eq:14}) in die Schrödingerartige Gleichung (\ref{eq:9}) ergibt

\begin{align}
  \label{eq:15}
   i\hbar \frac{\partial}{\partial t}|\alpha,t_0;t\rangle_I  &= V_I|\alpha,t_0;t\rangle_I \notag\\
 i\hbar \frac{\partial}{\partial t}  U_I(t,t_0)|\alpha,t_0;t_0\rangle_I &= V_I U_I(t,t_0)|\alpha,t_0;t_0\rangle_I  \notag\\
|\alpha,t_0;t_0\rangle_I \cdot i\hbar \frac{\partial}{\partial t}  U_I(t,t_0) &= V_I U_I(t,t_0)|\alpha,t_0;t_0\rangle_I \notag\\
 _I\langle \alpha,t_0;t_0| \cdot\Big|\qquad |\alpha,t_0;t_0\rangle_I \cdot i\hbar \frac{\partial}{\partial t}  U_I(t,t_0) &= V_I U_I(t,t_0)|\alpha,t_0;t_0\rangle_I
\end{align}

Damit erhalten wir eine DGL die nicht mehr vom Zustand \(|\alpha,t_0;t\rangle_I\) abhängig ist

\begin{align}
  \label{eq:16}
  \Rightarrow \boxed{i\hbar  \frac{\partial}{\partial t}U_I(t,t_0) = V_IU(t,t_0)}
\end{align}

Um diese DGL zu lösen integrieren wir die Gleichung (\ref{eq:16}) auf beiden Seiten von \(t_0\) bis \(t\) nach \(dt\) mit der Anfangbedingung \(U(t_0,t_0)=1\)

\begin{align}
  \label{eq:17}
i\hbar  \int_{t_0}^{t}dt' \frac{\partial}{\partial t}U_I(t',t_0) &= \int_{t_0}^{t}dt' V_IU(t',t_0)\notag \\
i\hbar \left(  U_I(t,t_0) -\underbrace{U_I(t_0,t_0) }_{1} \right)  &= \int_{t_0}^{t}dt' V_IU(t',t_0)
\end{align}

Damit erhalten wir eine Integralgleichung, die den Vorteil hat, da \(V_I\) klein ist, kann man sie iterativ lösen (damit kleine Glieder vernachlässigt werden können).

\begin{align}
  \label{eq:18}
  U_I^{(n)}(t,t_0) = 1 - \frac{i}{\hbar} \int_{t_0}^{t}dt' V_IU^{(n-1)}(t',t_0)
\end{align}

Damit lauten der Zeitevolutionsoperator in verschiedenen Störungsordnungen

\begin{align}
  U_I^{(0)}(t,t_0) &= U_I^{(0)}(t_0,t_0) = 1  \label{eq:19.0} \\
 U_I^{(1)}(t,t_0) &=  1 - \frac{i}{\hbar} \int_{t_0}^{t}dt' V_IU^{(0)}(t',t_0) =  1 - \frac{i}{\hbar} \int_{t_0}^{t}dt' V_I  \label{eq:19.1}\\
 U_I^{(2)}(t,t_0) &=  1 - \frac{i}{\hbar} \int_{t_0}^{t}dt' V_IU^{(1)}(t',t_0) =  1 - \frac{i}{\hbar} \int_{t_0}^{t}dt' V_I\left(  1 - \frac{i}{\hbar} \int_{t_0}^{t}dt'' V_I \right) \\
&= 1 - \frac{i}{\hbar} \int_{t_0}^{t}dt' V_I(t')  +  \frac{1}{\hbar^2} \int_{t_0}^{t}dt' V_I(t') \int_{t_0}^{t'}dt'' V_I(t'')  \label{eq:19.2}
\end{align}

Man erhält die sogenannte \textit{DYSON-Reihe} für \(U_I^{(\infty)}\)

\begin{align}
  \label{eq:20}
 \boxed{ U_I(t,t_0) = T \sum_{n=0}^\infty \frac{(-i)^n}{n! \hbar^n}\int_{t_0}^tdtV(t')\cdots\int_{t_0}^{t^n}dt^n V(t^n) = Te^{-\frac{i}{\hbar} \int_{t_0}^t dt' V(t') } }
\end{align}

Dabei ist \(T\) der Zeitordnungsoperator, der dafür sorgt, dass die späteren Zeiten nach links und die früheren nach recht kommen, d.h. er sortiert von höheren Zeiten zu kleineren Zeiten.

Wir wollen nun die Übergangswahrscheinlichkeit von einem Inertialzustand \(\ket{i}\) zu einem Endzustand \(\ket{n}\) bestimmen. Dazu betrachten wir den Inertialzustand bei \(t=t_0\) mit, den wir dann mit Hilfe des Zeitevolutionsoperators für beliebige Zeiten entwickeln (vergleiche mit Gleichung (\ref{eq:14}))

\begin{align}
  \label{eq:23}
  |i,t_0,t\rangle_I = U_I(t,t_0)|i\rangle = \mathbb 1\cdot U_I(t,t_0)|i\rangle = \sum_n |n\rangle \underbr{\langle n| U_I(t,t_0)|i\rangle}_{c_n} = \sum_n c_n(t) |n\rangle
\end{align}

Nun möchten wir die Übergangskoeffizienten \(c_n(t)\) des Zeitordnungsoperators \(U_I(t,t_0\) bestimmen.

\begin{align}
  \label{eq:24}
  c_n(t) = \langle n|U_I(t,t_0)|i\rangle = \langle n | Te^{-\frac{i}{\hbar}\int_{t_0}^tdt'V_I(t')}|i\rangle
\end{align}

Für \(U_I\) in 2 Ordnung Störungstheorie, siehe Gleichung (\ref{eq:19.2}), lautet \(c_n(t)\)

\begin{align}
  \label{eq:21}
   c_n(t) &= \langle n|i\rangle - \frac{i}{\hbar}\langle n |\int_{t_0}^t V_I(t')dt'|i\rangle + (\frac{i}{\hbar})^2\langle n |\int_{t_0}^t dt'\int_{t_0}^{t'} V_I(t')V_I(t'')dt'' |i\rangle \notag \\
 &= \langle n|i\rangle - \frac{i}{\hbar}\langle n |\int_{t_0}^t V_I(t')dt'|i\rangle + (\frac{i}{\hbar})^2\langle n |\int_{t_0}^t dt'\int_{t_0}^{t'} V_I(t')\cdot\sum_m |m\rangle \langle m|\cdot V_I(t'')dt'' |i\rangle\notag \\
&= \delta_{ni}+(\frac{-i}{\hbar})\int_{t_0}^t V_{ni}(t')e^{i\omega_{ni}t'}dt' + (\frac{-i}{\hbar})^2\sum_m\int_{t_0}^tdt'\int_{t_0}^{t'} V_{nm}(t')e^{i\omega_{ni}t'} V_{mi}(t'')e^{i\omega_{ni}t''} dt''\notag \\
&= c_n^{(0)}(t)+c_n^{(1)}(t)+c_n^{(2)}(t)
\end{align}

Damit erhalten wir eine Übergangswahrscheinlichkeit von Zustand \(\ket{i}\) zu einem beliebigen Zustand \(\ket{n}\) in 2-ter Näherung zeitabhängigen Störungstheorie

\begin{align}
  \label{eq:22}
  \boxed{P(i\rightarrow n) = |c^{(0)}_n + c_n^{(1)}(t)+c_n^{(2)}(t)|^2}
\end{align}

\subsection*{Beispiel: Konstante Störung}

Wir betrachten nun eine zeitlich konstante Störung \(V(t)\) für die gilt

\begin{align}
  \label{eq:19}
  V(t) =\begin{cases}0& t<0\\ V&t\ge 0    \end{cases}
\end{align}

Bestimme die Übergangswahrscheinlichkeit für \(n \ne i\) in erster Ordnung der zeitabhängigen Störungsrechnung. Laut Gleichung (\ref{eq:22}) gilt

\begin{align}
  \label{eq:25}
  P(i\rightarrow n) = |\underbr{c^{(0)}_n}_{=0} + c_n^{(1)}(t)|^2 = |c_n^{(1)}(t)|^2 = \left|(\frac{-i}{\hbar})\langle n |\int_{t_0}^t V_I(t')dt'|i\rangle\right|^2 =   \left|(\frac{-i}{\hbar})\int_{t_0}^t V_{ni}(t')e^{i\omega_{ni}t'}dt'\right|^2
\end{align}

Mit der Bedingung (\ref{eq:19}) \(t_0=0\) und \(V(t)=V\) lautet die Übergangswahrscheinlichkeit (\ref{eq:25})

\begin{align}
  \label{eq:26}
  P(i\rightarrow n) = \frac{1}{\hbar^2}| V_{ni} \int_{0}^t e^{i\omega_{ni}t'}dt'|^2
\end{align}

Wir machen eine Nebenrechnung für das Integral

\begin{align}
  \label{eq:27}
  \int_{0}^t e^{i\omega_{ni}t'}dt' &= \left[ \frac{e^{i\omega_{ni} t}}{i\omega_{ni}}  \right]_0^t = \frac{1}{i\omega_{ni}}\left( e^{i\omega_{ni} t}-1 \right) = \frac{1}{i\omega_{ni}}\left( e^{\frac{i\omega_{ni} t}{2}+\frac{i\omega_{ni} t}{2}}-1 \right) =  \frac{1}{i\omega_{ni}}\left( e^{\frac{i\omega_{ni} t}{2}}e^{\frac{i\omega_{ni} t}{2}}-1 \right) \notag \\
&=  \frac{1}{i\omega_{ni}}\left( e^{\frac{i\omega_{ni} t}{2}}-\frac{1}{e^{\frac{i\omega_{ni} t}{2}}} \right)e^{\frac{i\omega_{ni} t}{2}} = \frac{1}{i\omega_{ni}} \underbr{\left( e^{\frac{i\omega_{ni} t}{2}}-e^{-\frac{i\omega_{ni} t}{2}}\right)}_{2i\sin(\frac{\omega_{ni} t}{2})} e^{\frac{i\omega_{ni} t}{2}} = \frac{2}{\omega_{ni}}\sin(\frac{\omega_{ni} t}{2})e^{\frac{i\omega_{ni} t}{2}}
\end{align}

Die Nebenrechnung (\ref{eq:27}) in das Integral eingesetzt lautet die Übergangswahrscheinlichkeit nun

\begin{align}
  \label{eq:28}
  P(i\rightarrow n) = \frac{1}{\hbar^2} \left| V_{ni}\frac{2}{\omega_{ni}}\sin(\frac{\omega_{ni} t}{2})e^{\frac{i\omega_{ni} t}{2}} \right|^2 = \frac{4}{\hbar^2\omega_{ni}^2}|V_{ni}|^2\sin^2(\frac{\omega_{ni} t}{2})
\end{align}

Die Übergangswahrscheinlichkeit kann man wie folgt schreiben

\begin{align}
  \label{eq:29}
   P(i\rightarrow n) = |V_{ni}|^2 f(\omega_{ni}) 
\end{align}
\begin{figure}[!thb]
\begin{minipage}{1.0\linewidth}
  
  \centering
  % GNUPLOT: LaTeX picture
\setlength{\unitlength}{0.240900pt}
\ifx\plotpoint\undefined\newsavebox{\plotpoint}\fi
\sbox{\plotpoint}{\rule[-0.200pt]{0.400pt}{0.400pt}}%
\begin{picture}(1500,900)(0,0)
\sbox{\plotpoint}{\rule[-0.200pt]{0.400pt}{0.400pt}}%
\put(160.0,82.0){\rule[-0.200pt]{4.818pt}{0.400pt}}
\put(140,82){\makebox(0,0)[r]{$0$}}
\put(1419.0,82.0){\rule[-0.200pt]{4.818pt}{0.400pt}}
\put(160.0,152.0){\rule[-0.200pt]{4.818pt}{0.400pt}}
\put(140,152){\makebox(0,0)[r]{$0.1$}}
\put(1419.0,152.0){\rule[-0.200pt]{4.818pt}{0.400pt}}
\put(160.0,221.0){\rule[-0.200pt]{4.818pt}{0.400pt}}
\put(140,221){\makebox(0,0)[r]{$0.2$}}
\put(1419.0,221.0){\rule[-0.200pt]{4.818pt}{0.400pt}}
\put(160.0,291.0){\rule[-0.200pt]{4.818pt}{0.400pt}}
\put(140,291){\makebox(0,0)[r]{$0.3$}}
\put(1419.0,291.0){\rule[-0.200pt]{4.818pt}{0.400pt}}
\put(160.0,360.0){\rule[-0.200pt]{4.818pt}{0.400pt}}
\put(140,360){\makebox(0,0)[r]{$0.4$}}
\put(1419.0,360.0){\rule[-0.200pt]{4.818pt}{0.400pt}}
\put(160.0,430.0){\rule[-0.200pt]{4.818pt}{0.400pt}}
\put(140,430){\makebox(0,0)[r]{$0.5$}}
\put(1419.0,430.0){\rule[-0.200pt]{4.818pt}{0.400pt}}
\put(160.0,499.0){\rule[-0.200pt]{4.818pt}{0.400pt}}
\put(140,499){\makebox(0,0)[r]{$0.6$}}
\put(1419.0,499.0){\rule[-0.200pt]{4.818pt}{0.400pt}}
\put(160.0,568.0){\rule[-0.200pt]{4.818pt}{0.400pt}}
\put(140,568){\makebox(0,0)[r]{$0.7$}}
\put(1419.0,568.0){\rule[-0.200pt]{4.818pt}{0.400pt}}
\put(160.0,638.0){\rule[-0.200pt]{4.818pt}{0.400pt}}
\put(140,638){\makebox(0,0)[r]{$0.8$}}
\put(1419.0,638.0){\rule[-0.200pt]{4.818pt}{0.400pt}}
\put(160.0,707.0){\rule[-0.200pt]{4.818pt}{0.400pt}}
\put(140,707){\makebox(0,0)[r]{$0.9$}}
\put(1419.0,707.0){\rule[-0.200pt]{4.818pt}{0.400pt}}
\put(160.0,777.0){\rule[-0.200pt]{4.818pt}{0.400pt}}
\put(140,777){\makebox(0,0)[r]{$1$}}
\put(1419.0,777.0){\rule[-0.200pt]{4.818pt}{0.400pt}}
\put(160.0,82.0){\rule[-0.200pt]{0.400pt}{4.818pt}}
\put(160,41){\makebox(0,0){$-4\pi$}}
\put(160.0,757.0){\rule[-0.200pt]{0.400pt}{4.818pt}}
\put(320.0,82.0){\rule[-0.200pt]{0.400pt}{4.818pt}}
\put(320,41){\makebox(0,0){$-3\pi$}}
\put(320.0,757.0){\rule[-0.200pt]{0.400pt}{4.818pt}}
\put(480.0,82.0){\rule[-0.200pt]{0.400pt}{4.818pt}}
\put(480,41){\makebox(0,0){$-2\pi$}}
\put(480.0,757.0){\rule[-0.200pt]{0.400pt}{4.818pt}}
\put(799.0,82.0){\rule[-0.200pt]{0.400pt}{4.818pt}}
\put(799,41){\makebox(0,0){0}}
\put(799.0,757.0){\rule[-0.200pt]{0.400pt}{4.818pt}}
\put(1119.0,82.0){\rule[-0.200pt]{0.400pt}{4.818pt}}
\put(1119,41){\makebox(0,0){$2\pi$}}
\put(1119.0,757.0){\rule[-0.200pt]{0.400pt}{4.818pt}}
\put(1279.0,82.0){\rule[-0.200pt]{0.400pt}{4.818pt}}
\put(1279,41){\makebox(0,0){$3\pi$}}
\put(1279.0,757.0){\rule[-0.200pt]{0.400pt}{4.818pt}}
\put(1439.0,82.0){\rule[-0.200pt]{0.400pt}{4.818pt}}
\put(1439,41){\makebox(0,0){$4\pi$}}
\put(1439.0,757.0){\rule[-0.200pt]{0.400pt}{4.818pt}}
\put(160.0,82.0){\rule[-0.200pt]{0.400pt}{167.425pt}}
\put(160.0,82.0){\rule[-0.200pt]{308.111pt}{0.400pt}}
\put(1439.0,82.0){\rule[-0.200pt]{0.400pt}{167.425pt}}
\put(160.0,777.0){\rule[-0.200pt]{308.111pt}{0.400pt}}
\put(799,839){\makebox(0,0){$f(\omega)=\frac{\sin^2(\omega_{ni}t/2)}{\omega_{ni}^2}$ für $t=1$}}
\put(160,82){\usebox{\plotpoint}}
\put(173,81.67){\rule{3.132pt}{0.400pt}}
\multiput(173.00,81.17)(6.500,1.000){2}{\rule{1.566pt}{0.400pt}}
\put(186,83.17){\rule{2.700pt}{0.400pt}}
\multiput(186.00,82.17)(7.396,2.000){2}{\rule{1.350pt}{0.400pt}}
\put(199,85.17){\rule{2.700pt}{0.400pt}}
\multiput(199.00,84.17)(7.396,2.000){2}{\rule{1.350pt}{0.400pt}}
\multiput(212.00,87.61)(2.695,0.447){3}{\rule{1.833pt}{0.108pt}}
\multiput(212.00,86.17)(9.195,3.000){2}{\rule{0.917pt}{0.400pt}}
\multiput(225.00,90.61)(2.695,0.447){3}{\rule{1.833pt}{0.108pt}}
\multiput(225.00,89.17)(9.195,3.000){2}{\rule{0.917pt}{0.400pt}}
\multiput(238.00,93.61)(2.472,0.447){3}{\rule{1.700pt}{0.108pt}}
\multiput(238.00,92.17)(8.472,3.000){2}{\rule{0.850pt}{0.400pt}}
\multiput(250.00,96.60)(1.797,0.468){5}{\rule{1.400pt}{0.113pt}}
\multiput(250.00,95.17)(10.094,4.000){2}{\rule{0.700pt}{0.400pt}}
\multiput(263.00,100.60)(1.797,0.468){5}{\rule{1.400pt}{0.113pt}}
\multiput(263.00,99.17)(10.094,4.000){2}{\rule{0.700pt}{0.400pt}}
\multiput(276.00,104.61)(2.695,0.447){3}{\rule{1.833pt}{0.108pt}}
\multiput(276.00,103.17)(9.195,3.000){2}{\rule{0.917pt}{0.400pt}}
\multiput(289.00,107.61)(2.695,0.447){3}{\rule{1.833pt}{0.108pt}}
\multiput(289.00,106.17)(9.195,3.000){2}{\rule{0.917pt}{0.400pt}}
\multiput(302.00,110.61)(2.695,0.447){3}{\rule{1.833pt}{0.108pt}}
\multiput(302.00,109.17)(9.195,3.000){2}{\rule{0.917pt}{0.400pt}}
\put(315,112.67){\rule{3.132pt}{0.400pt}}
\multiput(315.00,112.17)(6.500,1.000){2}{\rule{1.566pt}{0.400pt}}
\put(328,113.67){\rule{3.132pt}{0.400pt}}
\multiput(328.00,113.17)(6.500,1.000){2}{\rule{1.566pt}{0.400pt}}
\put(341,113.67){\rule{3.132pt}{0.400pt}}
\multiput(341.00,114.17)(6.500,-1.000){2}{\rule{1.566pt}{0.400pt}}
\put(354,112.67){\rule{3.132pt}{0.400pt}}
\multiput(354.00,113.17)(6.500,-1.000){2}{\rule{1.566pt}{0.400pt}}
\multiput(367.00,111.95)(2.695,-0.447){3}{\rule{1.833pt}{0.108pt}}
\multiput(367.00,112.17)(9.195,-3.000){2}{\rule{0.917pt}{0.400pt}}
\multiput(380.00,108.95)(2.695,-0.447){3}{\rule{1.833pt}{0.108pt}}
\multiput(380.00,109.17)(9.195,-3.000){2}{\rule{0.917pt}{0.400pt}}
\multiput(393.00,105.94)(1.651,-0.468){5}{\rule{1.300pt}{0.113pt}}
\multiput(393.00,106.17)(9.302,-4.000){2}{\rule{0.650pt}{0.400pt}}
\multiput(405.00,101.93)(1.378,-0.477){7}{\rule{1.140pt}{0.115pt}}
\multiput(405.00,102.17)(10.634,-5.000){2}{\rule{0.570pt}{0.400pt}}
\multiput(418.00,96.93)(1.378,-0.477){7}{\rule{1.140pt}{0.115pt}}
\multiput(418.00,97.17)(10.634,-5.000){2}{\rule{0.570pt}{0.400pt}}
\multiput(431.00,91.94)(1.797,-0.468){5}{\rule{1.400pt}{0.113pt}}
\multiput(431.00,92.17)(10.094,-4.000){2}{\rule{0.700pt}{0.400pt}}
\multiput(444.00,87.94)(1.797,-0.468){5}{\rule{1.400pt}{0.113pt}}
\multiput(444.00,88.17)(10.094,-4.000){2}{\rule{0.700pt}{0.400pt}}
\put(457,83.17){\rule{2.700pt}{0.400pt}}
\multiput(457.00,84.17)(7.396,-2.000){2}{\rule{1.350pt}{0.400pt}}
\put(470,81.67){\rule{3.132pt}{0.400pt}}
\multiput(470.00,82.17)(6.500,-1.000){2}{\rule{1.566pt}{0.400pt}}
\put(483,82.17){\rule{2.700pt}{0.400pt}}
\multiput(483.00,81.17)(7.396,2.000){2}{\rule{1.350pt}{0.400pt}}
\multiput(496.00,84.59)(1.378,0.477){7}{\rule{1.140pt}{0.115pt}}
\multiput(496.00,83.17)(10.634,5.000){2}{\rule{0.570pt}{0.400pt}}
\multiput(509.00,89.59)(0.824,0.488){13}{\rule{0.750pt}{0.117pt}}
\multiput(509.00,88.17)(11.443,8.000){2}{\rule{0.375pt}{0.400pt}}
\multiput(522.00,97.58)(0.539,0.492){21}{\rule{0.533pt}{0.119pt}}
\multiput(522.00,96.17)(11.893,12.000){2}{\rule{0.267pt}{0.400pt}}
\multiput(535.58,109.00)(0.493,0.616){23}{\rule{0.119pt}{0.592pt}}
\multiput(534.17,109.00)(13.000,14.771){2}{\rule{0.400pt}{0.296pt}}
\multiput(548.58,125.00)(0.492,0.884){21}{\rule{0.119pt}{0.800pt}}
\multiput(547.17,125.00)(12.000,19.340){2}{\rule{0.400pt}{0.400pt}}
\multiput(560.58,146.00)(0.493,0.972){23}{\rule{0.119pt}{0.869pt}}
\multiput(559.17,146.00)(13.000,23.196){2}{\rule{0.400pt}{0.435pt}}
\multiput(573.58,171.00)(0.493,1.171){23}{\rule{0.119pt}{1.023pt}}
\multiput(572.17,171.00)(13.000,27.877){2}{\rule{0.400pt}{0.512pt}}
\multiput(586.58,201.00)(0.493,1.329){23}{\rule{0.119pt}{1.146pt}}
\multiput(585.17,201.00)(13.000,31.621){2}{\rule{0.400pt}{0.573pt}}
\multiput(599.58,235.00)(0.493,1.488){23}{\rule{0.119pt}{1.269pt}}
\multiput(598.17,235.00)(13.000,35.366){2}{\rule{0.400pt}{0.635pt}}
\multiput(612.58,273.00)(0.493,1.607){23}{\rule{0.119pt}{1.362pt}}
\multiput(611.17,273.00)(13.000,38.174){2}{\rule{0.400pt}{0.681pt}}
\multiput(625.58,314.00)(0.493,1.726){23}{\rule{0.119pt}{1.454pt}}
\multiput(624.17,314.00)(13.000,40.982){2}{\rule{0.400pt}{0.727pt}}
\multiput(638.58,358.00)(0.493,1.805){23}{\rule{0.119pt}{1.515pt}}
\multiput(637.17,358.00)(13.000,42.855){2}{\rule{0.400pt}{0.758pt}}
\multiput(651.58,404.00)(0.493,1.884){23}{\rule{0.119pt}{1.577pt}}
\multiput(650.17,404.00)(13.000,44.727){2}{\rule{0.400pt}{0.788pt}}
\multiput(664.58,452.00)(0.493,1.845){23}{\rule{0.119pt}{1.546pt}}
\multiput(663.17,452.00)(13.000,43.791){2}{\rule{0.400pt}{0.773pt}}
\multiput(677.58,499.00)(0.493,1.845){23}{\rule{0.119pt}{1.546pt}}
\multiput(676.17,499.00)(13.000,43.791){2}{\rule{0.400pt}{0.773pt}}
\multiput(690.58,546.00)(0.493,1.765){23}{\rule{0.119pt}{1.485pt}}
\multiput(689.17,546.00)(13.000,41.919){2}{\rule{0.400pt}{0.742pt}}
\multiput(703.58,591.00)(0.493,1.646){23}{\rule{0.119pt}{1.392pt}}
\multiput(702.17,591.00)(13.000,39.110){2}{\rule{0.400pt}{0.696pt}}
\multiput(716.58,633.00)(0.492,1.616){21}{\rule{0.119pt}{1.367pt}}
\multiput(715.17,633.00)(12.000,35.163){2}{\rule{0.400pt}{0.683pt}}
\multiput(728.58,671.00)(0.493,1.329){23}{\rule{0.119pt}{1.146pt}}
\multiput(727.17,671.00)(13.000,31.621){2}{\rule{0.400pt}{0.573pt}}
\multiput(741.58,705.00)(0.493,1.052){23}{\rule{0.119pt}{0.931pt}}
\multiput(740.17,705.00)(13.000,25.068){2}{\rule{0.400pt}{0.465pt}}
\multiput(754.58,732.00)(0.493,0.853){23}{\rule{0.119pt}{0.777pt}}
\multiput(753.17,732.00)(13.000,20.387){2}{\rule{0.400pt}{0.388pt}}
\multiput(767.58,754.00)(0.493,0.576){23}{\rule{0.119pt}{0.562pt}}
\multiput(766.17,754.00)(13.000,13.834){2}{\rule{0.400pt}{0.281pt}}
\multiput(780.00,769.59)(0.950,0.485){11}{\rule{0.843pt}{0.117pt}}
\multiput(780.00,768.17)(11.251,7.000){2}{\rule{0.421pt}{0.400pt}}
\put(160.0,82.0){\rule[-0.200pt]{3.132pt}{0.400pt}}
\multiput(806.00,774.93)(0.950,-0.485){11}{\rule{0.843pt}{0.117pt}}
\multiput(806.00,775.17)(11.251,-7.000){2}{\rule{0.421pt}{0.400pt}}
\multiput(819.58,766.67)(0.493,-0.576){23}{\rule{0.119pt}{0.562pt}}
\multiput(818.17,767.83)(13.000,-13.834){2}{\rule{0.400pt}{0.281pt}}
\multiput(832.58,750.77)(0.493,-0.853){23}{\rule{0.119pt}{0.777pt}}
\multiput(831.17,752.39)(13.000,-20.387){2}{\rule{0.400pt}{0.388pt}}
\multiput(845.58,728.14)(0.493,-1.052){23}{\rule{0.119pt}{0.931pt}}
\multiput(844.17,730.07)(13.000,-25.068){2}{\rule{0.400pt}{0.465pt}}
\multiput(858.58,700.24)(0.493,-1.329){23}{\rule{0.119pt}{1.146pt}}
\multiput(857.17,702.62)(13.000,-31.621){2}{\rule{0.400pt}{0.573pt}}
\multiput(871.58,665.33)(0.492,-1.616){21}{\rule{0.119pt}{1.367pt}}
\multiput(870.17,668.16)(12.000,-35.163){2}{\rule{0.400pt}{0.683pt}}
\multiput(883.58,627.22)(0.493,-1.646){23}{\rule{0.119pt}{1.392pt}}
\multiput(882.17,630.11)(13.000,-39.110){2}{\rule{0.400pt}{0.696pt}}
\multiput(896.58,584.84)(0.493,-1.765){23}{\rule{0.119pt}{1.485pt}}
\multiput(895.17,587.92)(13.000,-41.919){2}{\rule{0.400pt}{0.742pt}}
\multiput(909.58,539.58)(0.493,-1.845){23}{\rule{0.119pt}{1.546pt}}
\multiput(908.17,542.79)(13.000,-43.791){2}{\rule{0.400pt}{0.773pt}}
\multiput(922.58,492.58)(0.493,-1.845){23}{\rule{0.119pt}{1.546pt}}
\multiput(921.17,495.79)(13.000,-43.791){2}{\rule{0.400pt}{0.773pt}}
\multiput(935.58,445.45)(0.493,-1.884){23}{\rule{0.119pt}{1.577pt}}
\multiput(934.17,448.73)(13.000,-44.727){2}{\rule{0.400pt}{0.788pt}}
\multiput(948.58,397.71)(0.493,-1.805){23}{\rule{0.119pt}{1.515pt}}
\multiput(947.17,400.85)(13.000,-42.855){2}{\rule{0.400pt}{0.758pt}}
\multiput(961.58,351.96)(0.493,-1.726){23}{\rule{0.119pt}{1.454pt}}
\multiput(960.17,354.98)(13.000,-40.982){2}{\rule{0.400pt}{0.727pt}}
\multiput(974.58,308.35)(0.493,-1.607){23}{\rule{0.119pt}{1.362pt}}
\multiput(973.17,311.17)(13.000,-38.174){2}{\rule{0.400pt}{0.681pt}}
\multiput(987.58,267.73)(0.493,-1.488){23}{\rule{0.119pt}{1.269pt}}
\multiput(986.17,270.37)(13.000,-35.366){2}{\rule{0.400pt}{0.635pt}}
\multiput(1000.58,230.24)(0.493,-1.329){23}{\rule{0.119pt}{1.146pt}}
\multiput(999.17,232.62)(13.000,-31.621){2}{\rule{0.400pt}{0.573pt}}
\multiput(1013.58,196.75)(0.493,-1.171){23}{\rule{0.119pt}{1.023pt}}
\multiput(1012.17,198.88)(13.000,-27.877){2}{\rule{0.400pt}{0.512pt}}
\multiput(1026.58,167.39)(0.493,-0.972){23}{\rule{0.119pt}{0.869pt}}
\multiput(1025.17,169.20)(13.000,-23.196){2}{\rule{0.400pt}{0.435pt}}
\multiput(1039.58,142.68)(0.492,-0.884){21}{\rule{0.119pt}{0.800pt}}
\multiput(1038.17,144.34)(12.000,-19.340){2}{\rule{0.400pt}{0.400pt}}
\multiput(1051.58,122.54)(0.493,-0.616){23}{\rule{0.119pt}{0.592pt}}
\multiput(1050.17,123.77)(13.000,-14.771){2}{\rule{0.400pt}{0.296pt}}
\multiput(1064.00,107.92)(0.539,-0.492){21}{\rule{0.533pt}{0.119pt}}
\multiput(1064.00,108.17)(11.893,-12.000){2}{\rule{0.267pt}{0.400pt}}
\multiput(1077.00,95.93)(0.824,-0.488){13}{\rule{0.750pt}{0.117pt}}
\multiput(1077.00,96.17)(11.443,-8.000){2}{\rule{0.375pt}{0.400pt}}
\multiput(1090.00,87.93)(1.378,-0.477){7}{\rule{1.140pt}{0.115pt}}
\multiput(1090.00,88.17)(10.634,-5.000){2}{\rule{0.570pt}{0.400pt}}
\put(1103,82.17){\rule{2.700pt}{0.400pt}}
\multiput(1103.00,83.17)(7.396,-2.000){2}{\rule{1.350pt}{0.400pt}}
\put(1116,81.67){\rule{3.132pt}{0.400pt}}
\multiput(1116.00,81.17)(6.500,1.000){2}{\rule{1.566pt}{0.400pt}}
\put(1129,83.17){\rule{2.700pt}{0.400pt}}
\multiput(1129.00,82.17)(7.396,2.000){2}{\rule{1.350pt}{0.400pt}}
\multiput(1142.00,85.60)(1.797,0.468){5}{\rule{1.400pt}{0.113pt}}
\multiput(1142.00,84.17)(10.094,4.000){2}{\rule{0.700pt}{0.400pt}}
\multiput(1155.00,89.60)(1.797,0.468){5}{\rule{1.400pt}{0.113pt}}
\multiput(1155.00,88.17)(10.094,4.000){2}{\rule{0.700pt}{0.400pt}}
\multiput(1168.00,93.59)(1.378,0.477){7}{\rule{1.140pt}{0.115pt}}
\multiput(1168.00,92.17)(10.634,5.000){2}{\rule{0.570pt}{0.400pt}}
\multiput(1181.00,98.59)(1.378,0.477){7}{\rule{1.140pt}{0.115pt}}
\multiput(1181.00,97.17)(10.634,5.000){2}{\rule{0.570pt}{0.400pt}}
\multiput(1194.00,103.60)(1.651,0.468){5}{\rule{1.300pt}{0.113pt}}
\multiput(1194.00,102.17)(9.302,4.000){2}{\rule{0.650pt}{0.400pt}}
\multiput(1206.00,107.61)(2.695,0.447){3}{\rule{1.833pt}{0.108pt}}
\multiput(1206.00,106.17)(9.195,3.000){2}{\rule{0.917pt}{0.400pt}}
\multiput(1219.00,110.61)(2.695,0.447){3}{\rule{1.833pt}{0.108pt}}
\multiput(1219.00,109.17)(9.195,3.000){2}{\rule{0.917pt}{0.400pt}}
\put(1232,112.67){\rule{3.132pt}{0.400pt}}
\multiput(1232.00,112.17)(6.500,1.000){2}{\rule{1.566pt}{0.400pt}}
\put(1245,113.67){\rule{3.132pt}{0.400pt}}
\multiput(1245.00,113.17)(6.500,1.000){2}{\rule{1.566pt}{0.400pt}}
\put(1258,113.67){\rule{3.132pt}{0.400pt}}
\multiput(1258.00,114.17)(6.500,-1.000){2}{\rule{1.566pt}{0.400pt}}
\put(1271,112.67){\rule{3.132pt}{0.400pt}}
\multiput(1271.00,113.17)(6.500,-1.000){2}{\rule{1.566pt}{0.400pt}}
\multiput(1284.00,111.95)(2.695,-0.447){3}{\rule{1.833pt}{0.108pt}}
\multiput(1284.00,112.17)(9.195,-3.000){2}{\rule{0.917pt}{0.400pt}}
\multiput(1297.00,108.95)(2.695,-0.447){3}{\rule{1.833pt}{0.108pt}}
\multiput(1297.00,109.17)(9.195,-3.000){2}{\rule{0.917pt}{0.400pt}}
\multiput(1310.00,105.95)(2.695,-0.447){3}{\rule{1.833pt}{0.108pt}}
\multiput(1310.00,106.17)(9.195,-3.000){2}{\rule{0.917pt}{0.400pt}}
\multiput(1323.00,102.94)(1.797,-0.468){5}{\rule{1.400pt}{0.113pt}}
\multiput(1323.00,103.17)(10.094,-4.000){2}{\rule{0.700pt}{0.400pt}}
\multiput(1336.00,98.94)(1.797,-0.468){5}{\rule{1.400pt}{0.113pt}}
\multiput(1336.00,99.17)(10.094,-4.000){2}{\rule{0.700pt}{0.400pt}}
\multiput(1349.00,94.95)(2.472,-0.447){3}{\rule{1.700pt}{0.108pt}}
\multiput(1349.00,95.17)(8.472,-3.000){2}{\rule{0.850pt}{0.400pt}}
\multiput(1361.00,91.95)(2.695,-0.447){3}{\rule{1.833pt}{0.108pt}}
\multiput(1361.00,92.17)(9.195,-3.000){2}{\rule{0.917pt}{0.400pt}}
\multiput(1374.00,88.95)(2.695,-0.447){3}{\rule{1.833pt}{0.108pt}}
\multiput(1374.00,89.17)(9.195,-3.000){2}{\rule{0.917pt}{0.400pt}}
\put(1387,85.17){\rule{2.700pt}{0.400pt}}
\multiput(1387.00,86.17)(7.396,-2.000){2}{\rule{1.350pt}{0.400pt}}
\put(1400,83.17){\rule{2.700pt}{0.400pt}}
\multiput(1400.00,84.17)(7.396,-2.000){2}{\rule{1.350pt}{0.400pt}}
\put(1413,81.67){\rule{3.132pt}{0.400pt}}
\multiput(1413.00,82.17)(6.500,-1.000){2}{\rule{1.566pt}{0.400pt}}
\put(793.0,776.0){\rule[-0.200pt]{3.132pt}{0.400pt}}
\put(1426.0,82.0){\rule[-0.200pt]{3.132pt}{0.400pt}}
\put(160.0,82.0){\rule[-0.200pt]{0.400pt}{167.425pt}}
\put(160.0,82.0){\rule[-0.200pt]{308.111pt}{0.400pt}}
\put(1439.0,82.0){\rule[-0.200pt]{0.400pt}{167.425pt}}
\put(160.0,777.0){\rule[-0.200pt]{308.111pt}{0.400pt}}
\end{picture}

  % GNUPLOT: LaTeX picture
\setlength{\unitlength}{0.240900pt}
\ifx\plotpoint\undefined\newsavebox{\plotpoint}\fi
\sbox{\plotpoint}{\rule[-0.200pt]{0.400pt}{0.400pt}}%
\begin{picture}(1500,900)(0,0)
\sbox{\plotpoint}{\rule[-0.200pt]{0.400pt}{0.400pt}}%
\put(200.0,82.0){\rule[-0.200pt]{4.818pt}{0.400pt}}
\put(180,82){\makebox(0,0)[r]{$0$}}
\put(1419.0,82.0){\rule[-0.200pt]{4.818pt}{0.400pt}}
\put(200.0,152.0){\rule[-0.200pt]{4.818pt}{0.400pt}}
\put(180,152){\makebox(0,0)[r]{$1000$}}
\put(1419.0,152.0){\rule[-0.200pt]{4.818pt}{0.400pt}}
\put(200.0,221.0){\rule[-0.200pt]{4.818pt}{0.400pt}}
\put(180,221){\makebox(0,0)[r]{$2000$}}
\put(1419.0,221.0){\rule[-0.200pt]{4.818pt}{0.400pt}}
\put(200.0,291.0){\rule[-0.200pt]{4.818pt}{0.400pt}}
\put(180,291){\makebox(0,0)[r]{$3000$}}
\put(1419.0,291.0){\rule[-0.200pt]{4.818pt}{0.400pt}}
\put(200.0,360.0){\rule[-0.200pt]{4.818pt}{0.400pt}}
\put(180,360){\makebox(0,0)[r]{$4000$}}
\put(1419.0,360.0){\rule[-0.200pt]{4.818pt}{0.400pt}}
\put(200.0,430.0){\rule[-0.200pt]{4.818pt}{0.400pt}}
\put(180,430){\makebox(0,0)[r]{$5000$}}
\put(1419.0,430.0){\rule[-0.200pt]{4.818pt}{0.400pt}}
\put(200.0,499.0){\rule[-0.200pt]{4.818pt}{0.400pt}}
\put(180,499){\makebox(0,0)[r]{$6000$}}
\put(1419.0,499.0){\rule[-0.200pt]{4.818pt}{0.400pt}}
\put(200.0,569.0){\rule[-0.200pt]{4.818pt}{0.400pt}}
\put(180,569){\makebox(0,0)[r]{$7000$}}
\put(1419.0,569.0){\rule[-0.200pt]{4.818pt}{0.400pt}}
\put(200.0,638.0){\rule[-0.200pt]{4.818pt}{0.400pt}}
\put(180,638){\makebox(0,0)[r]{$8000$}}
\put(1419.0,638.0){\rule[-0.200pt]{4.818pt}{0.400pt}}
\put(200.0,708.0){\rule[-0.200pt]{4.818pt}{0.400pt}}
\put(180,708){\makebox(0,0)[r]{$9000$}}
\put(1419.0,708.0){\rule[-0.200pt]{4.818pt}{0.400pt}}
\put(200.0,777.0){\rule[-0.200pt]{4.818pt}{0.400pt}}
\put(180,777){\makebox(0,0)[r]{$10000$}}
\put(1419.0,777.0){\rule[-0.200pt]{4.818pt}{0.400pt}}
\put(200.0,82.0){\rule[-0.200pt]{0.400pt}{4.818pt}}
\put(200,41){\makebox(0,0){$-4\pi$}}
\put(200.0,757.0){\rule[-0.200pt]{0.400pt}{4.818pt}}
\put(355.0,82.0){\rule[-0.200pt]{0.400pt}{4.818pt}}
\put(355,41){\makebox(0,0){$-3\pi$}}
\put(355.0,757.0){\rule[-0.200pt]{0.400pt}{4.818pt}}
\put(510.0,82.0){\rule[-0.200pt]{0.400pt}{4.818pt}}
\put(510,41){\makebox(0,0){$-2\pi$}}
\put(510.0,757.0){\rule[-0.200pt]{0.400pt}{4.818pt}}
\put(819.0,82.0){\rule[-0.200pt]{0.400pt}{4.818pt}}
\put(819,41){\makebox(0,0){0}}
\put(819.0,757.0){\rule[-0.200pt]{0.400pt}{4.818pt}}
\put(1129.0,82.0){\rule[-0.200pt]{0.400pt}{4.818pt}}
\put(1129,41){\makebox(0,0){$2\pi$}}
\put(1129.0,757.0){\rule[-0.200pt]{0.400pt}{4.818pt}}
\put(1284.0,82.0){\rule[-0.200pt]{0.400pt}{4.818pt}}
\put(1284,41){\makebox(0,0){$3\pi$}}
\put(1284.0,757.0){\rule[-0.200pt]{0.400pt}{4.818pt}}
\put(1439.0,82.0){\rule[-0.200pt]{0.400pt}{4.818pt}}
\put(1439,41){\makebox(0,0){$4\pi$}}
\put(1439.0,757.0){\rule[-0.200pt]{0.400pt}{4.818pt}}
\put(200.0,82.0){\rule[-0.200pt]{0.400pt}{167.425pt}}
\put(200.0,82.0){\rule[-0.200pt]{298.475pt}{0.400pt}}
\put(1439.0,82.0){\rule[-0.200pt]{0.400pt}{167.425pt}}
\put(200.0,777.0){\rule[-0.200pt]{298.475pt}{0.400pt}}
\put(819,839){\makebox(0,0){$f(\omega)$ für $t=100$}}
\put(200,82){\usebox{\plotpoint}}
\put(200,82){\usebox{\plotpoint}}
\put(200,82){\usebox{\plotpoint}}
\put(200,82){\usebox{\plotpoint}}
\put(200,82){\usebox{\plotpoint}}
\put(200,82){\usebox{\plotpoint}}
\put(200,82){\usebox{\plotpoint}}
\put(200,82){\usebox{\plotpoint}}
\put(200,82){\usebox{\plotpoint}}
\put(200,82){\usebox{\plotpoint}}
\put(200,82){\usebox{\plotpoint}}
\put(200,82){\usebox{\plotpoint}}
\put(200,82){\usebox{\plotpoint}}
\put(200,82){\usebox{\plotpoint}}
\put(200,82){\usebox{\plotpoint}}
\put(200,82){\usebox{\plotpoint}}
\put(200,82){\usebox{\plotpoint}}
\put(200,82){\usebox{\plotpoint}}
\put(200,82){\usebox{\plotpoint}}
\put(200,82){\usebox{\plotpoint}}
\put(200,82){\usebox{\plotpoint}}
\put(200,82){\usebox{\plotpoint}}
\put(200,82){\usebox{\plotpoint}}
\put(200,82){\usebox{\plotpoint}}
\put(200,82){\usebox{\plotpoint}}
\put(200,82){\usebox{\plotpoint}}
\put(200,82){\usebox{\plotpoint}}
\put(200,82){\usebox{\plotpoint}}
\put(200,82){\usebox{\plotpoint}}
\put(200,82){\usebox{\plotpoint}}
\put(200,82){\usebox{\plotpoint}}
\put(200,82){\usebox{\plotpoint}}
\put(200,82){\usebox{\plotpoint}}
\put(200,82){\usebox{\plotpoint}}
\put(200,82){\usebox{\plotpoint}}
\put(200,82){\usebox{\plotpoint}}
\put(200,82){\usebox{\plotpoint}}
\put(200,82){\usebox{\plotpoint}}
\put(200,82){\usebox{\plotpoint}}
\put(200,82){\usebox{\plotpoint}}
\put(200,82){\usebox{\plotpoint}}
\put(200,82){\usebox{\plotpoint}}
\put(200,82){\usebox{\plotpoint}}
\put(200,82){\usebox{\plotpoint}}
\put(200,82){\usebox{\plotpoint}}
\put(200,82){\usebox{\plotpoint}}
\put(200,82){\usebox{\plotpoint}}
\put(200,82){\usebox{\plotpoint}}
\put(200,82){\usebox{\plotpoint}}
\put(200,82){\usebox{\plotpoint}}
\put(200,82){\usebox{\plotpoint}}
\put(200,82){\usebox{\plotpoint}}
\put(200,82){\usebox{\plotpoint}}
\put(200,82){\usebox{\plotpoint}}
\put(200,82){\usebox{\plotpoint}}
\put(200,82){\usebox{\plotpoint}}
\put(200,82){\usebox{\plotpoint}}
\put(200,82){\usebox{\plotpoint}}
\put(200,82){\usebox{\plotpoint}}
\put(200,82){\usebox{\plotpoint}}
\put(200,82){\usebox{\plotpoint}}
\put(200,82){\usebox{\plotpoint}}
\put(200,82){\usebox{\plotpoint}}
\put(200,82){\usebox{\plotpoint}}
\put(200,82){\usebox{\plotpoint}}
\put(200,82){\usebox{\plotpoint}}
\put(200,82){\usebox{\plotpoint}}
\put(200,82){\usebox{\plotpoint}}
\put(200,82){\usebox{\plotpoint}}
\put(200,82){\usebox{\plotpoint}}
\put(200,82){\usebox{\plotpoint}}
\put(200,82){\usebox{\plotpoint}}
\put(200,82){\usebox{\plotpoint}}
\put(200,82){\usebox{\plotpoint}}
\put(200,82){\usebox{\plotpoint}}
\put(200,82){\usebox{\plotpoint}}
\put(200,82){\usebox{\plotpoint}}
\put(200,82){\usebox{\plotpoint}}
\put(200,82){\usebox{\plotpoint}}
\put(200,82){\usebox{\plotpoint}}
\put(200,82){\usebox{\plotpoint}}
\put(200,82){\usebox{\plotpoint}}
\put(200,82){\usebox{\plotpoint}}
\put(200,82){\usebox{\plotpoint}}
\put(200,82){\usebox{\plotpoint}}
\put(200,82){\usebox{\plotpoint}}
\put(200,82){\usebox{\plotpoint}}
\put(200,82){\usebox{\plotpoint}}
\put(200,82){\usebox{\plotpoint}}
\put(200,82){\usebox{\plotpoint}}
\put(200,82){\usebox{\plotpoint}}
\put(200,82){\usebox{\plotpoint}}
\put(200,82){\usebox{\plotpoint}}
\put(200,82){\usebox{\plotpoint}}
\put(200,82){\usebox{\plotpoint}}
\put(200,82){\usebox{\plotpoint}}
\put(200,82){\usebox{\plotpoint}}
\put(200,82){\usebox{\plotpoint}}
\put(200,82){\usebox{\plotpoint}}
\put(200,82){\usebox{\plotpoint}}
\put(200,82){\usebox{\plotpoint}}
\put(200,82){\usebox{\plotpoint}}
\put(200,82){\usebox{\plotpoint}}
\put(200,82){\usebox{\plotpoint}}
\put(200,82){\usebox{\plotpoint}}
\put(200,82){\usebox{\plotpoint}}
\put(200,82){\usebox{\plotpoint}}
\put(200,82){\usebox{\plotpoint}}
\put(200,82){\usebox{\plotpoint}}
\put(200,82){\usebox{\plotpoint}}
\put(200,82){\usebox{\plotpoint}}
\put(200,82){\usebox{\plotpoint}}
\put(200,82){\usebox{\plotpoint}}
\put(200,82){\usebox{\plotpoint}}
\put(200,82){\usebox{\plotpoint}}
\put(200,82){\usebox{\plotpoint}}
\put(200,82){\usebox{\plotpoint}}
\put(200,82){\usebox{\plotpoint}}
\put(200,82){\usebox{\plotpoint}}
\put(200,82){\usebox{\plotpoint}}
\put(200,82){\usebox{\plotpoint}}
\put(200,82){\usebox{\plotpoint}}
\put(200,82){\usebox{\plotpoint}}
\put(200,82){\usebox{\plotpoint}}
\put(200,82){\usebox{\plotpoint}}
\put(200,82){\usebox{\plotpoint}}
\put(200,82){\usebox{\plotpoint}}
\put(200,82){\usebox{\plotpoint}}
\put(200,82){\usebox{\plotpoint}}
\put(200,82){\usebox{\plotpoint}}
\put(200,82){\usebox{\plotpoint}}
\put(200,82){\usebox{\plotpoint}}
\put(200,82){\usebox{\plotpoint}}
\put(200,82){\usebox{\plotpoint}}
\put(200,82){\usebox{\plotpoint}}
\put(200,82){\usebox{\plotpoint}}
\put(200,82){\usebox{\plotpoint}}
\put(200,82){\usebox{\plotpoint}}
\put(200,82){\usebox{\plotpoint}}
\put(200,82){\usebox{\plotpoint}}
\put(200,82){\usebox{\plotpoint}}
\put(200,82){\usebox{\plotpoint}}
\put(200,82){\usebox{\plotpoint}}
\put(200,82){\usebox{\plotpoint}}
\put(200,82){\usebox{\plotpoint}}
\put(200,82){\usebox{\plotpoint}}
\put(200,82){\usebox{\plotpoint}}
\put(200,82){\usebox{\plotpoint}}
\put(200,82){\usebox{\plotpoint}}
\put(200,82){\usebox{\plotpoint}}
\put(200,82){\usebox{\plotpoint}}
\put(200,82){\usebox{\plotpoint}}
\put(200,82){\usebox{\plotpoint}}
\put(200,82){\usebox{\plotpoint}}
\put(200,82){\usebox{\plotpoint}}
\put(200,82){\usebox{\plotpoint}}
\put(200,82){\usebox{\plotpoint}}
\put(200,82){\usebox{\plotpoint}}
\put(200,82){\usebox{\plotpoint}}
\put(200,82){\usebox{\plotpoint}}
\put(200,82){\usebox{\plotpoint}}
\put(200,82){\usebox{\plotpoint}}
\put(200,82){\usebox{\plotpoint}}
\put(200,82){\usebox{\plotpoint}}
\put(200,82){\usebox{\plotpoint}}
\put(200,82){\usebox{\plotpoint}}
\put(200,82){\usebox{\plotpoint}}
\put(200,82){\usebox{\plotpoint}}
\put(200,82){\usebox{\plotpoint}}
\put(200,82){\usebox{\plotpoint}}
\put(200,82){\usebox{\plotpoint}}
\put(200,82){\usebox{\plotpoint}}
\put(200,82){\usebox{\plotpoint}}
\put(200,82){\usebox{\plotpoint}}
\put(200,82){\usebox{\plotpoint}}
\put(200,82){\usebox{\plotpoint}}
\put(200,82){\usebox{\plotpoint}}
\put(200,82){\usebox{\plotpoint}}
\put(200,82){\usebox{\plotpoint}}
\put(200,82){\usebox{\plotpoint}}
\put(200,82){\usebox{\plotpoint}}
\put(200,82){\usebox{\plotpoint}}
\put(200,82){\usebox{\plotpoint}}
\put(200,82){\usebox{\plotpoint}}
\put(200,82){\usebox{\plotpoint}}
\put(200,82){\usebox{\plotpoint}}
\put(200,82){\usebox{\plotpoint}}
\put(200,82){\usebox{\plotpoint}}
\put(200,82){\usebox{\plotpoint}}
\put(200,82){\usebox{\plotpoint}}
\put(200,82){\usebox{\plotpoint}}
\put(200,82){\usebox{\plotpoint}}
\put(200,82){\usebox{\plotpoint}}
\put(200,82){\usebox{\plotpoint}}
\put(200,82){\usebox{\plotpoint}}
\put(200,82){\usebox{\plotpoint}}
\put(200,82){\usebox{\plotpoint}}
\put(200,82){\usebox{\plotpoint}}
\put(200,82){\usebox{\plotpoint}}
\put(200,82){\usebox{\plotpoint}}
\put(200,82){\usebox{\plotpoint}}
\put(200,82){\usebox{\plotpoint}}
\put(200,82){\usebox{\plotpoint}}
\put(200,82){\usebox{\plotpoint}}
\put(200,82){\usebox{\plotpoint}}
\put(200,82){\usebox{\plotpoint}}
\put(200,82){\usebox{\plotpoint}}
\put(200,82){\usebox{\plotpoint}}
\put(200,82){\usebox{\plotpoint}}
\put(200,82){\usebox{\plotpoint}}
\put(200,82){\usebox{\plotpoint}}
\put(200,82){\usebox{\plotpoint}}
\put(200,82){\usebox{\plotpoint}}
\put(200,82){\usebox{\plotpoint}}
\put(200,82){\usebox{\plotpoint}}
\put(200,82){\usebox{\plotpoint}}
\put(200,82){\usebox{\plotpoint}}
\put(200,82){\usebox{\plotpoint}}
\put(200,82){\usebox{\plotpoint}}
\put(200,82){\usebox{\plotpoint}}
\put(200,82){\usebox{\plotpoint}}
\put(200,82){\usebox{\plotpoint}}
\put(200,82){\usebox{\plotpoint}}
\put(200,82){\usebox{\plotpoint}}
\put(200,82){\usebox{\plotpoint}}
\put(200,82){\usebox{\plotpoint}}
\put(200,82){\usebox{\plotpoint}}
\put(200,82){\usebox{\plotpoint}}
\put(200,82){\usebox{\plotpoint}}
\put(200,82){\usebox{\plotpoint}}
\put(200,82){\usebox{\plotpoint}}
\put(200,82){\usebox{\plotpoint}}
\put(200,82){\usebox{\plotpoint}}
\put(200,82){\usebox{\plotpoint}}
\put(200,82){\usebox{\plotpoint}}
\put(200,82){\usebox{\plotpoint}}
\put(200,82){\usebox{\plotpoint}}
\put(200,82){\usebox{\plotpoint}}
\put(200,82){\usebox{\plotpoint}}
\put(200,82){\usebox{\plotpoint}}
\put(200,82){\usebox{\plotpoint}}
\put(200,82){\usebox{\plotpoint}}
\put(200,82){\usebox{\plotpoint}}
\put(200,82){\usebox{\plotpoint}}
\put(200,82){\usebox{\plotpoint}}
\put(200,82){\usebox{\plotpoint}}
\put(200,82){\usebox{\plotpoint}}
\put(200,82){\usebox{\plotpoint}}
\put(200,82){\usebox{\plotpoint}}
\put(200,82){\usebox{\plotpoint}}
\put(200,82){\usebox{\plotpoint}}
\put(200,82){\usebox{\plotpoint}}
\put(200,82){\usebox{\plotpoint}}
\put(200,82){\usebox{\plotpoint}}
\put(200,82){\usebox{\plotpoint}}
\put(200,82){\usebox{\plotpoint}}
\put(200,82){\usebox{\plotpoint}}
\put(200,82){\usebox{\plotpoint}}
\put(200,82){\usebox{\plotpoint}}
\put(200,82){\usebox{\plotpoint}}
\put(200,82){\usebox{\plotpoint}}
\put(200,82){\usebox{\plotpoint}}
\put(200,82){\usebox{\plotpoint}}
\put(200,82){\usebox{\plotpoint}}
\put(200,82){\usebox{\plotpoint}}
\put(200,82){\usebox{\plotpoint}}
\put(200,82){\usebox{\plotpoint}}
\put(200,82){\usebox{\plotpoint}}
\put(200,82){\usebox{\plotpoint}}
\put(200,82){\usebox{\plotpoint}}
\put(200,82){\usebox{\plotpoint}}
\put(200,82){\usebox{\plotpoint}}
\put(200,82){\usebox{\plotpoint}}
\put(200,82){\usebox{\plotpoint}}
\put(200,82){\usebox{\plotpoint}}
\put(200,82){\usebox{\plotpoint}}
\put(200,82){\usebox{\plotpoint}}
\put(200,82){\usebox{\plotpoint}}
\put(200,82){\usebox{\plotpoint}}
\put(200,82){\usebox{\plotpoint}}
\put(200,82){\usebox{\plotpoint}}
\put(200,82){\usebox{\plotpoint}}
\put(200,82){\usebox{\plotpoint}}
\put(200,82){\usebox{\plotpoint}}
\put(200,82){\usebox{\plotpoint}}
\put(200,82){\usebox{\plotpoint}}
\put(200,82){\usebox{\plotpoint}}
\put(200,82){\usebox{\plotpoint}}
\put(200,82){\usebox{\plotpoint}}
\put(200,82){\usebox{\plotpoint}}
\put(200,82){\usebox{\plotpoint}}
\put(200,82){\usebox{\plotpoint}}
\put(200,82){\usebox{\plotpoint}}
\put(200,82){\usebox{\plotpoint}}
\put(200,82){\usebox{\plotpoint}}
\put(200,82){\usebox{\plotpoint}}
\put(200,82){\usebox{\plotpoint}}
\put(200,82){\usebox{\plotpoint}}
\put(200,82){\usebox{\plotpoint}}
\put(200,82){\usebox{\plotpoint}}
\put(200,82){\usebox{\plotpoint}}
\put(200,82){\usebox{\plotpoint}}
\put(200,82){\usebox{\plotpoint}}
\put(200,82){\usebox{\plotpoint}}
\put(200,82){\usebox{\plotpoint}}
\put(200,82){\usebox{\plotpoint}}
\put(200,82){\usebox{\plotpoint}}
\put(200,82){\usebox{\plotpoint}}
\put(200,82){\usebox{\plotpoint}}
\put(200,82){\usebox{\plotpoint}}
\put(200,82){\usebox{\plotpoint}}
\put(200,82){\usebox{\plotpoint}}
\put(200,82){\usebox{\plotpoint}}
\put(200,82){\usebox{\plotpoint}}
\put(200,82){\usebox{\plotpoint}}
\put(200,82){\usebox{\plotpoint}}
\put(200,82){\usebox{\plotpoint}}
\put(200,82){\usebox{\plotpoint}}
\put(200,82){\usebox{\plotpoint}}
\put(200,82){\usebox{\plotpoint}}
\put(200,82){\usebox{\plotpoint}}
\put(200,82){\usebox{\plotpoint}}
\put(200,82){\usebox{\plotpoint}}
\put(200,82){\usebox{\plotpoint}}
\put(200,82){\usebox{\plotpoint}}
\put(200,82){\usebox{\plotpoint}}
\put(200,82){\usebox{\plotpoint}}
\put(200,82){\usebox{\plotpoint}}
\put(200,82){\usebox{\plotpoint}}
\put(200,82){\usebox{\plotpoint}}
\put(200,82){\usebox{\plotpoint}}
\put(200,82){\usebox{\plotpoint}}
\put(200,82){\usebox{\plotpoint}}
\put(200,82){\usebox{\plotpoint}}
\put(200,82){\usebox{\plotpoint}}
\put(200,82){\usebox{\plotpoint}}
\put(200,82){\usebox{\plotpoint}}
\put(200,82){\usebox{\plotpoint}}
\put(200,82){\usebox{\plotpoint}}
\put(200,82){\usebox{\plotpoint}}
\put(200,82){\usebox{\plotpoint}}
\put(200,82){\usebox{\plotpoint}}
\put(200,82){\usebox{\plotpoint}}
\put(200,82){\usebox{\plotpoint}}
\put(200,82){\usebox{\plotpoint}}
\put(200,82){\usebox{\plotpoint}}
\put(200,82){\usebox{\plotpoint}}
\put(200,82){\usebox{\plotpoint}}
\put(200,82){\usebox{\plotpoint}}
\put(200,82){\usebox{\plotpoint}}
\put(200,82){\usebox{\plotpoint}}
\put(200,82){\usebox{\plotpoint}}
\put(200,82){\usebox{\plotpoint}}
\put(200,82){\usebox{\plotpoint}}
\put(200,82){\usebox{\plotpoint}}
\put(200,82){\usebox{\plotpoint}}
\put(200,82){\usebox{\plotpoint}}
\put(200,82){\usebox{\plotpoint}}
\put(200,82){\usebox{\plotpoint}}
\put(200,82){\usebox{\plotpoint}}
\put(200,82){\usebox{\plotpoint}}
\put(200,82){\usebox{\plotpoint}}
\put(200,82){\usebox{\plotpoint}}
\put(200,82){\usebox{\plotpoint}}
\put(200,82){\usebox{\plotpoint}}
\put(200,82){\usebox{\plotpoint}}
\put(200,82){\usebox{\plotpoint}}
\put(200,82){\usebox{\plotpoint}}
\put(200,82){\usebox{\plotpoint}}
\put(200,82){\usebox{\plotpoint}}
\put(200,82){\usebox{\plotpoint}}
\put(200,82){\usebox{\plotpoint}}
\put(200,82){\usebox{\plotpoint}}
\put(200,82){\usebox{\plotpoint}}
\put(200,82){\usebox{\plotpoint}}
\put(200,82){\usebox{\plotpoint}}
\put(200,82){\usebox{\plotpoint}}
\put(200,82){\usebox{\plotpoint}}
\put(200,82){\usebox{\plotpoint}}
\put(200,82){\usebox{\plotpoint}}
\put(200,82){\usebox{\plotpoint}}
\put(200,82){\usebox{\plotpoint}}
\put(200,82){\usebox{\plotpoint}}
\put(200,82){\usebox{\plotpoint}}
\put(200,82){\usebox{\plotpoint}}
\put(200,82){\usebox{\plotpoint}}
\put(200,82){\usebox{\plotpoint}}
\put(200,82){\usebox{\plotpoint}}
\put(200,82){\usebox{\plotpoint}}
\put(200,82){\usebox{\plotpoint}}
\put(200,82){\usebox{\plotpoint}}
\put(200,82){\usebox{\plotpoint}}
\put(200,82){\usebox{\plotpoint}}
\put(200,82){\usebox{\plotpoint}}
\put(200,82){\usebox{\plotpoint}}
\put(200,82){\usebox{\plotpoint}}
\put(200,82){\usebox{\plotpoint}}
\put(200,82){\usebox{\plotpoint}}
\put(200,82){\usebox{\plotpoint}}
\put(200,82){\usebox{\plotpoint}}
\put(200,82){\usebox{\plotpoint}}
\put(200,82){\usebox{\plotpoint}}
\put(200,82){\usebox{\plotpoint}}
\put(200,82){\usebox{\plotpoint}}
\put(200.0,82.0){\rule[-0.200pt]{140.686pt}{0.400pt}}
\put(784.0,82.0){\usebox{\plotpoint}}
\put(784.0,82.0){\usebox{\plotpoint}}
\put(784.0,82.0){\rule[-0.200pt]{0.723pt}{0.400pt}}
\put(787.0,82.0){\usebox{\plotpoint}}
\put(787.0,82.0){\usebox{\plotpoint}}
\put(787.0,82.0){\rule[-0.200pt]{0.482pt}{0.400pt}}
\put(789.0,82.0){\usebox{\plotpoint}}
\put(789.0,83.0){\rule[-0.200pt]{0.482pt}{0.400pt}}
\put(791.0,82.0){\usebox{\plotpoint}}
\put(791.0,82.0){\usebox{\plotpoint}}
\put(792.0,82.0){\usebox{\plotpoint}}
\put(792.0,83.0){\rule[-0.200pt]{0.482pt}{0.400pt}}
\put(794.0,82.0){\usebox{\plotpoint}}
\put(794.0,82.0){\usebox{\plotpoint}}
\put(795.0,82.0){\usebox{\plotpoint}}
\put(795.0,83.0){\rule[-0.200pt]{0.482pt}{0.400pt}}
\put(797.0,82.0){\usebox{\plotpoint}}
\put(797.0,82.0){\usebox{\plotpoint}}
\put(798.0,82.0){\usebox{\plotpoint}}
\put(798.0,83.0){\usebox{\plotpoint}}
\put(799.0,83.0){\usebox{\plotpoint}}
\put(799.0,84.0){\usebox{\plotpoint}}
\put(800.0,82.0){\rule[-0.200pt]{0.400pt}{0.482pt}}
\put(800.0,82.0){\usebox{\plotpoint}}
\put(801.0,82.0){\usebox{\plotpoint}}
\put(801.0,83.0){\usebox{\plotpoint}}
\put(802.0,83.0){\usebox{\plotpoint}}
\put(802.0,84.0){\usebox{\plotpoint}}
\put(803.0,83.0){\usebox{\plotpoint}}
\put(803.0,83.0){\usebox{\plotpoint}}
\put(804.0,82.0){\usebox{\plotpoint}}
\put(804.0,82.0){\usebox{\plotpoint}}
\put(804.0,83.0){\usebox{\plotpoint}}
\put(805.0,83.0){\rule[-0.200pt]{0.400pt}{0.482pt}}
\put(805.0,85.0){\usebox{\plotpoint}}
\put(806.0,83.0){\rule[-0.200pt]{0.400pt}{0.482pt}}
\put(806.0,83.0){\usebox{\plotpoint}}
\put(807.0,82.0){\usebox{\plotpoint}}
\put(807.0,82.0){\usebox{\plotpoint}}
\put(807.0,83.0){\usebox{\plotpoint}}
\put(808.0,83.0){\rule[-0.200pt]{0.400pt}{0.964pt}}
\put(808.0,87.0){\usebox{\plotpoint}}
\put(809.0,87.0){\usebox{\plotpoint}}
\put(809.0,85.0){\rule[-0.200pt]{0.400pt}{0.723pt}}
\put(809.0,85.0){\usebox{\plotpoint}}
\put(810.0,82.0){\rule[-0.200pt]{0.400pt}{0.723pt}}
\put(810.0,82.0){\usebox{\plotpoint}}
\put(810.0,83.0){\usebox{\plotpoint}}
\put(811.0,83.0){\rule[-0.200pt]{0.400pt}{2.168pt}}
\put(811.0,92.0){\usebox{\plotpoint}}
\put(812.0,92.0){\usebox{\plotpoint}}
\put(812.0,89.0){\rule[-0.200pt]{0.400pt}{0.964pt}}
\put(812.0,89.0){\usebox{\plotpoint}}
\put(813.0,82.0){\rule[-0.200pt]{0.400pt}{1.686pt}}
\put(813.0,82.0){\usebox{\plotpoint}}
\put(813.0,83.0){\usebox{\plotpoint}}
\put(814.0,83.0){\rule[-0.200pt]{0.400pt}{5.541pt}}
\put(814.0,106.0){\usebox{\plotpoint}}
\put(815.0,106.0){\rule[-0.200pt]{0.400pt}{2.168pt}}
\put(815.0,109.0){\rule[-0.200pt]{0.400pt}{1.445pt}}
\put(815.0,109.0){\usebox{\plotpoint}}
\put(816.0,82.0){\rule[-0.200pt]{0.400pt}{6.504pt}}
\put(816.0,82.0){\usebox{\plotpoint}}
\put(816.0,83.0){\usebox{\plotpoint}}
\put(817.0,83.0){\rule[-0.200pt]{0.400pt}{32.521pt}}
\put(817.0,218.0){\usebox{\plotpoint}}
\put(818.0,218.0){\rule[-0.200pt]{0.400pt}{84.556pt}}
\put(818.0,569.0){\usebox{\plotpoint}}
\put(819.0,569.0){\rule[-0.200pt]{0.400pt}{50.107pt}}
\put(819.0,777.0){\usebox{\plotpoint}}
\put(820.0,569.0){\rule[-0.200pt]{0.400pt}{50.107pt}}
\put(820.0,569.0){\usebox{\plotpoint}}
\put(821.0,218.0){\rule[-0.200pt]{0.400pt}{84.556pt}}
\put(821.0,218.0){\usebox{\plotpoint}}
\put(822.0,83.0){\rule[-0.200pt]{0.400pt}{32.521pt}}
\put(822.0,83.0){\usebox{\plotpoint}}
\put(823.0,82.0){\usebox{\plotpoint}}
\put(823.0,82.0){\rule[-0.200pt]{0.400pt}{6.504pt}}
\put(823.0,109.0){\usebox{\plotpoint}}
\put(824.0,109.0){\rule[-0.200pt]{0.400pt}{1.445pt}}
\put(824.0,106.0){\rule[-0.200pt]{0.400pt}{2.168pt}}
\put(824.0,106.0){\usebox{\plotpoint}}
\put(825.0,83.0){\rule[-0.200pt]{0.400pt}{5.541pt}}
\put(825.0,83.0){\usebox{\plotpoint}}
\put(826.0,82.0){\usebox{\plotpoint}}
\put(826.0,82.0){\rule[-0.200pt]{0.400pt}{1.686pt}}
\put(826.0,89.0){\usebox{\plotpoint}}
\put(827.0,89.0){\rule[-0.200pt]{0.400pt}{0.964pt}}
\put(827.0,92.0){\usebox{\plotpoint}}
\put(827.0,92.0){\usebox{\plotpoint}}
\put(828.0,83.0){\rule[-0.200pt]{0.400pt}{2.168pt}}
\put(828.0,83.0){\usebox{\plotpoint}}
\put(829.0,82.0){\usebox{\plotpoint}}
\put(829.0,82.0){\rule[-0.200pt]{0.400pt}{0.723pt}}
\put(829.0,85.0){\usebox{\plotpoint}}
\put(830.0,85.0){\rule[-0.200pt]{0.400pt}{0.723pt}}
\put(830.0,87.0){\usebox{\plotpoint}}
\put(830.0,87.0){\usebox{\plotpoint}}
\put(831.0,83.0){\rule[-0.200pt]{0.400pt}{0.964pt}}
\put(831.0,83.0){\usebox{\plotpoint}}
\put(832.0,82.0){\usebox{\plotpoint}}
\put(832.0,82.0){\usebox{\plotpoint}}
\put(832.0,83.0){\usebox{\plotpoint}}
\put(833.0,83.0){\rule[-0.200pt]{0.400pt}{0.482pt}}
\put(833.0,85.0){\usebox{\plotpoint}}
\put(834.0,83.0){\rule[-0.200pt]{0.400pt}{0.482pt}}
\put(834.0,83.0){\usebox{\plotpoint}}
\put(835.0,82.0){\usebox{\plotpoint}}
\put(835.0,82.0){\usebox{\plotpoint}}
\put(835.0,83.0){\usebox{\plotpoint}}
\put(836.0,83.0){\usebox{\plotpoint}}
\put(836.0,84.0){\usebox{\plotpoint}}
\put(837.0,83.0){\usebox{\plotpoint}}
\put(837.0,83.0){\usebox{\plotpoint}}
\put(838.0,82.0){\usebox{\plotpoint}}
\put(838.0,82.0){\usebox{\plotpoint}}
\put(839.0,82.0){\rule[-0.200pt]{0.400pt}{0.482pt}}
\put(839.0,84.0){\usebox{\plotpoint}}
\put(840.0,83.0){\usebox{\plotpoint}}
\put(840.0,83.0){\usebox{\plotpoint}}
\put(841.0,82.0){\usebox{\plotpoint}}
\put(841.0,82.0){\usebox{\plotpoint}}
\put(842.0,82.0){\usebox{\plotpoint}}
\put(842.0,83.0){\rule[-0.200pt]{0.482pt}{0.400pt}}
\put(844.0,82.0){\usebox{\plotpoint}}
\put(844.0,82.0){\usebox{\plotpoint}}
\put(845.0,82.0){\usebox{\plotpoint}}
\put(845.0,83.0){\rule[-0.200pt]{0.482pt}{0.400pt}}
\put(847.0,82.0){\usebox{\plotpoint}}
\put(847.0,82.0){\usebox{\plotpoint}}
\put(848.0,82.0){\usebox{\plotpoint}}
\put(848.0,83.0){\rule[-0.200pt]{0.482pt}{0.400pt}}
\put(850.0,82.0){\usebox{\plotpoint}}
\put(850.0,82.0){\rule[-0.200pt]{0.482pt}{0.400pt}}
\put(852.0,82.0){\usebox{\plotpoint}}
\put(852.0,82.0){\usebox{\plotpoint}}
\put(852.0,82.0){\rule[-0.200pt]{0.723pt}{0.400pt}}
\put(855.0,82.0){\usebox{\plotpoint}}
\put(855.0,82.0){\usebox{\plotpoint}}
\put(855.0,82.0){\rule[-0.200pt]{140.686pt}{0.400pt}}
\put(200.0,82.0){\rule[-0.200pt]{0.400pt}{167.425pt}}
\put(200.0,82.0){\rule[-0.200pt]{298.475pt}{0.400pt}}
\put(1439.0,82.0){\rule[-0.200pt]{0.400pt}{167.425pt}}
\put(200.0,777.0){\rule[-0.200pt]{298.475pt}{0.400pt}}
\end{picture}

 % GNUPLOT: LaTeX picture
\setlength{\unitlength}{0.240900pt}
\ifx\plotpoint\undefined\newsavebox{\plotpoint}\fi
\sbox{\plotpoint}{\rule[-0.200pt]{0.400pt}{0.400pt}}%
\begin{picture}(1500,900)(0,0)
\sbox{\plotpoint}{\rule[-0.200pt]{0.400pt}{0.400pt}}%
\put(220.0,82.0){\rule[-0.200pt]{4.818pt}{0.400pt}}
\put(200,82){\makebox(0,0)[r]{$0$}}
\put(1419.0,82.0){\rule[-0.200pt]{4.818pt}{0.400pt}}
\put(220.0,151.0){\rule[-0.200pt]{4.818pt}{0.400pt}}
\put(200,151){\makebox(0,0)[r]{$100000$}}
\put(1419.0,151.0){\rule[-0.200pt]{4.818pt}{0.400pt}}
\put(220.0,221.0){\rule[-0.200pt]{4.818pt}{0.400pt}}
\put(200,221){\makebox(0,0)[r]{$200000$}}
\put(1419.0,221.0){\rule[-0.200pt]{4.818pt}{0.400pt}}
\put(220.0,290.0){\rule[-0.200pt]{4.818pt}{0.400pt}}
\put(200,290){\makebox(0,0)[r]{$300000$}}
\put(1419.0,290.0){\rule[-0.200pt]{4.818pt}{0.400pt}}
\put(220.0,360.0){\rule[-0.200pt]{4.818pt}{0.400pt}}
\put(200,360){\makebox(0,0)[r]{$400000$}}
\put(1419.0,360.0){\rule[-0.200pt]{4.818pt}{0.400pt}}
\put(220.0,429.0){\rule[-0.200pt]{4.818pt}{0.400pt}}
\put(200,429){\makebox(0,0)[r]{$500000$}}
\put(1419.0,429.0){\rule[-0.200pt]{4.818pt}{0.400pt}}
\put(220.0,499.0){\rule[-0.200pt]{4.818pt}{0.400pt}}
\put(200,499){\makebox(0,0)[r]{$600000$}}
\put(1419.0,499.0){\rule[-0.200pt]{4.818pt}{0.400pt}}
\put(220.0,568.0){\rule[-0.200pt]{4.818pt}{0.400pt}}
\put(200,568){\makebox(0,0)[r]{$700000$}}
\put(1419.0,568.0){\rule[-0.200pt]{4.818pt}{0.400pt}}
\put(220.0,638.0){\rule[-0.200pt]{4.818pt}{0.400pt}}
\put(200,638){\makebox(0,0)[r]{$800000$}}
\put(1419.0,638.0){\rule[-0.200pt]{4.818pt}{0.400pt}}
\put(220.0,707.0){\rule[-0.200pt]{4.818pt}{0.400pt}}
\put(200,707){\makebox(0,0)[r]{$900000$}}
\put(1419.0,707.0){\rule[-0.200pt]{4.818pt}{0.400pt}}
\put(220.0,777.0){\rule[-0.200pt]{4.818pt}{0.400pt}}
\put(200,777){\makebox(0,0)[r]{$1e+06$}}
\put(1419.0,777.0){\rule[-0.200pt]{4.818pt}{0.400pt}}
\put(220.0,82.0){\rule[-0.200pt]{0.400pt}{4.818pt}}
\put(220,41){\makebox(0,0){$-4\pi$}}
\put(220.0,757.0){\rule[-0.200pt]{0.400pt}{4.818pt}}
\put(372.0,82.0){\rule[-0.200pt]{0.400pt}{4.818pt}}
\put(372,41){\makebox(0,0){$-3\pi$}}
\put(372.0,757.0){\rule[-0.200pt]{0.400pt}{4.818pt}}
\put(525.0,82.0){\rule[-0.200pt]{0.400pt}{4.818pt}}
\put(525,41){\makebox(0,0){$-2\pi$}}
\put(525.0,757.0){\rule[-0.200pt]{0.400pt}{4.818pt}}
\put(830.0,82.0){\rule[-0.200pt]{0.400pt}{4.818pt}}
\put(830,41){\makebox(0,0){0}}
\put(830.0,757.0){\rule[-0.200pt]{0.400pt}{4.818pt}}
\put(1134.0,82.0){\rule[-0.200pt]{0.400pt}{4.818pt}}
\put(1134,41){\makebox(0,0){$2\pi$}}
\put(1134.0,757.0){\rule[-0.200pt]{0.400pt}{4.818pt}}
\put(1287.0,82.0){\rule[-0.200pt]{0.400pt}{4.818pt}}
\put(1287,41){\makebox(0,0){$3\pi$}}
\put(1287.0,757.0){\rule[-0.200pt]{0.400pt}{4.818pt}}
\put(1439.0,82.0){\rule[-0.200pt]{0.400pt}{4.818pt}}
\put(1439,41){\makebox(0,0){$4\pi$}}
\put(1439.0,757.0){\rule[-0.200pt]{0.400pt}{4.818pt}}
\put(220.0,82.0){\rule[-0.200pt]{0.400pt}{167.425pt}}
\put(220.0,82.0){\rule[-0.200pt]{293.657pt}{0.400pt}}
\put(1439.0,82.0){\rule[-0.200pt]{0.400pt}{167.425pt}}
\put(220.0,777.0){\rule[-0.200pt]{293.657pt}{0.400pt}}
\put(829,839){\makebox(0,0){$f(\omega)$ für $t=1000$}}
\put(220,82){\usebox{\plotpoint}}
\put(220,82){\usebox{\plotpoint}}
\put(220,82){\usebox{\plotpoint}}
\put(220,82){\usebox{\plotpoint}}
\put(220,82){\usebox{\plotpoint}}
\put(220,82){\usebox{\plotpoint}}
\put(220,82){\usebox{\plotpoint}}
\put(220,82){\usebox{\plotpoint}}
\put(220,82){\usebox{\plotpoint}}
\put(220,82){\usebox{\plotpoint}}
\put(220,82){\usebox{\plotpoint}}
\put(220,82){\usebox{\plotpoint}}
\put(220,82){\usebox{\plotpoint}}
\put(220,82){\usebox{\plotpoint}}
\put(220,82){\usebox{\plotpoint}}
\put(220,82){\usebox{\plotpoint}}
\put(220,82){\usebox{\plotpoint}}
\put(220,82){\usebox{\plotpoint}}
\put(220,82){\usebox{\plotpoint}}
\put(220,82){\usebox{\plotpoint}}
\put(220,82){\usebox{\plotpoint}}
\put(220,82){\usebox{\plotpoint}}
\put(220,82){\usebox{\plotpoint}}
\put(220,82){\usebox{\plotpoint}}
\put(220,82){\usebox{\plotpoint}}
\put(220,82){\usebox{\plotpoint}}
\put(220,82){\usebox{\plotpoint}}
\put(220,82){\usebox{\plotpoint}}
\put(220,82){\usebox{\plotpoint}}
\put(220,82){\usebox{\plotpoint}}
\put(220,82){\usebox{\plotpoint}}
\put(220,82){\usebox{\plotpoint}}
\put(220,82){\usebox{\plotpoint}}
\put(220,82){\usebox{\plotpoint}}
\put(220,82){\usebox{\plotpoint}}
\put(220,82){\usebox{\plotpoint}}
\put(220,82){\usebox{\plotpoint}}
\put(220,82){\usebox{\plotpoint}}
\put(220,82){\usebox{\plotpoint}}
\put(220,82){\usebox{\plotpoint}}
\put(220,82){\usebox{\plotpoint}}
\put(220,82){\usebox{\plotpoint}}
\put(220.0,82.0){\rule[-0.200pt]{145.985pt}{0.400pt}}
\put(826.0,82.0){\usebox{\plotpoint}}
\put(826.0,82.0){\usebox{\plotpoint}}
\put(826.0,82.0){\usebox{\plotpoint}}
\put(826.0,82.0){\usebox{\plotpoint}}
\put(826.0,82.0){\usebox{\plotpoint}}
\put(827.0,82.0){\usebox{\plotpoint}}
\put(827.0,82.0){\usebox{\plotpoint}}
\put(827.0,82.0){\usebox{\plotpoint}}
\put(827.0,82.0){\usebox{\plotpoint}}
\put(827.0,82.0){\usebox{\plotpoint}}
\put(827.0,82.0){\usebox{\plotpoint}}
\put(827.0,82.0){\rule[-0.200pt]{0.400pt}{0.482pt}}
\put(827.0,84.0){\usebox{\plotpoint}}
\put(828.0,82.0){\rule[-0.200pt]{0.400pt}{0.482pt}}
\put(828.0,82.0){\rule[-0.200pt]{0.400pt}{0.482pt}}
\put(828.0,82.0){\rule[-0.200pt]{0.400pt}{0.482pt}}
\put(828.0,82.0){\rule[-0.200pt]{0.400pt}{0.723pt}}
\put(828.0,82.0){\rule[-0.200pt]{0.400pt}{0.723pt}}
\put(828.0,82.0){\rule[-0.200pt]{0.400pt}{1.445pt}}
\put(828.0,86.0){\rule[-0.200pt]{0.400pt}{0.482pt}}
\put(828.0,86.0){\usebox{\plotpoint}}
\put(829.0,82.0){\rule[-0.200pt]{0.400pt}{0.964pt}}
\put(829.0,82.0){\rule[-0.200pt]{0.400pt}{2.650pt}}
\put(829.0,82.0){\rule[-0.200pt]{0.400pt}{2.650pt}}
\put(829.0,82.0){\rule[-0.200pt]{0.400pt}{7.950pt}}
\put(829.0,82.0){\rule[-0.200pt]{0.400pt}{7.950pt}}
\put(829.0,82.0){\rule[-0.200pt]{0.400pt}{167.185pt}}
\put(829.0,776.0){\usebox{\plotpoint}}
\put(830.0,82.0){\rule[-0.200pt]{0.400pt}{167.185pt}}
\put(830.0,82.0){\rule[-0.200pt]{0.400pt}{7.950pt}}
\put(830.0,82.0){\rule[-0.200pt]{0.400pt}{7.950pt}}
\put(830.0,82.0){\rule[-0.200pt]{0.400pt}{2.650pt}}
\put(830.0,82.0){\rule[-0.200pt]{0.400pt}{2.650pt}}
\put(830.0,82.0){\rule[-0.200pt]{0.400pt}{0.964pt}}
\put(830.0,86.0){\usebox{\plotpoint}}
\put(831.0,86.0){\rule[-0.200pt]{0.400pt}{0.482pt}}
\put(831.0,82.0){\rule[-0.200pt]{0.400pt}{1.445pt}}
\put(831.0,82.0){\rule[-0.200pt]{0.400pt}{0.723pt}}
\put(831.0,82.0){\rule[-0.200pt]{0.400pt}{0.723pt}}
\put(831.0,82.0){\rule[-0.200pt]{0.400pt}{0.482pt}}
\put(831.0,82.0){\rule[-0.200pt]{0.400pt}{0.482pt}}
\put(831.0,82.0){\rule[-0.200pt]{0.400pt}{0.482pt}}
\put(831.0,84.0){\usebox{\plotpoint}}
\put(832.0,82.0){\rule[-0.200pt]{0.400pt}{0.482pt}}
\put(832.0,82.0){\usebox{\plotpoint}}
\put(832.0,82.0){\usebox{\plotpoint}}
\put(832.0,82.0){\usebox{\plotpoint}}
\put(832.0,82.0){\usebox{\plotpoint}}
\put(832.0,82.0){\usebox{\plotpoint}}
\put(832.0,82.0){\usebox{\plotpoint}}
\put(832.0,82.0){\usebox{\plotpoint}}
\put(833.0,82.0){\usebox{\plotpoint}}
\put(833.0,82.0){\usebox{\plotpoint}}
\put(833.0,82.0){\usebox{\plotpoint}}
\put(833.0,82.0){\usebox{\plotpoint}}
\put(833.0,82.0){\rule[-0.200pt]{145.985pt}{0.400pt}}
\put(220.0,82.0){\rule[-0.200pt]{0.400pt}{167.425pt}}
\put(220.0,82.0){\rule[-0.200pt]{293.657pt}{0.400pt}}
\put(1439.0,82.0){\rule[-0.200pt]{0.400pt}{167.425pt}}
\put(220.0,777.0){\rule[-0.200pt]{293.657pt}{0.400pt}}
\end{picture}

 \caption{Die Funktion  \(f(\omega_{ni}) \) für verschiedene Zeiten \(t=1, t=100, t=1000\).  Man erkennt, dass die Funktion \(f(\omega_{ni})\) für größere Zeiten sich einer \(\delta\)-Funktion nähert. }
 \label{fig:1}

\end{minipage}
\end{figure}

Wobei es gilt 
\begin{align}
  \label{eq:30}
  f(\omega_{ni})= \frac{4}{\hbar^2\omega_{ni}^2}\sin^2(\frac{\omega_{ni} t}{2}) \quad \text{ mit } \omega_{ni}=\frac{E_n-E_i}{\hbar}
\end{align}

Wir wollen wir die Funktion \(f(\omega_{ni}) \) weiter Vereinfachen. Dazu betrachten wir sie für verschiedene \(t\). Siehe dazu Abbildung \ref{fig:1}.

Wie man in den Abbildung deutlich erkennt, nähert sich die Funktion \(f(\omega_{ni})\) für große \(t\) einer \(\delta\)-Funktion. D.h. es gilt

\begin{align}
  \label{eq:31}
  f(\omega_{ni}) \stackrel{t \to \infty}= c \delta(\omega_{ni})
\end{align}

Um die Konstante \(c\) zu bestimmen integrieren wir die Gleichung (\ref{eq:31}) auf beiden Seiten nach \(d\omega\) über das gesamte Intervall

\begin{align}
  \label{eq:32}
  \int_{-\infty}^\infty d\omega f(\omega_{ni}) = c \underbr{ \int_{-\infty}^\infty d\omega \delta(\omega_{ni}) }_{=1}
\end{align}

Das heißt, es gilt folgendes Integral zu berechnen

\begin{align}
  \label{eq:33}
  c &= \int_{-\infty}^\infty d\omega f(\omega_{ni}) \qquad \text{ mit } f(\omega_{ni})=\frac{4}{\hbar^2\omega_{ni}^2}\sin^2(\frac{\omega_{ni} t}{2}) \notag\\
&=\frac{4}{\hbar^2} \int_{-\infty}^\infty d\omega \frac{1}{\omega_{ni}^2}\sin^2(\frac{\omega_{ni} t}{2})
\end{align}

Zum Berechnen des Integrals ist eine Substitution des Sinus Arguments \(x= \frac{\omega_{ni} t}{2}\)  notwendig. Mit \(\omega = \frac{2x}{t} \) und \(d\omega = \frac{2dx}{t}\) eingesetzt in Gleichung (\ref{eq:33}) folgt

\begin{align}
  \label{eq:34}
  c = \frac{4}{\hbar^2} \int_{-\infty}^\infty dx \frac{2}{t} \frac{t^2}{4x^2}\sin^2(x) = \frac{2t}{\hbar^2} \underbr{\int_{-\infty}^\infty dx \frac{\sin^2(x)}{x^2}}_{\pi} = \frac{2t}{\hbar^2} \pi
\end{align}

Setzen wir \(c\) in die Gleichung (\ref{eq:31}) ein so vereinfacht sich die Funktion \(f(\omega_{ni})\) zu

\begin{align}
  \label{eq:35}
  f(\omega_{ni}) \stackrel{t \to \infty}= \frac{2\pi t}{\hbar^2} \delta(\omega_{ni})
\end{align}

Damit können wir die Übergangswahrscheinlichkeit Gleichung (\ref{eq:29}) für große Zeiten schreiben

\begin{align}
  \label{eq:36}
   P(i\rightarrow n) =& \frac{4}{\hbar^2\omega_{ni}^2}|V_{ni}|^2\sin^2(\frac{\omega_{ni} t}{2}) \notag\\
\stackrel{t\to\infty}=& \frac{2\pi t}{\hbar^2} |V_{ni}|^2  \delta(\omega_{ni}) = \frac{2\pi t}{\hbar^2} |V_{ni}|^2  \delta(\frac{E_n-E_i}{\hbar}) = \frac{2\pi t}{\hbar} |V_{ni}|^2  \delta(E_n-E_i)
\end{align}

Gerne verwendet man anstelle der Übergangswahrscheinlichkeit die Übergangsrate, die als Übergangswahrscheinlichkeit pro Zeiteinheit definiert ist \(w_{i\to n}= \diff_t P(i\rightarrow n)\). Damit gilt

\begin{align}
  \label{eq:37}
\boxed{  w_{i\to n}= |V_{ni}|^2 \frac{2\pi}{\hbar} \delta(E_n-E_i) }
\end{align}

Diese Gleichung (\ref{eq:37}) wird auch als \textbf{Fermis-Goldene-Regel} bezeichnet.
\\
Wie man aus der Gleichung (\ref{eq:36}) unschwer erkennen kann, gibt es nur eine Wahrscheinlichkeit für ein Übergang zwischen zwei Zuständen wenn ihre Energieniveaus gleich sind (wegen der \(\delta\)-Funktion, was der Energieerhaltung entspricht. Zum Beispiel bei der Streuung betrachtet man eine einfallende und gestreute Teilchen-Welle die zwei unterschiedliche Zustände repräsentieren. Jedoch ist die Energie der einfallenden und gestreuten Welle gleich. Oder beim Zerfall eines Neutrons in ein Proton, Elektron und ein Elektron-Antineutrino handelt es sich ebenso um zwei unterschiedliche Zustände, nämlich den Zustand des Neutrons \(\ket{i}\) und dem Zustand von den resultierenden drei Teilchen, die man mit dem Zustand \(\ket{n}\) beschreibt. In beiden Zuständen bleibt die Energie erhalten.

Normalerweise betrachtet man nicht die Übergangswahrscheinlichkeit zwischen zwei bestimmten Energieniveaus, sondern die Übergangswahrscheinlichkeit zwischen allen Zuständen in einem Energieniveau im Intervall \([E,E+dE]\). Das bezeichnet man als die \textit{totale Übergangswahrscheinlichkeit}. Für die gilt (für \(i\ne n\))

\begin{align}
  \label{eq:38}
  P\approx \sum_{E_i\approx E_n} P(i\rightarrow n) = \sum_{E_i\approx E_n} |V_{ni}|^2 \frac{2\pi t}{\hbar} \delta(E_n-E_i) =  |V_{ni}|^2 \frac{2\pi t}{\hbar} \underbr{ \sum_{E_i\approx E_n}\delta(E_n-E_i)}_{\rho(E_n) }
\end{align}

Mit der Zustandsdichte \(\rho\), die die Dichte der Energie-Zustände in einem Intervall \([E,E+dE]\) angibt. Für diese gilt

\begin{align}
  \label{eq:39}
  \rho(E_n) = \sum_{E_i\approx E_n}\delta(E_n-E_i) \equiv \int dE \delta(E_n-E_i)
\end{align}

Aus der Gleichung (\ref{eq:38}) folgt die totale Übergangsrate die eine andere Form der Fermis-Goldene-Regel darstellt

\begin{align}
  \label{eq:40}
  \boxed{  w_{i\to \{n\}}= \frac{2\pi }{\hbar} |V_{ni}|^2  \rho(E_n) }
\end{align}

\subsection{Harmonische Störung}

Wir betrachten eine Harmonische Störung mit dem allgemeinen Ansatz

\begin{align}
  \label{eq:41}
  V(t) = Ve^{i\omega t} + V^\dagger e^{-i\omega t}
\end{align}

Zunächst bestimmen wir das Matrixelement \(V_{ni}(t)\)

\begin{align}
  \label{eq:43}
   V_{ni}(t) &= \bra{n} V_I \ket{i} = \bra{n}e^{\frac{i}{\hbar}H_0 t}V(t) e^{ -\frac{i}{\hbar}H_0 t } \ket{i} = \underbr{\bra{n}e^{\frac{i}{\hbar}H_0 t}}_{\bra{n}e^{\frac{i}{\hbar}E_n t}} \left(Ve^{i\omega t} + V^\dagger e^{-i\omega t}\right) \underbr{ e^{ -\frac{i}{\hbar}H_0 t } \ket{i}}_{e^{ -\frac{i}{\hbar}E_i t } \ket{i}} \notag \\
 & = \bra{n}e^{\frac{i}{\hbar}(E_n-E_i) t} \left(Ve^{i\omega t} + V^\dagger e^{-i\omega t}\right) \ket{i} =  e^{i\omega_{ni} t} \left(\bra{n}V\ket{i} e^{i\omega t} +\bra{n} V^\dagger\ket{i} e^{-i\omega t}\right) \notag \\
 &=  V_{ni} e^{it(\omega_{ni}+\omega) }+ V^\dagger_{ni} e^{it(\omega_{ni}-\omega) }
\end{align}

Für diese Störung wollen wir nun die Übergangswahrscheinlichkeit und die Übergangsrate in ersten Ordnung der zeitabhängigen Störungstheorie berechnen für ein Übergang \(\ket{i}\to\ket{n}\) wobei \(i\ne n\). Um die Übergangswahrscheinlichkeit von \(\ket{i}\) nach \(\ket{n}\) zu bestimmen setzen wir den Störoperator in die Gleichung (\ref{eq:25}) ein

\begin{align}
  \label{eq:42}
 P(i\rightarrow n) &= \left|(\frac{-i}{\hbar})\langle n |\int_{t_0}^t V_I(t')dt'|i\rangle\right|^2 \notag \\
&= \frac{1}{\hbar^2}\left| \int_{0}^t \bra{n}  V_I(t) \ket{i} dt'\right|^2  \quad \text{mit Gleichung (\ref{eq:43})} \notag\\
&= \frac{1}{\hbar^2}\left| \int_{0}^t \left( V_{ni} e^{it(\omega_{ni}+\omega) }+ V^\dagger_{ni} e^{it(\omega_{ni}-\omega)} \right)  dt'\right|^2 \notag\\
&= \frac{1}{\hbar^2}  \Big|V_{ni}\underbrace{\int_{0}^t dt' e^{it(\omega_{ni}+\omega) }}_{  \frac{2}{\omega_{ni}+\omega}\sin(\frac{(\omega_{ni}+\omega) t}{2})e^{\frac{it(\omega_{ni}+\omega)}{2}}  }+ V^\dagger_{ni}\underbrace{ \int_{0}^t dt'  e^{it(\omega_{ni}-\omega)}}_{ \frac{2}{\omega_{ni}-\omega}\sin(\frac{(\omega_{ni}-\omega) t}{2})e^{\frac{it(\omega_{ni}-\omega)}{2}} }\Big|^2 \quad \text{ mit (\ref{eq:27})}\notag\\
&= \frac{1}{\hbar^2}  \Big|V_{ni} \frac{2}{\omega_{ni}+\omega}\sin\left(\frac{(\omega_{ni}+\omega) t}{2}\right)e^{\frac{it(\omega_{ni}+\omega)}{2}} + V^\dagger_{ni} \frac{2}{\omega_{ni}-\omega}\sin\left(\frac{(\omega_{ni}-\omega) t}{2}\right)e^{\frac{it(\omega_{ni}-\omega)}{2} }\Big|^2 \notag\\
\end{align}

Nun gibt es zwei Vorgehensweisen:

\begin{enumerate}
\item Man betrachte \(\omega_{ni}\approx - \omega\), dann sieht man, dass es eine Resonanz beim ersten Term gibt. Der zweite Term ist in diesem Fall zu vernachlässigen. Es gibt für \(\omega_{ni}\approx \omega\) eine Resonanz beim zweiten Term und in diesem Falle ist der erste Term zu vernachlässigen.
\item Berechne den Betragsquadrat aus. Wobei es gilt
  \begin{align}
    \label{eq:44}
    |a+b|^2 = (a+b)^\dagger \cdot(a+b) = (a^\dagger+b^\dagger)\cdot(a+b) = a^\dagger a+a^\dagger b + b^\dagger a + b^\dagger b = |a|^2+|b|^2 + a^\dagger b + b^\dagger a 
  \end{align}
\end{enumerate}

Im zweiten Fall erhalten wir aus der Gleichung (\ref{eq:42})

\begin{align}
  \label{eq:45}
  P(i\rightarrow n) &= \frac{1}{\hbar^2}  \Big| \underbr{V_{ni} \frac{2}{\omega_{ni}+\omega}\sin\left(\frac{(\omega_{ni}+\omega) t}{2}\right)e^{\frac{it(\omega_{ni}+\omega)}{2}}}_{a} +\underbr{ V^\dagger_{ni} \frac{2}{\omega_{ni}-\omega}\sin\left(\frac{(\omega_{ni}-\omega) t}{2}\right)e^{\frac{it(\omega_{ni}-\omega)}{2} }}_{b}\Big|^2 \notag\\
\end{align}

Nebenrechnung
\begin{align}
  \label{eq:46}
  a^\dagger b &= V_{ni}^\dagger \frac{2}{\omega_{ni}+\omega}\sin\left(\frac{(\omega_{ni}+\omega) t}{2}\right)e^{\frac{-it(\omega_{ni}+\omega)}{2}}\cdot  V^\dagger_{ni} \frac{2}{\omega_{ni}-\omega}\sin\left(\frac{(\omega_{ni}-\omega) t}{2}\right)e^{\frac{it(\omega_{ni}-\omega)}{2} } \notag\\
&=(V_{ni}^\dagger)^2 \frac{4}{\omega_{ni}^2 -\omega^2}\sin\left(\frac{(\omega_{ni}+\omega) t}{2}\right)\sin\left(\frac{(\omega_{ni}-\omega) t}{2}\right)e^{-i\omega t}
\end{align}

\begin{align}
  \label{eq:47}
    b^\dagger a &= V_{ni}\frac{2}{\omega_{ni}-\omega}\sin\left(\frac{(\omega_{ni}-\omega) t}{2}\right)e^{\frac{-it(\omega_{ni}-\omega)}{2}}\cdot  V_{ni} \frac{2}{\omega_{ni}+\omega}\sin\left(\frac{(\omega_{ni}+\omega) t}{2}\right)e^{\frac{it(\omega_{ni}+\omega)}{2} } \notag\\
&=V_{ni}^2 \frac{4}{\omega_{ni}^2 -\omega^2}\sin\left(\frac{(\omega_{ni}-\omega) t}{2}\right)\sin\left(\frac{(\omega_{ni}+\omega) t}{2}\right)e^{i\omega t}
\end{align}

\begin{align}
  \label{eq:48}
  a^\dagger b + b^\dagger a = \frac{4}{\omega_{ni}^2 -\omega^2}\sin\left(\frac{(\omega_{ni}-\omega) t}{2}\right)\sin\left(\frac{(\omega_{ni}+\omega) t}{2}\right)\left[(V_{ni}^\dagger)^2e^{-i\omega t} +   V_{ni}^2 e^{i\omega t}  \right]
\end{align}



Mit der Gleichung (\ref{eq:48}) und (\ref{eq:44}) lautet die gesamte Gleichung (\ref{eq:45})


\begin{align}
  \label{eq:49}
   P(i\rightarrow n) &= \frac{1}{\hbar^2}\Bigg\{  \Big|V_{ni} \frac{2}{\omega_{ni}+\omega}\sin\left(\frac{(\omega_{ni}+\omega) t}{2}\right)e^{\frac{it(\omega_{ni}+\omega)}{2}}\Big|^2 +  \Big|V^\dagger_{ni} \frac{2}{\omega_{ni}-\omega}\sin\left(\frac{(\omega_{ni}-\omega) t}{2}\right)e^{\frac{it(\omega_{ni}-\omega)}{2} }\Big|^2 \notag \\
 & + \frac{4}{\omega_{ni}^2 -\omega^2} \underbr{\sin\left(\frac{(\omega_{ni}-\omega) t}{2}\right)}_{\omega_{ni}\approx \omega \to 0}  \underbr{\sin\left(\frac{(\omega_{ni}+\omega) t}{2}\right)}_{\omega_{ni}\approx -\omega \to 0} \left[(V_{ni}^\dagger)^2e^{-i\omega t} +   V_{ni}^2 e^{i\omega t}  \right]  \Bigg\}
\end{align}

\begin{figure}[!thb]
  \centering
  % GNUPLOT: LaTeX picture
\setlength{\unitlength}{0.240900pt}
\ifx\plotpoint\undefined\newsavebox{\plotpoint}\fi
\sbox{\plotpoint}{\rule[-0.200pt]{0.400pt}{0.400pt}}%
\begin{picture}(1500,900)(0,0)
\sbox{\plotpoint}{\rule[-0.200pt]{0.400pt}{0.400pt}}%
\put(120.0,82.0){\rule[-0.200pt]{4.818pt}{0.400pt}}
\put(100,82){\makebox(0,0)[r]{$0$}}
\put(1419.0,82.0){\rule[-0.200pt]{4.818pt}{0.400pt}}
\put(120.0,181.0){\rule[-0.200pt]{4.818pt}{0.400pt}}
\put(100,181){\makebox(0,0)[r]{$1$}}
\put(1419.0,181.0){\rule[-0.200pt]{4.818pt}{0.400pt}}
\put(120.0,281.0){\rule[-0.200pt]{4.818pt}{0.400pt}}
\put(100,281){\makebox(0,0)[r]{$2$}}
\put(1419.0,281.0){\rule[-0.200pt]{4.818pt}{0.400pt}}
\put(120.0,380.0){\rule[-0.200pt]{4.818pt}{0.400pt}}
\put(100,380){\makebox(0,0)[r]{$3$}}
\put(1419.0,380.0){\rule[-0.200pt]{4.818pt}{0.400pt}}
\put(120.0,479.0){\rule[-0.200pt]{4.818pt}{0.400pt}}
\put(100,479){\makebox(0,0)[r]{$4$}}
\put(1419.0,479.0){\rule[-0.200pt]{4.818pt}{0.400pt}}
\put(120.0,578.0){\rule[-0.200pt]{4.818pt}{0.400pt}}
\put(100,578){\makebox(0,0)[r]{$5$}}
\put(1419.0,578.0){\rule[-0.200pt]{4.818pt}{0.400pt}}
\put(120.0,678.0){\rule[-0.200pt]{4.818pt}{0.400pt}}
\put(100,678){\makebox(0,0)[r]{$6$}}
\put(1419.0,678.0){\rule[-0.200pt]{4.818pt}{0.400pt}}
\put(120.0,777.0){\rule[-0.200pt]{4.818pt}{0.400pt}}
\put(100,777){\makebox(0,0)[r]{$7$}}
\put(1419.0,777.0){\rule[-0.200pt]{4.818pt}{0.400pt}}
\put(120.0,82.0){\rule[-0.200pt]{0.400pt}{4.818pt}}
\put(120,41){\makebox(0,0){$-4\pi$}}
\put(120.0,757.0){\rule[-0.200pt]{0.400pt}{4.818pt}}
\put(285.0,82.0){\rule[-0.200pt]{0.400pt}{4.818pt}}
\put(285,41){\makebox(0,0){$-3 \pi$}}
\put(285.0,757.0){\rule[-0.200pt]{0.400pt}{4.818pt}}
\put(450.0,82.0){\rule[-0.200pt]{0.400pt}{4.818pt}}
\put(450,41){\makebox(0,0){$-\omega$}}
\put(450.0,757.0){\rule[-0.200pt]{0.400pt}{4.818pt}}
\put(615.0,82.0){\rule[-0.200pt]{0.400pt}{4.818pt}}
\put(615,41){\makebox(0,0){$-\pi$}}
\put(615.0,757.0){\rule[-0.200pt]{0.400pt}{4.818pt}}
\put(780.0,82.0){\rule[-0.200pt]{0.400pt}{4.818pt}}
\put(780,41){\makebox(0,0){0}}
\put(780.0,757.0){\rule[-0.200pt]{0.400pt}{4.818pt}}
\put(944.0,82.0){\rule[-0.200pt]{0.400pt}{4.818pt}}
\put(944,41){\makebox(0,0){$\pi$}}
\put(944.0,757.0){\rule[-0.200pt]{0.400pt}{4.818pt}}
\put(1109.0,82.0){\rule[-0.200pt]{0.400pt}{4.818pt}}
\put(1109,41){\makebox(0,0){$\omega$}}
\put(1109.0,757.0){\rule[-0.200pt]{0.400pt}{4.818pt}}
\put(1274.0,82.0){\rule[-0.200pt]{0.400pt}{4.818pt}}
\put(1274,41){\makebox(0,0){$3\pi$}}
\put(1274.0,757.0){\rule[-0.200pt]{0.400pt}{4.818pt}}
\put(1439.0,82.0){\rule[-0.200pt]{0.400pt}{4.818pt}}
\put(1439,41){\makebox(0,0){$4\pi$}}
\put(1439.0,757.0){\rule[-0.200pt]{0.400pt}{4.818pt}}
\put(120.0,82.0){\rule[-0.200pt]{0.400pt}{167.425pt}}
\put(120.0,82.0){\rule[-0.200pt]{317.747pt}{0.400pt}}
\put(1439.0,82.0){\rule[-0.200pt]{0.400pt}{167.425pt}}
\put(120.0,777.0){\rule[-0.200pt]{317.747pt}{0.400pt}}
\put(779,839){\makebox(0,0){$ \frac{1}{(\omega_{ni}+\omega)^2}\sin^2\left(\frac{(\omega_{ni}+\omega) t}{2}\right)  +   \frac{1}{(\omega_{ni}-\omega)^2}\sin^2\left(\frac{(\omega_{ni}-\omega) t}{2}\right)$ für $t=5$}}
\put(120,82){\usebox{\plotpoint}}
\put(120,82){\usebox{\plotpoint}}
\put(120,82){\usebox{\plotpoint}}
\put(120,82){\usebox{\plotpoint}}
\put(120,82){\usebox{\plotpoint}}
\put(120,82){\usebox{\plotpoint}}
\put(120,82){\usebox{\plotpoint}}
\put(120,82){\usebox{\plotpoint}}
\put(120,82){\usebox{\plotpoint}}
\put(120,82){\usebox{\plotpoint}}
\put(120,82){\usebox{\plotpoint}}
\put(120,82){\usebox{\plotpoint}}
\put(120,82){\usebox{\plotpoint}}
\put(120,82){\usebox{\plotpoint}}
\put(120,82){\usebox{\plotpoint}}
\put(120,82){\usebox{\plotpoint}}
\put(120,82){\usebox{\plotpoint}}
\put(120,82){\usebox{\plotpoint}}
\put(120,82){\usebox{\plotpoint}}
\put(120,82){\usebox{\plotpoint}}
\put(120,82){\usebox{\plotpoint}}
\put(120,82){\usebox{\plotpoint}}
\put(120,82){\usebox{\plotpoint}}
\put(120,82){\usebox{\plotpoint}}
\put(120,82){\usebox{\plotpoint}}
\put(120,82){\usebox{\plotpoint}}
\put(120,82){\usebox{\plotpoint}}
\put(120,82){\usebox{\plotpoint}}
\put(120,82){\usebox{\plotpoint}}
\put(120,82){\usebox{\plotpoint}}
\put(120,82){\usebox{\plotpoint}}
\put(120,82){\usebox{\plotpoint}}
\put(120,82){\usebox{\plotpoint}}
\put(120,82){\usebox{\plotpoint}}
\put(120,82){\usebox{\plotpoint}}
\put(120,82){\usebox{\plotpoint}}
\put(120,82){\usebox{\plotpoint}}
\put(120,82){\usebox{\plotpoint}}
\put(120.0,82.0){\rule[-0.200pt]{2.168pt}{0.400pt}}
\put(129.0,82.0){\usebox{\plotpoint}}
\put(129.0,83.0){\rule[-0.200pt]{1.686pt}{0.400pt}}
\put(136.0,83.0){\usebox{\plotpoint}}
\put(136.0,84.0){\rule[-0.200pt]{1.686pt}{0.400pt}}
\put(143.0,84.0){\usebox{\plotpoint}}
\put(143.0,85.0){\rule[-0.200pt]{5.541pt}{0.400pt}}
\put(166.0,84.0){\usebox{\plotpoint}}
\put(166.0,84.0){\rule[-0.200pt]{1.445pt}{0.400pt}}
\put(172.0,83.0){\usebox{\plotpoint}}
\put(172.0,83.0){\rule[-0.200pt]{1.445pt}{0.400pt}}
\put(178.0,82.0){\usebox{\plotpoint}}
\put(178.0,82.0){\rule[-0.200pt]{3.613pt}{0.400pt}}
\put(193.0,82.0){\usebox{\plotpoint}}
\put(193.0,83.0){\rule[-0.200pt]{1.445pt}{0.400pt}}
\put(199.0,83.0){\usebox{\plotpoint}}
\put(199.0,84.0){\rule[-0.200pt]{0.964pt}{0.400pt}}
\put(203.0,84.0){\usebox{\plotpoint}}
\put(203.0,85.0){\rule[-0.200pt]{0.964pt}{0.400pt}}
\put(207.0,85.0){\usebox{\plotpoint}}
\put(207.0,86.0){\rule[-0.200pt]{0.964pt}{0.400pt}}
\put(211.0,86.0){\usebox{\plotpoint}}
\put(211.0,87.0){\rule[-0.200pt]{2.168pt}{0.400pt}}
\put(220.0,87.0){\usebox{\plotpoint}}
\put(220.0,88.0){\rule[-0.200pt]{0.482pt}{0.400pt}}
\put(222.0,87.0){\usebox{\plotpoint}}
\put(222.0,87.0){\rule[-0.200pt]{1.927pt}{0.400pt}}
\put(230.0,86.0){\usebox{\plotpoint}}
\put(230.0,86.0){\rule[-0.200pt]{0.964pt}{0.400pt}}
\put(234.0,85.0){\usebox{\plotpoint}}
\put(234.0,85.0){\rule[-0.200pt]{0.964pt}{0.400pt}}
\put(238.0,84.0){\usebox{\plotpoint}}
\put(238.0,84.0){\rule[-0.200pt]{0.964pt}{0.400pt}}
\put(242.0,83.0){\usebox{\plotpoint}}
\put(242.0,83.0){\rule[-0.200pt]{0.964pt}{0.400pt}}
\put(246.0,82.0){\usebox{\plotpoint}}
\put(246.0,82.0){\rule[-0.200pt]{2.650pt}{0.400pt}}
\put(257.0,82.0){\usebox{\plotpoint}}
\put(257.0,83.0){\rule[-0.200pt]{0.964pt}{0.400pt}}
\put(261.0,83.0){\usebox{\plotpoint}}
\put(261.0,84.0){\rule[-0.200pt]{0.723pt}{0.400pt}}
\put(264.0,84.0){\usebox{\plotpoint}}
\put(266,84.67){\rule{0.241pt}{0.400pt}}
\multiput(266.00,84.17)(0.500,1.000){2}{\rule{0.120pt}{0.400pt}}
\put(264.0,85.0){\rule[-0.200pt]{0.482pt}{0.400pt}}
\put(267,86){\usebox{\plotpoint}}
\put(267,86){\usebox{\plotpoint}}
\put(267,86){\usebox{\plotpoint}}
\put(267,86){\usebox{\plotpoint}}
\put(267,86){\usebox{\plotpoint}}
\put(267,86){\usebox{\plotpoint}}
\put(267,86){\usebox{\plotpoint}}
\put(267,86){\usebox{\plotpoint}}
\put(267,86){\usebox{\plotpoint}}
\put(267,86){\usebox{\plotpoint}}
\put(267,86){\usebox{\plotpoint}}
\put(267,86){\usebox{\plotpoint}}
\put(267,86){\usebox{\plotpoint}}
\put(267,86){\usebox{\plotpoint}}
\put(267,86){\usebox{\plotpoint}}
\put(267,86){\usebox{\plotpoint}}
\put(267,86){\usebox{\plotpoint}}
\put(267,86){\usebox{\plotpoint}}
\put(267,86){\usebox{\plotpoint}}
\put(267,86){\usebox{\plotpoint}}
\put(267,86){\usebox{\plotpoint}}
\put(267,86){\usebox{\plotpoint}}
\put(267,86){\usebox{\plotpoint}}
\put(267,86){\usebox{\plotpoint}}
\put(267,86){\usebox{\plotpoint}}
\put(267,86){\usebox{\plotpoint}}
\put(267,86){\usebox{\plotpoint}}
\put(267,86){\usebox{\plotpoint}}
\put(267,86){\usebox{\plotpoint}}
\put(267,86){\usebox{\plotpoint}}
\put(267,86){\usebox{\plotpoint}}
\put(267,86){\usebox{\plotpoint}}
\put(267,86){\usebox{\plotpoint}}
\put(267,86){\usebox{\plotpoint}}
\put(267,86){\usebox{\plotpoint}}
\put(267,86){\usebox{\plotpoint}}
\put(267,86){\usebox{\plotpoint}}
\put(267,86){\usebox{\plotpoint}}
\put(267,86){\usebox{\plotpoint}}
\put(267,86){\usebox{\plotpoint}}
\put(267,86){\usebox{\plotpoint}}
\put(267,86){\usebox{\plotpoint}}
\put(267,86){\usebox{\plotpoint}}
\put(267,86){\usebox{\plotpoint}}
\put(267,86){\usebox{\plotpoint}}
\put(267,86){\usebox{\plotpoint}}
\put(267,86){\usebox{\plotpoint}}
\put(267,86){\usebox{\plotpoint}}
\put(267,86){\usebox{\plotpoint}}
\put(267,86){\usebox{\plotpoint}}
\put(267,86){\usebox{\plotpoint}}
\put(267,86){\usebox{\plotpoint}}
\put(267,86){\usebox{\plotpoint}}
\put(267,86){\usebox{\plotpoint}}
\put(267,86){\usebox{\plotpoint}}
\put(267,86){\usebox{\plotpoint}}
\put(267,86){\usebox{\plotpoint}}
\put(267,86){\usebox{\plotpoint}}
\put(267,86){\usebox{\plotpoint}}
\put(267,86){\usebox{\plotpoint}}
\put(267,86){\usebox{\plotpoint}}
\put(267,86){\usebox{\plotpoint}}
\put(267,86){\usebox{\plotpoint}}
\put(267,86){\usebox{\plotpoint}}
\put(267,86){\usebox{\plotpoint}}
\put(267,86){\usebox{\plotpoint}}
\put(267,86){\usebox{\plotpoint}}
\put(267,86){\usebox{\plotpoint}}
\put(267,86){\usebox{\plotpoint}}
\put(267,86){\usebox{\plotpoint}}
\put(267,86){\usebox{\plotpoint}}
\put(267,86){\usebox{\plotpoint}}
\put(267,86){\usebox{\plotpoint}}
\put(267,86){\usebox{\plotpoint}}
\put(267,86){\usebox{\plotpoint}}
\put(267.0,86.0){\rule[-0.200pt]{0.482pt}{0.400pt}}
\put(269.0,86.0){\usebox{\plotpoint}}
\put(269.0,87.0){\rule[-0.200pt]{0.482pt}{0.400pt}}
\put(271.0,87.0){\usebox{\plotpoint}}
\put(271.0,88.0){\rule[-0.200pt]{0.482pt}{0.400pt}}
\put(273.0,88.0){\usebox{\plotpoint}}
\put(273.0,89.0){\rule[-0.200pt]{0.482pt}{0.400pt}}
\put(275.0,89.0){\usebox{\plotpoint}}
\put(275.0,90.0){\rule[-0.200pt]{0.723pt}{0.400pt}}
\put(278.0,90.0){\usebox{\plotpoint}}
\put(278.0,91.0){\rule[-0.200pt]{0.482pt}{0.400pt}}
\put(280.0,91.0){\usebox{\plotpoint}}
\put(280.0,92.0){\rule[-0.200pt]{1.204pt}{0.400pt}}
\put(285.0,92.0){\usebox{\plotpoint}}
\put(285.0,93.0){\rule[-0.200pt]{1.204pt}{0.400pt}}
\put(290.0,92.0){\usebox{\plotpoint}}
\put(290.0,92.0){\rule[-0.200pt]{0.964pt}{0.400pt}}
\put(294.0,91.0){\usebox{\plotpoint}}
\put(294.0,91.0){\rule[-0.200pt]{0.723pt}{0.400pt}}
\put(297.0,90.0){\usebox{\plotpoint}}
\put(297.0,90.0){\rule[-0.200pt]{0.482pt}{0.400pt}}
\put(299.0,89.0){\usebox{\plotpoint}}
\put(299.0,89.0){\rule[-0.200pt]{0.482pt}{0.400pt}}
\put(301.0,88.0){\usebox{\plotpoint}}
\put(301.0,88.0){\rule[-0.200pt]{0.482pt}{0.400pt}}
\put(303.0,87.0){\usebox{\plotpoint}}
\put(303.0,87.0){\rule[-0.200pt]{0.482pt}{0.400pt}}
\put(305.0,86.0){\usebox{\plotpoint}}
\put(305.0,86.0){\rule[-0.200pt]{0.482pt}{0.400pt}}
\put(307.0,85.0){\usebox{\plotpoint}}
\put(307.0,85.0){\rule[-0.200pt]{0.482pt}{0.400pt}}
\put(309.0,84.0){\usebox{\plotpoint}}
\put(309.0,84.0){\rule[-0.200pt]{0.482pt}{0.400pt}}
\put(311.0,83.0){\usebox{\plotpoint}}
\put(311.0,83.0){\rule[-0.200pt]{0.723pt}{0.400pt}}
\put(314.0,82.0){\usebox{\plotpoint}}
\put(314.0,82.0){\rule[-0.200pt]{1.686pt}{0.400pt}}
\put(321.0,82.0){\usebox{\plotpoint}}
\put(321.0,83.0){\rule[-0.200pt]{0.723pt}{0.400pt}}
\put(324.0,83.0){\usebox{\plotpoint}}
\put(324.0,84.0){\rule[-0.200pt]{0.482pt}{0.400pt}}
\put(326.0,84.0){\usebox{\plotpoint}}
\put(326.0,85.0){\usebox{\plotpoint}}
\put(327.0,85.0){\usebox{\plotpoint}}
\put(328,85.67){\rule{0.241pt}{0.400pt}}
\multiput(328.00,85.17)(0.500,1.000){2}{\rule{0.120pt}{0.400pt}}
\put(327.0,86.0){\usebox{\plotpoint}}
\put(329,87){\usebox{\plotpoint}}
\put(329,87){\usebox{\plotpoint}}
\put(329,87){\usebox{\plotpoint}}
\put(329,87){\usebox{\plotpoint}}
\put(329,87){\usebox{\plotpoint}}
\put(329,87){\usebox{\plotpoint}}
\put(329,87){\usebox{\plotpoint}}
\put(329,87){\usebox{\plotpoint}}
\put(329,87){\usebox{\plotpoint}}
\put(329,87){\usebox{\plotpoint}}
\put(329,87){\usebox{\plotpoint}}
\put(329,87){\usebox{\plotpoint}}
\put(329,87){\usebox{\plotpoint}}
\put(329,87){\usebox{\plotpoint}}
\put(329,87){\usebox{\plotpoint}}
\put(329,87){\usebox{\plotpoint}}
\put(329,87){\usebox{\plotpoint}}
\put(329,87){\usebox{\plotpoint}}
\put(329,87){\usebox{\plotpoint}}
\put(329,87){\usebox{\plotpoint}}
\put(329,87){\usebox{\plotpoint}}
\put(329,87){\usebox{\plotpoint}}
\put(329,87){\usebox{\plotpoint}}
\put(329,87){\usebox{\plotpoint}}
\put(329,87){\usebox{\plotpoint}}
\put(329,87){\usebox{\plotpoint}}
\put(329,87){\usebox{\plotpoint}}
\put(329,87){\usebox{\plotpoint}}
\put(329,87){\usebox{\plotpoint}}
\put(329,87){\usebox{\plotpoint}}
\put(329,87){\usebox{\plotpoint}}
\put(329,87){\usebox{\plotpoint}}
\put(329,87){\usebox{\plotpoint}}
\put(329,87){\usebox{\plotpoint}}
\put(329,87){\usebox{\plotpoint}}
\put(329,87){\usebox{\plotpoint}}
\put(329,87){\usebox{\plotpoint}}
\put(329,87){\usebox{\plotpoint}}
\put(329,87){\usebox{\plotpoint}}
\put(329,87){\usebox{\plotpoint}}
\put(329,87){\usebox{\plotpoint}}
\put(329,87){\usebox{\plotpoint}}
\put(329,87){\usebox{\plotpoint}}
\put(329,87){\usebox{\plotpoint}}
\put(329,87){\usebox{\plotpoint}}
\put(329,87){\usebox{\plotpoint}}
\put(329,87){\usebox{\plotpoint}}
\put(329,87){\usebox{\plotpoint}}
\put(329,87){\usebox{\plotpoint}}
\put(329,87){\usebox{\plotpoint}}
\put(329,87){\usebox{\plotpoint}}
\put(329,87){\usebox{\plotpoint}}
\put(329,87){\usebox{\plotpoint}}
\put(329,87){\usebox{\plotpoint}}
\put(329,87){\usebox{\plotpoint}}
\put(329,87){\usebox{\plotpoint}}
\put(329,87){\usebox{\plotpoint}}
\put(329,87){\usebox{\plotpoint}}
\put(329,87){\usebox{\plotpoint}}
\put(329,87){\usebox{\plotpoint}}
\put(329,87){\usebox{\plotpoint}}
\put(329,87){\usebox{\plotpoint}}
\put(329,87){\usebox{\plotpoint}}
\put(329,87){\usebox{\plotpoint}}
\put(329,87){\usebox{\plotpoint}}
\put(329,87){\usebox{\plotpoint}}
\put(329,87){\usebox{\plotpoint}}
\put(329,87){\usebox{\plotpoint}}
\put(329,87){\usebox{\plotpoint}}
\put(329,87){\usebox{\plotpoint}}
\put(329,87){\usebox{\plotpoint}}
\put(329,87){\usebox{\plotpoint}}
\put(329,87){\usebox{\plotpoint}}
\put(329,87){\usebox{\plotpoint}}
\put(329,87){\usebox{\plotpoint}}
\put(329.0,87.0){\usebox{\plotpoint}}
\put(330.0,87.0){\usebox{\plotpoint}}
\put(330.0,88.0){\usebox{\plotpoint}}
\put(331.0,88.0){\usebox{\plotpoint}}
\put(331.0,89.0){\usebox{\plotpoint}}
\put(332.0,89.0){\usebox{\plotpoint}}
\put(332.0,90.0){\usebox{\plotpoint}}
\put(333.0,90.0){\rule[-0.200pt]{0.400pt}{0.482pt}}
\put(333.0,92.0){\usebox{\plotpoint}}
\put(334.0,92.0){\usebox{\plotpoint}}
\put(334.0,93.0){\usebox{\plotpoint}}
\put(335.0,93.0){\usebox{\plotpoint}}
\put(335.0,94.0){\usebox{\plotpoint}}
\put(336.0,94.0){\usebox{\plotpoint}}
\put(336.0,95.0){\usebox{\plotpoint}}
\put(337.0,95.0){\usebox{\plotpoint}}
\put(337.0,96.0){\usebox{\plotpoint}}
\put(338.0,96.0){\rule[-0.200pt]{0.400pt}{0.482pt}}
\put(338.0,98.0){\usebox{\plotpoint}}
\put(339.0,98.0){\usebox{\plotpoint}}
\put(339.0,99.0){\usebox{\plotpoint}}
\put(340.0,99.0){\usebox{\plotpoint}}
\put(340.0,100.0){\usebox{\plotpoint}}
\put(341.0,100.0){\usebox{\plotpoint}}
\put(341.0,101.0){\usebox{\plotpoint}}
\put(342.0,101.0){\rule[-0.200pt]{0.400pt}{0.482pt}}
\put(342.0,103.0){\usebox{\plotpoint}}
\put(343.0,103.0){\usebox{\plotpoint}}
\put(343.0,104.0){\usebox{\plotpoint}}
\put(344.0,104.0){\usebox{\plotpoint}}
\put(344.0,105.0){\usebox{\plotpoint}}
\put(345.0,105.0){\usebox{\plotpoint}}
\put(345.0,106.0){\usebox{\plotpoint}}
\put(346.0,106.0){\usebox{\plotpoint}}
\put(346.0,107.0){\usebox{\plotpoint}}
\put(347.0,107.0){\usebox{\plotpoint}}
\put(347.0,108.0){\usebox{\plotpoint}}
\put(348.0,108.0){\usebox{\plotpoint}}
\put(348.0,109.0){\usebox{\plotpoint}}
\put(349.0,109.0){\usebox{\plotpoint}}
\put(349.0,110.0){\rule[-0.200pt]{0.482pt}{0.400pt}}
\put(351.0,110.0){\usebox{\plotpoint}}
\put(351.0,111.0){\rule[-0.200pt]{0.482pt}{0.400pt}}
\put(353.0,111.0){\usebox{\plotpoint}}
\put(353.0,112.0){\rule[-0.200pt]{0.964pt}{0.400pt}}
\put(357.0,111.0){\usebox{\plotpoint}}
\put(357.0,111.0){\rule[-0.200pt]{0.723pt}{0.400pt}}
\put(360.0,110.0){\usebox{\plotpoint}}
\put(360.0,110.0){\usebox{\plotpoint}}
\put(361.0,109.0){\usebox{\plotpoint}}
\put(361.0,109.0){\usebox{\plotpoint}}
\put(362.0,108.0){\usebox{\plotpoint}}
\put(362.0,108.0){\usebox{\plotpoint}}
\put(363.0,107.0){\usebox{\plotpoint}}
\put(363.0,107.0){\usebox{\plotpoint}}
\put(364.0,106.0){\usebox{\plotpoint}}
\put(364.0,106.0){\usebox{\plotpoint}}
\put(365.0,105.0){\usebox{\plotpoint}}
\put(365.0,105.0){\usebox{\plotpoint}}
\put(366.0,104.0){\usebox{\plotpoint}}
\put(366.0,104.0){\usebox{\plotpoint}}
\put(367.0,102.0){\rule[-0.200pt]{0.400pt}{0.482pt}}
\put(367.0,102.0){\usebox{\plotpoint}}
\put(368.0,101.0){\usebox{\plotpoint}}
\put(368.0,101.0){\usebox{\plotpoint}}
\put(369.0,99.0){\rule[-0.200pt]{0.400pt}{0.482pt}}
\put(369.0,99.0){\usebox{\plotpoint}}
\put(370.0,97.0){\rule[-0.200pt]{0.400pt}{0.482pt}}
\put(370.0,97.0){\usebox{\plotpoint}}
\put(371.0,96.0){\usebox{\plotpoint}}
\put(371.0,96.0){\usebox{\plotpoint}}
\put(372.0,94.0){\rule[-0.200pt]{0.400pt}{0.482pt}}
\put(372.0,94.0){\usebox{\plotpoint}}
\put(373.0,93.0){\usebox{\plotpoint}}
\put(373.0,93.0){\usebox{\plotpoint}}
\put(374.0,91.0){\rule[-0.200pt]{0.400pt}{0.482pt}}
\put(374.0,91.0){\usebox{\plotpoint}}
\put(375.0,89.0){\rule[-0.200pt]{0.400pt}{0.482pt}}
\put(375.0,89.0){\usebox{\plotpoint}}
\put(376.0,88.0){\usebox{\plotpoint}}
\put(376.0,88.0){\usebox{\plotpoint}}
\put(377.0,87.0){\usebox{\plotpoint}}
\put(377.0,87.0){\usebox{\plotpoint}}
\put(378.0,85.0){\rule[-0.200pt]{0.400pt}{0.482pt}}
\put(378.0,85.0){\usebox{\plotpoint}}
\put(379.0,84.0){\usebox{\plotpoint}}
\put(379.0,84.0){\usebox{\plotpoint}}
\put(380.0,83.0){\usebox{\plotpoint}}
\put(380.0,83.0){\rule[-0.200pt]{0.482pt}{0.400pt}}
\put(382.0,82.0){\usebox{\plotpoint}}
\put(382.0,82.0){\rule[-0.200pt]{0.964pt}{0.400pt}}
\put(386.0,82.0){\usebox{\plotpoint}}
\put(386.0,83.0){\usebox{\plotpoint}}
\put(387.0,83.0){\usebox{\plotpoint}}
\put(387.0,84.0){\usebox{\plotpoint}}
\put(388.0,84.0){\rule[-0.200pt]{0.400pt}{0.482pt}}
\put(388.0,86.0){\usebox{\plotpoint}}
\put(389.0,86.0){\usebox{\plotpoint}}
\put(389.0,87.0){\usebox{\plotpoint}}
\put(390.0,87.0){\rule[-0.200pt]{0.400pt}{0.723pt}}
\put(390.0,90.0){\usebox{\plotpoint}}
\put(391.0,90.0){\rule[-0.200pt]{0.400pt}{0.482pt}}
\put(391.0,92.0){\usebox{\plotpoint}}
\put(392.0,92.0){\rule[-0.200pt]{0.400pt}{0.964pt}}
\put(392.0,96.0){\usebox{\plotpoint}}
\put(393.0,96.0){\rule[-0.200pt]{0.400pt}{0.723pt}}
\put(393.0,99.0){\usebox{\plotpoint}}
\put(394,102.67){\rule{0.241pt}{0.400pt}}
\multiput(394.00,102.17)(0.500,1.000){2}{\rule{0.120pt}{0.400pt}}
\put(394.0,99.0){\rule[-0.200pt]{0.400pt}{0.964pt}}
\put(395,104){\usebox{\plotpoint}}
\put(395,104){\usebox{\plotpoint}}
\put(395,104){\usebox{\plotpoint}}
\put(395,104){\usebox{\plotpoint}}
\put(395,104){\usebox{\plotpoint}}
\put(395,104){\usebox{\plotpoint}}
\put(395,104){\usebox{\plotpoint}}
\put(395,104){\usebox{\plotpoint}}
\put(395,104){\usebox{\plotpoint}}
\put(395,104){\usebox{\plotpoint}}
\put(395,104){\usebox{\plotpoint}}
\put(395,104){\usebox{\plotpoint}}
\put(395,104){\usebox{\plotpoint}}
\put(395,104){\usebox{\plotpoint}}
\put(395,104){\usebox{\plotpoint}}
\put(395,104){\usebox{\plotpoint}}
\put(395.0,104.0){\rule[-0.200pt]{0.400pt}{0.964pt}}
\put(395.0,108.0){\usebox{\plotpoint}}
\put(396.0,108.0){\rule[-0.200pt]{0.400pt}{1.204pt}}
\put(396.0,113.0){\usebox{\plotpoint}}
\put(397.0,113.0){\rule[-0.200pt]{0.400pt}{1.445pt}}
\put(397.0,119.0){\usebox{\plotpoint}}
\put(398,124.67){\rule{0.241pt}{0.400pt}}
\multiput(398.00,124.17)(0.500,1.000){2}{\rule{0.120pt}{0.400pt}}
\put(398.0,119.0){\rule[-0.200pt]{0.400pt}{1.445pt}}
\put(399,126){\usebox{\plotpoint}}
\put(399,126){\usebox{\plotpoint}}
\put(399,126){\usebox{\plotpoint}}
\put(399,126){\usebox{\plotpoint}}
\put(399,126){\usebox{\plotpoint}}
\put(399,126){\usebox{\plotpoint}}
\put(399,126){\usebox{\plotpoint}}
\put(399,126){\usebox{\plotpoint}}
\put(399,126){\usebox{\plotpoint}}
\put(399,126){\usebox{\plotpoint}}
\put(399,126){\usebox{\plotpoint}}
\put(399.0,126.0){\rule[-0.200pt]{0.400pt}{1.445pt}}
\put(399.0,132.0){\usebox{\plotpoint}}
\put(400.0,132.0){\rule[-0.200pt]{0.400pt}{1.927pt}}
\put(400.0,140.0){\usebox{\plotpoint}}
\put(401.0,140.0){\rule[-0.200pt]{0.400pt}{1.927pt}}
\put(401.0,148.0){\usebox{\plotpoint}}
\put(402.0,148.0){\rule[-0.200pt]{0.400pt}{1.927pt}}
\put(402.0,156.0){\usebox{\plotpoint}}
\put(403,164.67){\rule{0.241pt}{0.400pt}}
\multiput(403.00,164.17)(0.500,1.000){2}{\rule{0.120pt}{0.400pt}}
\put(403.0,156.0){\rule[-0.200pt]{0.400pt}{2.168pt}}
\put(404,166){\usebox{\plotpoint}}
\put(404,166){\usebox{\plotpoint}}
\put(404,166){\usebox{\plotpoint}}
\put(404,166){\usebox{\plotpoint}}
\put(404,166){\usebox{\plotpoint}}
\put(404,166){\usebox{\plotpoint}}
\put(404,166){\usebox{\plotpoint}}
\put(404.0,166.0){\rule[-0.200pt]{0.400pt}{2.168pt}}
\put(404.0,175.0){\usebox{\plotpoint}}
\put(405,184.67){\rule{0.241pt}{0.400pt}}
\multiput(405.00,184.17)(0.500,1.000){2}{\rule{0.120pt}{0.400pt}}
\put(405.0,175.0){\rule[-0.200pt]{0.400pt}{2.409pt}}
\put(406,186){\usebox{\plotpoint}}
\put(406,186){\usebox{\plotpoint}}
\put(406,186){\usebox{\plotpoint}}
\put(406,186){\usebox{\plotpoint}}
\put(406,186){\usebox{\plotpoint}}
\put(406,186){\usebox{\plotpoint}}
\put(406,186){\usebox{\plotpoint}}
\put(406.0,186.0){\rule[-0.200pt]{0.400pt}{2.409pt}}
\put(406.0,196.0){\usebox{\plotpoint}}
\put(407.0,196.0){\rule[-0.200pt]{0.400pt}{2.891pt}}
\put(407.0,208.0){\usebox{\plotpoint}}
\put(408,218.67){\rule{0.241pt}{0.400pt}}
\multiput(408.00,218.17)(0.500,1.000){2}{\rule{0.120pt}{0.400pt}}
\put(408.0,208.0){\rule[-0.200pt]{0.400pt}{2.650pt}}
\put(409,220){\usebox{\plotpoint}}
\put(409,220){\usebox{\plotpoint}}
\put(409,220){\usebox{\plotpoint}}
\put(409,220){\usebox{\plotpoint}}
\put(409,220){\usebox{\plotpoint}}
\put(409.0,220.0){\rule[-0.200pt]{0.400pt}{2.891pt}}
\put(409.0,232.0){\usebox{\plotpoint}}
\put(410.0,232.0){\rule[-0.200pt]{0.400pt}{3.132pt}}
\put(410.0,245.0){\usebox{\plotpoint}}
\put(411.0,245.0){\rule[-0.200pt]{0.400pt}{3.132pt}}
\put(411.0,258.0){\usebox{\plotpoint}}
\put(412,270.67){\rule{0.241pt}{0.400pt}}
\multiput(412.00,270.17)(0.500,1.000){2}{\rule{0.120pt}{0.400pt}}
\put(412.0,258.0){\rule[-0.200pt]{0.400pt}{3.132pt}}
\put(413,272){\usebox{\plotpoint}}
\put(413,272){\usebox{\plotpoint}}
\put(413,272){\usebox{\plotpoint}}
\put(413,272){\usebox{\plotpoint}}
\put(413.0,272.0){\rule[-0.200pt]{0.400pt}{3.373pt}}
\put(413.0,286.0){\usebox{\plotpoint}}
\put(414.0,286.0){\rule[-0.200pt]{0.400pt}{3.373pt}}
\put(414.0,300.0){\usebox{\plotpoint}}
\put(415.0,300.0){\rule[-0.200pt]{0.400pt}{3.613pt}}
\put(415.0,315.0){\usebox{\plotpoint}}
\put(416.0,315.0){\rule[-0.200pt]{0.400pt}{3.613pt}}
\put(416.0,330.0){\usebox{\plotpoint}}
\put(417.0,330.0){\rule[-0.200pt]{0.400pt}{3.613pt}}
\put(417.0,345.0){\usebox{\plotpoint}}
\put(418.0,345.0){\rule[-0.200pt]{0.400pt}{3.854pt}}
\put(418.0,361.0){\usebox{\plotpoint}}
\put(419,375.67){\rule{0.241pt}{0.400pt}}
\multiput(419.00,375.17)(0.500,1.000){2}{\rule{0.120pt}{0.400pt}}
\put(419.0,361.0){\rule[-0.200pt]{0.400pt}{3.613pt}}
\put(420,377){\usebox{\plotpoint}}
\put(420,377){\usebox{\plotpoint}}
\put(420,377){\usebox{\plotpoint}}
\put(420,377){\usebox{\plotpoint}}
\put(420.0,377.0){\rule[-0.200pt]{0.400pt}{3.613pt}}
\put(420.0,392.0){\usebox{\plotpoint}}
\put(421.0,392.0){\rule[-0.200pt]{0.400pt}{3.854pt}}
\put(421.0,408.0){\usebox{\plotpoint}}
\put(422.0,408.0){\rule[-0.200pt]{0.400pt}{3.854pt}}
\put(422.0,424.0){\usebox{\plotpoint}}
\put(423.0,424.0){\rule[-0.200pt]{0.400pt}{3.854pt}}
\put(423.0,440.0){\usebox{\plotpoint}}
\put(424.0,440.0){\rule[-0.200pt]{0.400pt}{3.854pt}}
\put(424.0,456.0){\usebox{\plotpoint}}
\put(425.0,456.0){\rule[-0.200pt]{0.400pt}{3.854pt}}
\put(425.0,472.0){\usebox{\plotpoint}}
\put(426,486.67){\rule{0.241pt}{0.400pt}}
\multiput(426.00,486.17)(0.500,1.000){2}{\rule{0.120pt}{0.400pt}}
\put(426.0,472.0){\rule[-0.200pt]{0.400pt}{3.613pt}}
\put(427,488){\usebox{\plotpoint}}
\put(427,488){\usebox{\plotpoint}}
\put(427,488){\usebox{\plotpoint}}
\put(427,488){\usebox{\plotpoint}}
\put(427.0,488.0){\rule[-0.200pt]{0.400pt}{3.613pt}}
\put(427.0,503.0){\usebox{\plotpoint}}
\put(428.0,503.0){\rule[-0.200pt]{0.400pt}{3.613pt}}
\put(428.0,518.0){\usebox{\plotpoint}}
\put(429.0,518.0){\rule[-0.200pt]{0.400pt}{3.613pt}}
\put(429.0,533.0){\usebox{\plotpoint}}
\put(430.0,533.0){\rule[-0.200pt]{0.400pt}{3.373pt}}
\put(430.0,547.0){\usebox{\plotpoint}}
\put(431,560.67){\rule{0.241pt}{0.400pt}}
\multiput(431.00,560.17)(0.500,1.000){2}{\rule{0.120pt}{0.400pt}}
\put(431.0,547.0){\rule[-0.200pt]{0.400pt}{3.373pt}}
\put(432,562){\usebox{\plotpoint}}
\put(432,562){\usebox{\plotpoint}}
\put(432,562){\usebox{\plotpoint}}
\put(432,562){\usebox{\plotpoint}}
\put(432.0,562.0){\rule[-0.200pt]{0.400pt}{3.132pt}}
\put(432.0,575.0){\usebox{\plotpoint}}
\put(433.0,575.0){\rule[-0.200pt]{0.400pt}{3.132pt}}
\put(433.0,588.0){\usebox{\plotpoint}}
\put(434.0,588.0){\rule[-0.200pt]{0.400pt}{3.132pt}}
\put(434.0,601.0){\usebox{\plotpoint}}
\put(435.0,601.0){\rule[-0.200pt]{0.400pt}{2.891pt}}
\put(435.0,613.0){\usebox{\plotpoint}}
\put(436.0,613.0){\rule[-0.200pt]{0.400pt}{2.891pt}}
\put(436.0,625.0){\usebox{\plotpoint}}
\put(437,634.67){\rule{0.241pt}{0.400pt}}
\multiput(437.00,634.17)(0.500,1.000){2}{\rule{0.120pt}{0.400pt}}
\put(437.0,625.0){\rule[-0.200pt]{0.400pt}{2.409pt}}
\put(438,636){\usebox{\plotpoint}}
\put(438,636){\usebox{\plotpoint}}
\put(438,636){\usebox{\plotpoint}}
\put(438,636){\usebox{\plotpoint}}
\put(438,636){\usebox{\plotpoint}}
\put(438,636){\usebox{\plotpoint}}
\put(438,644.67){\rule{0.241pt}{0.400pt}}
\multiput(438.00,644.17)(0.500,1.000){2}{\rule{0.120pt}{0.400pt}}
\put(438.0,636.0){\rule[-0.200pt]{0.400pt}{2.168pt}}
\put(439,646){\usebox{\plotpoint}}
\put(439,646){\usebox{\plotpoint}}
\put(439,646){\usebox{\plotpoint}}
\put(439,646){\usebox{\plotpoint}}
\put(439,646){\usebox{\plotpoint}}
\put(439,646){\usebox{\plotpoint}}
\put(439,646){\usebox{\plotpoint}}
\put(439.0,646.0){\rule[-0.200pt]{0.400pt}{2.168pt}}
\put(439.0,655.0){\usebox{\plotpoint}}
\put(440,662.67){\rule{0.241pt}{0.400pt}}
\multiput(440.00,662.17)(0.500,1.000){2}{\rule{0.120pt}{0.400pt}}
\put(440.0,655.0){\rule[-0.200pt]{0.400pt}{1.927pt}}
\put(441,664){\usebox{\plotpoint}}
\put(441,664){\usebox{\plotpoint}}
\put(441,664){\usebox{\plotpoint}}
\put(441,664){\usebox{\plotpoint}}
\put(441,664){\usebox{\plotpoint}}
\put(441,664){\usebox{\plotpoint}}
\put(441,664){\usebox{\plotpoint}}
\put(441,664){\usebox{\plotpoint}}
\put(441.0,664.0){\rule[-0.200pt]{0.400pt}{1.686pt}}
\put(441.0,671.0){\usebox{\plotpoint}}
\put(442.0,671.0){\rule[-0.200pt]{0.400pt}{1.686pt}}
\put(442.0,678.0){\usebox{\plotpoint}}
\put(443.0,678.0){\rule[-0.200pt]{0.400pt}{1.445pt}}
\put(443.0,684.0){\usebox{\plotpoint}}
\put(444.0,684.0){\rule[-0.200pt]{0.400pt}{1.445pt}}
\put(444.0,690.0){\usebox{\plotpoint}}
\put(445.0,690.0){\rule[-0.200pt]{0.400pt}{0.964pt}}
\put(445.0,694.0){\usebox{\plotpoint}}
\put(446.0,694.0){\rule[-0.200pt]{0.400pt}{0.964pt}}
\put(446.0,698.0){\usebox{\plotpoint}}
\put(447.0,698.0){\rule[-0.200pt]{0.400pt}{0.482pt}}
\put(447.0,700.0){\usebox{\plotpoint}}
\put(448.0,700.0){\rule[-0.200pt]{0.400pt}{0.482pt}}
\put(448.0,702.0){\usebox{\plotpoint}}
\put(449.0,702.0){\usebox{\plotpoint}}
\put(449.0,703.0){\usebox{\plotpoint}}
\put(450.0,702.0){\usebox{\plotpoint}}
\put(450.0,702.0){\usebox{\plotpoint}}
\put(451.0,701.0){\usebox{\plotpoint}}
\put(451.0,701.0){\usebox{\plotpoint}}
\put(452.0,699.0){\rule[-0.200pt]{0.400pt}{0.482pt}}
\put(452.0,699.0){\usebox{\plotpoint}}
\put(453.0,696.0){\rule[-0.200pt]{0.400pt}{0.723pt}}
\put(453.0,696.0){\usebox{\plotpoint}}
\put(454.0,692.0){\rule[-0.200pt]{0.400pt}{0.964pt}}
\put(454.0,692.0){\usebox{\plotpoint}}
\put(455.0,687.0){\rule[-0.200pt]{0.400pt}{1.204pt}}
\put(455.0,687.0){\usebox{\plotpoint}}
\put(456,680.67){\rule{0.241pt}{0.400pt}}
\multiput(456.00,681.17)(0.500,-1.000){2}{\rule{0.120pt}{0.400pt}}
\put(456.0,682.0){\rule[-0.200pt]{0.400pt}{1.204pt}}
\put(457,681){\usebox{\plotpoint}}
\put(457,681){\usebox{\plotpoint}}
\put(457,681){\usebox{\plotpoint}}
\put(457,681){\usebox{\plotpoint}}
\put(457,681){\usebox{\plotpoint}}
\put(457,681){\usebox{\plotpoint}}
\put(457,681){\usebox{\plotpoint}}
\put(457,681){\usebox{\plotpoint}}
\put(457,681){\usebox{\plotpoint}}
\put(457,681){\usebox{\plotpoint}}
\put(457,681){\usebox{\plotpoint}}
\put(457.0,675.0){\rule[-0.200pt]{0.400pt}{1.445pt}}
\put(457.0,675.0){\usebox{\plotpoint}}
\put(458,666.67){\rule{0.241pt}{0.400pt}}
\multiput(458.00,667.17)(0.500,-1.000){2}{\rule{0.120pt}{0.400pt}}
\put(458.0,668.0){\rule[-0.200pt]{0.400pt}{1.686pt}}
\put(459,667){\usebox{\plotpoint}}
\put(459,667){\usebox{\plotpoint}}
\put(459,667){\usebox{\plotpoint}}
\put(459,667){\usebox{\plotpoint}}
\put(459,667){\usebox{\plotpoint}}
\put(459,667){\usebox{\plotpoint}}
\put(459,667){\usebox{\plotpoint}}
\put(459,667){\usebox{\plotpoint}}
\put(459,667){\usebox{\plotpoint}}
\put(459.0,659.0){\rule[-0.200pt]{0.400pt}{1.927pt}}
\put(459.0,659.0){\usebox{\plotpoint}}
\put(460.0,650.0){\rule[-0.200pt]{0.400pt}{2.168pt}}
\put(460.0,650.0){\usebox{\plotpoint}}
\put(461.0,641.0){\rule[-0.200pt]{0.400pt}{2.168pt}}
\put(461.0,641.0){\usebox{\plotpoint}}
\put(462.0,630.0){\rule[-0.200pt]{0.400pt}{2.650pt}}
\put(462.0,630.0){\usebox{\plotpoint}}
\put(463.0,619.0){\rule[-0.200pt]{0.400pt}{2.650pt}}
\put(463.0,619.0){\usebox{\plotpoint}}
\put(464.0,607.0){\rule[-0.200pt]{0.400pt}{2.891pt}}
\put(464.0,607.0){\usebox{\plotpoint}}
\put(465.0,595.0){\rule[-0.200pt]{0.400pt}{2.891pt}}
\put(465.0,595.0){\usebox{\plotpoint}}
\put(466.0,582.0){\rule[-0.200pt]{0.400pt}{3.132pt}}
\put(466.0,582.0){\usebox{\plotpoint}}
\put(467,567.67){\rule{0.241pt}{0.400pt}}
\multiput(467.00,568.17)(0.500,-1.000){2}{\rule{0.120pt}{0.400pt}}
\put(467.0,569.0){\rule[-0.200pt]{0.400pt}{3.132pt}}
\put(468,568){\usebox{\plotpoint}}
\put(468,568){\usebox{\plotpoint}}
\put(468,568){\usebox{\plotpoint}}
\put(468,568){\usebox{\plotpoint}}
\put(468,553.67){\rule{0.241pt}{0.400pt}}
\multiput(468.00,554.17)(0.500,-1.000){2}{\rule{0.120pt}{0.400pt}}
\put(468.0,555.0){\rule[-0.200pt]{0.400pt}{3.132pt}}
\put(469,554){\usebox{\plotpoint}}
\put(469,554){\usebox{\plotpoint}}
\put(469,554){\usebox{\plotpoint}}
\put(469,554){\usebox{\plotpoint}}
\put(469.0,540.0){\rule[-0.200pt]{0.400pt}{3.373pt}}
\put(469.0,540.0){\usebox{\plotpoint}}
\put(470,524.67){\rule{0.241pt}{0.400pt}}
\multiput(470.00,525.17)(0.500,-1.000){2}{\rule{0.120pt}{0.400pt}}
\put(470.0,526.0){\rule[-0.200pt]{0.400pt}{3.373pt}}
\put(471,525){\usebox{\plotpoint}}
\put(471,525){\usebox{\plotpoint}}
\put(471,525){\usebox{\plotpoint}}
\put(471,525){\usebox{\plotpoint}}
\put(471,509.67){\rule{0.241pt}{0.400pt}}
\multiput(471.00,510.17)(0.500,-1.000){2}{\rule{0.120pt}{0.400pt}}
\put(471.0,511.0){\rule[-0.200pt]{0.400pt}{3.373pt}}
\put(472,510){\usebox{\plotpoint}}
\put(472,510){\usebox{\plotpoint}}
\put(472,510){\usebox{\plotpoint}}
\put(472,510){\usebox{\plotpoint}}
\put(472.0,495.0){\rule[-0.200pt]{0.400pt}{3.613pt}}
\put(472.0,495.0){\usebox{\plotpoint}}
\put(473,478.67){\rule{0.241pt}{0.400pt}}
\multiput(473.00,479.17)(0.500,-1.000){2}{\rule{0.120pt}{0.400pt}}
\put(473.0,480.0){\rule[-0.200pt]{0.400pt}{3.613pt}}
\put(474,479){\usebox{\plotpoint}}
\put(474,479){\usebox{\plotpoint}}
\put(474,479){\usebox{\plotpoint}}
\put(474,479){\usebox{\plotpoint}}
\put(474.0,464.0){\rule[-0.200pt]{0.400pt}{3.613pt}}
\put(474.0,464.0){\usebox{\plotpoint}}
\put(475.0,448.0){\rule[-0.200pt]{0.400pt}{3.854pt}}
\put(475.0,448.0){\usebox{\plotpoint}}
\put(476.0,432.0){\rule[-0.200pt]{0.400pt}{3.854pt}}
\put(476.0,432.0){\usebox{\plotpoint}}
\put(477.0,416.0){\rule[-0.200pt]{0.400pt}{3.854pt}}
\put(477.0,416.0){\usebox{\plotpoint}}
\put(478.0,400.0){\rule[-0.200pt]{0.400pt}{3.854pt}}
\put(478.0,400.0){\usebox{\plotpoint}}
\put(479.0,384.0){\rule[-0.200pt]{0.400pt}{3.854pt}}
\put(479.0,384.0){\usebox{\plotpoint}}
\put(480,367.67){\rule{0.241pt}{0.400pt}}
\multiput(480.00,368.17)(0.500,-1.000){2}{\rule{0.120pt}{0.400pt}}
\put(480.0,369.0){\rule[-0.200pt]{0.400pt}{3.613pt}}
\put(481,368){\usebox{\plotpoint}}
\put(481,368){\usebox{\plotpoint}}
\put(481,368){\usebox{\plotpoint}}
\put(481,368){\usebox{\plotpoint}}
\put(481.0,353.0){\rule[-0.200pt]{0.400pt}{3.613pt}}
\put(481.0,353.0){\usebox{\plotpoint}}
\put(482.0,338.0){\rule[-0.200pt]{0.400pt}{3.613pt}}
\put(482.0,338.0){\usebox{\plotpoint}}
\put(483,321.67){\rule{0.241pt}{0.400pt}}
\multiput(483.00,322.17)(0.500,-1.000){2}{\rule{0.120pt}{0.400pt}}
\put(483.0,323.0){\rule[-0.200pt]{0.400pt}{3.613pt}}
\put(484,322){\usebox{\plotpoint}}
\put(484,322){\usebox{\plotpoint}}
\put(484,322){\usebox{\plotpoint}}
\put(484,322){\usebox{\plotpoint}}
\put(484,306.67){\rule{0.241pt}{0.400pt}}
\multiput(484.00,307.17)(0.500,-1.000){2}{\rule{0.120pt}{0.400pt}}
\put(484.0,308.0){\rule[-0.200pt]{0.400pt}{3.373pt}}
\put(485,307){\usebox{\plotpoint}}
\put(485,307){\usebox{\plotpoint}}
\put(485,307){\usebox{\plotpoint}}
\put(485,307){\usebox{\plotpoint}}
\put(485.0,293.0){\rule[-0.200pt]{0.400pt}{3.373pt}}
\put(485.0,293.0){\usebox{\plotpoint}}
\put(486.0,279.0){\rule[-0.200pt]{0.400pt}{3.373pt}}
\put(486.0,279.0){\usebox{\plotpoint}}
\put(487.0,265.0){\rule[-0.200pt]{0.400pt}{3.373pt}}
\put(487.0,265.0){\usebox{\plotpoint}}
\put(488.0,251.0){\rule[-0.200pt]{0.400pt}{3.373pt}}
\put(488.0,251.0){\usebox{\plotpoint}}
\put(489.0,238.0){\rule[-0.200pt]{0.400pt}{3.132pt}}
\put(489.0,238.0){\usebox{\plotpoint}}
\put(490.0,226.0){\rule[-0.200pt]{0.400pt}{2.891pt}}
\put(490.0,226.0){\usebox{\plotpoint}}
\put(491.0,214.0){\rule[-0.200pt]{0.400pt}{2.891pt}}
\put(491.0,214.0){\usebox{\plotpoint}}
\put(492.0,202.0){\rule[-0.200pt]{0.400pt}{2.891pt}}
\put(492.0,202.0){\usebox{\plotpoint}}
\put(493.0,191.0){\rule[-0.200pt]{0.400pt}{2.650pt}}
\put(493.0,191.0){\usebox{\plotpoint}}
\put(494.0,180.0){\rule[-0.200pt]{0.400pt}{2.650pt}}
\put(494.0,180.0){\usebox{\plotpoint}}
\put(495.0,170.0){\rule[-0.200pt]{0.400pt}{2.409pt}}
\put(495.0,170.0){\usebox{\plotpoint}}
\put(496.0,161.0){\rule[-0.200pt]{0.400pt}{2.168pt}}
\put(496.0,161.0){\usebox{\plotpoint}}
\put(497.0,152.0){\rule[-0.200pt]{0.400pt}{2.168pt}}
\put(497.0,152.0){\usebox{\plotpoint}}
\put(498.0,144.0){\rule[-0.200pt]{0.400pt}{1.927pt}}
\put(498.0,144.0){\usebox{\plotpoint}}
\put(499.0,136.0){\rule[-0.200pt]{0.400pt}{1.927pt}}
\put(499.0,136.0){\usebox{\plotpoint}}
\put(500.0,129.0){\rule[-0.200pt]{0.400pt}{1.686pt}}
\put(500.0,129.0){\usebox{\plotpoint}}
\put(501.0,122.0){\rule[-0.200pt]{0.400pt}{1.686pt}}
\put(501.0,122.0){\usebox{\plotpoint}}
\put(502.0,116.0){\rule[-0.200pt]{0.400pt}{1.445pt}}
\put(502.0,116.0){\usebox{\plotpoint}}
\put(503.0,111.0){\rule[-0.200pt]{0.400pt}{1.204pt}}
\put(503.0,111.0){\usebox{\plotpoint}}
\put(504.0,106.0){\rule[-0.200pt]{0.400pt}{1.204pt}}
\put(504.0,106.0){\usebox{\plotpoint}}
\put(505.0,101.0){\rule[-0.200pt]{0.400pt}{1.204pt}}
\put(505.0,101.0){\usebox{\plotpoint}}
\put(506.0,97.0){\rule[-0.200pt]{0.400pt}{0.964pt}}
\put(506.0,97.0){\usebox{\plotpoint}}
\put(507.0,94.0){\rule[-0.200pt]{0.400pt}{0.723pt}}
\put(507.0,94.0){\usebox{\plotpoint}}
\put(508.0,91.0){\rule[-0.200pt]{0.400pt}{0.723pt}}
\put(508.0,91.0){\usebox{\plotpoint}}
\put(509.0,89.0){\rule[-0.200pt]{0.400pt}{0.482pt}}
\put(509.0,89.0){\usebox{\plotpoint}}
\put(510,85.67){\rule{0.241pt}{0.400pt}}
\multiput(510.00,86.17)(0.500,-1.000){2}{\rule{0.120pt}{0.400pt}}
\put(510.0,87.0){\rule[-0.200pt]{0.400pt}{0.482pt}}
\put(511,86){\usebox{\plotpoint}}
\put(511,86){\usebox{\plotpoint}}
\put(511,86){\usebox{\plotpoint}}
\put(511,86){\usebox{\plotpoint}}
\put(511,86){\usebox{\plotpoint}}
\put(511,86){\usebox{\plotpoint}}
\put(511,86){\usebox{\plotpoint}}
\put(511,86){\usebox{\plotpoint}}
\put(511,86){\usebox{\plotpoint}}
\put(511,86){\usebox{\plotpoint}}
\put(511,86){\usebox{\plotpoint}}
\put(511,86){\usebox{\plotpoint}}
\put(511,86){\usebox{\plotpoint}}
\put(511,86){\usebox{\plotpoint}}
\put(511,86){\usebox{\plotpoint}}
\put(511,86){\usebox{\plotpoint}}
\put(511,86){\usebox{\plotpoint}}
\put(511,86){\usebox{\plotpoint}}
\put(511,86){\usebox{\plotpoint}}
\put(511,86){\usebox{\plotpoint}}
\put(511,86){\usebox{\plotpoint}}
\put(511,86){\usebox{\plotpoint}}
\put(511,86){\usebox{\plotpoint}}
\put(511,86){\usebox{\plotpoint}}
\put(511,86){\usebox{\plotpoint}}
\put(511,86){\usebox{\plotpoint}}
\put(511,86){\usebox{\plotpoint}}
\put(511,86){\usebox{\plotpoint}}
\put(511,86){\usebox{\plotpoint}}
\put(511,86){\usebox{\plotpoint}}
\put(511,86){\usebox{\plotpoint}}
\put(511,86){\usebox{\plotpoint}}
\put(511,86){\usebox{\plotpoint}}
\put(511,86){\usebox{\plotpoint}}
\put(511,86){\usebox{\plotpoint}}
\put(511,86){\usebox{\plotpoint}}
\put(511,86){\usebox{\plotpoint}}
\put(511,86){\usebox{\plotpoint}}
\put(511,86){\usebox{\plotpoint}}
\put(511,86){\usebox{\plotpoint}}
\put(511,86){\usebox{\plotpoint}}
\put(511,86){\usebox{\plotpoint}}
\put(511,86){\usebox{\plotpoint}}
\put(511.0,85.0){\usebox{\plotpoint}}
\put(511.0,85.0){\usebox{\plotpoint}}
\put(512.0,84.0){\usebox{\plotpoint}}
\put(512.0,84.0){\usebox{\plotpoint}}
\put(513.0,83.0){\usebox{\plotpoint}}
\put(513.0,83.0){\usebox{\plotpoint}}
\put(514.0,82.0){\usebox{\plotpoint}}
\put(514.0,82.0){\rule[-0.200pt]{0.964pt}{0.400pt}}
\put(518.0,82.0){\usebox{\plotpoint}}
\put(518.0,83.0){\usebox{\plotpoint}}
\put(519.0,83.0){\usebox{\plotpoint}}
\put(519.0,84.0){\usebox{\plotpoint}}
\put(520.0,84.0){\usebox{\plotpoint}}
\put(520.0,85.0){\usebox{\plotpoint}}
\put(521.0,85.0){\usebox{\plotpoint}}
\put(521.0,86.0){\usebox{\plotpoint}}
\put(522.0,86.0){\usebox{\plotpoint}}
\put(522.0,87.0){\usebox{\plotpoint}}
\put(523.0,87.0){\rule[-0.200pt]{0.400pt}{0.482pt}}
\put(523.0,89.0){\usebox{\plotpoint}}
\put(524.0,89.0){\usebox{\plotpoint}}
\put(524.0,90.0){\usebox{\plotpoint}}
\put(525.0,90.0){\rule[-0.200pt]{0.400pt}{0.482pt}}
\put(525.0,92.0){\usebox{\plotpoint}}
\put(526.0,92.0){\usebox{\plotpoint}}
\put(526.0,93.0){\usebox{\plotpoint}}
\put(527.0,93.0){\rule[-0.200pt]{0.400pt}{0.482pt}}
\put(527.0,95.0){\usebox{\plotpoint}}
\put(528.0,95.0){\rule[-0.200pt]{0.400pt}{0.482pt}}
\put(528.0,97.0){\usebox{\plotpoint}}
\put(529.0,97.0){\usebox{\plotpoint}}
\put(529.0,98.0){\usebox{\plotpoint}}
\put(530.0,98.0){\rule[-0.200pt]{0.400pt}{0.482pt}}
\put(530.0,100.0){\usebox{\plotpoint}}
\put(531,100.67){\rule{0.241pt}{0.400pt}}
\multiput(531.00,100.17)(0.500,1.000){2}{\rule{0.120pt}{0.400pt}}
\put(531.0,100.0){\usebox{\plotpoint}}
\put(532,102){\usebox{\plotpoint}}
\put(532,102){\usebox{\plotpoint}}
\put(532,102){\usebox{\plotpoint}}
\put(532,102){\usebox{\plotpoint}}
\put(532,102){\usebox{\plotpoint}}
\put(532,102){\usebox{\plotpoint}}
\put(532,102){\usebox{\plotpoint}}
\put(532,102){\usebox{\plotpoint}}
\put(532,102){\usebox{\plotpoint}}
\put(532,102){\usebox{\plotpoint}}
\put(532,102){\usebox{\plotpoint}}
\put(532,102){\usebox{\plotpoint}}
\put(532,102){\usebox{\plotpoint}}
\put(532,102){\usebox{\plotpoint}}
\put(532,102){\usebox{\plotpoint}}
\put(532,102){\usebox{\plotpoint}}
\put(532,102){\usebox{\plotpoint}}
\put(532,102){\usebox{\plotpoint}}
\put(532,102){\usebox{\plotpoint}}
\put(532,102){\usebox{\plotpoint}}
\put(532,102){\usebox{\plotpoint}}
\put(532,102){\usebox{\plotpoint}}
\put(532,102){\usebox{\plotpoint}}
\put(532,102){\usebox{\plotpoint}}
\put(532,102){\usebox{\plotpoint}}
\put(532,102){\usebox{\plotpoint}}
\put(532,102){\usebox{\plotpoint}}
\put(532,102){\usebox{\plotpoint}}
\put(532,102){\usebox{\plotpoint}}
\put(532,102){\usebox{\plotpoint}}
\put(532,102){\usebox{\plotpoint}}
\put(532,102){\usebox{\plotpoint}}
\put(532,102){\usebox{\plotpoint}}
\put(532,102){\usebox{\plotpoint}}
\put(532,102){\usebox{\plotpoint}}
\put(532,102){\usebox{\plotpoint}}
\put(532,102){\usebox{\plotpoint}}
\put(532,102){\usebox{\plotpoint}}
\put(532,102){\usebox{\plotpoint}}
\put(532,102){\usebox{\plotpoint}}
\put(532,102){\usebox{\plotpoint}}
\put(532,102){\usebox{\plotpoint}}
\put(532,102){\usebox{\plotpoint}}
\put(532,102){\usebox{\plotpoint}}
\put(532,102){\usebox{\plotpoint}}
\put(532,102){\usebox{\plotpoint}}
\put(532,102){\usebox{\plotpoint}}
\put(532,102){\usebox{\plotpoint}}
\put(532,102){\usebox{\plotpoint}}
\put(532,102){\usebox{\plotpoint}}
\put(532.0,102.0){\usebox{\plotpoint}}
\put(532.0,103.0){\usebox{\plotpoint}}
\put(533.0,103.0){\usebox{\plotpoint}}
\put(533.0,104.0){\usebox{\plotpoint}}
\put(534.0,104.0){\rule[-0.200pt]{0.400pt}{0.482pt}}
\put(534.0,106.0){\usebox{\plotpoint}}
\put(535.0,106.0){\usebox{\plotpoint}}
\put(535.0,107.0){\usebox{\plotpoint}}
\put(536.0,107.0){\usebox{\plotpoint}}
\put(536.0,108.0){\usebox{\plotpoint}}
\put(537.0,108.0){\usebox{\plotpoint}}
\put(537.0,109.0){\usebox{\plotpoint}}
\put(538.0,109.0){\usebox{\plotpoint}}
\put(538.0,110.0){\usebox{\plotpoint}}
\put(539.0,110.0){\usebox{\plotpoint}}
\put(539.0,111.0){\rule[-0.200pt]{0.482pt}{0.400pt}}
\put(541.0,111.0){\usebox{\plotpoint}}
\put(541.0,112.0){\rule[-0.200pt]{1.445pt}{0.400pt}}
\put(547.0,111.0){\usebox{\plotpoint}}
\put(547.0,111.0){\rule[-0.200pt]{0.482pt}{0.400pt}}
\put(549.0,110.0){\usebox{\plotpoint}}
\put(549.0,110.0){\rule[-0.200pt]{0.482pt}{0.400pt}}
\put(551.0,109.0){\usebox{\plotpoint}}
\put(551.0,109.0){\usebox{\plotpoint}}
\put(552.0,108.0){\usebox{\plotpoint}}
\put(552.0,108.0){\usebox{\plotpoint}}
\put(553.0,107.0){\usebox{\plotpoint}}
\put(553.0,107.0){\usebox{\plotpoint}}
\put(554.0,106.0){\usebox{\plotpoint}}
\put(554.0,106.0){\usebox{\plotpoint}}
\put(555.0,105.0){\usebox{\plotpoint}}
\put(555.0,105.0){\usebox{\plotpoint}}
\put(556.0,104.0){\usebox{\plotpoint}}
\put(556.0,104.0){\usebox{\plotpoint}}
\put(557.0,102.0){\rule[-0.200pt]{0.400pt}{0.482pt}}
\put(557.0,102.0){\usebox{\plotpoint}}
\put(558.0,101.0){\usebox{\plotpoint}}
\put(558.0,101.0){\usebox{\plotpoint}}
\put(559.0,100.0){\usebox{\plotpoint}}
\put(559.0,100.0){\usebox{\plotpoint}}
\put(560.0,99.0){\usebox{\plotpoint}}
\put(560.0,99.0){\usebox{\plotpoint}}
\put(561.0,97.0){\rule[-0.200pt]{0.400pt}{0.482pt}}
\put(561.0,97.0){\usebox{\plotpoint}}
\put(562.0,96.0){\usebox{\plotpoint}}
\put(562.0,96.0){\usebox{\plotpoint}}
\put(563.0,95.0){\usebox{\plotpoint}}
\put(563.0,95.0){\usebox{\plotpoint}}
\put(564.0,94.0){\usebox{\plotpoint}}
\put(564.0,94.0){\usebox{\plotpoint}}
\put(565.0,92.0){\rule[-0.200pt]{0.400pt}{0.482pt}}
\put(565.0,92.0){\usebox{\plotpoint}}
\put(566.0,91.0){\usebox{\plotpoint}}
\put(566.0,91.0){\usebox{\plotpoint}}
\put(567.0,90.0){\usebox{\plotpoint}}
\put(567.0,90.0){\usebox{\plotpoint}}
\put(568.0,89.0){\usebox{\plotpoint}}
\put(568.0,89.0){\usebox{\plotpoint}}
\put(569.0,88.0){\usebox{\plotpoint}}
\put(569.0,88.0){\usebox{\plotpoint}}
\put(570.0,87.0){\usebox{\plotpoint}}
\put(570.0,87.0){\usebox{\plotpoint}}
\put(571.0,86.0){\usebox{\plotpoint}}
\put(571.0,86.0){\usebox{\plotpoint}}
\put(572.0,85.0){\usebox{\plotpoint}}
\put(572.0,85.0){\rule[-0.200pt]{0.482pt}{0.400pt}}
\put(574.0,84.0){\usebox{\plotpoint}}
\put(574.0,84.0){\rule[-0.200pt]{0.482pt}{0.400pt}}
\put(576.0,83.0){\usebox{\plotpoint}}
\put(576.0,83.0){\rule[-0.200pt]{0.482pt}{0.400pt}}
\put(578.0,82.0){\usebox{\plotpoint}}
\put(578.0,82.0){\rule[-0.200pt]{1.686pt}{0.400pt}}
\put(585.0,82.0){\usebox{\plotpoint}}
\put(585.0,83.0){\rule[-0.200pt]{0.723pt}{0.400pt}}
\put(588.0,83.0){\usebox{\plotpoint}}
\put(588.0,84.0){\rule[-0.200pt]{0.723pt}{0.400pt}}
\put(591.0,84.0){\usebox{\plotpoint}}
\put(591.0,85.0){\usebox{\plotpoint}}
\put(592.0,85.0){\usebox{\plotpoint}}
\put(592.0,86.0){\rule[-0.200pt]{0.482pt}{0.400pt}}
\put(594.0,86.0){\usebox{\plotpoint}}
\put(594.0,87.0){\rule[-0.200pt]{0.482pt}{0.400pt}}
\put(596.0,87.0){\usebox{\plotpoint}}
\put(596.0,88.0){\rule[-0.200pt]{0.482pt}{0.400pt}}
\put(598.0,88.0){\usebox{\plotpoint}}
\put(598.0,89.0){\rule[-0.200pt]{0.482pt}{0.400pt}}
\put(600.0,89.0){\usebox{\plotpoint}}
\put(600.0,90.0){\usebox{\plotpoint}}
\put(601.0,90.0){\usebox{\plotpoint}}
\put(601.0,91.0){\rule[-0.200pt]{0.723pt}{0.400pt}}
\put(604.0,91.0){\usebox{\plotpoint}}
\put(604.0,92.0){\rule[-0.200pt]{0.723pt}{0.400pt}}
\put(607.0,92.0){\usebox{\plotpoint}}
\put(607.0,93.0){\rule[-0.200pt]{2.650pt}{0.400pt}}
\put(618.0,92.0){\usebox{\plotpoint}}
\put(618.0,92.0){\rule[-0.200pt]{0.723pt}{0.400pt}}
\put(621.0,91.0){\usebox{\plotpoint}}
\put(623,89.67){\rule{0.241pt}{0.400pt}}
\multiput(623.00,90.17)(0.500,-1.000){2}{\rule{0.120pt}{0.400pt}}
\put(621.0,91.0){\rule[-0.200pt]{0.482pt}{0.400pt}}
\put(624,90){\usebox{\plotpoint}}
\put(624,90){\usebox{\plotpoint}}
\put(624,90){\usebox{\plotpoint}}
\put(624,90){\usebox{\plotpoint}}
\put(624,90){\usebox{\plotpoint}}
\put(624,90){\usebox{\plotpoint}}
\put(624,90){\usebox{\plotpoint}}
\put(624,90){\usebox{\plotpoint}}
\put(624,90){\usebox{\plotpoint}}
\put(624,90){\usebox{\plotpoint}}
\put(624,90){\usebox{\plotpoint}}
\put(624,90){\usebox{\plotpoint}}
\put(624,90){\usebox{\plotpoint}}
\put(624,90){\usebox{\plotpoint}}
\put(624,90){\usebox{\plotpoint}}
\put(624,90){\usebox{\plotpoint}}
\put(624,90){\usebox{\plotpoint}}
\put(624,90){\usebox{\plotpoint}}
\put(624,90){\usebox{\plotpoint}}
\put(624,90){\usebox{\plotpoint}}
\put(624,90){\usebox{\plotpoint}}
\put(624,90){\usebox{\plotpoint}}
\put(624,90){\usebox{\plotpoint}}
\put(624,90){\usebox{\plotpoint}}
\put(624,90){\usebox{\plotpoint}}
\put(624,90){\usebox{\plotpoint}}
\put(624,90){\usebox{\plotpoint}}
\put(624,90){\usebox{\plotpoint}}
\put(624,90){\usebox{\plotpoint}}
\put(624,90){\usebox{\plotpoint}}
\put(624,90){\usebox{\plotpoint}}
\put(624,90){\usebox{\plotpoint}}
\put(624,90){\usebox{\plotpoint}}
\put(624,90){\usebox{\plotpoint}}
\put(624,90){\usebox{\plotpoint}}
\put(624,90){\usebox{\plotpoint}}
\put(624,90){\usebox{\plotpoint}}
\put(624,90){\usebox{\plotpoint}}
\put(624,90){\usebox{\plotpoint}}
\put(624,90){\usebox{\plotpoint}}
\put(624,90){\usebox{\plotpoint}}
\put(624,90){\usebox{\plotpoint}}
\put(624,90){\usebox{\plotpoint}}
\put(624,90){\usebox{\plotpoint}}
\put(624,90){\usebox{\plotpoint}}
\put(624,90){\usebox{\plotpoint}}
\put(624,90){\usebox{\plotpoint}}
\put(624,90){\usebox{\plotpoint}}
\put(624,90){\usebox{\plotpoint}}
\put(624,90){\usebox{\plotpoint}}
\put(624,90){\usebox{\plotpoint}}
\put(624,90){\usebox{\plotpoint}}
\put(624,90){\usebox{\plotpoint}}
\put(624,90){\usebox{\plotpoint}}
\put(624,90){\usebox{\plotpoint}}
\put(624,90){\usebox{\plotpoint}}
\put(624,90){\usebox{\plotpoint}}
\put(624,90){\usebox{\plotpoint}}
\put(624,90){\usebox{\plotpoint}}
\put(624,90){\usebox{\plotpoint}}
\put(624,90){\usebox{\plotpoint}}
\put(624,90){\usebox{\plotpoint}}
\put(624,90){\usebox{\plotpoint}}
\put(624,90){\usebox{\plotpoint}}
\put(624,90){\usebox{\plotpoint}}
\put(624,90){\usebox{\plotpoint}}
\put(624,90){\usebox{\plotpoint}}
\put(624,90){\usebox{\plotpoint}}
\put(624,90){\usebox{\plotpoint}}
\put(624,90){\usebox{\plotpoint}}
\put(624,90){\usebox{\plotpoint}}
\put(624,90){\usebox{\plotpoint}}
\put(624,90){\usebox{\plotpoint}}
\put(624,90){\usebox{\plotpoint}}
\put(624,90){\usebox{\plotpoint}}
\put(624.0,90.0){\rule[-0.200pt]{0.482pt}{0.400pt}}
\put(626.0,89.0){\usebox{\plotpoint}}
\put(626.0,89.0){\rule[-0.200pt]{0.482pt}{0.400pt}}
\put(628.0,88.0){\usebox{\plotpoint}}
\put(628.0,88.0){\rule[-0.200pt]{0.482pt}{0.400pt}}
\put(630.0,87.0){\usebox{\plotpoint}}
\put(630.0,87.0){\rule[-0.200pt]{0.482pt}{0.400pt}}
\put(632.0,86.0){\usebox{\plotpoint}}
\put(632.0,86.0){\rule[-0.200pt]{0.482pt}{0.400pt}}
\put(634.0,85.0){\usebox{\plotpoint}}
\put(634.0,85.0){\rule[-0.200pt]{0.482pt}{0.400pt}}
\put(636.0,84.0){\usebox{\plotpoint}}
\put(636.0,84.0){\rule[-0.200pt]{0.723pt}{0.400pt}}
\put(639.0,83.0){\usebox{\plotpoint}}
\put(642,81.67){\rule{0.241pt}{0.400pt}}
\multiput(642.00,82.17)(0.500,-1.000){2}{\rule{0.120pt}{0.400pt}}
\put(639.0,83.0){\rule[-0.200pt]{0.723pt}{0.400pt}}
\put(643,82){\usebox{\plotpoint}}
\put(643,82){\usebox{\plotpoint}}
\put(643,82){\usebox{\plotpoint}}
\put(643,82){\usebox{\plotpoint}}
\put(643,82){\usebox{\plotpoint}}
\put(643,82){\usebox{\plotpoint}}
\put(643,82){\usebox{\plotpoint}}
\put(643,82){\usebox{\plotpoint}}
\put(643,82){\usebox{\plotpoint}}
\put(643,82){\usebox{\plotpoint}}
\put(643,82){\usebox{\plotpoint}}
\put(643,82){\usebox{\plotpoint}}
\put(643,82){\usebox{\plotpoint}}
\put(643,82){\usebox{\plotpoint}}
\put(643,82){\usebox{\plotpoint}}
\put(643,82){\usebox{\plotpoint}}
\put(643,82){\usebox{\plotpoint}}
\put(643,82){\usebox{\plotpoint}}
\put(643,82){\usebox{\plotpoint}}
\put(643,82){\usebox{\plotpoint}}
\put(643,82){\usebox{\plotpoint}}
\put(643,82){\usebox{\plotpoint}}
\put(643,82){\usebox{\plotpoint}}
\put(643,82){\usebox{\plotpoint}}
\put(643,82){\usebox{\plotpoint}}
\put(643,82){\usebox{\plotpoint}}
\put(643,82){\usebox{\plotpoint}}
\put(643,82){\usebox{\plotpoint}}
\put(643,82){\usebox{\plotpoint}}
\put(643,82){\usebox{\plotpoint}}
\put(643,82){\usebox{\plotpoint}}
\put(643,82){\usebox{\plotpoint}}
\put(643,82){\usebox{\plotpoint}}
\put(643,82){\usebox{\plotpoint}}
\put(643,82){\usebox{\plotpoint}}
\put(643,82){\usebox{\plotpoint}}
\put(643,82){\usebox{\plotpoint}}
\put(643,82){\usebox{\plotpoint}}
\put(643,82){\usebox{\plotpoint}}
\put(643,82){\usebox{\plotpoint}}
\put(643,82){\usebox{\plotpoint}}
\put(643,82){\usebox{\plotpoint}}
\put(643,82){\usebox{\plotpoint}}
\put(643,82){\usebox{\plotpoint}}
\put(643,82){\usebox{\plotpoint}}
\put(643,82){\usebox{\plotpoint}}
\put(643,82){\usebox{\plotpoint}}
\put(643,82){\usebox{\plotpoint}}
\put(643,82){\usebox{\plotpoint}}
\put(643,82){\usebox{\plotpoint}}
\put(643,82){\usebox{\plotpoint}}
\put(643,82){\usebox{\plotpoint}}
\put(643,82){\usebox{\plotpoint}}
\put(643,82){\usebox{\plotpoint}}
\put(643,82){\usebox{\plotpoint}}
\put(643,82){\usebox{\plotpoint}}
\put(643,82){\usebox{\plotpoint}}
\put(643,82){\usebox{\plotpoint}}
\put(643,82){\usebox{\plotpoint}}
\put(643,82){\usebox{\plotpoint}}
\put(643,82){\usebox{\plotpoint}}
\put(643,82){\usebox{\plotpoint}}
\put(643,82){\usebox{\plotpoint}}
\put(643,82){\usebox{\plotpoint}}
\put(643,82){\usebox{\plotpoint}}
\put(643,82){\usebox{\plotpoint}}
\put(643,82){\usebox{\plotpoint}}
\put(643,82){\usebox{\plotpoint}}
\put(643,82){\usebox{\plotpoint}}
\put(643,82){\usebox{\plotpoint}}
\put(643,82){\usebox{\plotpoint}}
\put(643,82){\usebox{\plotpoint}}
\put(643,82){\usebox{\plotpoint}}
\put(643,82){\usebox{\plotpoint}}
\put(643,82){\usebox{\plotpoint}}
\put(643.0,82.0){\rule[-0.200pt]{2.409pt}{0.400pt}}
\put(653.0,82.0){\usebox{\plotpoint}}
\put(653.0,83.0){\rule[-0.200pt]{0.964pt}{0.400pt}}
\put(657.0,83.0){\usebox{\plotpoint}}
\put(657.0,84.0){\rule[-0.200pt]{0.964pt}{0.400pt}}
\put(661.0,84.0){\usebox{\plotpoint}}
\put(661.0,85.0){\rule[-0.200pt]{0.723pt}{0.400pt}}
\put(664.0,85.0){\usebox{\plotpoint}}
\put(664.0,86.0){\rule[-0.200pt]{0.723pt}{0.400pt}}
\put(667.0,86.0){\usebox{\plotpoint}}
\put(667.0,87.0){\rule[-0.200pt]{0.723pt}{0.400pt}}
\put(670.0,87.0){\usebox{\plotpoint}}
\put(670.0,88.0){\rule[-0.200pt]{1.445pt}{0.400pt}}
\put(676.0,88.0){\usebox{\plotpoint}}
\put(676.0,89.0){\rule[-0.200pt]{1.445pt}{0.400pt}}
\put(682.0,88.0){\usebox{\plotpoint}}
\put(682.0,88.0){\rule[-0.200pt]{1.445pt}{0.400pt}}
\put(688.0,87.0){\usebox{\plotpoint}}
\put(688.0,87.0){\rule[-0.200pt]{0.964pt}{0.400pt}}
\put(692.0,86.0){\usebox{\plotpoint}}
\put(692.0,86.0){\rule[-0.200pt]{0.964pt}{0.400pt}}
\put(696.0,85.0){\usebox{\plotpoint}}
\put(696.0,85.0){\rule[-0.200pt]{0.723pt}{0.400pt}}
\put(699.0,84.0){\usebox{\plotpoint}}
\put(702,82.67){\rule{0.241pt}{0.400pt}}
\multiput(702.00,83.17)(0.500,-1.000){2}{\rule{0.120pt}{0.400pt}}
\put(699.0,84.0){\rule[-0.200pt]{0.723pt}{0.400pt}}
\put(703,83){\usebox{\plotpoint}}
\put(703,83){\usebox{\plotpoint}}
\put(703,83){\usebox{\plotpoint}}
\put(703,83){\usebox{\plotpoint}}
\put(703,83){\usebox{\plotpoint}}
\put(703,83){\usebox{\plotpoint}}
\put(703,83){\usebox{\plotpoint}}
\put(703,83){\usebox{\plotpoint}}
\put(703,83){\usebox{\plotpoint}}
\put(703,83){\usebox{\plotpoint}}
\put(703,83){\usebox{\plotpoint}}
\put(703,83){\usebox{\plotpoint}}
\put(703,83){\usebox{\plotpoint}}
\put(703,83){\usebox{\plotpoint}}
\put(703,83){\usebox{\plotpoint}}
\put(703,83){\usebox{\plotpoint}}
\put(703,83){\usebox{\plotpoint}}
\put(703,83){\usebox{\plotpoint}}
\put(703,83){\usebox{\plotpoint}}
\put(703,83){\usebox{\plotpoint}}
\put(703,83){\usebox{\plotpoint}}
\put(703,83){\usebox{\plotpoint}}
\put(703,83){\usebox{\plotpoint}}
\put(703,83){\usebox{\plotpoint}}
\put(703,83){\usebox{\plotpoint}}
\put(703,83){\usebox{\plotpoint}}
\put(703,83){\usebox{\plotpoint}}
\put(703,83){\usebox{\plotpoint}}
\put(703,83){\usebox{\plotpoint}}
\put(703,83){\usebox{\plotpoint}}
\put(703,83){\usebox{\plotpoint}}
\put(703,83){\usebox{\plotpoint}}
\put(703,83){\usebox{\plotpoint}}
\put(703,83){\usebox{\plotpoint}}
\put(703,83){\usebox{\plotpoint}}
\put(703,83){\usebox{\plotpoint}}
\put(703,83){\usebox{\plotpoint}}
\put(703,83){\usebox{\plotpoint}}
\put(703,83){\usebox{\plotpoint}}
\put(703,83){\usebox{\plotpoint}}
\put(703,83){\usebox{\plotpoint}}
\put(703,83){\usebox{\plotpoint}}
\put(703,83){\usebox{\plotpoint}}
\put(703,83){\usebox{\plotpoint}}
\put(703,83){\usebox{\plotpoint}}
\put(703,83){\usebox{\plotpoint}}
\put(703,83){\usebox{\plotpoint}}
\put(703,83){\usebox{\plotpoint}}
\put(703,83){\usebox{\plotpoint}}
\put(703,83){\usebox{\plotpoint}}
\put(703,83){\usebox{\plotpoint}}
\put(703,83){\usebox{\plotpoint}}
\put(703,83){\usebox{\plotpoint}}
\put(703,83){\usebox{\plotpoint}}
\put(703,83){\usebox{\plotpoint}}
\put(703,83){\usebox{\plotpoint}}
\put(703,83){\usebox{\plotpoint}}
\put(703,83){\usebox{\plotpoint}}
\put(703,83){\usebox{\plotpoint}}
\put(703,83){\usebox{\plotpoint}}
\put(703,83){\usebox{\plotpoint}}
\put(703,83){\usebox{\plotpoint}}
\put(703,83){\usebox{\plotpoint}}
\put(703,83){\usebox{\plotpoint}}
\put(703,83){\usebox{\plotpoint}}
\put(703,83){\usebox{\plotpoint}}
\put(703,83){\usebox{\plotpoint}}
\put(703,83){\usebox{\plotpoint}}
\put(703,83){\usebox{\plotpoint}}
\put(703,83){\usebox{\plotpoint}}
\put(703,83){\usebox{\plotpoint}}
\put(703,83){\usebox{\plotpoint}}
\put(703,83){\usebox{\plotpoint}}
\put(703,83){\usebox{\plotpoint}}
\put(703,83){\usebox{\plotpoint}}
\put(703.0,83.0){\rule[-0.200pt]{0.964pt}{0.400pt}}
\put(707.0,82.0){\usebox{\plotpoint}}
\put(707.0,82.0){\rule[-0.200pt]{3.132pt}{0.400pt}}
\put(720.0,82.0){\usebox{\plotpoint}}
\put(720.0,83.0){\rule[-0.200pt]{1.204pt}{0.400pt}}
\put(725.0,83.0){\usebox{\plotpoint}}
\put(725.0,84.0){\rule[-0.200pt]{0.964pt}{0.400pt}}
\put(729.0,84.0){\usebox{\plotpoint}}
\put(729.0,85.0){\rule[-0.200pt]{0.964pt}{0.400pt}}
\put(733.0,85.0){\usebox{\plotpoint}}
\put(733.0,86.0){\rule[-0.200pt]{1.204pt}{0.400pt}}
\put(738.0,86.0){\usebox{\plotpoint}}
\put(738.0,87.0){\rule[-0.200pt]{3.854pt}{0.400pt}}
\put(754.0,86.0){\usebox{\plotpoint}}
\put(754.0,86.0){\rule[-0.200pt]{1.204pt}{0.400pt}}
\put(759.0,85.0){\usebox{\plotpoint}}
\put(759.0,85.0){\rule[-0.200pt]{0.964pt}{0.400pt}}
\put(763.0,84.0){\usebox{\plotpoint}}
\put(763.0,84.0){\rule[-0.200pt]{0.964pt}{0.400pt}}
\put(767.0,83.0){\usebox{\plotpoint}}
\put(767.0,83.0){\rule[-0.200pt]{1.445pt}{0.400pt}}
\put(773.0,82.0){\usebox{\plotpoint}}
\put(773.0,82.0){\rule[-0.200pt]{3.132pt}{0.400pt}}
\put(786.0,82.0){\usebox{\plotpoint}}
\put(786.0,83.0){\rule[-0.200pt]{1.445pt}{0.400pt}}
\put(792.0,83.0){\usebox{\plotpoint}}
\put(792.0,84.0){\rule[-0.200pt]{0.964pt}{0.400pt}}
\put(796.0,84.0){\usebox{\plotpoint}}
\put(796.0,85.0){\rule[-0.200pt]{0.964pt}{0.400pt}}
\put(800.0,85.0){\usebox{\plotpoint}}
\put(800.0,86.0){\rule[-0.200pt]{1.204pt}{0.400pt}}
\put(805.0,86.0){\usebox{\plotpoint}}
\put(805.0,87.0){\rule[-0.200pt]{3.854pt}{0.400pt}}
\put(821.0,86.0){\usebox{\plotpoint}}
\put(821.0,86.0){\rule[-0.200pt]{1.204pt}{0.400pt}}
\put(826.0,85.0){\usebox{\plotpoint}}
\put(826.0,85.0){\rule[-0.200pt]{0.964pt}{0.400pt}}
\put(830.0,84.0){\usebox{\plotpoint}}
\put(830.0,84.0){\rule[-0.200pt]{0.964pt}{0.400pt}}
\put(834.0,83.0){\usebox{\plotpoint}}
\put(834.0,83.0){\rule[-0.200pt]{1.204pt}{0.400pt}}
\put(839.0,82.0){\usebox{\plotpoint}}
\put(839.0,82.0){\rule[-0.200pt]{3.132pt}{0.400pt}}
\put(852.0,82.0){\usebox{\plotpoint}}
\put(856,82.67){\rule{0.241pt}{0.400pt}}
\multiput(856.00,82.17)(0.500,1.000){2}{\rule{0.120pt}{0.400pt}}
\put(852.0,83.0){\rule[-0.200pt]{0.964pt}{0.400pt}}
\put(857,84){\usebox{\plotpoint}}
\put(857,84){\usebox{\plotpoint}}
\put(857,84){\usebox{\plotpoint}}
\put(857,84){\usebox{\plotpoint}}
\put(857,84){\usebox{\plotpoint}}
\put(857,84){\usebox{\plotpoint}}
\put(857,84){\usebox{\plotpoint}}
\put(857,84){\usebox{\plotpoint}}
\put(857,84){\usebox{\plotpoint}}
\put(857,84){\usebox{\plotpoint}}
\put(857,84){\usebox{\plotpoint}}
\put(857,84){\usebox{\plotpoint}}
\put(857,84){\usebox{\plotpoint}}
\put(857,84){\usebox{\plotpoint}}
\put(857,84){\usebox{\plotpoint}}
\put(857,84){\usebox{\plotpoint}}
\put(857,84){\usebox{\plotpoint}}
\put(857,84){\usebox{\plotpoint}}
\put(857,84){\usebox{\plotpoint}}
\put(857,84){\usebox{\plotpoint}}
\put(857,84){\usebox{\plotpoint}}
\put(857,84){\usebox{\plotpoint}}
\put(857,84){\usebox{\plotpoint}}
\put(857,84){\usebox{\plotpoint}}
\put(857,84){\usebox{\plotpoint}}
\put(857,84){\usebox{\plotpoint}}
\put(857,84){\usebox{\plotpoint}}
\put(857,84){\usebox{\plotpoint}}
\put(857,84){\usebox{\plotpoint}}
\put(857,84){\usebox{\plotpoint}}
\put(857,84){\usebox{\plotpoint}}
\put(857,84){\usebox{\plotpoint}}
\put(857,84){\usebox{\plotpoint}}
\put(857,84){\usebox{\plotpoint}}
\put(857,84){\usebox{\plotpoint}}
\put(857,84){\usebox{\plotpoint}}
\put(857,84){\usebox{\plotpoint}}
\put(857,84){\usebox{\plotpoint}}
\put(857,84){\usebox{\plotpoint}}
\put(857,84){\usebox{\plotpoint}}
\put(857,84){\usebox{\plotpoint}}
\put(857,84){\usebox{\plotpoint}}
\put(857,84){\usebox{\plotpoint}}
\put(857,84){\usebox{\plotpoint}}
\put(857,84){\usebox{\plotpoint}}
\put(857,84){\usebox{\plotpoint}}
\put(857,84){\usebox{\plotpoint}}
\put(857,84){\usebox{\plotpoint}}
\put(857,84){\usebox{\plotpoint}}
\put(857,84){\usebox{\plotpoint}}
\put(857,84){\usebox{\plotpoint}}
\put(857,84){\usebox{\plotpoint}}
\put(857,84){\usebox{\plotpoint}}
\put(857,84){\usebox{\plotpoint}}
\put(857,84){\usebox{\plotpoint}}
\put(857,84){\usebox{\plotpoint}}
\put(857,84){\usebox{\plotpoint}}
\put(857,84){\usebox{\plotpoint}}
\put(857,84){\usebox{\plotpoint}}
\put(857,84){\usebox{\plotpoint}}
\put(857,84){\usebox{\plotpoint}}
\put(857,84){\usebox{\plotpoint}}
\put(857,84){\usebox{\plotpoint}}
\put(857,84){\usebox{\plotpoint}}
\put(857,84){\usebox{\plotpoint}}
\put(857,84){\usebox{\plotpoint}}
\put(857,84){\usebox{\plotpoint}}
\put(857,84){\usebox{\plotpoint}}
\put(857,84){\usebox{\plotpoint}}
\put(857,84){\usebox{\plotpoint}}
\put(857,84){\usebox{\plotpoint}}
\put(857,84){\usebox{\plotpoint}}
\put(857,84){\usebox{\plotpoint}}
\put(857,84){\usebox{\plotpoint}}
\put(857,84){\usebox{\plotpoint}}
\put(857.0,84.0){\rule[-0.200pt]{0.723pt}{0.400pt}}
\put(860.0,84.0){\usebox{\plotpoint}}
\put(860.0,85.0){\rule[-0.200pt]{0.723pt}{0.400pt}}
\put(863.0,85.0){\usebox{\plotpoint}}
\put(863.0,86.0){\rule[-0.200pt]{0.964pt}{0.400pt}}
\put(867.0,86.0){\usebox{\plotpoint}}
\put(867.0,87.0){\rule[-0.200pt]{0.964pt}{0.400pt}}
\put(871.0,87.0){\usebox{\plotpoint}}
\put(871.0,88.0){\rule[-0.200pt]{1.445pt}{0.400pt}}
\put(877.0,88.0){\usebox{\plotpoint}}
\put(877.0,89.0){\rule[-0.200pt]{1.445pt}{0.400pt}}
\put(883.0,88.0){\usebox{\plotpoint}}
\put(883.0,88.0){\rule[-0.200pt]{1.445pt}{0.400pt}}
\put(889.0,87.0){\usebox{\plotpoint}}
\put(889.0,87.0){\rule[-0.200pt]{0.723pt}{0.400pt}}
\put(892.0,86.0){\usebox{\plotpoint}}
\put(892.0,86.0){\rule[-0.200pt]{0.723pt}{0.400pt}}
\put(895.0,85.0){\usebox{\plotpoint}}
\put(895.0,85.0){\rule[-0.200pt]{0.723pt}{0.400pt}}
\put(898.0,84.0){\usebox{\plotpoint}}
\put(898.0,84.0){\rule[-0.200pt]{0.964pt}{0.400pt}}
\put(902.0,83.0){\usebox{\plotpoint}}
\put(902.0,83.0){\rule[-0.200pt]{0.964pt}{0.400pt}}
\put(906.0,82.0){\usebox{\plotpoint}}
\put(916,81.67){\rule{0.241pt}{0.400pt}}
\multiput(916.00,81.17)(0.500,1.000){2}{\rule{0.120pt}{0.400pt}}
\put(906.0,82.0){\rule[-0.200pt]{2.409pt}{0.400pt}}
\put(917,83){\usebox{\plotpoint}}
\put(917,83){\usebox{\plotpoint}}
\put(917,83){\usebox{\plotpoint}}
\put(917,83){\usebox{\plotpoint}}
\put(917,83){\usebox{\plotpoint}}
\put(917,83){\usebox{\plotpoint}}
\put(917,83){\usebox{\plotpoint}}
\put(917,83){\usebox{\plotpoint}}
\put(917,83){\usebox{\plotpoint}}
\put(917,83){\usebox{\plotpoint}}
\put(917,83){\usebox{\plotpoint}}
\put(917,83){\usebox{\plotpoint}}
\put(917,83){\usebox{\plotpoint}}
\put(917,83){\usebox{\plotpoint}}
\put(917,83){\usebox{\plotpoint}}
\put(917,83){\usebox{\plotpoint}}
\put(917,83){\usebox{\plotpoint}}
\put(917,83){\usebox{\plotpoint}}
\put(917,83){\usebox{\plotpoint}}
\put(917,83){\usebox{\plotpoint}}
\put(917,83){\usebox{\plotpoint}}
\put(917,83){\usebox{\plotpoint}}
\put(917,83){\usebox{\plotpoint}}
\put(917,83){\usebox{\plotpoint}}
\put(917,83){\usebox{\plotpoint}}
\put(917,83){\usebox{\plotpoint}}
\put(917,83){\usebox{\plotpoint}}
\put(917,83){\usebox{\plotpoint}}
\put(917,83){\usebox{\plotpoint}}
\put(917,83){\usebox{\plotpoint}}
\put(917,83){\usebox{\plotpoint}}
\put(917,83){\usebox{\plotpoint}}
\put(917,83){\usebox{\plotpoint}}
\put(917,83){\usebox{\plotpoint}}
\put(917,83){\usebox{\plotpoint}}
\put(917,83){\usebox{\plotpoint}}
\put(917,83){\usebox{\plotpoint}}
\put(917,83){\usebox{\plotpoint}}
\put(917,83){\usebox{\plotpoint}}
\put(917,83){\usebox{\plotpoint}}
\put(917,83){\usebox{\plotpoint}}
\put(917,83){\usebox{\plotpoint}}
\put(917,83){\usebox{\plotpoint}}
\put(917,83){\usebox{\plotpoint}}
\put(917,83){\usebox{\plotpoint}}
\put(917,83){\usebox{\plotpoint}}
\put(917,83){\usebox{\plotpoint}}
\put(917,83){\usebox{\plotpoint}}
\put(917,83){\usebox{\plotpoint}}
\put(917,83){\usebox{\plotpoint}}
\put(917,83){\usebox{\plotpoint}}
\put(917,83){\usebox{\plotpoint}}
\put(917,83){\usebox{\plotpoint}}
\put(917,83){\usebox{\plotpoint}}
\put(917,83){\usebox{\plotpoint}}
\put(917,83){\usebox{\plotpoint}}
\put(917,83){\usebox{\plotpoint}}
\put(917,83){\usebox{\plotpoint}}
\put(917,83){\usebox{\plotpoint}}
\put(917,83){\usebox{\plotpoint}}
\put(917,83){\usebox{\plotpoint}}
\put(917,83){\usebox{\plotpoint}}
\put(917,83){\usebox{\plotpoint}}
\put(917,83){\usebox{\plotpoint}}
\put(917,83){\usebox{\plotpoint}}
\put(917,83){\usebox{\plotpoint}}
\put(917,83){\usebox{\plotpoint}}
\put(917,83){\usebox{\plotpoint}}
\put(917,83){\usebox{\plotpoint}}
\put(917,83){\usebox{\plotpoint}}
\put(917,83){\usebox{\plotpoint}}
\put(917,83){\usebox{\plotpoint}}
\put(917,83){\usebox{\plotpoint}}
\put(917,83){\usebox{\plotpoint}}
\put(917.0,83.0){\rule[-0.200pt]{0.723pt}{0.400pt}}
\put(920.0,83.0){\usebox{\plotpoint}}
\put(920.0,84.0){\rule[-0.200pt]{0.723pt}{0.400pt}}
\put(923.0,84.0){\usebox{\plotpoint}}
\put(923.0,85.0){\rule[-0.200pt]{0.482pt}{0.400pt}}
\put(925.0,85.0){\usebox{\plotpoint}}
\put(925.0,86.0){\rule[-0.200pt]{0.482pt}{0.400pt}}
\put(927.0,86.0){\usebox{\plotpoint}}
\put(927.0,87.0){\rule[-0.200pt]{0.482pt}{0.400pt}}
\put(929.0,87.0){\usebox{\plotpoint}}
\put(929.0,88.0){\rule[-0.200pt]{0.482pt}{0.400pt}}
\put(931.0,88.0){\usebox{\plotpoint}}
\put(931.0,89.0){\rule[-0.200pt]{0.482pt}{0.400pt}}
\put(933.0,89.0){\usebox{\plotpoint}}
\put(935,89.67){\rule{0.241pt}{0.400pt}}
\multiput(935.00,89.17)(0.500,1.000){2}{\rule{0.120pt}{0.400pt}}
\put(933.0,90.0){\rule[-0.200pt]{0.482pt}{0.400pt}}
\put(936,91){\usebox{\plotpoint}}
\put(936,91){\usebox{\plotpoint}}
\put(936,91){\usebox{\plotpoint}}
\put(936,91){\usebox{\plotpoint}}
\put(936,91){\usebox{\plotpoint}}
\put(936,91){\usebox{\plotpoint}}
\put(936,91){\usebox{\plotpoint}}
\put(936,91){\usebox{\plotpoint}}
\put(936,91){\usebox{\plotpoint}}
\put(936,91){\usebox{\plotpoint}}
\put(936,91){\usebox{\plotpoint}}
\put(936,91){\usebox{\plotpoint}}
\put(936,91){\usebox{\plotpoint}}
\put(936,91){\usebox{\plotpoint}}
\put(936,91){\usebox{\plotpoint}}
\put(936,91){\usebox{\plotpoint}}
\put(936,91){\usebox{\plotpoint}}
\put(936,91){\usebox{\plotpoint}}
\put(936,91){\usebox{\plotpoint}}
\put(936,91){\usebox{\plotpoint}}
\put(936,91){\usebox{\plotpoint}}
\put(936,91){\usebox{\plotpoint}}
\put(936,91){\usebox{\plotpoint}}
\put(936,91){\usebox{\plotpoint}}
\put(936,91){\usebox{\plotpoint}}
\put(936,91){\usebox{\plotpoint}}
\put(936,91){\usebox{\plotpoint}}
\put(936,91){\usebox{\plotpoint}}
\put(936,91){\usebox{\plotpoint}}
\put(936,91){\usebox{\plotpoint}}
\put(936,91){\usebox{\plotpoint}}
\put(936,91){\usebox{\plotpoint}}
\put(936,91){\usebox{\plotpoint}}
\put(936,91){\usebox{\plotpoint}}
\put(936,91){\usebox{\plotpoint}}
\put(936,91){\usebox{\plotpoint}}
\put(936,91){\usebox{\plotpoint}}
\put(936,91){\usebox{\plotpoint}}
\put(936,91){\usebox{\plotpoint}}
\put(936,91){\usebox{\plotpoint}}
\put(936,91){\usebox{\plotpoint}}
\put(936,91){\usebox{\plotpoint}}
\put(936,91){\usebox{\plotpoint}}
\put(936,91){\usebox{\plotpoint}}
\put(936,91){\usebox{\plotpoint}}
\put(936,91){\usebox{\plotpoint}}
\put(936,91){\usebox{\plotpoint}}
\put(936,91){\usebox{\plotpoint}}
\put(936,91){\usebox{\plotpoint}}
\put(936,91){\usebox{\plotpoint}}
\put(936,91){\usebox{\plotpoint}}
\put(936,91){\usebox{\plotpoint}}
\put(936,91){\usebox{\plotpoint}}
\put(936,91){\usebox{\plotpoint}}
\put(936,91){\usebox{\plotpoint}}
\put(936,91){\usebox{\plotpoint}}
\put(936,91){\usebox{\plotpoint}}
\put(936,91){\usebox{\plotpoint}}
\put(936,91){\usebox{\plotpoint}}
\put(936,91){\usebox{\plotpoint}}
\put(936,91){\usebox{\plotpoint}}
\put(936,91){\usebox{\plotpoint}}
\put(936,91){\usebox{\plotpoint}}
\put(936,91){\usebox{\plotpoint}}
\put(936,91){\usebox{\plotpoint}}
\put(936,91){\usebox{\plotpoint}}
\put(936,91){\usebox{\plotpoint}}
\put(936,91){\usebox{\plotpoint}}
\put(936,91){\usebox{\plotpoint}}
\put(936,91){\usebox{\plotpoint}}
\put(936,91){\usebox{\plotpoint}}
\put(936,91){\usebox{\plotpoint}}
\put(936,91){\usebox{\plotpoint}}
\put(936,91){\usebox{\plotpoint}}
\put(936,91){\usebox{\plotpoint}}
\put(936.0,91.0){\rule[-0.200pt]{0.482pt}{0.400pt}}
\put(938.0,91.0){\usebox{\plotpoint}}
\put(938.0,92.0){\rule[-0.200pt]{0.723pt}{0.400pt}}
\put(941.0,92.0){\usebox{\plotpoint}}
\put(941.0,93.0){\rule[-0.200pt]{2.650pt}{0.400pt}}
\put(952.0,92.0){\usebox{\plotpoint}}
\put(952.0,92.0){\rule[-0.200pt]{0.723pt}{0.400pt}}
\put(955.0,91.0){\usebox{\plotpoint}}
\put(955.0,91.0){\rule[-0.200pt]{0.723pt}{0.400pt}}
\put(958.0,90.0){\usebox{\plotpoint}}
\put(958.0,90.0){\usebox{\plotpoint}}
\put(959.0,89.0){\usebox{\plotpoint}}
\put(959.0,89.0){\rule[-0.200pt]{0.482pt}{0.400pt}}
\put(961.0,88.0){\usebox{\plotpoint}}
\put(961.0,88.0){\rule[-0.200pt]{0.482pt}{0.400pt}}
\put(963.0,87.0){\usebox{\plotpoint}}
\put(963.0,87.0){\rule[-0.200pt]{0.482pt}{0.400pt}}
\put(965.0,86.0){\usebox{\plotpoint}}
\put(965.0,86.0){\rule[-0.200pt]{0.482pt}{0.400pt}}
\put(967.0,85.0){\usebox{\plotpoint}}
\put(967.0,85.0){\usebox{\plotpoint}}
\put(968.0,84.0){\usebox{\plotpoint}}
\put(968.0,84.0){\rule[-0.200pt]{0.723pt}{0.400pt}}
\put(971.0,83.0){\usebox{\plotpoint}}
\put(971.0,83.0){\rule[-0.200pt]{0.723pt}{0.400pt}}
\put(974.0,82.0){\usebox{\plotpoint}}
\put(974.0,82.0){\rule[-0.200pt]{1.686pt}{0.400pt}}
\put(981.0,82.0){\usebox{\plotpoint}}
\put(981.0,83.0){\rule[-0.200pt]{0.482pt}{0.400pt}}
\put(983.0,83.0){\usebox{\plotpoint}}
\put(983.0,84.0){\rule[-0.200pt]{0.482pt}{0.400pt}}
\put(985.0,84.0){\usebox{\plotpoint}}
\put(985.0,85.0){\rule[-0.200pt]{0.482pt}{0.400pt}}
\put(987.0,85.0){\usebox{\plotpoint}}
\put(987.0,86.0){\usebox{\plotpoint}}
\put(988.0,86.0){\usebox{\plotpoint}}
\put(988.0,87.0){\usebox{\plotpoint}}
\put(989.0,87.0){\usebox{\plotpoint}}
\put(989.0,88.0){\usebox{\plotpoint}}
\put(990.0,88.0){\usebox{\plotpoint}}
\put(990.0,89.0){\usebox{\plotpoint}}
\put(991.0,89.0){\usebox{\plotpoint}}
\put(991.0,90.0){\usebox{\plotpoint}}
\put(992.0,90.0){\usebox{\plotpoint}}
\put(992.0,91.0){\usebox{\plotpoint}}
\put(993.0,91.0){\usebox{\plotpoint}}
\put(993.0,92.0){\usebox{\plotpoint}}
\put(994.0,92.0){\rule[-0.200pt]{0.400pt}{0.482pt}}
\put(994.0,94.0){\usebox{\plotpoint}}
\put(995.0,94.0){\usebox{\plotpoint}}
\put(995.0,95.0){\usebox{\plotpoint}}
\put(996.0,95.0){\usebox{\plotpoint}}
\put(996.0,96.0){\usebox{\plotpoint}}
\put(997.0,96.0){\usebox{\plotpoint}}
\put(997.0,97.0){\usebox{\plotpoint}}
\put(998.0,97.0){\rule[-0.200pt]{0.400pt}{0.482pt}}
\put(998.0,99.0){\usebox{\plotpoint}}
\put(999.0,99.0){\usebox{\plotpoint}}
\put(999.0,100.0){\usebox{\plotpoint}}
\put(1000.0,100.0){\usebox{\plotpoint}}
\put(1000.0,101.0){\usebox{\plotpoint}}
\put(1001.0,101.0){\usebox{\plotpoint}}
\put(1001.0,102.0){\usebox{\plotpoint}}
\put(1002.0,102.0){\rule[-0.200pt]{0.400pt}{0.482pt}}
\put(1002.0,104.0){\usebox{\plotpoint}}
\put(1003.0,104.0){\usebox{\plotpoint}}
\put(1003.0,105.0){\usebox{\plotpoint}}
\put(1004.0,105.0){\usebox{\plotpoint}}
\put(1004.0,106.0){\usebox{\plotpoint}}
\put(1005.0,106.0){\usebox{\plotpoint}}
\put(1005.0,107.0){\usebox{\plotpoint}}
\put(1006.0,107.0){\usebox{\plotpoint}}
\put(1006.0,108.0){\usebox{\plotpoint}}
\put(1007.0,108.0){\usebox{\plotpoint}}
\put(1007.0,109.0){\usebox{\plotpoint}}
\put(1008.0,109.0){\usebox{\plotpoint}}
\put(1008.0,110.0){\rule[-0.200pt]{0.482pt}{0.400pt}}
\put(1010.0,110.0){\usebox{\plotpoint}}
\put(1010.0,111.0){\rule[-0.200pt]{0.482pt}{0.400pt}}
\put(1012.0,111.0){\usebox{\plotpoint}}
\put(1012.0,112.0){\rule[-0.200pt]{1.445pt}{0.400pt}}
\put(1018.0,111.0){\usebox{\plotpoint}}
\put(1018.0,111.0){\rule[-0.200pt]{0.482pt}{0.400pt}}
\put(1020.0,110.0){\usebox{\plotpoint}}
\put(1020.0,110.0){\usebox{\plotpoint}}
\put(1021.0,109.0){\usebox{\plotpoint}}
\put(1021.0,109.0){\usebox{\plotpoint}}
\put(1022.0,108.0){\usebox{\plotpoint}}
\put(1022.0,108.0){\usebox{\plotpoint}}
\put(1023.0,107.0){\usebox{\plotpoint}}
\put(1023.0,107.0){\usebox{\plotpoint}}
\put(1024.0,106.0){\usebox{\plotpoint}}
\put(1024.0,106.0){\usebox{\plotpoint}}
\put(1025.0,104.0){\rule[-0.200pt]{0.400pt}{0.482pt}}
\put(1025.0,104.0){\usebox{\plotpoint}}
\put(1026.0,103.0){\usebox{\plotpoint}}
\put(1026.0,103.0){\usebox{\plotpoint}}
\put(1027,100.67){\rule{0.241pt}{0.400pt}}
\multiput(1027.00,101.17)(0.500,-1.000){2}{\rule{0.120pt}{0.400pt}}
\put(1027.0,102.0){\usebox{\plotpoint}}
\put(1028,101){\usebox{\plotpoint}}
\put(1028,101){\usebox{\plotpoint}}
\put(1028,101){\usebox{\plotpoint}}
\put(1028,101){\usebox{\plotpoint}}
\put(1028,101){\usebox{\plotpoint}}
\put(1028,101){\usebox{\plotpoint}}
\put(1028,101){\usebox{\plotpoint}}
\put(1028,101){\usebox{\plotpoint}}
\put(1028,101){\usebox{\plotpoint}}
\put(1028,101){\usebox{\plotpoint}}
\put(1028,101){\usebox{\plotpoint}}
\put(1028,101){\usebox{\plotpoint}}
\put(1028,101){\usebox{\plotpoint}}
\put(1028,101){\usebox{\plotpoint}}
\put(1028,101){\usebox{\plotpoint}}
\put(1028,101){\usebox{\plotpoint}}
\put(1028,101){\usebox{\plotpoint}}
\put(1028,101){\usebox{\plotpoint}}
\put(1028,101){\usebox{\plotpoint}}
\put(1028,101){\usebox{\plotpoint}}
\put(1028,101){\usebox{\plotpoint}}
\put(1028,101){\usebox{\plotpoint}}
\put(1028,101){\usebox{\plotpoint}}
\put(1028,101){\usebox{\plotpoint}}
\put(1028,101){\usebox{\plotpoint}}
\put(1028,101){\usebox{\plotpoint}}
\put(1028,101){\usebox{\plotpoint}}
\put(1028,101){\usebox{\plotpoint}}
\put(1028,101){\usebox{\plotpoint}}
\put(1028,101){\usebox{\plotpoint}}
\put(1028,101){\usebox{\plotpoint}}
\put(1028,101){\usebox{\plotpoint}}
\put(1028,101){\usebox{\plotpoint}}
\put(1028,101){\usebox{\plotpoint}}
\put(1028,101){\usebox{\plotpoint}}
\put(1028,101){\usebox{\plotpoint}}
\put(1028,101){\usebox{\plotpoint}}
\put(1028,101){\usebox{\plotpoint}}
\put(1028,101){\usebox{\plotpoint}}
\put(1028,101){\usebox{\plotpoint}}
\put(1028,101){\usebox{\plotpoint}}
\put(1028,101){\usebox{\plotpoint}}
\put(1028,101){\usebox{\plotpoint}}
\put(1028,101){\usebox{\plotpoint}}
\put(1028,101){\usebox{\plotpoint}}
\put(1028,101){\usebox{\plotpoint}}
\put(1028,101){\usebox{\plotpoint}}
\put(1028,101){\usebox{\plotpoint}}
\put(1028.0,100.0){\usebox{\plotpoint}}
\put(1028.0,100.0){\usebox{\plotpoint}}
\put(1029.0,98.0){\rule[-0.200pt]{0.400pt}{0.482pt}}
\put(1029.0,98.0){\usebox{\plotpoint}}
\put(1030.0,97.0){\usebox{\plotpoint}}
\put(1030.0,97.0){\usebox{\plotpoint}}
\put(1031.0,95.0){\rule[-0.200pt]{0.400pt}{0.482pt}}
\put(1031.0,95.0){\usebox{\plotpoint}}
\put(1032.0,93.0){\rule[-0.200pt]{0.400pt}{0.482pt}}
\put(1032.0,93.0){\usebox{\plotpoint}}
\put(1033.0,92.0){\usebox{\plotpoint}}
\put(1033.0,92.0){\usebox{\plotpoint}}
\put(1034.0,90.0){\rule[-0.200pt]{0.400pt}{0.482pt}}
\put(1034.0,90.0){\usebox{\plotpoint}}
\put(1035.0,89.0){\usebox{\plotpoint}}
\put(1035.0,89.0){\usebox{\plotpoint}}
\put(1036.0,87.0){\rule[-0.200pt]{0.400pt}{0.482pt}}
\put(1036.0,87.0){\usebox{\plotpoint}}
\put(1037.0,86.0){\usebox{\plotpoint}}
\put(1037.0,86.0){\usebox{\plotpoint}}
\put(1038.0,85.0){\usebox{\plotpoint}}
\put(1038.0,85.0){\usebox{\plotpoint}}
\put(1039.0,84.0){\usebox{\plotpoint}}
\put(1039.0,84.0){\usebox{\plotpoint}}
\put(1040.0,83.0){\usebox{\plotpoint}}
\put(1040.0,83.0){\usebox{\plotpoint}}
\put(1041.0,82.0){\usebox{\plotpoint}}
\put(1041.0,82.0){\rule[-0.200pt]{0.964pt}{0.400pt}}
\put(1045.0,82.0){\usebox{\plotpoint}}
\put(1045.0,83.0){\usebox{\plotpoint}}
\put(1046.0,83.0){\usebox{\plotpoint}}
\put(1046.0,84.0){\usebox{\plotpoint}}
\put(1047.0,84.0){\usebox{\plotpoint}}
\put(1047.0,85.0){\usebox{\plotpoint}}
\put(1048,85.67){\rule{0.241pt}{0.400pt}}
\multiput(1048.00,85.17)(0.500,1.000){2}{\rule{0.120pt}{0.400pt}}
\put(1048.0,85.0){\usebox{\plotpoint}}
\put(1049,87){\usebox{\plotpoint}}
\put(1049,87){\usebox{\plotpoint}}
\put(1049,87){\usebox{\plotpoint}}
\put(1049,87){\usebox{\plotpoint}}
\put(1049,87){\usebox{\plotpoint}}
\put(1049,87){\usebox{\plotpoint}}
\put(1049,87){\usebox{\plotpoint}}
\put(1049,87){\usebox{\plotpoint}}
\put(1049,87){\usebox{\plotpoint}}
\put(1049,87){\usebox{\plotpoint}}
\put(1049,87){\usebox{\plotpoint}}
\put(1049,87){\usebox{\plotpoint}}
\put(1049,87){\usebox{\plotpoint}}
\put(1049,87){\usebox{\plotpoint}}
\put(1049,87){\usebox{\plotpoint}}
\put(1049,87){\usebox{\plotpoint}}
\put(1049,87){\usebox{\plotpoint}}
\put(1049,87){\usebox{\plotpoint}}
\put(1049,87){\usebox{\plotpoint}}
\put(1049,87){\usebox{\plotpoint}}
\put(1049,87){\usebox{\plotpoint}}
\put(1049,87){\usebox{\plotpoint}}
\put(1049,87){\usebox{\plotpoint}}
\put(1049,87){\usebox{\plotpoint}}
\put(1049,87){\usebox{\plotpoint}}
\put(1049,87){\usebox{\plotpoint}}
\put(1049,87){\usebox{\plotpoint}}
\put(1049,87){\usebox{\plotpoint}}
\put(1049,87){\usebox{\plotpoint}}
\put(1049,87){\usebox{\plotpoint}}
\put(1049,87){\usebox{\plotpoint}}
\put(1049,87){\usebox{\plotpoint}}
\put(1049,87){\usebox{\plotpoint}}
\put(1049,87){\usebox{\plotpoint}}
\put(1049,87){\usebox{\plotpoint}}
\put(1049,87){\usebox{\plotpoint}}
\put(1049,87){\usebox{\plotpoint}}
\put(1049,87){\usebox{\plotpoint}}
\put(1049.0,87.0){\rule[-0.200pt]{0.400pt}{0.482pt}}
\put(1049.0,89.0){\usebox{\plotpoint}}
\put(1050.0,89.0){\rule[-0.200pt]{0.400pt}{0.482pt}}
\put(1050.0,91.0){\usebox{\plotpoint}}
\put(1051.0,91.0){\rule[-0.200pt]{0.400pt}{0.723pt}}
\put(1051.0,94.0){\usebox{\plotpoint}}
\put(1052.0,94.0){\rule[-0.200pt]{0.400pt}{0.723pt}}
\put(1052.0,97.0){\usebox{\plotpoint}}
\put(1053.0,97.0){\rule[-0.200pt]{0.400pt}{0.964pt}}
\put(1053.0,101.0){\usebox{\plotpoint}}
\put(1054.0,101.0){\rule[-0.200pt]{0.400pt}{1.204pt}}
\put(1054.0,106.0){\usebox{\plotpoint}}
\put(1055.0,106.0){\rule[-0.200pt]{0.400pt}{1.204pt}}
\put(1055.0,111.0){\usebox{\plotpoint}}
\put(1056.0,111.0){\rule[-0.200pt]{0.400pt}{1.204pt}}
\put(1056.0,116.0){\usebox{\plotpoint}}
\put(1057.0,116.0){\rule[-0.200pt]{0.400pt}{1.445pt}}
\put(1057.0,122.0){\usebox{\plotpoint}}
\put(1058.0,122.0){\rule[-0.200pt]{0.400pt}{1.686pt}}
\put(1058.0,129.0){\usebox{\plotpoint}}
\put(1059.0,129.0){\rule[-0.200pt]{0.400pt}{1.686pt}}
\put(1059.0,136.0){\usebox{\plotpoint}}
\put(1060.0,136.0){\rule[-0.200pt]{0.400pt}{1.927pt}}
\put(1060.0,144.0){\usebox{\plotpoint}}
\put(1061.0,144.0){\rule[-0.200pt]{0.400pt}{1.927pt}}
\put(1061.0,152.0){\usebox{\plotpoint}}
\put(1062.0,152.0){\rule[-0.200pt]{0.400pt}{2.168pt}}
\put(1062.0,161.0){\usebox{\plotpoint}}
\put(1063.0,161.0){\rule[-0.200pt]{0.400pt}{2.168pt}}
\put(1063.0,170.0){\usebox{\plotpoint}}
\put(1064.0,170.0){\rule[-0.200pt]{0.400pt}{2.409pt}}
\put(1064.0,180.0){\usebox{\plotpoint}}
\put(1065.0,180.0){\rule[-0.200pt]{0.400pt}{2.650pt}}
\put(1065.0,191.0){\usebox{\plotpoint}}
\put(1066.0,191.0){\rule[-0.200pt]{0.400pt}{2.650pt}}
\put(1066.0,202.0){\usebox{\plotpoint}}
\put(1067.0,202.0){\rule[-0.200pt]{0.400pt}{2.891pt}}
\put(1067.0,214.0){\usebox{\plotpoint}}
\put(1068.0,214.0){\rule[-0.200pt]{0.400pt}{2.891pt}}
\put(1068.0,226.0){\usebox{\plotpoint}}
\put(1069.0,226.0){\rule[-0.200pt]{0.400pt}{2.891pt}}
\put(1069.0,238.0){\usebox{\plotpoint}}
\put(1070.0,238.0){\rule[-0.200pt]{0.400pt}{3.132pt}}
\put(1070.0,251.0){\usebox{\plotpoint}}
\put(1071.0,251.0){\rule[-0.200pt]{0.400pt}{3.373pt}}
\put(1071.0,265.0){\usebox{\plotpoint}}
\put(1072.0,265.0){\rule[-0.200pt]{0.400pt}{3.373pt}}
\put(1072.0,279.0){\usebox{\plotpoint}}
\put(1073.0,279.0){\rule[-0.200pt]{0.400pt}{3.373pt}}
\put(1073.0,293.0){\usebox{\plotpoint}}
\put(1074,306.67){\rule{0.241pt}{0.400pt}}
\multiput(1074.00,306.17)(0.500,1.000){2}{\rule{0.120pt}{0.400pt}}
\put(1074.0,293.0){\rule[-0.200pt]{0.400pt}{3.373pt}}
\put(1075,308){\usebox{\plotpoint}}
\put(1075,308){\usebox{\plotpoint}}
\put(1075,308){\usebox{\plotpoint}}
\put(1075,308){\usebox{\plotpoint}}
\put(1075,321.67){\rule{0.241pt}{0.400pt}}
\multiput(1075.00,321.17)(0.500,1.000){2}{\rule{0.120pt}{0.400pt}}
\put(1075.0,308.0){\rule[-0.200pt]{0.400pt}{3.373pt}}
\put(1076,323){\usebox{\plotpoint}}
\put(1076,323){\usebox{\plotpoint}}
\put(1076,323){\usebox{\plotpoint}}
\put(1076,323){\usebox{\plotpoint}}
\put(1076.0,323.0){\rule[-0.200pt]{0.400pt}{3.613pt}}
\put(1076.0,338.0){\usebox{\plotpoint}}
\put(1077.0,338.0){\rule[-0.200pt]{0.400pt}{3.613pt}}
\put(1077.0,353.0){\usebox{\plotpoint}}
\put(1078,367.67){\rule{0.241pt}{0.400pt}}
\multiput(1078.00,367.17)(0.500,1.000){2}{\rule{0.120pt}{0.400pt}}
\put(1078.0,353.0){\rule[-0.200pt]{0.400pt}{3.613pt}}
\put(1079,369){\usebox{\plotpoint}}
\put(1079,369){\usebox{\plotpoint}}
\put(1079,369){\usebox{\plotpoint}}
\put(1079,369){\usebox{\plotpoint}}
\put(1079.0,369.0){\rule[-0.200pt]{0.400pt}{3.613pt}}
\put(1079.0,384.0){\usebox{\plotpoint}}
\put(1080.0,384.0){\rule[-0.200pt]{0.400pt}{3.854pt}}
\put(1080.0,400.0){\usebox{\plotpoint}}
\put(1081.0,400.0){\rule[-0.200pt]{0.400pt}{3.854pt}}
\put(1081.0,416.0){\usebox{\plotpoint}}
\put(1082.0,416.0){\rule[-0.200pt]{0.400pt}{3.854pt}}
\put(1082.0,432.0){\usebox{\plotpoint}}
\put(1083.0,432.0){\rule[-0.200pt]{0.400pt}{3.854pt}}
\put(1083.0,448.0){\usebox{\plotpoint}}
\put(1084.0,448.0){\rule[-0.200pt]{0.400pt}{3.854pt}}
\put(1084.0,464.0){\usebox{\plotpoint}}
\put(1085,478.67){\rule{0.241pt}{0.400pt}}
\multiput(1085.00,478.17)(0.500,1.000){2}{\rule{0.120pt}{0.400pt}}
\put(1085.0,464.0){\rule[-0.200pt]{0.400pt}{3.613pt}}
\put(1086,480){\usebox{\plotpoint}}
\put(1086,480){\usebox{\plotpoint}}
\put(1086,480){\usebox{\plotpoint}}
\put(1086.0,480.0){\rule[-0.200pt]{0.400pt}{3.613pt}}
\put(1086.0,495.0){\usebox{\plotpoint}}
\put(1087,509.67){\rule{0.241pt}{0.400pt}}
\multiput(1087.00,509.17)(0.500,1.000){2}{\rule{0.120pt}{0.400pt}}
\put(1087.0,495.0){\rule[-0.200pt]{0.400pt}{3.613pt}}
\put(1088,511){\usebox{\plotpoint}}
\put(1088,511){\usebox{\plotpoint}}
\put(1088,511){\usebox{\plotpoint}}
\put(1088,511){\usebox{\plotpoint}}
\put(1088,524.67){\rule{0.241pt}{0.400pt}}
\multiput(1088.00,524.17)(0.500,1.000){2}{\rule{0.120pt}{0.400pt}}
\put(1088.0,511.0){\rule[-0.200pt]{0.400pt}{3.373pt}}
\put(1089,526){\usebox{\plotpoint}}
\put(1089,526){\usebox{\plotpoint}}
\put(1089,526){\usebox{\plotpoint}}
\put(1089,526){\usebox{\plotpoint}}
\put(1089.0,526.0){\rule[-0.200pt]{0.400pt}{3.373pt}}
\put(1089.0,540.0){\usebox{\plotpoint}}
\put(1090,553.67){\rule{0.241pt}{0.400pt}}
\multiput(1090.00,553.17)(0.500,1.000){2}{\rule{0.120pt}{0.400pt}}
\put(1090.0,540.0){\rule[-0.200pt]{0.400pt}{3.373pt}}
\put(1091,555){\usebox{\plotpoint}}
\put(1091,555){\usebox{\plotpoint}}
\put(1091,555){\usebox{\plotpoint}}
\put(1091,555){\usebox{\plotpoint}}
\put(1091,555){\usebox{\plotpoint}}
\put(1091,567.67){\rule{0.241pt}{0.400pt}}
\multiput(1091.00,567.17)(0.500,1.000){2}{\rule{0.120pt}{0.400pt}}
\put(1091.0,555.0){\rule[-0.200pt]{0.400pt}{3.132pt}}
\put(1092,569){\usebox{\plotpoint}}
\put(1092,569){\usebox{\plotpoint}}
\put(1092,569){\usebox{\plotpoint}}
\put(1092,569){\usebox{\plotpoint}}
\put(1092,569){\usebox{\plotpoint}}
\put(1092.0,569.0){\rule[-0.200pt]{0.400pt}{3.132pt}}
\put(1092.0,582.0){\usebox{\plotpoint}}
\put(1093.0,582.0){\rule[-0.200pt]{0.400pt}{3.132pt}}
\put(1093.0,595.0){\usebox{\plotpoint}}
\put(1094.0,595.0){\rule[-0.200pt]{0.400pt}{2.891pt}}
\put(1094.0,607.0){\usebox{\plotpoint}}
\put(1095.0,607.0){\rule[-0.200pt]{0.400pt}{2.891pt}}
\put(1095.0,619.0){\usebox{\plotpoint}}
\put(1096.0,619.0){\rule[-0.200pt]{0.400pt}{2.650pt}}
\put(1096.0,630.0){\usebox{\plotpoint}}
\put(1097.0,630.0){\rule[-0.200pt]{0.400pt}{2.650pt}}
\put(1097.0,641.0){\usebox{\plotpoint}}
\put(1098.0,641.0){\rule[-0.200pt]{0.400pt}{2.168pt}}
\put(1098.0,650.0){\usebox{\plotpoint}}
\put(1099.0,650.0){\rule[-0.200pt]{0.400pt}{2.168pt}}
\put(1099.0,659.0){\usebox{\plotpoint}}
\put(1100,666.67){\rule{0.241pt}{0.400pt}}
\multiput(1100.00,666.17)(0.500,1.000){2}{\rule{0.120pt}{0.400pt}}
\put(1100.0,659.0){\rule[-0.200pt]{0.400pt}{1.927pt}}
\put(1101,668){\usebox{\plotpoint}}
\put(1101,668){\usebox{\plotpoint}}
\put(1101,668){\usebox{\plotpoint}}
\put(1101,668){\usebox{\plotpoint}}
\put(1101,668){\usebox{\plotpoint}}
\put(1101,668){\usebox{\plotpoint}}
\put(1101,668){\usebox{\plotpoint}}
\put(1101,668){\usebox{\plotpoint}}
\put(1101,668){\usebox{\plotpoint}}
\put(1101.0,668.0){\rule[-0.200pt]{0.400pt}{1.686pt}}
\put(1101.0,675.0){\usebox{\plotpoint}}
\put(1102,680.67){\rule{0.241pt}{0.400pt}}
\multiput(1102.00,680.17)(0.500,1.000){2}{\rule{0.120pt}{0.400pt}}
\put(1102.0,675.0){\rule[-0.200pt]{0.400pt}{1.445pt}}
\put(1103,682){\usebox{\plotpoint}}
\put(1103,682){\usebox{\plotpoint}}
\put(1103,682){\usebox{\plotpoint}}
\put(1103,682){\usebox{\plotpoint}}
\put(1103,682){\usebox{\plotpoint}}
\put(1103,682){\usebox{\plotpoint}}
\put(1103,682){\usebox{\plotpoint}}
\put(1103,682){\usebox{\plotpoint}}
\put(1103,682){\usebox{\plotpoint}}
\put(1103,682){\usebox{\plotpoint}}
\put(1103,682){\usebox{\plotpoint}}
\put(1103.0,682.0){\rule[-0.200pt]{0.400pt}{1.204pt}}
\put(1103.0,687.0){\usebox{\plotpoint}}
\put(1104.0,687.0){\rule[-0.200pt]{0.400pt}{1.204pt}}
\put(1104.0,692.0){\usebox{\plotpoint}}
\put(1105.0,692.0){\rule[-0.200pt]{0.400pt}{0.964pt}}
\put(1105.0,696.0){\usebox{\plotpoint}}
\put(1106.0,696.0){\rule[-0.200pt]{0.400pt}{0.723pt}}
\put(1106.0,699.0){\usebox{\plotpoint}}
\put(1107.0,699.0){\rule[-0.200pt]{0.400pt}{0.482pt}}
\put(1107.0,701.0){\usebox{\plotpoint}}
\put(1108.0,701.0){\usebox{\plotpoint}}
\put(1108.0,702.0){\usebox{\plotpoint}}
\put(1109.0,702.0){\usebox{\plotpoint}}
\put(1109.0,703.0){\usebox{\plotpoint}}
\put(1110.0,702.0){\usebox{\plotpoint}}
\put(1110.0,702.0){\usebox{\plotpoint}}
\put(1111.0,700.0){\rule[-0.200pt]{0.400pt}{0.482pt}}
\put(1111.0,700.0){\usebox{\plotpoint}}
\put(1112.0,698.0){\rule[-0.200pt]{0.400pt}{0.482pt}}
\put(1112.0,698.0){\usebox{\plotpoint}}
\put(1113.0,694.0){\rule[-0.200pt]{0.400pt}{0.964pt}}
\put(1113.0,694.0){\usebox{\plotpoint}}
\put(1114.0,690.0){\rule[-0.200pt]{0.400pt}{0.964pt}}
\put(1114.0,690.0){\usebox{\plotpoint}}
\put(1115.0,684.0){\rule[-0.200pt]{0.400pt}{1.445pt}}
\put(1115.0,684.0){\usebox{\plotpoint}}
\put(1116.0,678.0){\rule[-0.200pt]{0.400pt}{1.445pt}}
\put(1116.0,678.0){\usebox{\plotpoint}}
\put(1117.0,671.0){\rule[-0.200pt]{0.400pt}{1.686pt}}
\put(1117.0,671.0){\usebox{\plotpoint}}
\put(1118,662.67){\rule{0.241pt}{0.400pt}}
\multiput(1118.00,663.17)(0.500,-1.000){2}{\rule{0.120pt}{0.400pt}}
\put(1118.0,664.0){\rule[-0.200pt]{0.400pt}{1.686pt}}
\put(1119,663){\usebox{\plotpoint}}
\put(1119,663){\usebox{\plotpoint}}
\put(1119,663){\usebox{\plotpoint}}
\put(1119,663){\usebox{\plotpoint}}
\put(1119,663){\usebox{\plotpoint}}
\put(1119,663){\usebox{\plotpoint}}
\put(1119,663){\usebox{\plotpoint}}
\put(1119,663){\usebox{\plotpoint}}
\put(1119.0,655.0){\rule[-0.200pt]{0.400pt}{1.927pt}}
\put(1119.0,655.0){\usebox{\plotpoint}}
\put(1120,644.67){\rule{0.241pt}{0.400pt}}
\multiput(1120.00,645.17)(0.500,-1.000){2}{\rule{0.120pt}{0.400pt}}
\put(1120.0,646.0){\rule[-0.200pt]{0.400pt}{2.168pt}}
\put(1121,645){\usebox{\plotpoint}}
\put(1121,645){\usebox{\plotpoint}}
\put(1121,645){\usebox{\plotpoint}}
\put(1121,645){\usebox{\plotpoint}}
\put(1121,645){\usebox{\plotpoint}}
\put(1121,645){\usebox{\plotpoint}}
\put(1121,645){\usebox{\plotpoint}}
\put(1121,634.67){\rule{0.241pt}{0.400pt}}
\multiput(1121.00,635.17)(0.500,-1.000){2}{\rule{0.120pt}{0.400pt}}
\put(1121.0,636.0){\rule[-0.200pt]{0.400pt}{2.168pt}}
\put(1122,635){\usebox{\plotpoint}}
\put(1122,635){\usebox{\plotpoint}}
\put(1122,635){\usebox{\plotpoint}}
\put(1122,635){\usebox{\plotpoint}}
\put(1122,635){\usebox{\plotpoint}}
\put(1122,635){\usebox{\plotpoint}}
\put(1122.0,625.0){\rule[-0.200pt]{0.400pt}{2.409pt}}
\put(1122.0,625.0){\usebox{\plotpoint}}
\put(1123.0,613.0){\rule[-0.200pt]{0.400pt}{2.891pt}}
\put(1123.0,613.0){\usebox{\plotpoint}}
\put(1124.0,601.0){\rule[-0.200pt]{0.400pt}{2.891pt}}
\put(1124.0,601.0){\usebox{\plotpoint}}
\put(1125.0,588.0){\rule[-0.200pt]{0.400pt}{3.132pt}}
\put(1125.0,588.0){\usebox{\plotpoint}}
\put(1126.0,575.0){\rule[-0.200pt]{0.400pt}{3.132pt}}
\put(1126.0,575.0){\usebox{\plotpoint}}
\put(1127,560.67){\rule{0.241pt}{0.400pt}}
\multiput(1127.00,561.17)(0.500,-1.000){2}{\rule{0.120pt}{0.400pt}}
\put(1127.0,562.0){\rule[-0.200pt]{0.400pt}{3.132pt}}
\put(1128,561){\usebox{\plotpoint}}
\put(1128,561){\usebox{\plotpoint}}
\put(1128,561){\usebox{\plotpoint}}
\put(1128,561){\usebox{\plotpoint}}
\put(1128,561){\usebox{\plotpoint}}
\put(1128.0,547.0){\rule[-0.200pt]{0.400pt}{3.373pt}}
\put(1128.0,547.0){\usebox{\plotpoint}}
\put(1129.0,533.0){\rule[-0.200pt]{0.400pt}{3.373pt}}
\put(1129.0,533.0){\usebox{\plotpoint}}
\put(1130.0,518.0){\rule[-0.200pt]{0.400pt}{3.613pt}}
\put(1130.0,518.0){\usebox{\plotpoint}}
\put(1131.0,503.0){\rule[-0.200pt]{0.400pt}{3.613pt}}
\put(1131.0,503.0){\usebox{\plotpoint}}
\put(1132,486.67){\rule{0.241pt}{0.400pt}}
\multiput(1132.00,487.17)(0.500,-1.000){2}{\rule{0.120pt}{0.400pt}}
\put(1132.0,488.0){\rule[-0.200pt]{0.400pt}{3.613pt}}
\put(1133,487){\usebox{\plotpoint}}
\put(1133,487){\usebox{\plotpoint}}
\put(1133,487){\usebox{\plotpoint}}
\put(1133.0,472.0){\rule[-0.200pt]{0.400pt}{3.613pt}}
\put(1133.0,472.0){\usebox{\plotpoint}}
\put(1134.0,456.0){\rule[-0.200pt]{0.400pt}{3.854pt}}
\put(1134.0,456.0){\usebox{\plotpoint}}
\put(1135.0,440.0){\rule[-0.200pt]{0.400pt}{3.854pt}}
\put(1135.0,440.0){\usebox{\plotpoint}}
\put(1136.0,424.0){\rule[-0.200pt]{0.400pt}{3.854pt}}
\put(1136.0,424.0){\usebox{\plotpoint}}
\put(1137.0,408.0){\rule[-0.200pt]{0.400pt}{3.854pt}}
\put(1137.0,408.0){\usebox{\plotpoint}}
\put(1138.0,392.0){\rule[-0.200pt]{0.400pt}{3.854pt}}
\put(1138.0,392.0){\usebox{\plotpoint}}
\put(1139,375.67){\rule{0.241pt}{0.400pt}}
\multiput(1139.00,376.17)(0.500,-1.000){2}{\rule{0.120pt}{0.400pt}}
\put(1139.0,377.0){\rule[-0.200pt]{0.400pt}{3.613pt}}
\put(1140,376){\usebox{\plotpoint}}
\put(1140,376){\usebox{\plotpoint}}
\put(1140,376){\usebox{\plotpoint}}
\put(1140.0,361.0){\rule[-0.200pt]{0.400pt}{3.613pt}}
\put(1140.0,361.0){\usebox{\plotpoint}}
\put(1141.0,345.0){\rule[-0.200pt]{0.400pt}{3.854pt}}
\put(1141.0,345.0){\usebox{\plotpoint}}
\put(1142.0,330.0){\rule[-0.200pt]{0.400pt}{3.613pt}}
\put(1142.0,330.0){\usebox{\plotpoint}}
\put(1143.0,315.0){\rule[-0.200pt]{0.400pt}{3.613pt}}
\put(1143.0,315.0){\usebox{\plotpoint}}
\put(1144.0,300.0){\rule[-0.200pt]{0.400pt}{3.613pt}}
\put(1144.0,300.0){\usebox{\plotpoint}}
\put(1145.0,286.0){\rule[-0.200pt]{0.400pt}{3.373pt}}
\put(1145.0,286.0){\usebox{\plotpoint}}
\put(1146,270.67){\rule{0.241pt}{0.400pt}}
\multiput(1146.00,271.17)(0.500,-1.000){2}{\rule{0.120pt}{0.400pt}}
\put(1146.0,272.0){\rule[-0.200pt]{0.400pt}{3.373pt}}
\put(1147,271){\usebox{\plotpoint}}
\put(1147,271){\usebox{\plotpoint}}
\put(1147,271){\usebox{\plotpoint}}
\put(1147,271){\usebox{\plotpoint}}
\put(1147,271){\usebox{\plotpoint}}
\put(1147.0,258.0){\rule[-0.200pt]{0.400pt}{3.132pt}}
\put(1147.0,258.0){\usebox{\plotpoint}}
\put(1148.0,245.0){\rule[-0.200pt]{0.400pt}{3.132pt}}
\put(1148.0,245.0){\usebox{\plotpoint}}
\put(1149.0,232.0){\rule[-0.200pt]{0.400pt}{3.132pt}}
\put(1149.0,232.0){\usebox{\plotpoint}}
\put(1150,218.67){\rule{0.241pt}{0.400pt}}
\multiput(1150.00,219.17)(0.500,-1.000){2}{\rule{0.120pt}{0.400pt}}
\put(1150.0,220.0){\rule[-0.200pt]{0.400pt}{2.891pt}}
\put(1151,219){\usebox{\plotpoint}}
\put(1151,219){\usebox{\plotpoint}}
\put(1151,219){\usebox{\plotpoint}}
\put(1151,219){\usebox{\plotpoint}}
\put(1151,219){\usebox{\plotpoint}}
\put(1151.0,208.0){\rule[-0.200pt]{0.400pt}{2.650pt}}
\put(1151.0,208.0){\usebox{\plotpoint}}
\put(1152.0,196.0){\rule[-0.200pt]{0.400pt}{2.891pt}}
\put(1152.0,196.0){\usebox{\plotpoint}}
\put(1153,184.67){\rule{0.241pt}{0.400pt}}
\multiput(1153.00,185.17)(0.500,-1.000){2}{\rule{0.120pt}{0.400pt}}
\put(1153.0,186.0){\rule[-0.200pt]{0.400pt}{2.409pt}}
\put(1154,185){\usebox{\plotpoint}}
\put(1154,185){\usebox{\plotpoint}}
\put(1154,185){\usebox{\plotpoint}}
\put(1154,185){\usebox{\plotpoint}}
\put(1154,185){\usebox{\plotpoint}}
\put(1154,185){\usebox{\plotpoint}}
\put(1154.0,175.0){\rule[-0.200pt]{0.400pt}{2.409pt}}
\put(1154.0,175.0){\usebox{\plotpoint}}
\put(1155,164.67){\rule{0.241pt}{0.400pt}}
\multiput(1155.00,165.17)(0.500,-1.000){2}{\rule{0.120pt}{0.400pt}}
\put(1155.0,166.0){\rule[-0.200pt]{0.400pt}{2.168pt}}
\put(1156,165){\usebox{\plotpoint}}
\put(1156,165){\usebox{\plotpoint}}
\put(1156,165){\usebox{\plotpoint}}
\put(1156,165){\usebox{\plotpoint}}
\put(1156,165){\usebox{\plotpoint}}
\put(1156,165){\usebox{\plotpoint}}
\put(1156,165){\usebox{\plotpoint}}
\put(1156,165){\usebox{\plotpoint}}
\put(1156.0,156.0){\rule[-0.200pt]{0.400pt}{2.168pt}}
\put(1156.0,156.0){\usebox{\plotpoint}}
\put(1157.0,148.0){\rule[-0.200pt]{0.400pt}{1.927pt}}
\put(1157.0,148.0){\usebox{\plotpoint}}
\put(1158.0,140.0){\rule[-0.200pt]{0.400pt}{1.927pt}}
\put(1158.0,140.0){\usebox{\plotpoint}}
\put(1159.0,132.0){\rule[-0.200pt]{0.400pt}{1.927pt}}
\put(1159.0,132.0){\usebox{\plotpoint}}
\put(1160,124.67){\rule{0.241pt}{0.400pt}}
\multiput(1160.00,125.17)(0.500,-1.000){2}{\rule{0.120pt}{0.400pt}}
\put(1160.0,126.0){\rule[-0.200pt]{0.400pt}{1.445pt}}
\put(1161,125){\usebox{\plotpoint}}
\put(1161,125){\usebox{\plotpoint}}
\put(1161,125){\usebox{\plotpoint}}
\put(1161,125){\usebox{\plotpoint}}
\put(1161,125){\usebox{\plotpoint}}
\put(1161,125){\usebox{\plotpoint}}
\put(1161,125){\usebox{\plotpoint}}
\put(1161,125){\usebox{\plotpoint}}
\put(1161,125){\usebox{\plotpoint}}
\put(1161,125){\usebox{\plotpoint}}
\put(1161.0,119.0){\rule[-0.200pt]{0.400pt}{1.445pt}}
\put(1161.0,119.0){\usebox{\plotpoint}}
\put(1162.0,113.0){\rule[-0.200pt]{0.400pt}{1.445pt}}
\put(1162.0,113.0){\usebox{\plotpoint}}
\put(1163.0,108.0){\rule[-0.200pt]{0.400pt}{1.204pt}}
\put(1163.0,108.0){\usebox{\plotpoint}}
\put(1164,102.67){\rule{0.241pt}{0.400pt}}
\multiput(1164.00,103.17)(0.500,-1.000){2}{\rule{0.120pt}{0.400pt}}
\put(1164.0,104.0){\rule[-0.200pt]{0.400pt}{0.964pt}}
\put(1165,103){\usebox{\plotpoint}}
\put(1165,103){\usebox{\plotpoint}}
\put(1165,103){\usebox{\plotpoint}}
\put(1165,103){\usebox{\plotpoint}}
\put(1165,103){\usebox{\plotpoint}}
\put(1165,103){\usebox{\plotpoint}}
\put(1165,103){\usebox{\plotpoint}}
\put(1165,103){\usebox{\plotpoint}}
\put(1165,103){\usebox{\plotpoint}}
\put(1165,103){\usebox{\plotpoint}}
\put(1165,103){\usebox{\plotpoint}}
\put(1165,103){\usebox{\plotpoint}}
\put(1165,103){\usebox{\plotpoint}}
\put(1165,103){\usebox{\plotpoint}}
\put(1165,103){\usebox{\plotpoint}}
\put(1165,103){\usebox{\plotpoint}}
\put(1165.0,99.0){\rule[-0.200pt]{0.400pt}{0.964pt}}
\put(1165.0,99.0){\usebox{\plotpoint}}
\put(1166.0,96.0){\rule[-0.200pt]{0.400pt}{0.723pt}}
\put(1166.0,96.0){\usebox{\plotpoint}}
\put(1167.0,92.0){\rule[-0.200pt]{0.400pt}{0.964pt}}
\put(1167.0,92.0){\usebox{\plotpoint}}
\put(1168.0,90.0){\rule[-0.200pt]{0.400pt}{0.482pt}}
\put(1168.0,90.0){\usebox{\plotpoint}}
\put(1169.0,87.0){\rule[-0.200pt]{0.400pt}{0.723pt}}
\put(1169.0,87.0){\usebox{\plotpoint}}
\put(1170.0,86.0){\usebox{\plotpoint}}
\put(1170.0,86.0){\usebox{\plotpoint}}
\put(1171.0,84.0){\rule[-0.200pt]{0.400pt}{0.482pt}}
\put(1171.0,84.0){\usebox{\plotpoint}}
\put(1172.0,83.0){\usebox{\plotpoint}}
\put(1172.0,83.0){\usebox{\plotpoint}}
\put(1173.0,82.0){\usebox{\plotpoint}}
\put(1173.0,82.0){\rule[-0.200pt]{0.964pt}{0.400pt}}
\put(1177.0,82.0){\usebox{\plotpoint}}
\put(1177.0,83.0){\rule[-0.200pt]{0.482pt}{0.400pt}}
\put(1179.0,83.0){\usebox{\plotpoint}}
\put(1179.0,84.0){\usebox{\plotpoint}}
\put(1180.0,84.0){\usebox{\plotpoint}}
\put(1180.0,85.0){\usebox{\plotpoint}}
\put(1181.0,85.0){\rule[-0.200pt]{0.400pt}{0.482pt}}
\put(1181.0,87.0){\usebox{\plotpoint}}
\put(1182.0,87.0){\usebox{\plotpoint}}
\put(1182.0,88.0){\usebox{\plotpoint}}
\put(1183.0,88.0){\usebox{\plotpoint}}
\put(1183.0,89.0){\usebox{\plotpoint}}
\put(1184.0,89.0){\rule[-0.200pt]{0.400pt}{0.482pt}}
\put(1184.0,91.0){\usebox{\plotpoint}}
\put(1185.0,91.0){\rule[-0.200pt]{0.400pt}{0.482pt}}
\put(1185.0,93.0){\usebox{\plotpoint}}
\put(1186.0,93.0){\usebox{\plotpoint}}
\put(1186.0,94.0){\usebox{\plotpoint}}
\put(1187.0,94.0){\rule[-0.200pt]{0.400pt}{0.482pt}}
\put(1187.0,96.0){\usebox{\plotpoint}}
\put(1188.0,96.0){\usebox{\plotpoint}}
\put(1188.0,97.0){\usebox{\plotpoint}}
\put(1189.0,97.0){\rule[-0.200pt]{0.400pt}{0.482pt}}
\put(1189.0,99.0){\usebox{\plotpoint}}
\put(1190.0,99.0){\rule[-0.200pt]{0.400pt}{0.482pt}}
\put(1190.0,101.0){\usebox{\plotpoint}}
\put(1191.0,101.0){\usebox{\plotpoint}}
\put(1191.0,102.0){\usebox{\plotpoint}}
\put(1192.0,102.0){\rule[-0.200pt]{0.400pt}{0.482pt}}
\put(1192.0,104.0){\usebox{\plotpoint}}
\put(1193.0,104.0){\usebox{\plotpoint}}
\put(1193.0,105.0){\usebox{\plotpoint}}
\put(1194.0,105.0){\usebox{\plotpoint}}
\put(1194.0,106.0){\usebox{\plotpoint}}
\put(1195.0,106.0){\usebox{\plotpoint}}
\put(1195.0,107.0){\usebox{\plotpoint}}
\put(1196.0,107.0){\usebox{\plotpoint}}
\put(1196.0,108.0){\usebox{\plotpoint}}
\put(1197.0,108.0){\usebox{\plotpoint}}
\put(1197.0,109.0){\usebox{\plotpoint}}
\put(1198.0,109.0){\usebox{\plotpoint}}
\put(1198.0,110.0){\usebox{\plotpoint}}
\put(1199.0,110.0){\usebox{\plotpoint}}
\put(1199.0,111.0){\rule[-0.200pt]{0.723pt}{0.400pt}}
\put(1202.0,111.0){\usebox{\plotpoint}}
\put(1202.0,112.0){\rule[-0.200pt]{0.964pt}{0.400pt}}
\put(1206.0,111.0){\usebox{\plotpoint}}
\put(1206.0,111.0){\rule[-0.200pt]{0.482pt}{0.400pt}}
\put(1208.0,110.0){\usebox{\plotpoint}}
\put(1208.0,110.0){\rule[-0.200pt]{0.482pt}{0.400pt}}
\put(1210.0,109.0){\usebox{\plotpoint}}
\put(1210.0,109.0){\usebox{\plotpoint}}
\put(1211.0,108.0){\usebox{\plotpoint}}
\put(1211.0,108.0){\usebox{\plotpoint}}
\put(1212.0,107.0){\usebox{\plotpoint}}
\put(1212.0,107.0){\usebox{\plotpoint}}
\put(1213.0,106.0){\usebox{\plotpoint}}
\put(1213.0,106.0){\usebox{\plotpoint}}
\put(1214.0,105.0){\usebox{\plotpoint}}
\put(1214.0,105.0){\usebox{\plotpoint}}
\put(1215.0,104.0){\usebox{\plotpoint}}
\put(1215.0,104.0){\usebox{\plotpoint}}
\put(1216.0,103.0){\usebox{\plotpoint}}
\put(1216.0,103.0){\usebox{\plotpoint}}
\put(1217.0,101.0){\rule[-0.200pt]{0.400pt}{0.482pt}}
\put(1217.0,101.0){\usebox{\plotpoint}}
\put(1218.0,100.0){\usebox{\plotpoint}}
\put(1218.0,100.0){\usebox{\plotpoint}}
\put(1219.0,99.0){\usebox{\plotpoint}}
\put(1219.0,99.0){\usebox{\plotpoint}}
\put(1220.0,98.0){\usebox{\plotpoint}}
\put(1220.0,98.0){\usebox{\plotpoint}}
\put(1221.0,96.0){\rule[-0.200pt]{0.400pt}{0.482pt}}
\put(1221.0,96.0){\usebox{\plotpoint}}
\put(1222.0,95.0){\usebox{\plotpoint}}
\put(1222.0,95.0){\usebox{\plotpoint}}
\put(1223.0,94.0){\usebox{\plotpoint}}
\put(1223.0,94.0){\usebox{\plotpoint}}
\put(1224.0,93.0){\usebox{\plotpoint}}
\put(1224.0,93.0){\usebox{\plotpoint}}
\put(1225.0,92.0){\usebox{\plotpoint}}
\put(1225.0,92.0){\usebox{\plotpoint}}
\put(1226.0,90.0){\rule[-0.200pt]{0.400pt}{0.482pt}}
\put(1226.0,90.0){\usebox{\plotpoint}}
\put(1227.0,89.0){\usebox{\plotpoint}}
\put(1227.0,89.0){\usebox{\plotpoint}}
\put(1228.0,88.0){\usebox{\plotpoint}}
\put(1228.0,88.0){\usebox{\plotpoint}}
\put(1229.0,87.0){\usebox{\plotpoint}}
\put(1230,85.67){\rule{0.241pt}{0.400pt}}
\multiput(1230.00,86.17)(0.500,-1.000){2}{\rule{0.120pt}{0.400pt}}
\put(1229.0,87.0){\usebox{\plotpoint}}
\put(1231,86){\usebox{\plotpoint}}
\put(1231,86){\usebox{\plotpoint}}
\put(1231,86){\usebox{\plotpoint}}
\put(1231,86){\usebox{\plotpoint}}
\put(1231,86){\usebox{\plotpoint}}
\put(1231,86){\usebox{\plotpoint}}
\put(1231,86){\usebox{\plotpoint}}
\put(1231,86){\usebox{\plotpoint}}
\put(1231,86){\usebox{\plotpoint}}
\put(1231,86){\usebox{\plotpoint}}
\put(1231,86){\usebox{\plotpoint}}
\put(1231,86){\usebox{\plotpoint}}
\put(1231,86){\usebox{\plotpoint}}
\put(1231,86){\usebox{\plotpoint}}
\put(1231,86){\usebox{\plotpoint}}
\put(1231,86){\usebox{\plotpoint}}
\put(1231,86){\usebox{\plotpoint}}
\put(1231,86){\usebox{\plotpoint}}
\put(1231,86){\usebox{\plotpoint}}
\put(1231,86){\usebox{\plotpoint}}
\put(1231,86){\usebox{\plotpoint}}
\put(1231,86){\usebox{\plotpoint}}
\put(1231,86){\usebox{\plotpoint}}
\put(1231,86){\usebox{\plotpoint}}
\put(1231,86){\usebox{\plotpoint}}
\put(1231,86){\usebox{\plotpoint}}
\put(1231,86){\usebox{\plotpoint}}
\put(1231,86){\usebox{\plotpoint}}
\put(1231,86){\usebox{\plotpoint}}
\put(1231,86){\usebox{\plotpoint}}
\put(1231,86){\usebox{\plotpoint}}
\put(1231,86){\usebox{\plotpoint}}
\put(1231,86){\usebox{\plotpoint}}
\put(1231,86){\usebox{\plotpoint}}
\put(1231,86){\usebox{\plotpoint}}
\put(1231,86){\usebox{\plotpoint}}
\put(1231,86){\usebox{\plotpoint}}
\put(1231,86){\usebox{\plotpoint}}
\put(1231,86){\usebox{\plotpoint}}
\put(1231,86){\usebox{\plotpoint}}
\put(1231,86){\usebox{\plotpoint}}
\put(1231,86){\usebox{\plotpoint}}
\put(1231,86){\usebox{\plotpoint}}
\put(1231,86){\usebox{\plotpoint}}
\put(1231,86){\usebox{\plotpoint}}
\put(1231,86){\usebox{\plotpoint}}
\put(1231,86){\usebox{\plotpoint}}
\put(1231,86){\usebox{\plotpoint}}
\put(1231,86){\usebox{\plotpoint}}
\put(1231,86){\usebox{\plotpoint}}
\put(1231,86){\usebox{\plotpoint}}
\put(1231,86){\usebox{\plotpoint}}
\put(1231,86){\usebox{\plotpoint}}
\put(1231,86){\usebox{\plotpoint}}
\put(1231,86){\usebox{\plotpoint}}
\put(1231,86){\usebox{\plotpoint}}
\put(1231,86){\usebox{\plotpoint}}
\put(1231,86){\usebox{\plotpoint}}
\put(1231,86){\usebox{\plotpoint}}
\put(1231,86){\usebox{\plotpoint}}
\put(1231,86){\usebox{\plotpoint}}
\put(1231,86){\usebox{\plotpoint}}
\put(1231,86){\usebox{\plotpoint}}
\put(1231,86){\usebox{\plotpoint}}
\put(1231,86){\usebox{\plotpoint}}
\put(1231,86){\usebox{\plotpoint}}
\put(1231,86){\usebox{\plotpoint}}
\put(1231,86){\usebox{\plotpoint}}
\put(1231,86){\usebox{\plotpoint}}
\put(1231,86){\usebox{\plotpoint}}
\put(1231,86){\usebox{\plotpoint}}
\put(1231,86){\usebox{\plotpoint}}
\put(1231,86){\usebox{\plotpoint}}
\put(1231,86){\usebox{\plotpoint}}
\put(1231,86){\usebox{\plotpoint}}
\put(1231.0,86.0){\usebox{\plotpoint}}
\put(1232.0,85.0){\usebox{\plotpoint}}
\put(1232.0,85.0){\usebox{\plotpoint}}
\put(1233.0,84.0){\usebox{\plotpoint}}
\put(1233.0,84.0){\rule[-0.200pt]{0.482pt}{0.400pt}}
\put(1235.0,83.0){\usebox{\plotpoint}}
\put(1235.0,83.0){\rule[-0.200pt]{0.723pt}{0.400pt}}
\put(1238.0,82.0){\usebox{\plotpoint}}
\put(1238.0,82.0){\rule[-0.200pt]{1.686pt}{0.400pt}}
\put(1245.0,82.0){\usebox{\plotpoint}}
\put(1245.0,83.0){\rule[-0.200pt]{0.723pt}{0.400pt}}
\put(1248.0,83.0){\usebox{\plotpoint}}
\put(1248.0,84.0){\rule[-0.200pt]{0.482pt}{0.400pt}}
\put(1250.0,84.0){\usebox{\plotpoint}}
\put(1250.0,85.0){\rule[-0.200pt]{0.482pt}{0.400pt}}
\put(1252.0,85.0){\usebox{\plotpoint}}
\put(1252.0,86.0){\rule[-0.200pt]{0.482pt}{0.400pt}}
\put(1254.0,86.0){\usebox{\plotpoint}}
\put(1254.0,87.0){\rule[-0.200pt]{0.482pt}{0.400pt}}
\put(1256.0,87.0){\usebox{\plotpoint}}
\put(1256.0,88.0){\rule[-0.200pt]{0.482pt}{0.400pt}}
\put(1258.0,88.0){\usebox{\plotpoint}}
\put(1258.0,89.0){\rule[-0.200pt]{0.482pt}{0.400pt}}
\put(1260.0,89.0){\usebox{\plotpoint}}
\put(1260.0,90.0){\rule[-0.200pt]{0.482pt}{0.400pt}}
\put(1262.0,90.0){\usebox{\plotpoint}}
\put(1262.0,91.0){\rule[-0.200pt]{0.723pt}{0.400pt}}
\put(1265.0,91.0){\usebox{\plotpoint}}
\put(1265.0,92.0){\rule[-0.200pt]{0.964pt}{0.400pt}}
\put(1269.0,92.0){\usebox{\plotpoint}}
\put(1269.0,93.0){\rule[-0.200pt]{1.204pt}{0.400pt}}
\put(1274.0,92.0){\usebox{\plotpoint}}
\put(1274.0,92.0){\rule[-0.200pt]{1.204pt}{0.400pt}}
\put(1279.0,91.0){\usebox{\plotpoint}}
\put(1279.0,91.0){\rule[-0.200pt]{0.482pt}{0.400pt}}
\put(1281.0,90.0){\usebox{\plotpoint}}
\put(1281.0,90.0){\rule[-0.200pt]{0.723pt}{0.400pt}}
\put(1284.0,89.0){\usebox{\plotpoint}}
\put(1284.0,89.0){\rule[-0.200pt]{0.482pt}{0.400pt}}
\put(1286.0,88.0){\usebox{\plotpoint}}
\put(1286.0,88.0){\rule[-0.200pt]{0.482pt}{0.400pt}}
\put(1288.0,87.0){\usebox{\plotpoint}}
\put(1288.0,87.0){\rule[-0.200pt]{0.482pt}{0.400pt}}
\put(1290.0,86.0){\usebox{\plotpoint}}
\put(1292,84.67){\rule{0.241pt}{0.400pt}}
\multiput(1292.00,85.17)(0.500,-1.000){2}{\rule{0.120pt}{0.400pt}}
\put(1290.0,86.0){\rule[-0.200pt]{0.482pt}{0.400pt}}
\put(1293,85){\usebox{\plotpoint}}
\put(1293,85){\usebox{\plotpoint}}
\put(1293,85){\usebox{\plotpoint}}
\put(1293,85){\usebox{\plotpoint}}
\put(1293,85){\usebox{\plotpoint}}
\put(1293,85){\usebox{\plotpoint}}
\put(1293,85){\usebox{\plotpoint}}
\put(1293,85){\usebox{\plotpoint}}
\put(1293,85){\usebox{\plotpoint}}
\put(1293,85){\usebox{\plotpoint}}
\put(1293,85){\usebox{\plotpoint}}
\put(1293,85){\usebox{\plotpoint}}
\put(1293,85){\usebox{\plotpoint}}
\put(1293,85){\usebox{\plotpoint}}
\put(1293,85){\usebox{\plotpoint}}
\put(1293,85){\usebox{\plotpoint}}
\put(1293,85){\usebox{\plotpoint}}
\put(1293,85){\usebox{\plotpoint}}
\put(1293,85){\usebox{\plotpoint}}
\put(1293,85){\usebox{\plotpoint}}
\put(1293,85){\usebox{\plotpoint}}
\put(1293,85){\usebox{\plotpoint}}
\put(1293,85){\usebox{\plotpoint}}
\put(1293,85){\usebox{\plotpoint}}
\put(1293,85){\usebox{\plotpoint}}
\put(1293,85){\usebox{\plotpoint}}
\put(1293,85){\usebox{\plotpoint}}
\put(1293,85){\usebox{\plotpoint}}
\put(1293,85){\usebox{\plotpoint}}
\put(1293,85){\usebox{\plotpoint}}
\put(1293,85){\usebox{\plotpoint}}
\put(1293,85){\usebox{\plotpoint}}
\put(1293,85){\usebox{\plotpoint}}
\put(1293,85){\usebox{\plotpoint}}
\put(1293,85){\usebox{\plotpoint}}
\put(1293,85){\usebox{\plotpoint}}
\put(1293,85){\usebox{\plotpoint}}
\put(1293,85){\usebox{\plotpoint}}
\put(1293,85){\usebox{\plotpoint}}
\put(1293,85){\usebox{\plotpoint}}
\put(1293,85){\usebox{\plotpoint}}
\put(1293,85){\usebox{\plotpoint}}
\put(1293,85){\usebox{\plotpoint}}
\put(1293,85){\usebox{\plotpoint}}
\put(1293,85){\usebox{\plotpoint}}
\put(1293,85){\usebox{\plotpoint}}
\put(1293,85){\usebox{\plotpoint}}
\put(1293,85){\usebox{\plotpoint}}
\put(1293,85){\usebox{\plotpoint}}
\put(1293,85){\usebox{\plotpoint}}
\put(1293,85){\usebox{\plotpoint}}
\put(1293,85){\usebox{\plotpoint}}
\put(1293,85){\usebox{\plotpoint}}
\put(1293,85){\usebox{\plotpoint}}
\put(1293,85){\usebox{\plotpoint}}
\put(1293,85){\usebox{\plotpoint}}
\put(1293,85){\usebox{\plotpoint}}
\put(1293,85){\usebox{\plotpoint}}
\put(1293,85){\usebox{\plotpoint}}
\put(1293,85){\usebox{\plotpoint}}
\put(1293,85){\usebox{\plotpoint}}
\put(1293,85){\usebox{\plotpoint}}
\put(1293,85){\usebox{\plotpoint}}
\put(1293,85){\usebox{\plotpoint}}
\put(1293,85){\usebox{\plotpoint}}
\put(1293,85){\usebox{\plotpoint}}
\put(1293,85){\usebox{\plotpoint}}
\put(1293,85){\usebox{\plotpoint}}
\put(1293,85){\usebox{\plotpoint}}
\put(1293,85){\usebox{\plotpoint}}
\put(1293,85){\usebox{\plotpoint}}
\put(1293,85){\usebox{\plotpoint}}
\put(1293,85){\usebox{\plotpoint}}
\put(1293,85){\usebox{\plotpoint}}
\put(1293,85){\usebox{\plotpoint}}
\put(1293.0,85.0){\rule[-0.200pt]{0.482pt}{0.400pt}}
\put(1295.0,84.0){\usebox{\plotpoint}}
\put(1295.0,84.0){\rule[-0.200pt]{0.723pt}{0.400pt}}
\put(1298.0,83.0){\usebox{\plotpoint}}
\put(1298.0,83.0){\rule[-0.200pt]{0.964pt}{0.400pt}}
\put(1302.0,82.0){\usebox{\plotpoint}}
\put(1302.0,82.0){\rule[-0.200pt]{2.650pt}{0.400pt}}
\put(1313.0,82.0){\usebox{\plotpoint}}
\put(1313.0,83.0){\rule[-0.200pt]{0.964pt}{0.400pt}}
\put(1317.0,83.0){\usebox{\plotpoint}}
\put(1317.0,84.0){\rule[-0.200pt]{0.964pt}{0.400pt}}
\put(1321.0,84.0){\usebox{\plotpoint}}
\put(1321.0,85.0){\rule[-0.200pt]{0.964pt}{0.400pt}}
\put(1325.0,85.0){\usebox{\plotpoint}}
\put(1325.0,86.0){\rule[-0.200pt]{0.964pt}{0.400pt}}
\put(1329.0,86.0){\usebox{\plotpoint}}
\put(1329.0,87.0){\rule[-0.200pt]{1.927pt}{0.400pt}}
\put(1337.0,87.0){\usebox{\plotpoint}}
\put(1337.0,88.0){\rule[-0.200pt]{0.482pt}{0.400pt}}
\put(1339.0,87.0){\usebox{\plotpoint}}
\put(1339.0,87.0){\rule[-0.200pt]{2.168pt}{0.400pt}}
\put(1348.0,86.0){\usebox{\plotpoint}}
\put(1348.0,86.0){\rule[-0.200pt]{0.964pt}{0.400pt}}
\put(1352.0,85.0){\usebox{\plotpoint}}
\put(1352.0,85.0){\rule[-0.200pt]{0.964pt}{0.400pt}}
\put(1356.0,84.0){\usebox{\plotpoint}}
\put(1356.0,84.0){\rule[-0.200pt]{0.964pt}{0.400pt}}
\put(1360.0,83.0){\usebox{\plotpoint}}
\put(1360.0,83.0){\rule[-0.200pt]{1.445pt}{0.400pt}}
\put(1366.0,82.0){\usebox{\plotpoint}}
\put(1366.0,82.0){\rule[-0.200pt]{3.613pt}{0.400pt}}
\put(1381.0,82.0){\usebox{\plotpoint}}
\put(1381.0,83.0){\rule[-0.200pt]{1.445pt}{0.400pt}}
\put(1387.0,83.0){\usebox{\plotpoint}}
\put(1387.0,84.0){\rule[-0.200pt]{1.445pt}{0.400pt}}
\put(1393.0,84.0){\usebox{\plotpoint}}
\put(1393.0,85.0){\rule[-0.200pt]{5.541pt}{0.400pt}}
\put(1416.0,84.0){\usebox{\plotpoint}}
\put(1416.0,84.0){\rule[-0.200pt]{1.686pt}{0.400pt}}
\put(1423.0,83.0){\usebox{\plotpoint}}
\put(1423.0,83.0){\rule[-0.200pt]{1.686pt}{0.400pt}}
\put(1430.0,82.0){\usebox{\plotpoint}}
\put(1430.0,82.0){\rule[-0.200pt]{2.168pt}{0.400pt}}
\put(120.0,82.0){\rule[-0.200pt]{0.400pt}{167.425pt}}
\put(120.0,82.0){\rule[-0.200pt]{317.747pt}{0.400pt}}
\put(1439.0,82.0){\rule[-0.200pt]{0.400pt}{167.425pt}}
\put(120.0,777.0){\rule[-0.200pt]{317.747pt}{0.400pt}}
\end{picture}

  \caption{Plot der Übergangswahrscheinlichkeit über \(\omega_{ni}\)  für ein festes \(t\). Man erhält zwei Peaks die bei \(\pm \omega\)}
 \label{fig:2}
\end{figure}

In der Gleichung (\ref{eq:49}) sieht man dass der dritte Term, der sowieso ein oszillierender Term ist, für beiden Resonanzbedingung \(\omega_{ni}\approx \pm \omega\) verschwindet. Also können wir die Übergangswahrscheinlichkeit näherungsweise schreiben


\begin{align}
  \label{eq:50}
   P(i\rightarrow n) &=  \frac{1}{\hbar^2}  \Big|V_{ni} \frac{2}{\omega_{ni}+\omega}\sin\left(\frac{(\omega_{ni}+\omega) t}{2}\right)e^{\frac{it(\omega_{ni}+\omega)}{2}}\Big|^2 +  \Big|V^\dagger_{ni} \frac{2}{\omega_{ni}-\omega}\sin\left(\frac{(\omega_{ni}-\omega) t}{2}\right)e^{\frac{it(\omega_{ni}-\omega)}{2} }\Big|^2 \notag\\
&=  \frac{4}{\hbar^2} \left[  \frac{|V_{ni}|^2}{(\omega_{ni}+\omega)^2}\sin^2\left(\frac{(\omega_{ni}+\omega) t}{2}\right)  +   \frac{|V^\dagger_{ni}|^2}{(\omega_{ni}-\omega)^2}\sin^2\left(\frac{(\omega_{ni}-\omega) t}{2}\right) \right]
\end{align}

Diese Übergangswahrscheinlichkeit Gleichung (\ref{eq:50}) ist in der Abbildung \ref{fig:2} für ein festes \(t\) geplotet, woraus man bereits zwei Resonanzen bei \(\pm\omega\) ersehen kann.\\
\\
Betrachten wir analog zu der konstanten Störung den Limes \(t\to\infty\) so erhalten wir eine Übergangswahrscheinlichkeit

\begin{align}
  \label{eq:51}
   P(i\rightarrow n) &= \frac{2\pi t}{\hbar^2}|V_{ni}|^2 \delta(\omega_{ni}+\omega) + \frac{2\pi t}{\hbar^2}|V_{ni}^\dagger|^2 \delta(\omega_{ni}-\omega)
\end{align}

Beziehungsweise die Delta-Funktionen in Energie ausgedrückt

\begin{align}
  \label{eq:52}
     P(i\rightarrow n) &= \frac{2\pi t}{\hbar}|V_{ni}|^2 \delta(E_n - E_i + \hbar\omega ) + \frac{2\pi t}{\hbar}|V_{ni}^\dagger|^2 \delta(E_n - E_i - \hbar\omega)
\end{align}

Aus der Übergangswahrscheinlichkeit folgt die Übergangsrate

\begin{align}
  \label{eq:53}
  w_{i\rightarrow n} &= \frac{2\pi}{\hbar}|V_{ni}|^2 \delta(E_n - E_i + \hbar\omega ) + \frac{2\pi}{\hbar}|V_{ni}^\dagger|^2 \delta(E_n - E_i - \hbar\omega)
\end{align}

Dabei stellt der erste Term eine stimulierte \textbf{Emission} eines Photons der Energie \(\hbar \omega\) dar und der zweite Term eine \textbf{Absorption} eines Photons der selben Energie \(\hbar\omega\). Die Abbildung \ref{fig:3} verdeutlicht diese zwei Erscheinungen.

\begin{figure}[!thb]
  \centering
  \input{stoerung_pics/Abs_Ems.pdf_t}
  \caption{Emission und Absorption eines Photons der Energie \(\hbar\omega\)}
 \label{fig:3}
\end{figure}

\section*{Anwendung}
\subsection*{Wechselwirkung mit klassischen Strahlungsfeld}

Wir betrachten ein Elektron in einem elektromagnetischen Feld. Der Hamilton-Operator des gesamten Systems sieht folgendermaßen aus

\begin{align}
  \label{eq:54}
  H = \frac{(\vec p - e\vec A)^2}{2m} + e\phi = \frac{p^2}{2m} - \underbr{\frac{e^2\vec A^2}{2m}}_{\text{Ornung }e^2\text{ sehr klein}} - \frac{e}{2m}( \vec p\cdot \vec A + \vec A \cdot\vec p ) + e\phi
\end{align}

Allgemein gilt

\begin{align}
  \label{eq:56}
  \vec p \cdot\vec A +  \vec A \cdot\vec p = \vec p\vec A + \vec A\vec p + \vec A\vec p - \vec A\vec p = [\vec p,\vec A] + 2\vec A\vec p 
\end{align}

Nebenrechnung:
\begin{align}
  \label{eq:57}
  [\vec p,\vec A]\psi = \frac{\hbar}{i}[\vec\nabla,\vec A]\psi =  \frac{\hbar}{i}(\vec\nabla(\vec A\psi) - \vec A\vec\nabla(\psi)  ) = \frac{\hbar}{i}(\psi \vec\nabla(\vec A)+ \cancel{\vec A \vec\nabla(\psi)} - \cancel{\vec A\vec\nabla(\psi)  )} = \frac{\hbar}{i}( \vec\nabla(\vec A) )\psi
\end{align}
Also lautet die Gleichung (\ref{eq:56})

\begin{align}
  \label{eq:58}
   \vec p \cdot\vec A +  \vec A \cdot\vec p  = [\vec p,\vec A] + 2\vec A\vec p =  \underbr{\frac{\hbar}{i}(\vec \nabla\vec A)}_{\text{Coulombeichung: }\vec \nabla \vec A=0}+ 2\vec A \cdot\vec p 
\end{align}

Damit verkürzt sich der Hamilton-Operator zu

\begin{align}
  \label{eq:55}
  \rightarrow  H \approx \underbr{\frac{p^2}{2m} + e\phi}_{H_0} \underbr{- \frac{e}{m} \vec A\cdot \vec p }_{V}
\end{align}

Zusätzlich ist das Teilchen nicht in an einen Potential gebunden, damit ist \(\phi=0\) und wir erhalten für ein Teilchen im elektromagetischen Feld

\begin{align}
  \label{eq:59}
  H = \frac{p^2}{2m} + V\qquad \text{mit } V = - \frac{e}{m} \vec A\cdot \vec p
\end{align}

Betrachte nun das klassische Vektorpotential als eine ebene Welle

\begin{align}
  \label{eq:60}
  \vec A = A'_0 \hat\epsilon \cos(\vec k\cdot \vec x - \omega t) = \underbr{\frac{A'_0}{2}}_{A_0}\hat\epsilon \left( e^{i(\vec k\cdot \vec x - \omega t) }+ e^{- i(\vec k\cdot \vec x - \omega t) }  \right) =  A_0\hat\epsilon e^{i\vec k\cdot \vec x}e^{ - i\omega t }+  A_0\hat\epsilon e^{- i \vec k\cdot \vec x} e^{ + i \omega t } 
\end{align}

Wobei \(\hat\epsilon\) ein  Einheitsvektor  ist, der die Richtung des Vektorpotentials und somit die Polarisation angibt. Das Vektorpotential in den Störoperator \(V\) eingesetzt ergibt

\begin{align}
  \label{eq:61}
  V(t) = - \frac{e}{m} \vec A\cdot \vec p = \underbr{- \frac{e}{m} A_0 e^{- i \vec k\cdot \vec x} \hat\epsilon\cdot \vec p}_{V} \cdot e^{ + i \omega t } \underbr{- \frac{e}{m}  A_0 e^{i\vec k\cdot \vec x} \hat\epsilon \cdot \vec p }_{V^\dagger} \cdot e^{ - i\omega t }
\end{align}

Vergleiche mit der harmonischen Störoperator Gleichung (\ref{eq:41}). Somit können wir mit Hilfe der Gleichung (\ref{eq:53}) eine Übergangsrate für Absorption oder Emission ausrechnen. Für die Absorptionsrate gilt zum Beispiel

\begin{align}
  \label{eq:62}
   w_{i\rightarrow n} &= \frac{2\pi}{\hbar}|V_{ni}^\dagger|^2 \delta(E_n - E_i - \hbar\omega)
\end{align}

Mit \(V^\dagger\) aus Gleichung (\ref{eq:61}) erhalten wir

\begin{align}
  \label{eq:63}
  w_{i\rightarrow n} &= \frac{2\pi e^2 A_0^2}{\hbar m^2}\Big| \bra{n} e^{i\vec k\cdot \vec x} \hat\epsilon \cdot \vec p \ket{i} \Big|^2 \delta(E_n - E_i - \hbar\omega)
\end{align}

\subsection*{Absorptionswirkungsquerschnitt}

Wir wollen den Absorptionswirkungsquerschnitt für das Teilchen im Strahlungsfeld berechnen. Der totale Wirkungsquerschnitt lautet

\begin{align}
  \label{eq:64}
  \sigma_{\text{abs}} = \frac{\text{Übergangsrate}}{\text{Photonenfluss}} = \frac{\text{Übergangsrate}}{\frac{\text{Anzahl Photonen}}{\text{Fläche}\cdot\text{Zeit}}}
\end{align}

Die Übergangsrate können wir aus der Gleichung (\ref{eq:63}) entnehmen. Der Photonenfluss ist uns aber unbekannt. Deswegen erweitern wir die Gleichung (\ref{eq:64}) mit \(\hbar\omega\) (Energie eines einzelnen Photons) sodass wir im Nenner ein Energiefluss haben. Die Gleichung (\ref{eq:64}) sieht dann wie folgt aus

\begin{align}
  \label{eq:65}
  \sigma_{\text{abs}} = \frac{\text{Übergangsrate}}{\text{Photonenfluss}}\cdot \frac{\hbar\omega}{\hbar\omega}
\end{align}

Nun wollen wir uns den Nenner genau ansehen

\begin{align}
  \label{eq:66}
  \text{Photonenfluss}\cdot\hbar\omega &= \text{Energiefluß} = \frac{\text{Energie}\cdot \text{Lichtgeschwindigkeit}}{\text{Fläche}\cdot\underbr{\text{Zeit}\cdot \text{Lichtgeschwindigkeit}}_{\text{Länge}}} = \frac{\text{Energie}\cdot \text{Lichtgeschwindigkeit}}{\text{Volumen}} \notag \\
&= \text{Energiedichte} \cdot \text{Lichtgeschwindigkeit}
\end{align}

Für die Energie eines klassischen Strahlungsfeldes können wir angeben

\begin{align}
  \label{eq:67}
  H = \frac{1}{2}\int d^3\vec x \left( \epsilon_0 \vec E^2 + \frac{1}{\mu_0} \vec B^2 \right)
\end{align}

Die Energiedichte ergibt sich aus der Ableitung nach allen drei Raumrichtungen

\begin{align}
  \label{eq:68}
  \text{Energiedichte} = \vec \nabla\cdot H = \frac{1}{2}\left(\epsilon_0 \vec E^2 + \frac{1}{\mu_0} \vec B^2 \right)
\end{align}

Wir drücken die Energiedichte nur durch das \(E\)-Feld aus mit der Ausnutzung, dass das Energiefeld des Elektrischen Feldes und des Magnetischen Feldes im Strahlungsfeld wegen der Energieerhaltung gleich groß ist. Ersetze \(\vec B^2 = \frac{1}{c^2}\vec E^2\)

\begin{align}
  \label{eq:69}
  \text{Energiedichte} = \vec \nabla\cdot H = \frac{1}{2}\left(\epsilon_0 \vec E^2 + \frac{1}{c^2\mu_0} \vec E^2  \right) = \epsilon_0 \vec E^2
\end{align}

Das \(\vec E\)-Feld ergibt sich aus der Zeitlichen Ableitung des Vektorpotentials. Mit Hilfe der Gleichung (\ref{eq:60}) ergibt sich für das \(\vec E\)-Feld

\begin{align}
  \label{eq:70}
  \vec E = - \pdiff_t \vec A = - A'_0\, \omega\, \hat\epsilon\, \sin(\vec k\cdot \vec x - \omega t) = - 2 A_0\, \omega\, \hat\epsilon\, \sin(\vec k\cdot \vec x - \omega t)
\end{align}

Bisher haben wir das \(\vec E\) und \(\vec B\)-Feld klassisch betrachtet. Für die Quantenmechanik benötigen wir aber den Erwartungswert. Energiedichte in Gleichung (\ref{eq:69}) ist also definiert als

\begin{align}
  \label{eq:71}
  \text{Energiedichte} = \epsilon_0 \langle \vec E^2 \rangle =  \epsilon_0 4 |A_0|^2\, \omega^2\, \underbr{\langle \sin^2(\vec k\cdot \vec x - \omega t)}_{= \frac{1}{2}} \rangle = 2 \epsilon_0 |A_0|^2\, \omega^2
\end{align}

Eine \(sin^2(x)\)-Funktion ist eine periodische Funktion, die um den Wert \(\frac{1}{2}\) oszilliert. D.h. der Mittelwert über die Zeit gemittelt ist \(\frac{1}{2}\). Für den matematik Begeisterten gibt es eine Analytische Lösung mittels des Residuen-Satzen. Die Rechnung für den Erwartungswert \(\langle \sin^2(\vec k\cdot \vec x - \omega t)=\frac{1}{2}\) findet sich ausführlich in dem PDF \url{http://github.com/wernwa/theo-fragen/raw/master/qm/sin^2-erwartungswert.pdf}.\\
\\
Damit können wir mit Hilfe Übergangsrate aus Gleichung (\ref{eq:63}) und der Energiedichte aus Gleichung (\ref{eq:71}) den Absorptions-Wirktungsquerschnitt aus Gleichung (\ref{eq:65}) schlussendlich berechnen

\begin{align}
  \label{eq:72}
  \sigma_{\text{abs}} &= \frac{\text{Übergangsrate} \cdot\hbar\omega}{\text{Energiedichte}\cdot\text{Lichtgeschwindigkeit} } = \frac{ \frac{2\pi e^2 A_0^2}{\hbar m^2}\Big| \bra{n} e^{i\vec k\cdot \vec x} \hat\epsilon \cdot \vec p \ket{i} \Big|^2 \delta(E_n - E_i - \hbar\omega) \cdot\hbar\omega }{ 2 \epsilon_0 |A_0|^2\, \omega^2 \cdot c} \notag\\
&= \frac{2\pi e^2 A_0^2 \hbar\omega }{2 \hbar m^2 \epsilon_0 |A_0|^2\, \omega^2 c}\Big| \bra{n} e^{i\vec k\cdot \vec x} \hat\epsilon \cdot \vec p \ket{i} \Big|^2 \delta(E_n - E_i - \hbar\omega)  \notag\\
&= \frac{\pi e^2 }{ m^2 \epsilon_0 \omega c}\Big| \bra{n} e^{i\vec k\cdot \vec x} \hat\epsilon \cdot \vec p \ket{i} \Big|^2 \delta(E_n - E_i - \hbar\omega)  \notag\\
\end{align}

In der Literatur wir häufig der Absorptions-Wirktungsquerschnitt mit Hilfe der Feinstrukturkonstante

\begin{align}
  \label{eq:73}
  \alpha = \frac{e^2}{4\pi\cdot\epsilon_0\cdot\hbar\cdot c}
\end{align}

ausgedrückt. Damit sieht die Gleichung (\ref{eq:72}) wie folgt aus

\begin{align}
  \label{eq:74}
  \boxed{\sigma_{\text{abs}} = \frac{4 \pi^2 \hbar }{ m^2 \omega} \cdot \alpha \cdot \Big| \bra{n} e^{i\vec k\cdot \vec x} \hat\epsilon \cdot \vec p \ket{i} \Big|^2 \delta(E_n - E_i - \hbar\omega)}
\end{align}

Der Wirkungsquerschnitt \(\sigma_{\text{abs}}\) hat die Einheit einer Fläche \(m^2\). Demnach wird dem Wirkungsquerschnitt eine fiktive Fläche zugeordnet, so dass jeder einfallende Strom der auf diese Fläche trifft, von ihr absorbiert wird.

Um das Matrixelement \(\Big| \bra{n} e^{i\vec k\cdot \vec x} \hat\epsilon \cdot \vec p \ket{i} \Big|^2 \) zumindest näherungsweise zu berechnen, führt man eine sogenante Dipol-Approximation durch.

\subsection*{Elektrische Dipol-Approximation}

Bei einer Dipol-Approximation betrachtet man zum Beispiel ein gebundenes Elektron in einem Atom, der von einer Strahlungsquelle angestrahlt wird. Es gilt

\begin{align}
  \label{eq:75}
   e^{i\vec k\cdot \vec x} \approx 1
\end{align}

Um diese Behauptung zu belegen betrachten wir die Dimesionen von \(\vec k\cdot \vec x\). Betrachte einfallende Strahlung als Niveau-Übergang von Zustand \(\ket{i}\) nach Zustand \(\ket{n}\). Damit folgt für \(|\vec k|\)

\begin{align}
  \label{eq:76}
  |\vec k| = \frac{\omega_{ni}}{c} = \frac{E_n - E_i}{c \hbar} < \frac{E_{\text{Ry}}}{\hbar c}
\end{align}

Zur Errinerung, die Rydberg-Ernergie \(E_{\text{Ry}}\) ist Bindungsenergie für das Elektron in einem Wasserstoffatom im Grundzustand. Für alle weiteren Bindungszustände gilt

\begin{align}
  \label{eq:77}
  E_n = -\frac{e^2}{4\pi \epsilon_0 n^2} =  -E_{\text{Ry}}\frac{1}{n^2}
\end{align}

Da jedes höhere Nievaue mit \(\frac{1}{n^2}\) im Vergleich zu der Rydberg-Energie abfällt ist die Ungleichung \(E_n - E_i < E_{\text{Ry}}\) in Gleichung (\ref{eq:76}) immer gerechtfertigt. Mit der Annahme dass das Photon das Atom durchquert, kann man für \(|\vec x|\approx a_0\) ansetzen, so gilt für Das Produckt \(\vec k\cdot \vec x\)

\begin{align}
  \label{eq:78}
  |\vec k|\cdot |\vec x| \approx |\vec k|\cdot a_0 \approx \frac{E_n - E_i}{c \hbar} \cdot a_0 \lesssim \frac{E_{\text{Ry}} a_0}{\hbar c} = \frac{\alpha}{2} \approx \frac{1}{137\cdot 2} \ll 1
\end{align}

Damit wäre die Behauptung (\ref{eq:75}) gerechtfertigt. Wir können nun die Wirkungsquerschnitt aus Gleichung (\ref{eq:74}) in der Dipol-Approximation schreiben

\begin{align}
  \label{eq:79}
  \sigma_{\text{abs}} = \frac{4 \pi^2 \hbar }{ m^2 \omega} \cdot \alpha \cdot \Big| \bra{n} \hat\epsilon \cdot \vec p \ket{i} \Big|^2 \delta(E_n - E_i - \hbar\omega)
\end{align}

Um das weiter zu vereinfachen, betrachte die Welle in der \(x\)-Richtung. Damit gilt \(\hat\epsilon \cdot \vec p = p_x \)

\begin{align}
  \label{eq:80}
    \sigma_{\text{abs}} = \frac{4 \pi^2 \hbar }{ m^2 \omega} \cdot \alpha \cdot \Big| \bra{n} p_x \ket{i} \Big|^2 \delta(E_n - E_i - \hbar\omega)
\end{align}

Wir errinern uns an folgende Kommutator-Relation

\begin{align}
  \label{eq:81}
  [x,H_0] = [x,\frac{p^2_x}{2m}] = \frac{1}{2m}[x,p^2_x] = \frac{1}{2m}(p_x\underbr{[x,p_x]}_{i\hbar} + \underbr{[x,p_x]}_{i\hbar}p_x) = \frac{i\hbar}{m}p_x \Leftrightarrow p_x = \frac{m}{i\hbar}[x,H_0]
\end{align}

Setzen wir nun den Impuls \(p_x\) aus Gleichung (\ref{eq:81}) in die Gleichung (\ref{eq:80}) ein, so erhalten wir

\begin{align}
  \label{eq:82}
    \sigma_{\text{abs}} &= \frac{4 \pi^2 \hbar }{ m^2 \omega} \frac{m^2}{\hbar^2} \cdot \alpha \cdot \Big| \bra{n}[x,H_0]  \ket{i} \Big|^2 \delta(E_n - E_i - \hbar\omega) \notag \\
&= \frac{4 \pi^2  }{\hbar \omega} \cdot \alpha \cdot \Big| \bra{n} (xH_0-H_0x)  \ket{i} \Big|^2 \delta(E_n - E_i - \hbar\omega) \notag \\
&= \frac{4 \pi^2  }{\hbar \omega} \cdot \alpha \cdot \Big| \bra{n}xH_0\ket{i}- \bra{n} H_0x \ket{i} \Big|^2 \delta(E_n - E_i - \hbar\omega) \notag \\
&= \frac{4 \pi^2  }{\hbar \omega} \cdot \alpha \cdot \Big|E_i \bra{n}x\ket{i}- E_n \bra{n} x \ket{i} \Big|^2 \delta(E_n - E_i - \hbar\omega) \notag \\
&= \frac{4 \pi^2  }{\hbar \omega} \cdot \alpha \cdot \Big|\bra{n}x\ket{i}( E_i - E_n) \Big|^2 \delta(E_n - E_i - \hbar\omega) \notag \\
&= \frac{4 \pi^2  }{\hbar \omega} \cdot \alpha \cdot \Big|( E_i - E_n) \Big|^2 \Big|\bra{n}x\ket{i}\Big|^2 \delta(E_n - E_i - \hbar\omega) \notag \\
&= \frac{4 \pi^2  }{\hbar \omega} \cdot \alpha \cdot \Big|\underbr{( E_n - E_i)}_{\omega_{ni}\hbar} \Big|^2 \Big|\bra{n}x\ket{i}\Big|^2 \delta(E_n - E_i - \hbar\omega) \notag \\
\end{align}

Aus dieser Gleichung folgt für die elektrische Dipol-Approximation

\begin{align}
  \label{eq:83}
 \boxed{ \sigma_{\text{abs}}  = \frac{4 \pi^2 \hbar }{ \omega} \cdot \alpha \cdot \omega_{ni}^2  \Big|\bra{n}x\ket{i}\Big|^2 \delta(E_n - E_i - \hbar\omega) }
\end{align}

Zur Erläuterung warum das elektrische Dipolapproximation heist. Die Definition eines physikalischen Dipols besteht aus zwei gegensätzlichen Ladungen und hinreihend kurzen Abstandes \(\vec x\). So gilt für das Dipolmoment

\begin{align}
  \label{eq:84}
  \vec d = q\cdot\vec x
\end{align}

Setzen wir die Feinstrukturkonstante in die Gleichung (\ref{eq:83}) ein

\begin{align}
  \label{eq:85}
  \sigma_{\text{abs}}  = \frac{4 \pi^2 }{4\pi \epsilon_0 c \omega} \cdot  \omega_{ni}^2 \cdot\underbr{ e^2\cdot \Big|\bra{n}x\ket{i}\Big|^2}_{|\vec d|^2} \delta(E_n - E_i - \hbar\omega) 
\end{align}

so sieht man dass die Fläche des Wirkungsquerschnitts \( \sigma_{\text{abs}} \) aus dem Dipolmoment von den Einheiten abhängt. Deswegen nennt man diese Approximation auch die Dipol-Approximation. 

\end{document}


