\input{../headers/header_script.tex}



\begin{document}

\section*{Zeitabhängige Störungstheorie}

Wir betrachten einen Hamiltonoperator der aus einem zeitunabhänigen Teil \(H_0\) und einer zeitabhängigen Störung \(V(t)\) besteht.

\begin{align}
  \label{eq:2}
  H=H_0+V(t)
\end{align}

Die Eigenzustände von \(H_0\) sind gegeben durch

\begin{align}
  \label{eq:1}
   H_0 |n\rangle = E_n|n\rangle
\end{align}

Da der gesamte Hamiltonoperator zeitabhängig ist gibt es keine stationäre Zustände. Deswegen betrachten wir die Übergangswahrscheinlichkeiten von einem Zustand \(\ket{n}\) zu einem Zustand \(\ket{m}\). Wir definieren den Zustand \(\ket{\alpha}\) den wir dann nach den Eigenzuständen \(\ket{n}\) des \(H_0\)-Operators entwickeln

\begin{align}
  \label{eq:5}
  \ket{\alpha} = \mathds 1 \ket{\alpha} = \sum_n \ket{n}\underbr{\bra{n} \ket{\alpha}}_{c_n} = \sum_n c_n \ket{n}
\end{align}

Die Zeitenwicklung des Zustands \(\ket{\alpha}\) ist gegeben durch

\begin{align}
  \label{eq:6}
  \ket{\alpha,t} &= e^{-\frac{i}{\hbar}Ht}\ket{\alpha} =  e^{-\frac{i}{\hbar}H_0t} e^{-\frac{i}{\hbar}V(t)t}\ket{\alpha} \stackrel{\eqref{eq:5}}= e^{-\frac{i}{\hbar}H_0t} e^{-\frac{i}{\hbar}V(t)t} \sum_n c_n \ket{n} \\
&=\sum_n c_n  e^{-\frac{i}{\hbar}V(t)t} e^{-\frac{i}{\hbar}H_0t} \ket{n} \\
&=\sum_n \underbr{c_n  e^{-\frac{i}{\hbar}V(t)t} }_{c_n(t)}e^{-\frac{i}{\hbar}E_nt} \ket{n}
\end{align}

Damit lassen sich die zeitabhängigen Eigenzustände des gesamten Hamiltonoperators schreiben als

\begin{align}
  \label{eq:7}
 \ket{\alpha,t} =\sum_n c_n(t) e^{-\frac{i}{\hbar}E_nt} \ket{n} 
\end{align}

Aus der Gleichung (\ref{eq:6}) sieht man dass die Zeitabhängigkeit von \(c_n\) nur von \(V(t)\) verursacht wird. Desweiteren lässt sich die Wahrscheinlichkeit den Zustand \(\ket{n}\) zu finden mit \(|c_n(t)|^2\) berechnen.

\subsection*{Wechselwirkungsbild}

In der Zeitabhängigen Störungstheorie ist es zweckmäßig vom Schrödingerbild in Wechselwirkungsbild zu wechseln. Dabei hat das WW-Bild volgende Eigenschaften. Für ein Zustand im WW-Bild gilt

\begin{align}
  \label{eq:3}
  |\alpha,t\rangle_I = e^{iH_0t/\hbar}|\alpha,t\rangle_S
\end{align}

Für ein Operator gilt

\begin{align}
  \label{eq:4}
   A_I(t) =  e^{iH_0t/\hbar} A_S e^{-iH_0t/\hbar}
\end{align}

Wir wollen eine schrödinger-artige Gleichung im WW-Bild herleiten

\begin{align}
  \label{eq:8}
 i\hbar \frac{\partial}{\partial t}|\alpha,t_0;t\rangle_I &= i\hbar \frac{\partial}{\partial t}(e^{\frac{i}{\hbar}H_0t}|\alpha,t_0,t\rangle_S) \notag\\
&= i\hbar\left(\frac{i}{\hbar}H_0 e^{\frac{i}{\hbar}H_0 t}|\alpha,t_i,t\rangle_S +  e^{\frac{i}{\hbar}H_0t}\underbrace{\frac{\partial}{\partial t}|\alpha,t_0,t\rangle_S}_{\frac{1}{i\hbar}(H_0+V)|\alpha,t_0,t\rangle_S} \right) \quad |\text{mit SG:}\quad H|\psi(t)\rangle=i\hbar \frac{\partial}{\partial t}|\psi(t)\rangle  \notag  \\
&= -H_0 e^{\frac{i}{\hbar}H_0 t}|\alpha,t_i,t\rangle_S + e^{\frac{i}{\hbar}H_0t}(H_0+V)|\alpha,t_0;t\rangle_S \notag \\
&= e^{\frac{i}{\hbar}H_0t}V\cdot\mathbb 1\cdot|\alpha,t_0;t\rangle_S\notag\\
&= \underbrace{e^{\frac{i}{\hbar}H_0t}Ve^{-\frac{i}{\hbar}H_0t}}_{V_I}\cdot \underbrace{e^{\frac{i}{\hbar}H_0t}|\alpha,t_0;t\rangle_S}_{|\alpha,t_0;t\rangle_I}
\end{align}

Damit lautet die schrödinger-artige Gleichung im WW-Bild

\begin{align}
  \label{eq:9}
  \boxed{i\hbar \frac{\partial}{\partial t} |\alpha,t_0;t\rangle_I = V_I|\alpha,t_0;t\rangle_I}
\end{align}
Man sieht dass diese Gleichung unabhängig von dem stationäre Anteil des Hamiltonoperators \(H_0\) ist.

\subsection*{Lösung der schrödinger-artigen Gleichung}

Um die zeitabhängigen Koeffizienten \(c_n(t)\) zu bestimmen und damit auch die Wahrscheinlichkeit das System in einem bestimmen Zustand n berechnen zu können müssen die schrödinger-artigen Gleichung (\ref{eq:9}) wie folgt umschreiben

\begin{align}
  \label{eq:10}
  \bra{n}\cdot|\qquad  i\hbar \frac{\partial}{\partial t}|\alpha,t_0,t\rangle_I &= V_I|\alpha,t_0,t\rangle_I \notag \\
 i\hbar \frac{\partial}{\partial t}\braket{n}{\alpha,t_0,t}_I &= \bra{n}V_I|\mathds 1|\alpha,t_0,t\rangle_I \notag \\
i\hbar \frac{\partial}{\partial t}\underbr{\braket{n}{\alpha,t_0,t}_I}_{c_n(t)} &= \sum_m\bra{n}V_I \ket{m}\underbr{\bra{m}\alpha,t_0,t\rangle_I}_{c_m(t)} \notag \\
i\hbar \frac{\partial}{\partial t}c_n(t) &= \sum_m\bra{n}V_I \ket{m}c_m(t) 
\end{align}

Sehen uns das Matrixelement \(\bra{n}V_I \ket{m}c_m(t) \) genauer an

\begin{align}
  \label{eq:11}
  \bra{n}V_I \ket{m} &=  \underbrace{\langle n | e^{\frac{i}{\hbar}H_0t}}_{\langle n|e^{\frac{i}{\hbar}E_nt}}V(t)\underbrace{e^{-\frac{i}{\hbar}H_0t}|m\rangle}_{e^{-\frac{i}{\hbar}E_mt}|m\rangle } \notag\\
& = \langle n|V(t)|m\rangle e^{\frac{i}{\hbar}(E_n-E_m)t} \notag \\
&= V_{nm}(t) e^{i\omega_{nm} t}
\end{align}

Damit erhalten wir mit der Abkürzung \(\omega_{nm} = - \omega_{mn} = \frac{1}{\hbar}(E_n - E_m)\) ein System gekoppelter Differentialgleichungen das es zu lösen gilt

\begin{align}
  \label{eq:12}
  \boxed{i\hbar \frac{\partial}{\partial t}c_n(t) = \sum_m V_{nm}(t) e^{i\omega_{nm}t}c_m(t)}
\end{align}

In Matrixschreibwese sieht die Gleichung (\ref{eq:12}) folgendermaßen aus

\begin{align}
  \label{eq:13}
  i\hbar \begin{pmatrix}\dot c_1\\\dot c_2\\.\\.\\.\end{pmatrix} =
\begin{pmatrix}
V_{11}&V_{12}e^{i\omega_{12}t}&.&.&.\\
V_{21}e^{i\omega_{21}t}&V_{22}&.&.&.\\
  .&.&.&.&.\\
  .&.&.&.&.\\
  .&.&.&.&.
\end{pmatrix}\cdot\begin{pmatrix}c_1\\c_2\\.\\.\\.\end{pmatrix}
\end{align}

Die gekoppelte Differentialgleichung (\ref{eq:12}) ist für hinreichend einfache Systeme mit endlich vielen Zuständen eventuell exakt lösbar. Für Systeme die nicht exakt lösbar sind wendet man die Zeitabhängige Störungsrechnung an.

\subsection*{Zeitabhängige Störungsrechnung}

Wir führen den Zeitevolutionsoperator \(U(t,t_0)\) ein, der im WW-Bild eine Zeittransformation eines zeitunabhängigen Ket durchführt

\begin{align}
  \label{eq:14}
  |\alpha,t_0;t\rangle_I = U_I(t,t_0)|\alpha,t_0;t_0\rangle_I
\end{align}

Einsetzen in der Gleichung (\ref{eq:14}) in die Schrödingerartige Gleichung (\ref{eq:9}) ergibt

\begin{align}
  \label{eq:15}
   i\hbar \frac{\partial}{\partial t}|\alpha,t_0;t\rangle_I  &= V_I|\alpha,t_0;t\rangle_I \notag\\
 i\hbar \frac{\partial}{\partial t}  U_I(t,t_0)|\alpha,t_0;t_0\rangle_I &= V_I U_I(t,t_0)|\alpha,t_0;t_0\rangle_I  \notag\\
|\alpha,t_0;t_0\rangle_I \cdot i\hbar \frac{\partial}{\partial t}  U_I(t,t_0) &= V_I U_I(t,t_0)|\alpha,t_0;t_0\rangle_I \notag\\
 _I\langle \alpha,t_0;t_0| \cdot\Big|\qquad |\alpha,t_0;t_0\rangle_I \cdot i\hbar \frac{\partial}{\partial t}  U_I(t,t_0) &= V_I U_I(t,t_0)|\alpha,t_0;t_0\rangle_I
\end{align}

Damit erhalten wir eine DGL die nicht mehr vom Zustand \(|\alpha,t_0;t\rangle_I\) abhängig ist

\begin{align}
  \label{eq:16}
  \Rightarrow \boxed{i\hbar  \frac{\partial}{\partial t}U_I(t,t_0) = V_IU(t,t_0)}
\end{align}

Um diese DGL zu lösen integrieren wir die Gleichung (\ref{eq:16}) auf beiden Seiten von \(t_0\) bis \(t\) nach \(dt\) mit der Anfangbedingung \(U(t_0,t_0)=1\)

\begin{align}
  \label{eq:17}
i\hbar  \int_{t_0}^{t}dt' \frac{\partial}{\partial t}U_I(t',t_0) &= \int_{t_0}^{t}dt' V_IU(t',t_0)\notag \\
i\hbar \left(  U_I(t,t_0) -\underbrace{U_I(t_0,t_0) }_{1} \right)  &= \int_{t_0}^{t}dt' V_IU(t',t_0)
\end{align}

Damit erhalten wir eine Integralgleichung, die den Vorteil hat, da \(V_I\) klein ist, kann man sie iterativ lösen (damit kleine Glieder vernachlässigt werden können).

\begin{align}
  \label{eq:18}
  U_I^{(n)}(t,t_0) = 1 - \frac{i}{\hbar} \int_{t_0}^{t}dt' V_IU^{(n-1)}(t',t_0)
\end{align}

Damit lauten der Zeitevolutionsoperator in verschiedenen Störungsordnungen

\begin{align}
  U_I^{(0)}(t,t_0) &= U_I^{(0)}(t_0,t_0) = 1  \label{eq:19.0} \\
 U_I^{(1)}(t,t_0) &=  1 - \frac{i}{\hbar} \int_{t_0}^{t}dt' V_IU^{(0)}(t',t_0) =  1 - \frac{i}{\hbar} \int_{t_0}^{t}dt' V_I  \label{eq:19.1}\\
 U_I^{(2)}(t,t_0) &=  1 - \frac{i}{\hbar} \int_{t_0}^{t}dt' V_IU^{(1)}(t',t_0) =  1 - \frac{i}{\hbar} \int_{t_0}^{t}dt' V_I\left(  1 - \frac{i}{\hbar} \int_{t_0}^{t}dt'' V_I \right) \\
&= 1 - \frac{i}{\hbar} \int_{t_0}^{t}dt' V_I(t')  +  \frac{1}{\hbar^2} \int_{t_0}^{t}dt' V_I(t') \int_{t_0}^{t'}dt'' V_I(t'')  \label{eq:19.2}
\end{align}

Man erhält die sogenannte \textit{DYSON-Reihe} für \(U_I^{(\infty)}\)

\begin{align}
  \label{eq:20}
 \boxed{ U_I(t,t_0) = T \sum_{n=0}^\infty \frac{(-i)^n}{n! \hbar^n}\int_{t_0}^tdtV(t')\cdots\int_{t_0}^{t^n}dt^n V(t^n) = Te^{-\frac{i}{\hbar} \int_{t_0}^t dt' V(t') } }
\end{align}

Dabei ist \(T\) der Zeitordnungsoperator, der dafür sorgt, dass die späteren Zeiten nach links und die früheren nach recht kommen, d.h. er sortiert von höheren Zeiten zu kleineren Zeiten.

Wir wollen nun die Übergangswahrscheinlichkeit von einem Inertialzustand \(\ket{i}\) zu einem Endzustand \(\ket{n}\) bestimmen. Dazu betrachten wir den Inertialzustand bei \(t=t_0\) mit, den wir dann mit Hilfe des Zeitevolutionsoperators für beliebige Zeiten entwickeln (vergleiche mit Gleichung (\ref{eq:14}))

\begin{align}
  \label{eq:23}
  |i,t_0,t\rangle_I = U_I(t,t_0)|i\rangle = \mathbb 1\cdot U_I(t,t_0)|i\rangle = \sum_n |n\rangle \underbr{\langle n| U_I(t,t_0)|i\rangle}_{c_n} = \sum_n c_n(t) |n\rangle
\end{align}

Nun möchten wir die Übergangskoeffizienten \(c_n(t)\) des Zeitordnungsoperators \(U_I(t,t_0\) bestimmen.

\begin{align}
  \label{eq:24}
  c_n(t) = \langle n|U_I(t,t_0)|i\rangle = \langle n | Te^{-\frac{i}{\hbar}\int_{t_0}^tdt'V_I(t')}|i\rangle
\end{align}

Für \(U_I\) in 2 Ordnung Störungstheorie, siehe Gleichung (\ref{eq:19.2}), lautet \(c_n(t)\)

\begin{align}
  \label{eq:21}
   c_n(t) &= \langle n|i\rangle - \frac{i}{\hbar}\langle n |\int_{t_0}^t V_I(t')dt'|i\rangle + (\frac{i}{\hbar})^2\langle n |\int_{t_0}^t dt'\int_{t_0}^{t'} V_I(t')V_I(t'')dt'' |i\rangle \notag \\
 &= \langle n|i\rangle - \frac{i}{\hbar}\langle n |\int_{t_0}^t V_I(t')dt'|i\rangle + (\frac{i}{\hbar})^2\langle n |\int_{t_0}^t dt'\int_{t_0}^{t'} V_I(t')\cdot\sum_m |m\rangle \langle m|\cdot V_I(t'')dt'' |i\rangle\notag \\
&= \delta_{ni}+(\frac{-i}{\hbar})\int_{t_0}^t V_{ni}(t')e^{i\omega_{ni}t'}dt' + (\frac{-i}{\hbar})^2\sum_m\int_{t_0}^tdt'\int_{t_0}^{t'} V_{nm}(t')e^{i\omega_{ni}t'} V_{mi}(t'')e^{i\omega_{ni}t''} dt''\notag \\
&= c_n^{(0)}(t)+c_n^{(1)}(t)+c_n^{(2)}(t)
\end{align}

Damit erhalten wir eine Übergangswahrscheinlichkeit von Zustand \(\ket{i}\) zu einem beliebigen Zustand \(\ket{n}\) in 2-ter Näherung zeitabhängigen Störungstheorie

\begin{align}
  \label{eq:22}
  \boxed{P(i\rightarrow n) = |c^{(0)}_n + c_n^{(1)}(t)+c_n^{(2)}(t)|^2}
\end{align}

\subsection*{Beispiel: Konstante Störung}

Wir betrachten nun eine zeitlich konstante Störung \(V(t)\) für die gilt

\begin{align}
  \label{eq:19}
  V(t) =\begin{cases}0& t<0\\ V&t\ge 0    \end{cases}
\end{align}

Bestimme die Übergangswahrscheinlichkeit für \(n \ne i\) in erster Ordnung der zeitabhängigen Störungsrechnung. Laut Gleichung (\ref{eq:22}) gilt

\begin{align}
  \label{eq:25}
  P(i\rightarrow n) = |\underbr{c^{(0)}_n}_{=0} + c_n^{(1)}(t)|^2 = |c_n^{(1)}(t)|^2 = |(\frac{-i}{\hbar})\int_{t_0}^t V_{ni}(t')e^{i\omega_{ni}t'}dt'|^2
\end{align}

Mit der Bedingung (\ref{eq:19}) \(t_0=0\) und \(V(t)=V\) lautet die Übergangswahrscheinlichkeit (\ref{eq:25})

\begin{align}
  \label{eq:26}
  P(i\rightarrow n) = \frac{1}{\hbar^2}| V_{ni} \int_{0}^t e^{i\omega_{ni}t'}dt'|^2
\end{align}

Wir machen eine Nebenrechnung für das Integral

\begin{align}
  \label{eq:27}
  \int_{0}^t e^{i\omega_{ni}t'}dt' &= \left[ \frac{e^{i\omega_{ni} t}}{i\omega_{ni}}  \right]_0^t = \frac{1}{i\omega_{ni}}\left( e^{i\omega_{ni} t}-1 \right) = \frac{1}{i\omega_{ni}}\left( e^{\frac{i\omega_{ni} t}{2}+\frac{i\omega_{ni} t}{2}}-1 \right) =  \frac{1}{i\omega_{ni}}\left( e^{\frac{i\omega_{ni} t}{2}}e^{\frac{i\omega_{ni} t}{2}}-1 \right) \notag \\
&=  \frac{1}{i\omega_{ni}}\left( e^{\frac{i\omega_{ni} t}{2}}-\frac{1}{e^{\frac{i\omega_{ni} t}{2}}} \right)e^{\frac{i\omega_{ni} t}{2}} = \frac{1}{i\omega_{ni}} \underbr{\left( e^{\frac{i\omega_{ni} t}{2}}-e^{-\frac{i\omega_{ni} t}{2}}\right)}_{2i\sin(\frac{\omega_{ni} t}{2})} e^{\frac{i\omega_{ni} t}{2}} = \frac{2}{\omega_{ni}}\sin(\frac{\omega_{ni} t}{2})e^{\frac{i\omega_{ni} t}{2}}
\end{align}

Die Nebenrechnung (\ref{eq:27}) in das Integral eingesetzt lautet die Übergangswahrscheinlichkeit nun

\begin{align}
  \label{eq:28}
  P(i\rightarrow n) = \frac{1}{\hbar^2} \left| V_{ni}\frac{2}{\omega_{ni}}\sin(\frac{\omega_{ni} t}{2})e^{\frac{i\omega_{ni} t}{2}} \right|^2 = \frac{4}{\hbar^2\omega_{ni}^2}|V_{ni}|^2\sin^2(\frac{\omega_{ni} t}{2})
\end{align}

Die Übergangswahrscheinlichkeit kann man wie folgt schreiben

\begin{align}
  \label{eq:29}
   P(i\rightarrow n) = |V_{ni}|^2 f(\omega_{ni}) 
\end{align}
\begin{figure}[!thb]
\begin{minipage}{1.0\linewidth}
  
  \centering
  % GNUPLOT: LaTeX picture
\setlength{\unitlength}{0.240900pt}
\ifx\plotpoint\undefined\newsavebox{\plotpoint}\fi
\sbox{\plotpoint}{\rule[-0.200pt]{0.400pt}{0.400pt}}%
\begin{picture}(1500,900)(0,0)
\sbox{\plotpoint}{\rule[-0.200pt]{0.400pt}{0.400pt}}%
\put(160.0,82.0){\rule[-0.200pt]{4.818pt}{0.400pt}}
\put(140,82){\makebox(0,0)[r]{$0$}}
\put(1419.0,82.0){\rule[-0.200pt]{4.818pt}{0.400pt}}
\put(160.0,152.0){\rule[-0.200pt]{4.818pt}{0.400pt}}
\put(140,152){\makebox(0,0)[r]{$0.1$}}
\put(1419.0,152.0){\rule[-0.200pt]{4.818pt}{0.400pt}}
\put(160.0,221.0){\rule[-0.200pt]{4.818pt}{0.400pt}}
\put(140,221){\makebox(0,0)[r]{$0.2$}}
\put(1419.0,221.0){\rule[-0.200pt]{4.818pt}{0.400pt}}
\put(160.0,291.0){\rule[-0.200pt]{4.818pt}{0.400pt}}
\put(140,291){\makebox(0,0)[r]{$0.3$}}
\put(1419.0,291.0){\rule[-0.200pt]{4.818pt}{0.400pt}}
\put(160.0,360.0){\rule[-0.200pt]{4.818pt}{0.400pt}}
\put(140,360){\makebox(0,0)[r]{$0.4$}}
\put(1419.0,360.0){\rule[-0.200pt]{4.818pt}{0.400pt}}
\put(160.0,430.0){\rule[-0.200pt]{4.818pt}{0.400pt}}
\put(140,430){\makebox(0,0)[r]{$0.5$}}
\put(1419.0,430.0){\rule[-0.200pt]{4.818pt}{0.400pt}}
\put(160.0,499.0){\rule[-0.200pt]{4.818pt}{0.400pt}}
\put(140,499){\makebox(0,0)[r]{$0.6$}}
\put(1419.0,499.0){\rule[-0.200pt]{4.818pt}{0.400pt}}
\put(160.0,568.0){\rule[-0.200pt]{4.818pt}{0.400pt}}
\put(140,568){\makebox(0,0)[r]{$0.7$}}
\put(1419.0,568.0){\rule[-0.200pt]{4.818pt}{0.400pt}}
\put(160.0,638.0){\rule[-0.200pt]{4.818pt}{0.400pt}}
\put(140,638){\makebox(0,0)[r]{$0.8$}}
\put(1419.0,638.0){\rule[-0.200pt]{4.818pt}{0.400pt}}
\put(160.0,707.0){\rule[-0.200pt]{4.818pt}{0.400pt}}
\put(140,707){\makebox(0,0)[r]{$0.9$}}
\put(1419.0,707.0){\rule[-0.200pt]{4.818pt}{0.400pt}}
\put(160.0,777.0){\rule[-0.200pt]{4.818pt}{0.400pt}}
\put(140,777){\makebox(0,0)[r]{$1$}}
\put(1419.0,777.0){\rule[-0.200pt]{4.818pt}{0.400pt}}
\put(160.0,82.0){\rule[-0.200pt]{0.400pt}{4.818pt}}
\put(160,41){\makebox(0,0){$-4\pi$}}
\put(160.0,757.0){\rule[-0.200pt]{0.400pt}{4.818pt}}
\put(320.0,82.0){\rule[-0.200pt]{0.400pt}{4.818pt}}
\put(320,41){\makebox(0,0){$-3\pi$}}
\put(320.0,757.0){\rule[-0.200pt]{0.400pt}{4.818pt}}
\put(480.0,82.0){\rule[-0.200pt]{0.400pt}{4.818pt}}
\put(480,41){\makebox(0,0){$-2\pi$}}
\put(480.0,757.0){\rule[-0.200pt]{0.400pt}{4.818pt}}
\put(799.0,82.0){\rule[-0.200pt]{0.400pt}{4.818pt}}
\put(799,41){\makebox(0,0){0}}
\put(799.0,757.0){\rule[-0.200pt]{0.400pt}{4.818pt}}
\put(1119.0,82.0){\rule[-0.200pt]{0.400pt}{4.818pt}}
\put(1119,41){\makebox(0,0){$2\pi$}}
\put(1119.0,757.0){\rule[-0.200pt]{0.400pt}{4.818pt}}
\put(1279.0,82.0){\rule[-0.200pt]{0.400pt}{4.818pt}}
\put(1279,41){\makebox(0,0){$3\pi$}}
\put(1279.0,757.0){\rule[-0.200pt]{0.400pt}{4.818pt}}
\put(1439.0,82.0){\rule[-0.200pt]{0.400pt}{4.818pt}}
\put(1439,41){\makebox(0,0){$4\pi$}}
\put(1439.0,757.0){\rule[-0.200pt]{0.400pt}{4.818pt}}
\put(160.0,82.0){\rule[-0.200pt]{0.400pt}{167.425pt}}
\put(160.0,82.0){\rule[-0.200pt]{308.111pt}{0.400pt}}
\put(1439.0,82.0){\rule[-0.200pt]{0.400pt}{167.425pt}}
\put(160.0,777.0){\rule[-0.200pt]{308.111pt}{0.400pt}}
\put(799,839){\makebox(0,0){$f(\omega)=\frac{\sin^2(\omega_{ni}t/2)}{\omega_{ni}^2}$ für $t=1$}}
\put(160,82){\usebox{\plotpoint}}
\put(173,81.67){\rule{3.132pt}{0.400pt}}
\multiput(173.00,81.17)(6.500,1.000){2}{\rule{1.566pt}{0.400pt}}
\put(186,83.17){\rule{2.700pt}{0.400pt}}
\multiput(186.00,82.17)(7.396,2.000){2}{\rule{1.350pt}{0.400pt}}
\put(199,85.17){\rule{2.700pt}{0.400pt}}
\multiput(199.00,84.17)(7.396,2.000){2}{\rule{1.350pt}{0.400pt}}
\multiput(212.00,87.61)(2.695,0.447){3}{\rule{1.833pt}{0.108pt}}
\multiput(212.00,86.17)(9.195,3.000){2}{\rule{0.917pt}{0.400pt}}
\multiput(225.00,90.61)(2.695,0.447){3}{\rule{1.833pt}{0.108pt}}
\multiput(225.00,89.17)(9.195,3.000){2}{\rule{0.917pt}{0.400pt}}
\multiput(238.00,93.61)(2.472,0.447){3}{\rule{1.700pt}{0.108pt}}
\multiput(238.00,92.17)(8.472,3.000){2}{\rule{0.850pt}{0.400pt}}
\multiput(250.00,96.60)(1.797,0.468){5}{\rule{1.400pt}{0.113pt}}
\multiput(250.00,95.17)(10.094,4.000){2}{\rule{0.700pt}{0.400pt}}
\multiput(263.00,100.60)(1.797,0.468){5}{\rule{1.400pt}{0.113pt}}
\multiput(263.00,99.17)(10.094,4.000){2}{\rule{0.700pt}{0.400pt}}
\multiput(276.00,104.61)(2.695,0.447){3}{\rule{1.833pt}{0.108pt}}
\multiput(276.00,103.17)(9.195,3.000){2}{\rule{0.917pt}{0.400pt}}
\multiput(289.00,107.61)(2.695,0.447){3}{\rule{1.833pt}{0.108pt}}
\multiput(289.00,106.17)(9.195,3.000){2}{\rule{0.917pt}{0.400pt}}
\multiput(302.00,110.61)(2.695,0.447){3}{\rule{1.833pt}{0.108pt}}
\multiput(302.00,109.17)(9.195,3.000){2}{\rule{0.917pt}{0.400pt}}
\put(315,112.67){\rule{3.132pt}{0.400pt}}
\multiput(315.00,112.17)(6.500,1.000){2}{\rule{1.566pt}{0.400pt}}
\put(328,113.67){\rule{3.132pt}{0.400pt}}
\multiput(328.00,113.17)(6.500,1.000){2}{\rule{1.566pt}{0.400pt}}
\put(341,113.67){\rule{3.132pt}{0.400pt}}
\multiput(341.00,114.17)(6.500,-1.000){2}{\rule{1.566pt}{0.400pt}}
\put(354,112.67){\rule{3.132pt}{0.400pt}}
\multiput(354.00,113.17)(6.500,-1.000){2}{\rule{1.566pt}{0.400pt}}
\multiput(367.00,111.95)(2.695,-0.447){3}{\rule{1.833pt}{0.108pt}}
\multiput(367.00,112.17)(9.195,-3.000){2}{\rule{0.917pt}{0.400pt}}
\multiput(380.00,108.95)(2.695,-0.447){3}{\rule{1.833pt}{0.108pt}}
\multiput(380.00,109.17)(9.195,-3.000){2}{\rule{0.917pt}{0.400pt}}
\multiput(393.00,105.94)(1.651,-0.468){5}{\rule{1.300pt}{0.113pt}}
\multiput(393.00,106.17)(9.302,-4.000){2}{\rule{0.650pt}{0.400pt}}
\multiput(405.00,101.93)(1.378,-0.477){7}{\rule{1.140pt}{0.115pt}}
\multiput(405.00,102.17)(10.634,-5.000){2}{\rule{0.570pt}{0.400pt}}
\multiput(418.00,96.93)(1.378,-0.477){7}{\rule{1.140pt}{0.115pt}}
\multiput(418.00,97.17)(10.634,-5.000){2}{\rule{0.570pt}{0.400pt}}
\multiput(431.00,91.94)(1.797,-0.468){5}{\rule{1.400pt}{0.113pt}}
\multiput(431.00,92.17)(10.094,-4.000){2}{\rule{0.700pt}{0.400pt}}
\multiput(444.00,87.94)(1.797,-0.468){5}{\rule{1.400pt}{0.113pt}}
\multiput(444.00,88.17)(10.094,-4.000){2}{\rule{0.700pt}{0.400pt}}
\put(457,83.17){\rule{2.700pt}{0.400pt}}
\multiput(457.00,84.17)(7.396,-2.000){2}{\rule{1.350pt}{0.400pt}}
\put(470,81.67){\rule{3.132pt}{0.400pt}}
\multiput(470.00,82.17)(6.500,-1.000){2}{\rule{1.566pt}{0.400pt}}
\put(483,82.17){\rule{2.700pt}{0.400pt}}
\multiput(483.00,81.17)(7.396,2.000){2}{\rule{1.350pt}{0.400pt}}
\multiput(496.00,84.59)(1.378,0.477){7}{\rule{1.140pt}{0.115pt}}
\multiput(496.00,83.17)(10.634,5.000){2}{\rule{0.570pt}{0.400pt}}
\multiput(509.00,89.59)(0.824,0.488){13}{\rule{0.750pt}{0.117pt}}
\multiput(509.00,88.17)(11.443,8.000){2}{\rule{0.375pt}{0.400pt}}
\multiput(522.00,97.58)(0.539,0.492){21}{\rule{0.533pt}{0.119pt}}
\multiput(522.00,96.17)(11.893,12.000){2}{\rule{0.267pt}{0.400pt}}
\multiput(535.58,109.00)(0.493,0.616){23}{\rule{0.119pt}{0.592pt}}
\multiput(534.17,109.00)(13.000,14.771){2}{\rule{0.400pt}{0.296pt}}
\multiput(548.58,125.00)(0.492,0.884){21}{\rule{0.119pt}{0.800pt}}
\multiput(547.17,125.00)(12.000,19.340){2}{\rule{0.400pt}{0.400pt}}
\multiput(560.58,146.00)(0.493,0.972){23}{\rule{0.119pt}{0.869pt}}
\multiput(559.17,146.00)(13.000,23.196){2}{\rule{0.400pt}{0.435pt}}
\multiput(573.58,171.00)(0.493,1.171){23}{\rule{0.119pt}{1.023pt}}
\multiput(572.17,171.00)(13.000,27.877){2}{\rule{0.400pt}{0.512pt}}
\multiput(586.58,201.00)(0.493,1.329){23}{\rule{0.119pt}{1.146pt}}
\multiput(585.17,201.00)(13.000,31.621){2}{\rule{0.400pt}{0.573pt}}
\multiput(599.58,235.00)(0.493,1.488){23}{\rule{0.119pt}{1.269pt}}
\multiput(598.17,235.00)(13.000,35.366){2}{\rule{0.400pt}{0.635pt}}
\multiput(612.58,273.00)(0.493,1.607){23}{\rule{0.119pt}{1.362pt}}
\multiput(611.17,273.00)(13.000,38.174){2}{\rule{0.400pt}{0.681pt}}
\multiput(625.58,314.00)(0.493,1.726){23}{\rule{0.119pt}{1.454pt}}
\multiput(624.17,314.00)(13.000,40.982){2}{\rule{0.400pt}{0.727pt}}
\multiput(638.58,358.00)(0.493,1.805){23}{\rule{0.119pt}{1.515pt}}
\multiput(637.17,358.00)(13.000,42.855){2}{\rule{0.400pt}{0.758pt}}
\multiput(651.58,404.00)(0.493,1.884){23}{\rule{0.119pt}{1.577pt}}
\multiput(650.17,404.00)(13.000,44.727){2}{\rule{0.400pt}{0.788pt}}
\multiput(664.58,452.00)(0.493,1.845){23}{\rule{0.119pt}{1.546pt}}
\multiput(663.17,452.00)(13.000,43.791){2}{\rule{0.400pt}{0.773pt}}
\multiput(677.58,499.00)(0.493,1.845){23}{\rule{0.119pt}{1.546pt}}
\multiput(676.17,499.00)(13.000,43.791){2}{\rule{0.400pt}{0.773pt}}
\multiput(690.58,546.00)(0.493,1.765){23}{\rule{0.119pt}{1.485pt}}
\multiput(689.17,546.00)(13.000,41.919){2}{\rule{0.400pt}{0.742pt}}
\multiput(703.58,591.00)(0.493,1.646){23}{\rule{0.119pt}{1.392pt}}
\multiput(702.17,591.00)(13.000,39.110){2}{\rule{0.400pt}{0.696pt}}
\multiput(716.58,633.00)(0.492,1.616){21}{\rule{0.119pt}{1.367pt}}
\multiput(715.17,633.00)(12.000,35.163){2}{\rule{0.400pt}{0.683pt}}
\multiput(728.58,671.00)(0.493,1.329){23}{\rule{0.119pt}{1.146pt}}
\multiput(727.17,671.00)(13.000,31.621){2}{\rule{0.400pt}{0.573pt}}
\multiput(741.58,705.00)(0.493,1.052){23}{\rule{0.119pt}{0.931pt}}
\multiput(740.17,705.00)(13.000,25.068){2}{\rule{0.400pt}{0.465pt}}
\multiput(754.58,732.00)(0.493,0.853){23}{\rule{0.119pt}{0.777pt}}
\multiput(753.17,732.00)(13.000,20.387){2}{\rule{0.400pt}{0.388pt}}
\multiput(767.58,754.00)(0.493,0.576){23}{\rule{0.119pt}{0.562pt}}
\multiput(766.17,754.00)(13.000,13.834){2}{\rule{0.400pt}{0.281pt}}
\multiput(780.00,769.59)(0.950,0.485){11}{\rule{0.843pt}{0.117pt}}
\multiput(780.00,768.17)(11.251,7.000){2}{\rule{0.421pt}{0.400pt}}
\put(160.0,82.0){\rule[-0.200pt]{3.132pt}{0.400pt}}
\multiput(806.00,774.93)(0.950,-0.485){11}{\rule{0.843pt}{0.117pt}}
\multiput(806.00,775.17)(11.251,-7.000){2}{\rule{0.421pt}{0.400pt}}
\multiput(819.58,766.67)(0.493,-0.576){23}{\rule{0.119pt}{0.562pt}}
\multiput(818.17,767.83)(13.000,-13.834){2}{\rule{0.400pt}{0.281pt}}
\multiput(832.58,750.77)(0.493,-0.853){23}{\rule{0.119pt}{0.777pt}}
\multiput(831.17,752.39)(13.000,-20.387){2}{\rule{0.400pt}{0.388pt}}
\multiput(845.58,728.14)(0.493,-1.052){23}{\rule{0.119pt}{0.931pt}}
\multiput(844.17,730.07)(13.000,-25.068){2}{\rule{0.400pt}{0.465pt}}
\multiput(858.58,700.24)(0.493,-1.329){23}{\rule{0.119pt}{1.146pt}}
\multiput(857.17,702.62)(13.000,-31.621){2}{\rule{0.400pt}{0.573pt}}
\multiput(871.58,665.33)(0.492,-1.616){21}{\rule{0.119pt}{1.367pt}}
\multiput(870.17,668.16)(12.000,-35.163){2}{\rule{0.400pt}{0.683pt}}
\multiput(883.58,627.22)(0.493,-1.646){23}{\rule{0.119pt}{1.392pt}}
\multiput(882.17,630.11)(13.000,-39.110){2}{\rule{0.400pt}{0.696pt}}
\multiput(896.58,584.84)(0.493,-1.765){23}{\rule{0.119pt}{1.485pt}}
\multiput(895.17,587.92)(13.000,-41.919){2}{\rule{0.400pt}{0.742pt}}
\multiput(909.58,539.58)(0.493,-1.845){23}{\rule{0.119pt}{1.546pt}}
\multiput(908.17,542.79)(13.000,-43.791){2}{\rule{0.400pt}{0.773pt}}
\multiput(922.58,492.58)(0.493,-1.845){23}{\rule{0.119pt}{1.546pt}}
\multiput(921.17,495.79)(13.000,-43.791){2}{\rule{0.400pt}{0.773pt}}
\multiput(935.58,445.45)(0.493,-1.884){23}{\rule{0.119pt}{1.577pt}}
\multiput(934.17,448.73)(13.000,-44.727){2}{\rule{0.400pt}{0.788pt}}
\multiput(948.58,397.71)(0.493,-1.805){23}{\rule{0.119pt}{1.515pt}}
\multiput(947.17,400.85)(13.000,-42.855){2}{\rule{0.400pt}{0.758pt}}
\multiput(961.58,351.96)(0.493,-1.726){23}{\rule{0.119pt}{1.454pt}}
\multiput(960.17,354.98)(13.000,-40.982){2}{\rule{0.400pt}{0.727pt}}
\multiput(974.58,308.35)(0.493,-1.607){23}{\rule{0.119pt}{1.362pt}}
\multiput(973.17,311.17)(13.000,-38.174){2}{\rule{0.400pt}{0.681pt}}
\multiput(987.58,267.73)(0.493,-1.488){23}{\rule{0.119pt}{1.269pt}}
\multiput(986.17,270.37)(13.000,-35.366){2}{\rule{0.400pt}{0.635pt}}
\multiput(1000.58,230.24)(0.493,-1.329){23}{\rule{0.119pt}{1.146pt}}
\multiput(999.17,232.62)(13.000,-31.621){2}{\rule{0.400pt}{0.573pt}}
\multiput(1013.58,196.75)(0.493,-1.171){23}{\rule{0.119pt}{1.023pt}}
\multiput(1012.17,198.88)(13.000,-27.877){2}{\rule{0.400pt}{0.512pt}}
\multiput(1026.58,167.39)(0.493,-0.972){23}{\rule{0.119pt}{0.869pt}}
\multiput(1025.17,169.20)(13.000,-23.196){2}{\rule{0.400pt}{0.435pt}}
\multiput(1039.58,142.68)(0.492,-0.884){21}{\rule{0.119pt}{0.800pt}}
\multiput(1038.17,144.34)(12.000,-19.340){2}{\rule{0.400pt}{0.400pt}}
\multiput(1051.58,122.54)(0.493,-0.616){23}{\rule{0.119pt}{0.592pt}}
\multiput(1050.17,123.77)(13.000,-14.771){2}{\rule{0.400pt}{0.296pt}}
\multiput(1064.00,107.92)(0.539,-0.492){21}{\rule{0.533pt}{0.119pt}}
\multiput(1064.00,108.17)(11.893,-12.000){2}{\rule{0.267pt}{0.400pt}}
\multiput(1077.00,95.93)(0.824,-0.488){13}{\rule{0.750pt}{0.117pt}}
\multiput(1077.00,96.17)(11.443,-8.000){2}{\rule{0.375pt}{0.400pt}}
\multiput(1090.00,87.93)(1.378,-0.477){7}{\rule{1.140pt}{0.115pt}}
\multiput(1090.00,88.17)(10.634,-5.000){2}{\rule{0.570pt}{0.400pt}}
\put(1103,82.17){\rule{2.700pt}{0.400pt}}
\multiput(1103.00,83.17)(7.396,-2.000){2}{\rule{1.350pt}{0.400pt}}
\put(1116,81.67){\rule{3.132pt}{0.400pt}}
\multiput(1116.00,81.17)(6.500,1.000){2}{\rule{1.566pt}{0.400pt}}
\put(1129,83.17){\rule{2.700pt}{0.400pt}}
\multiput(1129.00,82.17)(7.396,2.000){2}{\rule{1.350pt}{0.400pt}}
\multiput(1142.00,85.60)(1.797,0.468){5}{\rule{1.400pt}{0.113pt}}
\multiput(1142.00,84.17)(10.094,4.000){2}{\rule{0.700pt}{0.400pt}}
\multiput(1155.00,89.60)(1.797,0.468){5}{\rule{1.400pt}{0.113pt}}
\multiput(1155.00,88.17)(10.094,4.000){2}{\rule{0.700pt}{0.400pt}}
\multiput(1168.00,93.59)(1.378,0.477){7}{\rule{1.140pt}{0.115pt}}
\multiput(1168.00,92.17)(10.634,5.000){2}{\rule{0.570pt}{0.400pt}}
\multiput(1181.00,98.59)(1.378,0.477){7}{\rule{1.140pt}{0.115pt}}
\multiput(1181.00,97.17)(10.634,5.000){2}{\rule{0.570pt}{0.400pt}}
\multiput(1194.00,103.60)(1.651,0.468){5}{\rule{1.300pt}{0.113pt}}
\multiput(1194.00,102.17)(9.302,4.000){2}{\rule{0.650pt}{0.400pt}}
\multiput(1206.00,107.61)(2.695,0.447){3}{\rule{1.833pt}{0.108pt}}
\multiput(1206.00,106.17)(9.195,3.000){2}{\rule{0.917pt}{0.400pt}}
\multiput(1219.00,110.61)(2.695,0.447){3}{\rule{1.833pt}{0.108pt}}
\multiput(1219.00,109.17)(9.195,3.000){2}{\rule{0.917pt}{0.400pt}}
\put(1232,112.67){\rule{3.132pt}{0.400pt}}
\multiput(1232.00,112.17)(6.500,1.000){2}{\rule{1.566pt}{0.400pt}}
\put(1245,113.67){\rule{3.132pt}{0.400pt}}
\multiput(1245.00,113.17)(6.500,1.000){2}{\rule{1.566pt}{0.400pt}}
\put(1258,113.67){\rule{3.132pt}{0.400pt}}
\multiput(1258.00,114.17)(6.500,-1.000){2}{\rule{1.566pt}{0.400pt}}
\put(1271,112.67){\rule{3.132pt}{0.400pt}}
\multiput(1271.00,113.17)(6.500,-1.000){2}{\rule{1.566pt}{0.400pt}}
\multiput(1284.00,111.95)(2.695,-0.447){3}{\rule{1.833pt}{0.108pt}}
\multiput(1284.00,112.17)(9.195,-3.000){2}{\rule{0.917pt}{0.400pt}}
\multiput(1297.00,108.95)(2.695,-0.447){3}{\rule{1.833pt}{0.108pt}}
\multiput(1297.00,109.17)(9.195,-3.000){2}{\rule{0.917pt}{0.400pt}}
\multiput(1310.00,105.95)(2.695,-0.447){3}{\rule{1.833pt}{0.108pt}}
\multiput(1310.00,106.17)(9.195,-3.000){2}{\rule{0.917pt}{0.400pt}}
\multiput(1323.00,102.94)(1.797,-0.468){5}{\rule{1.400pt}{0.113pt}}
\multiput(1323.00,103.17)(10.094,-4.000){2}{\rule{0.700pt}{0.400pt}}
\multiput(1336.00,98.94)(1.797,-0.468){5}{\rule{1.400pt}{0.113pt}}
\multiput(1336.00,99.17)(10.094,-4.000){2}{\rule{0.700pt}{0.400pt}}
\multiput(1349.00,94.95)(2.472,-0.447){3}{\rule{1.700pt}{0.108pt}}
\multiput(1349.00,95.17)(8.472,-3.000){2}{\rule{0.850pt}{0.400pt}}
\multiput(1361.00,91.95)(2.695,-0.447){3}{\rule{1.833pt}{0.108pt}}
\multiput(1361.00,92.17)(9.195,-3.000){2}{\rule{0.917pt}{0.400pt}}
\multiput(1374.00,88.95)(2.695,-0.447){3}{\rule{1.833pt}{0.108pt}}
\multiput(1374.00,89.17)(9.195,-3.000){2}{\rule{0.917pt}{0.400pt}}
\put(1387,85.17){\rule{2.700pt}{0.400pt}}
\multiput(1387.00,86.17)(7.396,-2.000){2}{\rule{1.350pt}{0.400pt}}
\put(1400,83.17){\rule{2.700pt}{0.400pt}}
\multiput(1400.00,84.17)(7.396,-2.000){2}{\rule{1.350pt}{0.400pt}}
\put(1413,81.67){\rule{3.132pt}{0.400pt}}
\multiput(1413.00,82.17)(6.500,-1.000){2}{\rule{1.566pt}{0.400pt}}
\put(793.0,776.0){\rule[-0.200pt]{3.132pt}{0.400pt}}
\put(1426.0,82.0){\rule[-0.200pt]{3.132pt}{0.400pt}}
\put(160.0,82.0){\rule[-0.200pt]{0.400pt}{167.425pt}}
\put(160.0,82.0){\rule[-0.200pt]{308.111pt}{0.400pt}}
\put(1439.0,82.0){\rule[-0.200pt]{0.400pt}{167.425pt}}
\put(160.0,777.0){\rule[-0.200pt]{308.111pt}{0.400pt}}
\end{picture}

  % GNUPLOT: LaTeX picture
\setlength{\unitlength}{0.240900pt}
\ifx\plotpoint\undefined\newsavebox{\plotpoint}\fi
\sbox{\plotpoint}{\rule[-0.200pt]{0.400pt}{0.400pt}}%
\begin{picture}(1500,900)(0,0)
\sbox{\plotpoint}{\rule[-0.200pt]{0.400pt}{0.400pt}}%
\put(200.0,82.0){\rule[-0.200pt]{4.818pt}{0.400pt}}
\put(180,82){\makebox(0,0)[r]{$0$}}
\put(1419.0,82.0){\rule[-0.200pt]{4.818pt}{0.400pt}}
\put(200.0,152.0){\rule[-0.200pt]{4.818pt}{0.400pt}}
\put(180,152){\makebox(0,0)[r]{$1000$}}
\put(1419.0,152.0){\rule[-0.200pt]{4.818pt}{0.400pt}}
\put(200.0,221.0){\rule[-0.200pt]{4.818pt}{0.400pt}}
\put(180,221){\makebox(0,0)[r]{$2000$}}
\put(1419.0,221.0){\rule[-0.200pt]{4.818pt}{0.400pt}}
\put(200.0,291.0){\rule[-0.200pt]{4.818pt}{0.400pt}}
\put(180,291){\makebox(0,0)[r]{$3000$}}
\put(1419.0,291.0){\rule[-0.200pt]{4.818pt}{0.400pt}}
\put(200.0,360.0){\rule[-0.200pt]{4.818pt}{0.400pt}}
\put(180,360){\makebox(0,0)[r]{$4000$}}
\put(1419.0,360.0){\rule[-0.200pt]{4.818pt}{0.400pt}}
\put(200.0,430.0){\rule[-0.200pt]{4.818pt}{0.400pt}}
\put(180,430){\makebox(0,0)[r]{$5000$}}
\put(1419.0,430.0){\rule[-0.200pt]{4.818pt}{0.400pt}}
\put(200.0,499.0){\rule[-0.200pt]{4.818pt}{0.400pt}}
\put(180,499){\makebox(0,0)[r]{$6000$}}
\put(1419.0,499.0){\rule[-0.200pt]{4.818pt}{0.400pt}}
\put(200.0,569.0){\rule[-0.200pt]{4.818pt}{0.400pt}}
\put(180,569){\makebox(0,0)[r]{$7000$}}
\put(1419.0,569.0){\rule[-0.200pt]{4.818pt}{0.400pt}}
\put(200.0,638.0){\rule[-0.200pt]{4.818pt}{0.400pt}}
\put(180,638){\makebox(0,0)[r]{$8000$}}
\put(1419.0,638.0){\rule[-0.200pt]{4.818pt}{0.400pt}}
\put(200.0,708.0){\rule[-0.200pt]{4.818pt}{0.400pt}}
\put(180,708){\makebox(0,0)[r]{$9000$}}
\put(1419.0,708.0){\rule[-0.200pt]{4.818pt}{0.400pt}}
\put(200.0,777.0){\rule[-0.200pt]{4.818pt}{0.400pt}}
\put(180,777){\makebox(0,0)[r]{$10000$}}
\put(1419.0,777.0){\rule[-0.200pt]{4.818pt}{0.400pt}}
\put(200.0,82.0){\rule[-0.200pt]{0.400pt}{4.818pt}}
\put(200,41){\makebox(0,0){$-4\pi$}}
\put(200.0,757.0){\rule[-0.200pt]{0.400pt}{4.818pt}}
\put(355.0,82.0){\rule[-0.200pt]{0.400pt}{4.818pt}}
\put(355,41){\makebox(0,0){$-3\pi$}}
\put(355.0,757.0){\rule[-0.200pt]{0.400pt}{4.818pt}}
\put(510.0,82.0){\rule[-0.200pt]{0.400pt}{4.818pt}}
\put(510,41){\makebox(0,0){$-2\pi$}}
\put(510.0,757.0){\rule[-0.200pt]{0.400pt}{4.818pt}}
\put(819.0,82.0){\rule[-0.200pt]{0.400pt}{4.818pt}}
\put(819,41){\makebox(0,0){0}}
\put(819.0,757.0){\rule[-0.200pt]{0.400pt}{4.818pt}}
\put(1129.0,82.0){\rule[-0.200pt]{0.400pt}{4.818pt}}
\put(1129,41){\makebox(0,0){$2\pi$}}
\put(1129.0,757.0){\rule[-0.200pt]{0.400pt}{4.818pt}}
\put(1284.0,82.0){\rule[-0.200pt]{0.400pt}{4.818pt}}
\put(1284,41){\makebox(0,0){$3\pi$}}
\put(1284.0,757.0){\rule[-0.200pt]{0.400pt}{4.818pt}}
\put(1439.0,82.0){\rule[-0.200pt]{0.400pt}{4.818pt}}
\put(1439,41){\makebox(0,0){$4\pi$}}
\put(1439.0,757.0){\rule[-0.200pt]{0.400pt}{4.818pt}}
\put(200.0,82.0){\rule[-0.200pt]{0.400pt}{167.425pt}}
\put(200.0,82.0){\rule[-0.200pt]{298.475pt}{0.400pt}}
\put(1439.0,82.0){\rule[-0.200pt]{0.400pt}{167.425pt}}
\put(200.0,777.0){\rule[-0.200pt]{298.475pt}{0.400pt}}
\put(819,839){\makebox(0,0){$f(\omega)$ für $t=100$}}
\put(200,82){\usebox{\plotpoint}}
\put(200,82){\usebox{\plotpoint}}
\put(200,82){\usebox{\plotpoint}}
\put(200,82){\usebox{\plotpoint}}
\put(200,82){\usebox{\plotpoint}}
\put(200,82){\usebox{\plotpoint}}
\put(200,82){\usebox{\plotpoint}}
\put(200,82){\usebox{\plotpoint}}
\put(200,82){\usebox{\plotpoint}}
\put(200,82){\usebox{\plotpoint}}
\put(200,82){\usebox{\plotpoint}}
\put(200,82){\usebox{\plotpoint}}
\put(200,82){\usebox{\plotpoint}}
\put(200,82){\usebox{\plotpoint}}
\put(200,82){\usebox{\plotpoint}}
\put(200,82){\usebox{\plotpoint}}
\put(200,82){\usebox{\plotpoint}}
\put(200,82){\usebox{\plotpoint}}
\put(200,82){\usebox{\plotpoint}}
\put(200,82){\usebox{\plotpoint}}
\put(200,82){\usebox{\plotpoint}}
\put(200,82){\usebox{\plotpoint}}
\put(200,82){\usebox{\plotpoint}}
\put(200,82){\usebox{\plotpoint}}
\put(200,82){\usebox{\plotpoint}}
\put(200,82){\usebox{\plotpoint}}
\put(200,82){\usebox{\plotpoint}}
\put(200,82){\usebox{\plotpoint}}
\put(200,82){\usebox{\plotpoint}}
\put(200,82){\usebox{\plotpoint}}
\put(200,82){\usebox{\plotpoint}}
\put(200,82){\usebox{\plotpoint}}
\put(200,82){\usebox{\plotpoint}}
\put(200,82){\usebox{\plotpoint}}
\put(200,82){\usebox{\plotpoint}}
\put(200,82){\usebox{\plotpoint}}
\put(200,82){\usebox{\plotpoint}}
\put(200,82){\usebox{\plotpoint}}
\put(200,82){\usebox{\plotpoint}}
\put(200,82){\usebox{\plotpoint}}
\put(200,82){\usebox{\plotpoint}}
\put(200,82){\usebox{\plotpoint}}
\put(200,82){\usebox{\plotpoint}}
\put(200,82){\usebox{\plotpoint}}
\put(200,82){\usebox{\plotpoint}}
\put(200,82){\usebox{\plotpoint}}
\put(200,82){\usebox{\plotpoint}}
\put(200,82){\usebox{\plotpoint}}
\put(200,82){\usebox{\plotpoint}}
\put(200,82){\usebox{\plotpoint}}
\put(200,82){\usebox{\plotpoint}}
\put(200,82){\usebox{\plotpoint}}
\put(200,82){\usebox{\plotpoint}}
\put(200,82){\usebox{\plotpoint}}
\put(200,82){\usebox{\plotpoint}}
\put(200,82){\usebox{\plotpoint}}
\put(200,82){\usebox{\plotpoint}}
\put(200,82){\usebox{\plotpoint}}
\put(200,82){\usebox{\plotpoint}}
\put(200,82){\usebox{\plotpoint}}
\put(200,82){\usebox{\plotpoint}}
\put(200,82){\usebox{\plotpoint}}
\put(200,82){\usebox{\plotpoint}}
\put(200,82){\usebox{\plotpoint}}
\put(200,82){\usebox{\plotpoint}}
\put(200,82){\usebox{\plotpoint}}
\put(200,82){\usebox{\plotpoint}}
\put(200,82){\usebox{\plotpoint}}
\put(200,82){\usebox{\plotpoint}}
\put(200,82){\usebox{\plotpoint}}
\put(200,82){\usebox{\plotpoint}}
\put(200,82){\usebox{\plotpoint}}
\put(200,82){\usebox{\plotpoint}}
\put(200,82){\usebox{\plotpoint}}
\put(200,82){\usebox{\plotpoint}}
\put(200,82){\usebox{\plotpoint}}
\put(200,82){\usebox{\plotpoint}}
\put(200,82){\usebox{\plotpoint}}
\put(200,82){\usebox{\plotpoint}}
\put(200,82){\usebox{\plotpoint}}
\put(200,82){\usebox{\plotpoint}}
\put(200,82){\usebox{\plotpoint}}
\put(200,82){\usebox{\plotpoint}}
\put(200,82){\usebox{\plotpoint}}
\put(200,82){\usebox{\plotpoint}}
\put(200,82){\usebox{\plotpoint}}
\put(200,82){\usebox{\plotpoint}}
\put(200,82){\usebox{\plotpoint}}
\put(200,82){\usebox{\plotpoint}}
\put(200,82){\usebox{\plotpoint}}
\put(200,82){\usebox{\plotpoint}}
\put(200,82){\usebox{\plotpoint}}
\put(200,82){\usebox{\plotpoint}}
\put(200,82){\usebox{\plotpoint}}
\put(200,82){\usebox{\plotpoint}}
\put(200,82){\usebox{\plotpoint}}
\put(200,82){\usebox{\plotpoint}}
\put(200,82){\usebox{\plotpoint}}
\put(200,82){\usebox{\plotpoint}}
\put(200,82){\usebox{\plotpoint}}
\put(200,82){\usebox{\plotpoint}}
\put(200,82){\usebox{\plotpoint}}
\put(200,82){\usebox{\plotpoint}}
\put(200,82){\usebox{\plotpoint}}
\put(200,82){\usebox{\plotpoint}}
\put(200,82){\usebox{\plotpoint}}
\put(200,82){\usebox{\plotpoint}}
\put(200,82){\usebox{\plotpoint}}
\put(200,82){\usebox{\plotpoint}}
\put(200,82){\usebox{\plotpoint}}
\put(200,82){\usebox{\plotpoint}}
\put(200,82){\usebox{\plotpoint}}
\put(200,82){\usebox{\plotpoint}}
\put(200,82){\usebox{\plotpoint}}
\put(200,82){\usebox{\plotpoint}}
\put(200,82){\usebox{\plotpoint}}
\put(200,82){\usebox{\plotpoint}}
\put(200,82){\usebox{\plotpoint}}
\put(200,82){\usebox{\plotpoint}}
\put(200,82){\usebox{\plotpoint}}
\put(200,82){\usebox{\plotpoint}}
\put(200,82){\usebox{\plotpoint}}
\put(200,82){\usebox{\plotpoint}}
\put(200,82){\usebox{\plotpoint}}
\put(200,82){\usebox{\plotpoint}}
\put(200,82){\usebox{\plotpoint}}
\put(200,82){\usebox{\plotpoint}}
\put(200,82){\usebox{\plotpoint}}
\put(200,82){\usebox{\plotpoint}}
\put(200,82){\usebox{\plotpoint}}
\put(200,82){\usebox{\plotpoint}}
\put(200,82){\usebox{\plotpoint}}
\put(200,82){\usebox{\plotpoint}}
\put(200,82){\usebox{\plotpoint}}
\put(200,82){\usebox{\plotpoint}}
\put(200,82){\usebox{\plotpoint}}
\put(200,82){\usebox{\plotpoint}}
\put(200,82){\usebox{\plotpoint}}
\put(200,82){\usebox{\plotpoint}}
\put(200,82){\usebox{\plotpoint}}
\put(200,82){\usebox{\plotpoint}}
\put(200,82){\usebox{\plotpoint}}
\put(200,82){\usebox{\plotpoint}}
\put(200,82){\usebox{\plotpoint}}
\put(200,82){\usebox{\plotpoint}}
\put(200,82){\usebox{\plotpoint}}
\put(200,82){\usebox{\plotpoint}}
\put(200,82){\usebox{\plotpoint}}
\put(200,82){\usebox{\plotpoint}}
\put(200,82){\usebox{\plotpoint}}
\put(200,82){\usebox{\plotpoint}}
\put(200,82){\usebox{\plotpoint}}
\put(200,82){\usebox{\plotpoint}}
\put(200,82){\usebox{\plotpoint}}
\put(200,82){\usebox{\plotpoint}}
\put(200,82){\usebox{\plotpoint}}
\put(200,82){\usebox{\plotpoint}}
\put(200,82){\usebox{\plotpoint}}
\put(200,82){\usebox{\plotpoint}}
\put(200,82){\usebox{\plotpoint}}
\put(200,82){\usebox{\plotpoint}}
\put(200,82){\usebox{\plotpoint}}
\put(200,82){\usebox{\plotpoint}}
\put(200,82){\usebox{\plotpoint}}
\put(200,82){\usebox{\plotpoint}}
\put(200,82){\usebox{\plotpoint}}
\put(200,82){\usebox{\plotpoint}}
\put(200,82){\usebox{\plotpoint}}
\put(200,82){\usebox{\plotpoint}}
\put(200,82){\usebox{\plotpoint}}
\put(200,82){\usebox{\plotpoint}}
\put(200,82){\usebox{\plotpoint}}
\put(200,82){\usebox{\plotpoint}}
\put(200,82){\usebox{\plotpoint}}
\put(200,82){\usebox{\plotpoint}}
\put(200,82){\usebox{\plotpoint}}
\put(200,82){\usebox{\plotpoint}}
\put(200,82){\usebox{\plotpoint}}
\put(200,82){\usebox{\plotpoint}}
\put(200,82){\usebox{\plotpoint}}
\put(200,82){\usebox{\plotpoint}}
\put(200,82){\usebox{\plotpoint}}
\put(200,82){\usebox{\plotpoint}}
\put(200,82){\usebox{\plotpoint}}
\put(200,82){\usebox{\plotpoint}}
\put(200,82){\usebox{\plotpoint}}
\put(200,82){\usebox{\plotpoint}}
\put(200,82){\usebox{\plotpoint}}
\put(200,82){\usebox{\plotpoint}}
\put(200,82){\usebox{\plotpoint}}
\put(200,82){\usebox{\plotpoint}}
\put(200,82){\usebox{\plotpoint}}
\put(200,82){\usebox{\plotpoint}}
\put(200,82){\usebox{\plotpoint}}
\put(200,82){\usebox{\plotpoint}}
\put(200,82){\usebox{\plotpoint}}
\put(200,82){\usebox{\plotpoint}}
\put(200,82){\usebox{\plotpoint}}
\put(200,82){\usebox{\plotpoint}}
\put(200,82){\usebox{\plotpoint}}
\put(200,82){\usebox{\plotpoint}}
\put(200,82){\usebox{\plotpoint}}
\put(200,82){\usebox{\plotpoint}}
\put(200,82){\usebox{\plotpoint}}
\put(200,82){\usebox{\plotpoint}}
\put(200,82){\usebox{\plotpoint}}
\put(200,82){\usebox{\plotpoint}}
\put(200,82){\usebox{\plotpoint}}
\put(200,82){\usebox{\plotpoint}}
\put(200,82){\usebox{\plotpoint}}
\put(200,82){\usebox{\plotpoint}}
\put(200,82){\usebox{\plotpoint}}
\put(200,82){\usebox{\plotpoint}}
\put(200,82){\usebox{\plotpoint}}
\put(200,82){\usebox{\plotpoint}}
\put(200,82){\usebox{\plotpoint}}
\put(200,82){\usebox{\plotpoint}}
\put(200,82){\usebox{\plotpoint}}
\put(200,82){\usebox{\plotpoint}}
\put(200,82){\usebox{\plotpoint}}
\put(200,82){\usebox{\plotpoint}}
\put(200,82){\usebox{\plotpoint}}
\put(200,82){\usebox{\plotpoint}}
\put(200,82){\usebox{\plotpoint}}
\put(200,82){\usebox{\plotpoint}}
\put(200,82){\usebox{\plotpoint}}
\put(200,82){\usebox{\plotpoint}}
\put(200,82){\usebox{\plotpoint}}
\put(200,82){\usebox{\plotpoint}}
\put(200,82){\usebox{\plotpoint}}
\put(200,82){\usebox{\plotpoint}}
\put(200,82){\usebox{\plotpoint}}
\put(200,82){\usebox{\plotpoint}}
\put(200,82){\usebox{\plotpoint}}
\put(200,82){\usebox{\plotpoint}}
\put(200,82){\usebox{\plotpoint}}
\put(200,82){\usebox{\plotpoint}}
\put(200,82){\usebox{\plotpoint}}
\put(200,82){\usebox{\plotpoint}}
\put(200,82){\usebox{\plotpoint}}
\put(200,82){\usebox{\plotpoint}}
\put(200,82){\usebox{\plotpoint}}
\put(200,82){\usebox{\plotpoint}}
\put(200,82){\usebox{\plotpoint}}
\put(200,82){\usebox{\plotpoint}}
\put(200,82){\usebox{\plotpoint}}
\put(200,82){\usebox{\plotpoint}}
\put(200,82){\usebox{\plotpoint}}
\put(200,82){\usebox{\plotpoint}}
\put(200,82){\usebox{\plotpoint}}
\put(200,82){\usebox{\plotpoint}}
\put(200,82){\usebox{\plotpoint}}
\put(200,82){\usebox{\plotpoint}}
\put(200,82){\usebox{\plotpoint}}
\put(200,82){\usebox{\plotpoint}}
\put(200,82){\usebox{\plotpoint}}
\put(200,82){\usebox{\plotpoint}}
\put(200,82){\usebox{\plotpoint}}
\put(200,82){\usebox{\plotpoint}}
\put(200,82){\usebox{\plotpoint}}
\put(200,82){\usebox{\plotpoint}}
\put(200,82){\usebox{\plotpoint}}
\put(200,82){\usebox{\plotpoint}}
\put(200,82){\usebox{\plotpoint}}
\put(200,82){\usebox{\plotpoint}}
\put(200,82){\usebox{\plotpoint}}
\put(200,82){\usebox{\plotpoint}}
\put(200,82){\usebox{\plotpoint}}
\put(200,82){\usebox{\plotpoint}}
\put(200,82){\usebox{\plotpoint}}
\put(200,82){\usebox{\plotpoint}}
\put(200,82){\usebox{\plotpoint}}
\put(200,82){\usebox{\plotpoint}}
\put(200,82){\usebox{\plotpoint}}
\put(200,82){\usebox{\plotpoint}}
\put(200,82){\usebox{\plotpoint}}
\put(200,82){\usebox{\plotpoint}}
\put(200,82){\usebox{\plotpoint}}
\put(200,82){\usebox{\plotpoint}}
\put(200,82){\usebox{\plotpoint}}
\put(200,82){\usebox{\plotpoint}}
\put(200,82){\usebox{\plotpoint}}
\put(200,82){\usebox{\plotpoint}}
\put(200,82){\usebox{\plotpoint}}
\put(200,82){\usebox{\plotpoint}}
\put(200,82){\usebox{\plotpoint}}
\put(200,82){\usebox{\plotpoint}}
\put(200,82){\usebox{\plotpoint}}
\put(200,82){\usebox{\plotpoint}}
\put(200,82){\usebox{\plotpoint}}
\put(200,82){\usebox{\plotpoint}}
\put(200,82){\usebox{\plotpoint}}
\put(200,82){\usebox{\plotpoint}}
\put(200,82){\usebox{\plotpoint}}
\put(200,82){\usebox{\plotpoint}}
\put(200,82){\usebox{\plotpoint}}
\put(200,82){\usebox{\plotpoint}}
\put(200,82){\usebox{\plotpoint}}
\put(200,82){\usebox{\plotpoint}}
\put(200,82){\usebox{\plotpoint}}
\put(200,82){\usebox{\plotpoint}}
\put(200,82){\usebox{\plotpoint}}
\put(200,82){\usebox{\plotpoint}}
\put(200,82){\usebox{\plotpoint}}
\put(200,82){\usebox{\plotpoint}}
\put(200,82){\usebox{\plotpoint}}
\put(200,82){\usebox{\plotpoint}}
\put(200,82){\usebox{\plotpoint}}
\put(200,82){\usebox{\plotpoint}}
\put(200,82){\usebox{\plotpoint}}
\put(200,82){\usebox{\plotpoint}}
\put(200,82){\usebox{\plotpoint}}
\put(200,82){\usebox{\plotpoint}}
\put(200,82){\usebox{\plotpoint}}
\put(200,82){\usebox{\plotpoint}}
\put(200,82){\usebox{\plotpoint}}
\put(200,82){\usebox{\plotpoint}}
\put(200,82){\usebox{\plotpoint}}
\put(200,82){\usebox{\plotpoint}}
\put(200,82){\usebox{\plotpoint}}
\put(200,82){\usebox{\plotpoint}}
\put(200,82){\usebox{\plotpoint}}
\put(200,82){\usebox{\plotpoint}}
\put(200,82){\usebox{\plotpoint}}
\put(200,82){\usebox{\plotpoint}}
\put(200,82){\usebox{\plotpoint}}
\put(200,82){\usebox{\plotpoint}}
\put(200,82){\usebox{\plotpoint}}
\put(200,82){\usebox{\plotpoint}}
\put(200,82){\usebox{\plotpoint}}
\put(200,82){\usebox{\plotpoint}}
\put(200,82){\usebox{\plotpoint}}
\put(200,82){\usebox{\plotpoint}}
\put(200,82){\usebox{\plotpoint}}
\put(200,82){\usebox{\plotpoint}}
\put(200,82){\usebox{\plotpoint}}
\put(200,82){\usebox{\plotpoint}}
\put(200,82){\usebox{\plotpoint}}
\put(200,82){\usebox{\plotpoint}}
\put(200,82){\usebox{\plotpoint}}
\put(200,82){\usebox{\plotpoint}}
\put(200,82){\usebox{\plotpoint}}
\put(200,82){\usebox{\plotpoint}}
\put(200,82){\usebox{\plotpoint}}
\put(200,82){\usebox{\plotpoint}}
\put(200,82){\usebox{\plotpoint}}
\put(200,82){\usebox{\plotpoint}}
\put(200,82){\usebox{\plotpoint}}
\put(200,82){\usebox{\plotpoint}}
\put(200,82){\usebox{\plotpoint}}
\put(200,82){\usebox{\plotpoint}}
\put(200,82){\usebox{\plotpoint}}
\put(200,82){\usebox{\plotpoint}}
\put(200,82){\usebox{\plotpoint}}
\put(200,82){\usebox{\plotpoint}}
\put(200,82){\usebox{\plotpoint}}
\put(200,82){\usebox{\plotpoint}}
\put(200,82){\usebox{\plotpoint}}
\put(200,82){\usebox{\plotpoint}}
\put(200,82){\usebox{\plotpoint}}
\put(200,82){\usebox{\plotpoint}}
\put(200,82){\usebox{\plotpoint}}
\put(200,82){\usebox{\plotpoint}}
\put(200,82){\usebox{\plotpoint}}
\put(200,82){\usebox{\plotpoint}}
\put(200,82){\usebox{\plotpoint}}
\put(200,82){\usebox{\plotpoint}}
\put(200,82){\usebox{\plotpoint}}
\put(200,82){\usebox{\plotpoint}}
\put(200,82){\usebox{\plotpoint}}
\put(200,82){\usebox{\plotpoint}}
\put(200,82){\usebox{\plotpoint}}
\put(200,82){\usebox{\plotpoint}}
\put(200,82){\usebox{\plotpoint}}
\put(200,82){\usebox{\plotpoint}}
\put(200,82){\usebox{\plotpoint}}
\put(200,82){\usebox{\plotpoint}}
\put(200,82){\usebox{\plotpoint}}
\put(200,82){\usebox{\plotpoint}}
\put(200,82){\usebox{\plotpoint}}
\put(200,82){\usebox{\plotpoint}}
\put(200,82){\usebox{\plotpoint}}
\put(200,82){\usebox{\plotpoint}}
\put(200,82){\usebox{\plotpoint}}
\put(200,82){\usebox{\plotpoint}}
\put(200,82){\usebox{\plotpoint}}
\put(200,82){\usebox{\plotpoint}}
\put(200,82){\usebox{\plotpoint}}
\put(200,82){\usebox{\plotpoint}}
\put(200,82){\usebox{\plotpoint}}
\put(200,82){\usebox{\plotpoint}}
\put(200,82){\usebox{\plotpoint}}
\put(200,82){\usebox{\plotpoint}}
\put(200,82){\usebox{\plotpoint}}
\put(200,82){\usebox{\plotpoint}}
\put(200,82){\usebox{\plotpoint}}
\put(200,82){\usebox{\plotpoint}}
\put(200,82){\usebox{\plotpoint}}
\put(200,82){\usebox{\plotpoint}}
\put(200,82){\usebox{\plotpoint}}
\put(200,82){\usebox{\plotpoint}}
\put(200,82){\usebox{\plotpoint}}
\put(200,82){\usebox{\plotpoint}}
\put(200,82){\usebox{\plotpoint}}
\put(200.0,82.0){\rule[-0.200pt]{140.686pt}{0.400pt}}
\put(784.0,82.0){\usebox{\plotpoint}}
\put(784.0,82.0){\usebox{\plotpoint}}
\put(784.0,82.0){\rule[-0.200pt]{0.723pt}{0.400pt}}
\put(787.0,82.0){\usebox{\plotpoint}}
\put(787.0,82.0){\usebox{\plotpoint}}
\put(787.0,82.0){\rule[-0.200pt]{0.482pt}{0.400pt}}
\put(789.0,82.0){\usebox{\plotpoint}}
\put(789.0,83.0){\rule[-0.200pt]{0.482pt}{0.400pt}}
\put(791.0,82.0){\usebox{\plotpoint}}
\put(791.0,82.0){\usebox{\plotpoint}}
\put(792.0,82.0){\usebox{\plotpoint}}
\put(792.0,83.0){\rule[-0.200pt]{0.482pt}{0.400pt}}
\put(794.0,82.0){\usebox{\plotpoint}}
\put(794.0,82.0){\usebox{\plotpoint}}
\put(795.0,82.0){\usebox{\plotpoint}}
\put(795.0,83.0){\rule[-0.200pt]{0.482pt}{0.400pt}}
\put(797.0,82.0){\usebox{\plotpoint}}
\put(797.0,82.0){\usebox{\plotpoint}}
\put(798.0,82.0){\usebox{\plotpoint}}
\put(798.0,83.0){\usebox{\plotpoint}}
\put(799.0,83.0){\usebox{\plotpoint}}
\put(799.0,84.0){\usebox{\plotpoint}}
\put(800.0,82.0){\rule[-0.200pt]{0.400pt}{0.482pt}}
\put(800.0,82.0){\usebox{\plotpoint}}
\put(801.0,82.0){\usebox{\plotpoint}}
\put(801.0,83.0){\usebox{\plotpoint}}
\put(802.0,83.0){\usebox{\plotpoint}}
\put(802.0,84.0){\usebox{\plotpoint}}
\put(803.0,83.0){\usebox{\plotpoint}}
\put(803.0,83.0){\usebox{\plotpoint}}
\put(804.0,82.0){\usebox{\plotpoint}}
\put(804.0,82.0){\usebox{\plotpoint}}
\put(804.0,83.0){\usebox{\plotpoint}}
\put(805.0,83.0){\rule[-0.200pt]{0.400pt}{0.482pt}}
\put(805.0,85.0){\usebox{\plotpoint}}
\put(806.0,83.0){\rule[-0.200pt]{0.400pt}{0.482pt}}
\put(806.0,83.0){\usebox{\plotpoint}}
\put(807.0,82.0){\usebox{\plotpoint}}
\put(807.0,82.0){\usebox{\plotpoint}}
\put(807.0,83.0){\usebox{\plotpoint}}
\put(808.0,83.0){\rule[-0.200pt]{0.400pt}{0.964pt}}
\put(808.0,87.0){\usebox{\plotpoint}}
\put(809.0,87.0){\usebox{\plotpoint}}
\put(809.0,85.0){\rule[-0.200pt]{0.400pt}{0.723pt}}
\put(809.0,85.0){\usebox{\plotpoint}}
\put(810.0,82.0){\rule[-0.200pt]{0.400pt}{0.723pt}}
\put(810.0,82.0){\usebox{\plotpoint}}
\put(810.0,83.0){\usebox{\plotpoint}}
\put(811.0,83.0){\rule[-0.200pt]{0.400pt}{2.168pt}}
\put(811.0,92.0){\usebox{\plotpoint}}
\put(812.0,92.0){\usebox{\plotpoint}}
\put(812.0,89.0){\rule[-0.200pt]{0.400pt}{0.964pt}}
\put(812.0,89.0){\usebox{\plotpoint}}
\put(813.0,82.0){\rule[-0.200pt]{0.400pt}{1.686pt}}
\put(813.0,82.0){\usebox{\plotpoint}}
\put(813.0,83.0){\usebox{\plotpoint}}
\put(814.0,83.0){\rule[-0.200pt]{0.400pt}{5.541pt}}
\put(814.0,106.0){\usebox{\plotpoint}}
\put(815.0,106.0){\rule[-0.200pt]{0.400pt}{2.168pt}}
\put(815.0,109.0){\rule[-0.200pt]{0.400pt}{1.445pt}}
\put(815.0,109.0){\usebox{\plotpoint}}
\put(816.0,82.0){\rule[-0.200pt]{0.400pt}{6.504pt}}
\put(816.0,82.0){\usebox{\plotpoint}}
\put(816.0,83.0){\usebox{\plotpoint}}
\put(817.0,83.0){\rule[-0.200pt]{0.400pt}{32.521pt}}
\put(817.0,218.0){\usebox{\plotpoint}}
\put(818.0,218.0){\rule[-0.200pt]{0.400pt}{84.556pt}}
\put(818.0,569.0){\usebox{\plotpoint}}
\put(819.0,569.0){\rule[-0.200pt]{0.400pt}{50.107pt}}
\put(819.0,777.0){\usebox{\plotpoint}}
\put(820.0,569.0){\rule[-0.200pt]{0.400pt}{50.107pt}}
\put(820.0,569.0){\usebox{\plotpoint}}
\put(821.0,218.0){\rule[-0.200pt]{0.400pt}{84.556pt}}
\put(821.0,218.0){\usebox{\plotpoint}}
\put(822.0,83.0){\rule[-0.200pt]{0.400pt}{32.521pt}}
\put(822.0,83.0){\usebox{\plotpoint}}
\put(823.0,82.0){\usebox{\plotpoint}}
\put(823.0,82.0){\rule[-0.200pt]{0.400pt}{6.504pt}}
\put(823.0,109.0){\usebox{\plotpoint}}
\put(824.0,109.0){\rule[-0.200pt]{0.400pt}{1.445pt}}
\put(824.0,106.0){\rule[-0.200pt]{0.400pt}{2.168pt}}
\put(824.0,106.0){\usebox{\plotpoint}}
\put(825.0,83.0){\rule[-0.200pt]{0.400pt}{5.541pt}}
\put(825.0,83.0){\usebox{\plotpoint}}
\put(826.0,82.0){\usebox{\plotpoint}}
\put(826.0,82.0){\rule[-0.200pt]{0.400pt}{1.686pt}}
\put(826.0,89.0){\usebox{\plotpoint}}
\put(827.0,89.0){\rule[-0.200pt]{0.400pt}{0.964pt}}
\put(827.0,92.0){\usebox{\plotpoint}}
\put(827.0,92.0){\usebox{\plotpoint}}
\put(828.0,83.0){\rule[-0.200pt]{0.400pt}{2.168pt}}
\put(828.0,83.0){\usebox{\plotpoint}}
\put(829.0,82.0){\usebox{\plotpoint}}
\put(829.0,82.0){\rule[-0.200pt]{0.400pt}{0.723pt}}
\put(829.0,85.0){\usebox{\plotpoint}}
\put(830.0,85.0){\rule[-0.200pt]{0.400pt}{0.723pt}}
\put(830.0,87.0){\usebox{\plotpoint}}
\put(830.0,87.0){\usebox{\plotpoint}}
\put(831.0,83.0){\rule[-0.200pt]{0.400pt}{0.964pt}}
\put(831.0,83.0){\usebox{\plotpoint}}
\put(832.0,82.0){\usebox{\plotpoint}}
\put(832.0,82.0){\usebox{\plotpoint}}
\put(832.0,83.0){\usebox{\plotpoint}}
\put(833.0,83.0){\rule[-0.200pt]{0.400pt}{0.482pt}}
\put(833.0,85.0){\usebox{\plotpoint}}
\put(834.0,83.0){\rule[-0.200pt]{0.400pt}{0.482pt}}
\put(834.0,83.0){\usebox{\plotpoint}}
\put(835.0,82.0){\usebox{\plotpoint}}
\put(835.0,82.0){\usebox{\plotpoint}}
\put(835.0,83.0){\usebox{\plotpoint}}
\put(836.0,83.0){\usebox{\plotpoint}}
\put(836.0,84.0){\usebox{\plotpoint}}
\put(837.0,83.0){\usebox{\plotpoint}}
\put(837.0,83.0){\usebox{\plotpoint}}
\put(838.0,82.0){\usebox{\plotpoint}}
\put(838.0,82.0){\usebox{\plotpoint}}
\put(839.0,82.0){\rule[-0.200pt]{0.400pt}{0.482pt}}
\put(839.0,84.0){\usebox{\plotpoint}}
\put(840.0,83.0){\usebox{\plotpoint}}
\put(840.0,83.0){\usebox{\plotpoint}}
\put(841.0,82.0){\usebox{\plotpoint}}
\put(841.0,82.0){\usebox{\plotpoint}}
\put(842.0,82.0){\usebox{\plotpoint}}
\put(842.0,83.0){\rule[-0.200pt]{0.482pt}{0.400pt}}
\put(844.0,82.0){\usebox{\plotpoint}}
\put(844.0,82.0){\usebox{\plotpoint}}
\put(845.0,82.0){\usebox{\plotpoint}}
\put(845.0,83.0){\rule[-0.200pt]{0.482pt}{0.400pt}}
\put(847.0,82.0){\usebox{\plotpoint}}
\put(847.0,82.0){\usebox{\plotpoint}}
\put(848.0,82.0){\usebox{\plotpoint}}
\put(848.0,83.0){\rule[-0.200pt]{0.482pt}{0.400pt}}
\put(850.0,82.0){\usebox{\plotpoint}}
\put(850.0,82.0){\rule[-0.200pt]{0.482pt}{0.400pt}}
\put(852.0,82.0){\usebox{\plotpoint}}
\put(852.0,82.0){\usebox{\plotpoint}}
\put(852.0,82.0){\rule[-0.200pt]{0.723pt}{0.400pt}}
\put(855.0,82.0){\usebox{\plotpoint}}
\put(855.0,82.0){\usebox{\plotpoint}}
\put(855.0,82.0){\rule[-0.200pt]{140.686pt}{0.400pt}}
\put(200.0,82.0){\rule[-0.200pt]{0.400pt}{167.425pt}}
\put(200.0,82.0){\rule[-0.200pt]{298.475pt}{0.400pt}}
\put(1439.0,82.0){\rule[-0.200pt]{0.400pt}{167.425pt}}
\put(200.0,777.0){\rule[-0.200pt]{298.475pt}{0.400pt}}
\end{picture}

 % GNUPLOT: LaTeX picture
\setlength{\unitlength}{0.240900pt}
\ifx\plotpoint\undefined\newsavebox{\plotpoint}\fi
\sbox{\plotpoint}{\rule[-0.200pt]{0.400pt}{0.400pt}}%
\begin{picture}(1500,900)(0,0)
\sbox{\plotpoint}{\rule[-0.200pt]{0.400pt}{0.400pt}}%
\put(220.0,82.0){\rule[-0.200pt]{4.818pt}{0.400pt}}
\put(200,82){\makebox(0,0)[r]{$0$}}
\put(1419.0,82.0){\rule[-0.200pt]{4.818pt}{0.400pt}}
\put(220.0,151.0){\rule[-0.200pt]{4.818pt}{0.400pt}}
\put(200,151){\makebox(0,0)[r]{$100000$}}
\put(1419.0,151.0){\rule[-0.200pt]{4.818pt}{0.400pt}}
\put(220.0,221.0){\rule[-0.200pt]{4.818pt}{0.400pt}}
\put(200,221){\makebox(0,0)[r]{$200000$}}
\put(1419.0,221.0){\rule[-0.200pt]{4.818pt}{0.400pt}}
\put(220.0,290.0){\rule[-0.200pt]{4.818pt}{0.400pt}}
\put(200,290){\makebox(0,0)[r]{$300000$}}
\put(1419.0,290.0){\rule[-0.200pt]{4.818pt}{0.400pt}}
\put(220.0,360.0){\rule[-0.200pt]{4.818pt}{0.400pt}}
\put(200,360){\makebox(0,0)[r]{$400000$}}
\put(1419.0,360.0){\rule[-0.200pt]{4.818pt}{0.400pt}}
\put(220.0,429.0){\rule[-0.200pt]{4.818pt}{0.400pt}}
\put(200,429){\makebox(0,0)[r]{$500000$}}
\put(1419.0,429.0){\rule[-0.200pt]{4.818pt}{0.400pt}}
\put(220.0,499.0){\rule[-0.200pt]{4.818pt}{0.400pt}}
\put(200,499){\makebox(0,0)[r]{$600000$}}
\put(1419.0,499.0){\rule[-0.200pt]{4.818pt}{0.400pt}}
\put(220.0,568.0){\rule[-0.200pt]{4.818pt}{0.400pt}}
\put(200,568){\makebox(0,0)[r]{$700000$}}
\put(1419.0,568.0){\rule[-0.200pt]{4.818pt}{0.400pt}}
\put(220.0,638.0){\rule[-0.200pt]{4.818pt}{0.400pt}}
\put(200,638){\makebox(0,0)[r]{$800000$}}
\put(1419.0,638.0){\rule[-0.200pt]{4.818pt}{0.400pt}}
\put(220.0,707.0){\rule[-0.200pt]{4.818pt}{0.400pt}}
\put(200,707){\makebox(0,0)[r]{$900000$}}
\put(1419.0,707.0){\rule[-0.200pt]{4.818pt}{0.400pt}}
\put(220.0,777.0){\rule[-0.200pt]{4.818pt}{0.400pt}}
\put(200,777){\makebox(0,0)[r]{$1e+06$}}
\put(1419.0,777.0){\rule[-0.200pt]{4.818pt}{0.400pt}}
\put(220.0,82.0){\rule[-0.200pt]{0.400pt}{4.818pt}}
\put(220,41){\makebox(0,0){$-4\pi$}}
\put(220.0,757.0){\rule[-0.200pt]{0.400pt}{4.818pt}}
\put(372.0,82.0){\rule[-0.200pt]{0.400pt}{4.818pt}}
\put(372,41){\makebox(0,0){$-3\pi$}}
\put(372.0,757.0){\rule[-0.200pt]{0.400pt}{4.818pt}}
\put(525.0,82.0){\rule[-0.200pt]{0.400pt}{4.818pt}}
\put(525,41){\makebox(0,0){$-2\pi$}}
\put(525.0,757.0){\rule[-0.200pt]{0.400pt}{4.818pt}}
\put(830.0,82.0){\rule[-0.200pt]{0.400pt}{4.818pt}}
\put(830,41){\makebox(0,0){0}}
\put(830.0,757.0){\rule[-0.200pt]{0.400pt}{4.818pt}}
\put(1134.0,82.0){\rule[-0.200pt]{0.400pt}{4.818pt}}
\put(1134,41){\makebox(0,0){$2\pi$}}
\put(1134.0,757.0){\rule[-0.200pt]{0.400pt}{4.818pt}}
\put(1287.0,82.0){\rule[-0.200pt]{0.400pt}{4.818pt}}
\put(1287,41){\makebox(0,0){$3\pi$}}
\put(1287.0,757.0){\rule[-0.200pt]{0.400pt}{4.818pt}}
\put(1439.0,82.0){\rule[-0.200pt]{0.400pt}{4.818pt}}
\put(1439,41){\makebox(0,0){$4\pi$}}
\put(1439.0,757.0){\rule[-0.200pt]{0.400pt}{4.818pt}}
\put(220.0,82.0){\rule[-0.200pt]{0.400pt}{167.425pt}}
\put(220.0,82.0){\rule[-0.200pt]{293.657pt}{0.400pt}}
\put(1439.0,82.0){\rule[-0.200pt]{0.400pt}{167.425pt}}
\put(220.0,777.0){\rule[-0.200pt]{293.657pt}{0.400pt}}
\put(829,839){\makebox(0,0){$f(\omega)$ für $t=1000$}}
\put(220,82){\usebox{\plotpoint}}
\put(220,82){\usebox{\plotpoint}}
\put(220,82){\usebox{\plotpoint}}
\put(220,82){\usebox{\plotpoint}}
\put(220,82){\usebox{\plotpoint}}
\put(220,82){\usebox{\plotpoint}}
\put(220,82){\usebox{\plotpoint}}
\put(220,82){\usebox{\plotpoint}}
\put(220,82){\usebox{\plotpoint}}
\put(220,82){\usebox{\plotpoint}}
\put(220,82){\usebox{\plotpoint}}
\put(220,82){\usebox{\plotpoint}}
\put(220,82){\usebox{\plotpoint}}
\put(220,82){\usebox{\plotpoint}}
\put(220,82){\usebox{\plotpoint}}
\put(220,82){\usebox{\plotpoint}}
\put(220,82){\usebox{\plotpoint}}
\put(220,82){\usebox{\plotpoint}}
\put(220,82){\usebox{\plotpoint}}
\put(220,82){\usebox{\plotpoint}}
\put(220,82){\usebox{\plotpoint}}
\put(220,82){\usebox{\plotpoint}}
\put(220,82){\usebox{\plotpoint}}
\put(220,82){\usebox{\plotpoint}}
\put(220,82){\usebox{\plotpoint}}
\put(220,82){\usebox{\plotpoint}}
\put(220,82){\usebox{\plotpoint}}
\put(220,82){\usebox{\plotpoint}}
\put(220,82){\usebox{\plotpoint}}
\put(220,82){\usebox{\plotpoint}}
\put(220,82){\usebox{\plotpoint}}
\put(220,82){\usebox{\plotpoint}}
\put(220,82){\usebox{\plotpoint}}
\put(220,82){\usebox{\plotpoint}}
\put(220,82){\usebox{\plotpoint}}
\put(220,82){\usebox{\plotpoint}}
\put(220,82){\usebox{\plotpoint}}
\put(220,82){\usebox{\plotpoint}}
\put(220,82){\usebox{\plotpoint}}
\put(220,82){\usebox{\plotpoint}}
\put(220,82){\usebox{\plotpoint}}
\put(220,82){\usebox{\plotpoint}}
\put(220.0,82.0){\rule[-0.200pt]{145.985pt}{0.400pt}}
\put(826.0,82.0){\usebox{\plotpoint}}
\put(826.0,82.0){\usebox{\plotpoint}}
\put(826.0,82.0){\usebox{\plotpoint}}
\put(826.0,82.0){\usebox{\plotpoint}}
\put(826.0,82.0){\usebox{\plotpoint}}
\put(827.0,82.0){\usebox{\plotpoint}}
\put(827.0,82.0){\usebox{\plotpoint}}
\put(827.0,82.0){\usebox{\plotpoint}}
\put(827.0,82.0){\usebox{\plotpoint}}
\put(827.0,82.0){\usebox{\plotpoint}}
\put(827.0,82.0){\usebox{\plotpoint}}
\put(827.0,82.0){\rule[-0.200pt]{0.400pt}{0.482pt}}
\put(827.0,84.0){\usebox{\plotpoint}}
\put(828.0,82.0){\rule[-0.200pt]{0.400pt}{0.482pt}}
\put(828.0,82.0){\rule[-0.200pt]{0.400pt}{0.482pt}}
\put(828.0,82.0){\rule[-0.200pt]{0.400pt}{0.482pt}}
\put(828.0,82.0){\rule[-0.200pt]{0.400pt}{0.723pt}}
\put(828.0,82.0){\rule[-0.200pt]{0.400pt}{0.723pt}}
\put(828.0,82.0){\rule[-0.200pt]{0.400pt}{1.445pt}}
\put(828.0,86.0){\rule[-0.200pt]{0.400pt}{0.482pt}}
\put(828.0,86.0){\usebox{\plotpoint}}
\put(829.0,82.0){\rule[-0.200pt]{0.400pt}{0.964pt}}
\put(829.0,82.0){\rule[-0.200pt]{0.400pt}{2.650pt}}
\put(829.0,82.0){\rule[-0.200pt]{0.400pt}{2.650pt}}
\put(829.0,82.0){\rule[-0.200pt]{0.400pt}{7.950pt}}
\put(829.0,82.0){\rule[-0.200pt]{0.400pt}{7.950pt}}
\put(829.0,82.0){\rule[-0.200pt]{0.400pt}{167.185pt}}
\put(829.0,776.0){\usebox{\plotpoint}}
\put(830.0,82.0){\rule[-0.200pt]{0.400pt}{167.185pt}}
\put(830.0,82.0){\rule[-0.200pt]{0.400pt}{7.950pt}}
\put(830.0,82.0){\rule[-0.200pt]{0.400pt}{7.950pt}}
\put(830.0,82.0){\rule[-0.200pt]{0.400pt}{2.650pt}}
\put(830.0,82.0){\rule[-0.200pt]{0.400pt}{2.650pt}}
\put(830.0,82.0){\rule[-0.200pt]{0.400pt}{0.964pt}}
\put(830.0,86.0){\usebox{\plotpoint}}
\put(831.0,86.0){\rule[-0.200pt]{0.400pt}{0.482pt}}
\put(831.0,82.0){\rule[-0.200pt]{0.400pt}{1.445pt}}
\put(831.0,82.0){\rule[-0.200pt]{0.400pt}{0.723pt}}
\put(831.0,82.0){\rule[-0.200pt]{0.400pt}{0.723pt}}
\put(831.0,82.0){\rule[-0.200pt]{0.400pt}{0.482pt}}
\put(831.0,82.0){\rule[-0.200pt]{0.400pt}{0.482pt}}
\put(831.0,82.0){\rule[-0.200pt]{0.400pt}{0.482pt}}
\put(831.0,84.0){\usebox{\plotpoint}}
\put(832.0,82.0){\rule[-0.200pt]{0.400pt}{0.482pt}}
\put(832.0,82.0){\usebox{\plotpoint}}
\put(832.0,82.0){\usebox{\plotpoint}}
\put(832.0,82.0){\usebox{\plotpoint}}
\put(832.0,82.0){\usebox{\plotpoint}}
\put(832.0,82.0){\usebox{\plotpoint}}
\put(832.0,82.0){\usebox{\plotpoint}}
\put(832.0,82.0){\usebox{\plotpoint}}
\put(833.0,82.0){\usebox{\plotpoint}}
\put(833.0,82.0){\usebox{\plotpoint}}
\put(833.0,82.0){\usebox{\plotpoint}}
\put(833.0,82.0){\usebox{\plotpoint}}
\put(833.0,82.0){\rule[-0.200pt]{145.985pt}{0.400pt}}
\put(220.0,82.0){\rule[-0.200pt]{0.400pt}{167.425pt}}
\put(220.0,82.0){\rule[-0.200pt]{293.657pt}{0.400pt}}
\put(1439.0,82.0){\rule[-0.200pt]{0.400pt}{167.425pt}}
\put(220.0,777.0){\rule[-0.200pt]{293.657pt}{0.400pt}}
\end{picture}

 \caption{Die Funktion  \(f(\omega_{ni}) \) für verschiedene Zeiten \(t=1, t=100, t=1000\).  Man erkennt, dass die Funktion \(f(\omega_{ni})\) für größere Zeiten sich einer \(\delta\)-Funktion nähert. }
 \label{fig:1}

\end{minipage}
\end{figure}

Wobei es gilt 
\begin{align}
  \label{eq:30}
  f(\omega_{ni})= \frac{4}{\hbar^2\omega_{ni}^2}\sin^2(\frac{\omega_{ni} t}{2}) \quad \text{ mit } \omega_{ni}=\frac{E_n-E_i}{\hbar}
\end{align}

Wir wollen wir die Funktion \(f(\omega_{ni}) \) weiter Vereinfachen. Dazu betrachten wir sie für verschiedene \(t\). Siehe dazu Abbildung \ref{fig:1}.

Wie man in den Abbildung deutlich erkennt, nähert sich die Funktion \(f(\omega_{ni})\) für große \(t\) einer \(\delta\)-Funktion. D.h. es gilt

\begin{align}
  \label{eq:31}
  f(\omega_{ni}) \stackrel{t \to \infty}= c \delta(\omega_{ni})
\end{align}

Um die Konstante \(c\) zu bestimmen integrieren wir die Gleichung (\ref{eq:31}) auf beiden Seiten nach \(d\omega\) über das gesamte Intervall

\begin{align}
  \label{eq:32}
  \int_{-\infty}^\infty d\omega f(\omega_{ni}) = c \underbr{ \int_{-\infty}^\infty d\omega \delta(\omega_{ni}) }_{=1}
\end{align}

Das heißt, es gilt folgendes Integral zu berechnen

\begin{align}
  \label{eq:33}
  c &= \int_{-\infty}^\infty d\omega f(\omega_{ni}) \qquad \text{ mit } f(\omega_{ni})=\frac{4}{\hbar^2\omega_{ni}^2}\sin^2(\frac{\omega_{ni} t}{2}) \notag\\
&=\frac{4}{\hbar^2} \int_{-\infty}^\infty d\omega \frac{1}{\omega_{ni}^2}\sin^2(\frac{\omega_{ni} t}{2})
\end{align}

Zum Berechnen des Integrals ist eine Substitution des Sinus Arguments \(x= \frac{\omega_{ni} t}{2}\)  notwendig. Mit \(\omega = \frac{2x}{t} \) und \(d\omega = \frac{2dx}{t}\) eingesetzt in Gleichung (\ref{eq:33}) folgt

\begin{align}
  \label{eq:34}
  c = \frac{4}{\hbar^2} \int_{-\infty}^\infty dx \frac{2}{t} \frac{t^2}{4x^2}\sin^2(x) = \frac{2t}{\hbar^2} \underbr{\int_{-\infty}^\infty dx \frac{\sin^2(x)}{x^2}}_{\pi} = \frac{2t}{\hbar^2} \pi
\end{align}

Setzen wir \(c\) in die Gleichung (\ref{eq:31}) ein so vereinfacht sich die Funktion \(f(\omega_{ni})\) zu

\begin{align}
  \label{eq:35}
  f(\omega_{ni}) \stackrel{t \to \infty}= \frac{2\pi t}{\hbar^2} \delta(\omega_{ni})
\end{align}

Damit können wir die Übergangswahrscheinlichkeit Gleichung (\ref{eq:29}) für große Zeiten schreiben

\begin{align}
  \label{eq:36}
   P(i\rightarrow n) = \frac{2\pi t}{\hbar^2} |V_{ni}|^2  \delta(\omega_{ni}) = \frac{2\pi t}{\hbar^2} |V_{ni}|^2  \delta(\frac{E_n-E_i}{\hbar}) = \frac{2\pi t}{\hbar} |V_{ni}|^2  \delta(E_n-E_i)
\end{align}

Gerne verwendet man anstelle der Übergangswahrscheinlichkeit die Übergangsrate, die als Übergangswahrscheinlichkeit pro Zeiteinheit definiert ist \(w_{i\to n}= \diff_t P(i\rightarrow n)\). Damit gilt

\begin{align}
  \label{eq:37}
\boxed{  w_{i\to n}= |V_{ni}|^2 \frac{2\pi}{\hbar} \delta(E_n-E_i) }
\end{align}

Diese Gleichung (\ref{eq:37}) wird auch als \textbf{Fermis-Goldene-Regel} bezeichnet.
\\
Wie man aus der Gleichung (\ref{eq:36}) unschwer erkennen kann, gibt es nur eine Wahrscheinlichkeit für ein Übergang zwischen zwei Zuständen wenn ihre Energieniveaus gleich sind (wegen der \(\delta\)-Funktion, was der Energieerhaltung entspricht. Zum Beispiel bei der Streuung betrachtet man eine einfallende und gestreute Teilchen-Welle die zwei unterschiedliche Zustände repräsentieren. Jedoch ist die Energie der einfallenden und gestreuten Welle gleich. Oder beim Zerfall eines Neutrons in ein Proton, Elektron und ein Elektron-Antineutrino handelt es sich ebenso um zwei unterschiedliche Zustände, nämlich den Zustand des Neutrons \(\ket{i}\) und dem Zustand von den resultierenden drei Teilchen, die man mit dem Zustand \(\ket{n}\) beschreibt. In beiden Zuständen bleibt die Energie erhalten.

Normalerweise betrachtet man nicht die Übergangswahrscheinlichkeit zwischen zwei bestimmten Energieniveaus, sondern die Übergangswahrscheinlichkeit zwischen allen Zuständen in einem Energieniveau im Intervall \([E,E+dE]\). Das bezeichnet man als die \textit{totale Übergangswahrscheinlichkeit}. Für die gilt (für \(i\ne n\))

\begin{align}
  \label{eq:38}
  P\approx \sum_{E_i\approx E_n} P(i\rightarrow n) = \sum_{E_i\approx E_n} |V_{ni}|^2 \frac{2\pi t}{\hbar} \delta(E_n-E_i) =  |V_{ni}|^2 \frac{2\pi t}{\hbar} \underbr{ \sum_{E_i\approx E_n}\delta(E_n-E_i)}_{\rho(E_n) }
\end{align}

Mit der Zustandsdichte \(\rho\), die die Dichte der Energie-Zustände in einem Intervall \([E,E+dE]\) angibt. Für diese gilt

\begin{align}
  \label{eq:39}
  \rho(E_n) = \sum_{E_i\approx E_n}\delta(E_n-E_i) \equiv \int dE \delta(E_n-E_i)
\end{align}

Aus der Gleichung (\ref{eq:38}) folgt die totale Übergangsrate die eine andere Form der Fermis-Goldene-Regel darstellt

\begin{align}
  \label{eq:40}
  \boxed{  w_{i\to \{n\}}= \frac{2\pi }{\hbar} |V_{ni}|^2  \rho(E_n) }
\end{align}

\end{document}
