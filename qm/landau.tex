\input{../headers/header_script.tex}
%\includegraphics[width=0.75\textwidth]{thepic.png}

\begin{document}

\section*{Landau Niveaus}

Wir betrachten ein Teilchen im Magnetfeld. Das konstante Magnetfeld zeigt in z-Richtung \(\vec B=(0,0,B_0)\). Das Vektorpotential ist nach Landau-Eichung (\(\nabla\cdot\vec A = 0\)) somit \(\vec A = (-yB_0,0,0)\). Der Hamiltonoperator lautet:

\[H=\frac{1}{2m}\left(\vec p - \frac{q}{c}\vec A\right)^2 \]


In Quantenmechanischer Schreibweise:

\[H=\frac{1}{2m}\left(\frac{\hbar}{i}\nabla - \frac{q}{c}\vec A\right)^2 \]

Einsetzen des Hamiltonoperators in die Schrödinger-Gleichung:

\[H\psi = E\psi\]

\[ \frac{1}{2m}\left(\frac{\hbar}{i}\nabla - \frac{q}{c}\vec A\right)\left(\frac{\hbar}{i}\nabla - \frac{q}{c}\vec A\right)\psi = E\psi \]

\[ \frac{1}{2m}\left(-\hbar^2\nabla^2 -\frac{\hbar q}{ic}\nabla \vec A -\frac{\hbar q}{ic} \vec A \nabla  + \frac{q^2}{c^2}\vec A^2 \right) \psi = E\psi \]


\[ \frac{1}{2m}\left(-\hbar^2\nabla^2\psi -\frac{\hbar q}{ic}\underbrace{\nabla \vec A\psi}_{(\nabla \vec A)\psi + \vec A\cdot(\nabla\psi)} -\frac{\hbar q}{ic} \vec A \nabla\psi  + \frac{q^2}{c^2}\vec A^2\psi \right) = E\psi \]

Nach der Landau Eichung ist der Term \((\nabla \vec A)\psi = 0\)

\[ \frac{1}{2m}\left(-\hbar^2\nabla^2\psi -\frac{\hbar q}{ic}\vec A\cdot(\nabla\psi) -\frac{\hbar q}{ic} \vec A \nabla\psi  + \frac{q^2}{c^2}\vec A^2\psi \right) = E\psi \]

\[ \frac{1}{2m}\left(-\hbar^2\nabla^2\psi - \frac{2 \hbar q}{ic}\vec A\cdot(\nabla\psi) + \frac{q^2}{c^2}\vec A^2\psi \right) = E\psi \]


Einsetzen des Vektorfeldes ergibt nur noch eine Ableitung in die x-Richtung:


\[ \frac{1}{2m}\left(-\hbar^2\nabla^2\psi + \frac{2 \hbar q}{ic} yB_0 \frac{\partial}{\partial x} \psi + \frac{q^2}{c^2}y^2B_0^2\psi \right) = E\psi \]

Mit der Annahme vom \(\psi(x,y,z) = e^{\frac{i}{\hbar}p_xx}\cdot \phi(y)\cdot e^{\frac{i}{\hbar}p_zz}\) können wir erstmal \(\nabla^2\psi\) ausrechnen NR:

\[\nabla\psi = \begin{pmatrix} \frac{ip_x}{\hbar}e^{ip_xx/\hbar}\cdot\phi(y)e^{\frac{i}{\hbar}p_zz} \\ e^{ip_xx/\hbar}\cdot\frac{\partial}{\partial y}\phi(y)e^{\frac{i}{\hbar}p_zz}\\\frac{ip_z}{\hbar}e^{ip_xx/\hbar}\cdot\phi(y)e^{\frac{i}{\hbar}p_zz} \end{pmatrix}  \]
\[\nabla^2\psi =- \frac{p_x^2}{\hbar^2}e^{ip_xx/\hbar}\phi(y)e^{\frac{i}{\hbar}p_zz} + e^{ip_xx/\hbar}\frac{\partial^2}{\partial y^2}\phi(y)e^{\frac{i}{\hbar}p_zz}  - \frac{p_z^2}{\hbar^2}e^{ip_xx/\hbar}\phi(y)e^{\frac{i}{\hbar}p_zz}\]

Ebenso die Ableitung nach x NR:

\[\frac{\partial}{\partial x}\psi =\frac{i}{\hbar}p_x e^{\frac{i}{\hbar}p_xx}\cdot \phi(y) e^{\frac{i}{\hbar}p_zz} \]


Die NR in wieder in die SGL einsetzen:


\[\begin{split} \frac{1}{2m}\left(p_x^2e^{ip_xx/\hbar}\phi(y)e^{\frac{i}{\hbar}p_zz} -\hbar^2e^{ip_xx/\hbar}\frac{\partial^2}{\partial y^2}\phi(y)e^{\frac{i}{\hbar}p_zz} + p_z^2e^{ip_xx/\hbar}\phi(y)e^{\frac{i}{\hbar}p_zz}  \right. \\
\left.  + \frac{2 \hbar q}{ic} yB_0\frac{i}{\hbar}p_x e^{\frac{i}{\hbar}p_xx}\cdot \phi(y)e^{\frac{i}{\hbar}p_zz}  + \frac{q^2}{c^2}y^2B_0^2\psi \right) = E\psi \end{split} \]

Gleichung durch \( e^{\frac{i}{\hbar}p_xx} e^{\frac{i}{\hbar}p_zz} \) teilen:

\[ \frac{1}{2m}\left(p_x^2\phi(y) - \hbar^2 \frac{\partial^2}{\partial y^2}\phi(y) + p_z^2\phi(y) + \frac{2 q}{c} yB_0p_x  \phi(y)  + \frac{q^2}{c^2}y^2B_0^2\phi(y) \right) = E \phi(y) \]


\[  \frac{1}{2m}\left(- \hbar^2\frac{\partial^2}{\partial y^2}\phi(y) +  \underbrace{\left[p_x^2 + \frac{2 q}{c} yB_0p_x   + \frac{q^2}{c^2}y^2B_0^2\right]}_{\text{Binomische Formel}}\phi(y) + p_z^2\phi(y) \right) = E \phi(y) \]

\[  \frac{1}{2m}\left(-\hbar^2 \frac{\partial^2}{\partial y^2}\phi(y) +  \left[\frac{q}{c}B_0\cdot y +  p_x \right]^2\phi(y) + p_z^2\phi(y) \right) = E \phi(y) \]


\[ -\frac{ \hbar^2}{2m} \frac{\partial^2}{\partial y^2}\phi(y) + \frac{q^2 B_0^2}{2mc^2} \left[y +  \frac{p_xc}{qB_0} \right]^2\phi(y) +\frac{p_z^2}{2m}\phi(y)  = E \phi(y) \]


Mit der Zyklotronfrequenz \( \boxed{\omega_c = \frac{qB_0}{cm}} \) und \(y_0 = \frac{p_x}{m\omega_c}\) kann die SGL in eine HO und die eines freien Teilchens vergleichbare Form gebracht werden:

\[  \left( \underbrace{ \frac{- \hbar^2}{2m} \frac{\partial^2}{\partial y^2} + \frac{m\omega_c^2}{2} (y +  y_0 )^2}_{\text{Harmonischer Oszillator}} +\underbrace{ \frac{p_z^2}{2m}}_{\text{freies Teilchen}}  \right)\phi(y) = E \phi(y) \]


Vergleiche mit dem Harmonischen Oszillator in y-Richtung: \( H = \frac{p^2_y}{2m} + \frac{m\omega^2}{2}y^2 \) Ergeben sich die Eigenwerte dazu:

\begin{equation}
\label{eq:42}
 \boxed{E_n^{\text{Landau}} = \hbar\omega_c(n+\frac{1}{2}) + \frac{p_z^2}{2m} }
\end{equation}
\\
Wir haben eine Wellenfunktion die in x und z Richtung eine Ebene Welle ist und in y Richtung die Lösung eines Harmonischen Oszillators darstellt. Die Auszeichnung der y Richtung liegt an der von uns gewählten Eichung bzw. dem gewählten Vektorpotential  \(\vec A = (-yB_0,0,0)\). Wie auch immer man die Eichung wählt, das entscheidende ist, dass das Magneteld die Ausbreitung der freier ebener Wellen nur in Feldrichtung erlaubt. In unserem Fall ist es die z Richtung. Wie man aus der Gleichung \eqref{eq:42} sieht, sind die Energieeigenwerte unendlichfach entartet. Nur die x-y-Ebene ist quantisiert.




\end{document}
