\documentclass[10pt,a4paper,oneside,fleqn]{article}
\usepackage{geometry}
\geometry{a4paper,left=20mm,right=20mm,top=1cm,bottom=2cm}
\usepackage[utf8]{inputenc}
%\usepackage{ngerman}
\usepackage{amsmath}                % brauche ich um dir Formel zu umrahmen.
\usepackage{amsfonts}                % brauche ich für die Mengensymbole
\usepackage{graphicx}
\setlength{\parindent}{0px}
\setlength{\mathindent}{10mm}
\usepackage{bbold}                    %brauche ich für die doppel Zahlen Darstellung (Einheitsmatrix z.B)



\usepackage{color}
\usepackage{titlesec} %sudo apt-get install texlive-latex-extra

\definecolor{darkblue}{rgb}{0.1,0.1,0.55}
\definecolor{verydarkblue}{rgb}{0.1,0.1,0.35}
\definecolor{darkred}{rgb}{0.55,0.2,0.2}

%hyperref Link color
\usepackage[colorlinks=true,
        linkcolor=darkblue,
        citecolor=darkblue,
        filecolor=darkblue,
        pagecolor=darkblue,
        urlcolor=darkblue,
        bookmarks=true,
        bookmarksopen=true,
        bookmarksopenlevel=3,
        plainpages=false,
        pdfpagelabels=true]{hyperref}

\titleformat{\chapter}[display]{\color{darkred}\normalfont\huge\bfseries}{\chaptertitlename\
\thechapter}{20pt}{\Huge}

\titleformat{\section}{\color{darkblue}\normalfont\Large\bfseries}{\thesection}{1em}{}
\titleformat{\subsection}{\color{verydarkblue}\normalfont\large\bfseries}{\thesubsection}{1em}{}

% Notiz Box
\usepackage{fancybox}
\newcommand{\notiz}[1]{\vspace{5mm}\ovalbox{\begin{minipage}{1\textwidth}#1\end{minipage}}\vspace{5mm}}

\usepackage{cancel}
\setcounter{secnumdepth}{3}
\setcounter{tocdepth}{3}





%-------------------------------------------------------------------------------
%Diff-Makro:
%Das Diff-Makro stellt einen Differentialoperator da.
%
%Benutzung:
% \diff  ->  d
% \diff f  ->  df
% \diff^2 f  ->  d^2 f
% \diff_x  ->  d/dx
% \diff^2_x  ->  d^2/dx^2
% \diff f_x  ->  df/dx
% \diff^2 f_x  ->  d^2f/dx^2
% \diff^2{f(x^5)}_x  ->  d^2(f(x^5))/dx^2
%
%Ersetzt man \diff durch \pdiff, so wird der partieller
%Differentialoperator dargestellt.
%
\makeatletter
\def\diff@n^#1{\@ifnextchar{_}{\diff@n@d^#1}{\diff@n@fun^#1}}
\def\diff@n@d^#1_#2{\frac{\textrm{d}^#1}{\textrm{d}#2^#1}}
\def\diff@n@fun^#1#2{\@ifnextchar{_}{\diff@n@fun@d^#1#2}{\textrm{d}^#1#2}}
\def\diff@n@fun@d^#1#2_#3{\frac{\textrm{d}^#1 #2}{\textrm{d}#3^#1}}
\def\diff@one@d_#1{\frac{\textrm{d}}{\textrm{d}#1}}
\def\diff@one@fun#1{\@ifnextchar{_}{\diff@one@fun@d #1}{\textrm{d}#1}}
\def\diff@one@fun@d#1_#2{\frac{\textrm{d}#1}{\textrm{d}#2}}
\newcommand*{\diff}{\@ifnextchar{^}{\diff@n}
  {\@ifnextchar{_}{\diff@one@d}{\diff@one@fun}}}
%
%Partieller Diff-Operator.
\def\pdiff@n^#1{\@ifnextchar{_}{\pdiff@n@d^#1}{\pdiff@n@fun^#1}}
\def\pdiff@n@d^#1_#2{\frac{\partial^#1}{\partial#2^#1}}
\def\pdiff@n@fun^#1#2{\@ifnextchar{_}{\pdiff@n@fun@d^#1#2}{\partial^#1#2}}
\def\pdiff@n@fun@d^#1#2_#3{\frac{\partial^#1 #2}{\partial#3^#1}}
\def\pdiff@one@d_#1{\frac{\partial}{\partial #1}}
\def\pdiff@one@fun#1{\@ifnextchar{_}{\pdiff@one@fun@d #1}{\partial#1}}
\def\pdiff@one@fun@d#1_#2{\frac{\partial#1}{\partial#2}}
\newcommand*{\pdiff}{\@ifnextchar{^}{\pdiff@n}
  {\@ifnextchar{_}{\pdiff@one@d}{\pdiff@one@fun}}}
\makeatother
%
%Das gleich nur mit etwas andere Syntax. Die Potenz der Differentiation wird erst
%zum Schluss angegeben. Somit lautet die Syntax:
%
% \diff_x^2  ->  d^2/dx^2
% \diff f_x^2  ->  d^2f/dx^2
% \diff{f(x^5)}_x^2  ->  d^2(f(x^5))/dx^2
% Ansonsten wie Oben.
%
%Ersetzt man \diff durch \pdiff, so wird der partieller
%Differentialoperator dargestellt.
%
%\makeatletter
%\def\diff@#1{\@ifnextchar{_}{\diff@fun#1}{\textrm{d} #1}}
%\def\diff@one_#1{\@ifnextchar{^}{\diff@n{#1}}%
%  {\frac{\textrm d}{\textrm{d} #1}}}
%\def\diff@fun#1_#2{\@ifnextchar{^}{\diff@fun@n#1_#2}%
%  {\frac{\textrm d #1}{\textrm{d} #2}}}
%\def\diff@n#1^#2{\frac{\textrm d^#2}{\textrm{d}#1^#2}}
%\def\diff@fun@n#1_#2^#3{\frac{\textrm d^#3 #1}%
%  {\textrm{d}#2^#3}}
%\def\diff{\@ifnextchar{_}{\diff@one}{\diff@}}
%\newcommand*{\diff}{\@ifnextchar{_}{\diff@one}{\diff@}}
%
%Partieller Diff-Operator.
%\def\pdiff@#1{\@ifnextchar{_}{\pdiff@fun#1}{\partial #1}}
%\def\pdiff@one_#1{\@ifnextchar{^}{\pdiff@n{#1}}%
%  {\frac{\partial}{\partial #1}}}
%\def\pdiff@fun#1_#2{\@ifnextchar{^}{\pdiff@fun@n#1_#2}%
%  {\frac{\partial #1}{\partial #2}}}
%\def\pdiff@n#1^#2{\frac{\partial^#2}{\partial #1^#2}}
%\def\pdiff@fun@n#1_#2^#3{\frac{\partial^#3 #1}%
%  {\partial #2^#3}}
%\newcommand*{\pdiff}{\@ifnextchar{_}{\pdiff@one}{\pdiff@}}
%\makeatother

%-------------------------------------------------------------------------------
%%Nützliche Makros um in der Quantenmechanik Bras, Kets und das Skalarprodukt
%%zwischen den beiden darzustellen.
%%Benutzung:
%% \bra{x}  ->    < x |
%% \ket{x}  ->    | x >
%% \braket{x}{y} ->   < x | y >

\newcommand\bra[1]{\left\langle #1 \right|}
\newcommand\ket[1]{\left| #1 \right\rangle}
\newcommand\braket[2]{%
  \left\langle #1\vphantom{#2} \right.%
  \left|\vphantom{#1#2}\right.%
  \left. \vphantom{#1}#2 \right\rangle}%

%-------------------------------------------------------------------------------
%%Aus dem Buch:
%%Titel:  Latex in Naturwissenschaften und Mathematik
%%Autor:  Herbert Voß
%%Verlag: Franzis Verlag, 2006
%%ISBN:   3772374190, 9783772374197
%%
%%Hier werden drei Makros definiert:\mathllap, \mathclap und \mathrlap, welche
%%analog zu den aus Latex bekannten \rlap und \llap arbeiten, d.h. selbst
%%keinerlei horizontalen Platz benötigen, aber dennoch zentriert zum aktuellen
%%Punkt erscheinen.

\newcommand*\mathllap{\mathstrut\mathpalette\mathllapinternal}
\newcommand*\mathllapinternal[2]{\llap{$\mathsurround=0pt#1{#2}$}}
\newcommand*\clap[1]{\hbox to 0pt{\hss#1\hss}}
\newcommand*\mathclap{\mathpalette\mathclapinternal}
\newcommand*\mathclapinternal[2]{\clap{$\mathsurround=0pt#1{#2}$}}
\newcommand*\mathrlap{\mathpalette\mathrlapinternal}
\newcommand*\mathrlapinternal[2]{\rlap{$\mathsurround=0pt#1{#2}$}}

%%Das Gleiche nur mit \def statt \newcommand.
%\def\mathllap{\mathpalette\mathllapinternal}
%\def\mathllapinternal#1#2{%
%  \llap{$\mathsurround=0pt#1{#2}$}% $
%}
%\def\clap#1{\hbox to 0pt{\hss#1\hss}}
%\def\mathclap{\mathpalette\mathclapinternal}
%\def\mathclapinternal#1#2{%
%  \clap{$\mathsurround=0pt#1{#2}$}%
%}
%\def\mathrlap{\mathpalette\mathrlapinternal}
%\def\mathrlapinternal#1#2{%
%  \rlap{$\mathsurround=0pt#1{#2}$}% $
%}

%-------------------------------------------------------------------------------
%%Hier werden zwei neue Makros definiert \overbr und \underbr welche analog zu
%%\overbrace und \underbrace funktionieren jedoch die Gleichung nicht
%%'zerreißen'. Dies wird ermöglicht durch das \mathclap Makro.

\def\overbr#1^#2{\overbrace{#1}^{\mathclap{#2}}}
\def\underbr#1_#2{\underbrace{#1}_{\mathclap{#2}}}
%\includegraphics[width=0.75\textwidth]{thepic.png}

\begin{document}

\section*{Landau Niveaus}

Wir betrachten ein Teilchen im Magnetfeld. Das konstante Magnetfeld zeigt in z-Richtung \(\vec B=(0,0,B_0)\). Das Vektorpotential ist nach Landau-Eichung (\(\nabla\cdot\vec A = 0\)) somit \(\vec A = (-yB_0,0,0)\). Der Hamiltonoperator lautet:

\[H=\frac{1}{2m}\left(\vec p - \frac{q}{c}\vec A\right)^2 \]


In Quantenmechanischer Schreibweise:

\[H=\frac{1}{2m}\left(\frac{\hbar}{i}\nabla - \frac{q}{c}\vec A\right)^2 \]

Einsetzen des Hamiltonoperators in die Schrödinger-Gleichung:

\[H\psi = E\psi\]

\[ \frac{1}{2m}\left(\frac{\hbar}{i}\nabla - \frac{q}{c}\vec A\right)\left(\frac{\hbar}{i}\nabla - \frac{q}{c}\vec A\right)\psi = E\psi \]

\[ \frac{1}{2m}\left(-\hbar^2\nabla^2 -\frac{\hbar q}{ic}\nabla \vec A -\frac{\hbar q}{ic} \vec A \nabla  + \frac{q^2}{c^2}\vec A^2 \right) \psi = E\psi \]


\[ \frac{1}{2m}\left(-\hbar^2\nabla^2\psi -\frac{\hbar q}{ic}\underbrace{\nabla \vec A\psi}_{(\nabla \vec A)\psi + \vec A\cdot(\nabla\psi)} -\frac{\hbar q}{ic} \vec A \nabla\psi  + \frac{q^2}{c^2}\vec A^2\psi \right) = E\psi \]

Nach der Landau Eichung ist der Term \((\nabla \vec A)\psi = 0\)

\[ \frac{1}{2m}\left(-\hbar^2\nabla^2\psi -\frac{\hbar q}{ic}\vec A\cdot(\nabla\psi) -\frac{\hbar q}{ic} \vec A \nabla\psi  + \frac{q^2}{c^2}\vec A^2\psi \right) = E\psi \]

\[ \frac{1}{2m}\left(-\hbar^2\nabla^2\psi - \frac{2 \hbar q}{ic}\vec A\cdot(\nabla\psi) + \frac{q^2}{c^2}\vec A^2\psi \right) = E\psi \]


Einsetzen des Vektorfeldes ergibt nur noch eine Ableitung in die x-Richtung:


\[ \frac{1}{2m}\left(-\hbar^2\nabla^2\psi + \frac{2 \hbar q}{ic} yB_0 \frac{\partial}{\partial x} \psi + \frac{q^2}{c^2}y^2B_0^2\psi \right) = E\psi \]

Mit der Annahme vom \(\psi(x,y,z) = e^{\frac{i}{\hbar}p_xx}\cdot \phi(y)\cdot e^{\frac{i}{\hbar}p_zz}\) können wir erstmal \(\nabla^2\psi\) ausrechnen NR:

\[\nabla\psi = \begin{pmatrix} \frac{ip_x}{\hbar}e^{ip_xx/\hbar}\cdot\phi(y)e^{\frac{i}{\hbar}p_zz} \\ e^{ip_xx/\hbar}\cdot\frac{\partial}{\partial y}\phi(y)e^{\frac{i}{\hbar}p_zz}\\\frac{ip_z}{\hbar}e^{ip_xx/\hbar}\cdot\phi(y)e^{\frac{i}{\hbar}p_zz} \end{pmatrix}  \]
\[\nabla^2\psi =- \frac{p_x^2}{\hbar^2}e^{ip_xx/\hbar}\phi(y)e^{\frac{i}{\hbar}p_zz} + e^{ip_xx/\hbar}\frac{\partial^2}{\partial y^2}\phi(y)e^{\frac{i}{\hbar}p_zz}  - \frac{p_z^2}{\hbar^2}e^{ip_xx/\hbar}\phi(y)e^{\frac{i}{\hbar}p_zz}\]

Ebenso die Ableitung nach x NR:

\[\frac{\partial}{\partial x}\psi =\frac{i}{\hbar}p_x e^{\frac{i}{\hbar}p_xx}\cdot \phi(y) e^{\frac{i}{\hbar}p_zz} \]


Die NR in wieder in die SGL einsetzen:


\[\begin{split} \frac{1}{2m}\left(p_x^2e^{ip_xx/\hbar}\phi(y)e^{\frac{i}{\hbar}p_zz} -\hbar^2e^{ip_xx/\hbar}\frac{\partial^2}{\partial y^2}\phi(y)e^{\frac{i}{\hbar}p_zz} + p_z^2e^{ip_xx/\hbar}\phi(y)e^{\frac{i}{\hbar}p_zz}  \right. \\
\left.  + \frac{2 \hbar q}{ic} yB_0\frac{i}{\hbar}p_x e^{\frac{i}{\hbar}p_xx}\cdot \phi(y)e^{\frac{i}{\hbar}p_zz}  + \frac{q^2}{c^2}y^2B_0^2\psi \right) = E\psi \end{split} \]

Gleichung durch \( e^{\frac{i}{\hbar}p_0x} e^{\frac{i}{\hbar}p_0z} \) teilen:

\[ \frac{1}{2m}\left(p_x^2\phi(y) - \hbar^2 \frac{\partial^2}{\partial y^2}\phi(y) + p_z^2\phi(y) + \frac{2 q}{c} yB_0p_x  \phi(y)  + \frac{q^2}{c^2}y^2B_0^2\phi(y) \right) = E \phi(y) \]


\[  \frac{1}{2m}\left(- \hbar^2\frac{\partial^2}{\partial y^2}\phi(y) +  \underbrace{\left[p_x^2 + \frac{2 q}{c} yB_0p_x   + \frac{q^2}{c^2}y^2B_0^2\right]}_{\text{Binomische Formel}}\phi(y) + p_z^2\phi(y) \right) = E \phi(y) \]

\[  \frac{1}{2m}\left(-\hbar^2 \frac{\partial^2}{\partial y^2}\phi(y) +  (\frac{q}{c}B_0\cdot y +  p_x )^2\phi(y) + p_z^2\phi(y) \right) = E \phi(y) \]


\[ - \hbar^2\frac{1}{2m} \frac{\partial^2}{\partial y^2}\phi(y) + \frac{q^2 B_0^2}{2mc^2} (y +  \frac{p_xc}{qB_0} )^2\phi(y) +\frac{p_z^2}{2m}\phi(y)  = E \phi(y) \]


Mit der Zyklotronfrequenz \( \omega_c = \frac{qB_0}{cm} \) und \(y_0 = \frac{p_x}{m\omega_c}\)

\[  - \hbar^2\left( \underbrace{ \frac{1}{2m} \frac{\partial^2}{\partial y^2} + \frac{m\omega_c^2}{2} (y +  y_0 )^2}_{\text{Harmonischer Oszillator}} +\underbrace{ \frac{p_z^2}{2m}}_{\text{freies Teilchen}}  \right)\phi(y) = E \phi(y) \]


Vergleiche mit dem Harmonischen Oszillator in y-Richtung: \( H = \frac{p^2_y}{2m} + \frac{m\omega^2}{2}y^2 \) Ergeben sich die Eigenwerte dazu:

\begin{equation}
\label{eq:42}
 \boxed{E_n^{\text{Landau}} = \hbar\omega_c(n+\frac{1}{2}) + \frac{p_z^2}{2m} }
\end{equation}
\\
Wir haben eine Wellenfunktion die in x und z Richtung eine Ebene Welle ist und in y Richtung die Lösung eines Harmonischen Oszillators darstellt. Die Auszeichnung der y Richtung liegt an der von uns gewählten Eichung bzw. dem gewählten Vektorpotential  \(\vec A = (-yB_0,0,0)\). Wie auch immer man die Eichung wählt, das entscheidende ist, dass das Magneteld die Ausbreitung der freier ebener Wellen nur in Feldrichtung erlaubt. In unserem Fall ist es die z Richtung. Wie man aus der Gleichung \eqref{eq:42} sieht, sind die Energieeigenwerte unendlichfach entartet. Nur die x-y-Ebene ist quantisiert.




\end{document}
